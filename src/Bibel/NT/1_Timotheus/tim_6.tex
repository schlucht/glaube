\subsection*{Kapitel 6}
\addcontentsline{toc}{subsection}{Kapitel 6}
\verstab{1}{Schlachter}{Diejenigen, die als Knechte unter dem Joch sind, sollen ihre eigenen Herren aller Ehre wert halten, damit nicht der Name Gottes und die Lehre verlästert werden.}
\verstab{2}{Schlachter}{Die aber, welche gläubige Herren haben, sollen diese darum nicht gering schätzen, weil sie Brüder sind, sondern ihnen umso lieber dienen, weil es Gläubige und Geliebte sind, die darauf bedacht sind, Gutes zu tun. Dies sollst du lehren und dazu ermahnen!}
\verstab{3}{Schlachter}{Wenn jemand fremde Lehren verbreitet und nicht die gesunden Worte unseres Herrn Jesus Christus annimmt und die Lehre, die der Gottesfurcht entspricht,}
\verstab{4}{Schlachter}{so ist er aufgeblasen und versteht doch nichts, sondern krankt an Streitfragen und Wortgefechten, woraus Neid, Zwietracht, Lästerung, böse Verdächtigungen entstehen,}
\verstab{5}{Schlachter}{unnütze Streitgespräche von Menschen, die eine verdorbene Gesinnung haben und der Wahrheit beraubt sind und meinen, die Gottesfurcht sei ein Mittel zur Bereicherung — von solchen halte dich fern!}
\verstab{6}{Schlachter}{Es ist allerdings die Gottesfurcht eine große Bereicherung, wenn sie mit Genügsamkeit verbunden wird.}
\verstab{7}{Schlachter}{Denn wir haben nichts in die Welt hineingebracht, und es ist klar, dass wir auch nichts hinausbringen können.}
\verstab{8}{Schlachter}{Wenn wir aber Nahrung und Kleidung haben, soll uns das genügen!}
\verstab{9}{Schlachter}{enn die, welche reich werden wollen, fallen in Versuchung und Fallstricke und viele törichte und schädliche Begierden, welche die Menschen in Untergang und Verderben stürzen.}
\verstab{10}{Schlachter}{Denn die Geldgier ist eine Wurzel alles Bösen; etliche, die sich ihr hingegeben haben, sind vom Glauben abgeirrt und haben sich selbst viel Schmerzen verursacht.}
\verstab{11}{Schlachter}{Du aber, o Mensch Gottes, fliehe diese Dinge, jage aber nach Gerechtigkeit, Gottesfurcht, Glauben, Liebe, Geduld, Sanftmut!}
\verstab{12}{Schlachter}{Kämpfe den guten Kampf des Glaubens; ergreife das ewige Leben, zu dem du auch berufen bist und worüber du das gute Bekenntnis vor vielen Zeugen abgelegt hast.}
\verstab{13}{Schlachter}{Ich gebiete dir vor Gott, der alles lebendig macht, und vor Christus Jesus, der vor Pontius Pilatus das gute Bekenntnis bezeugt hat,}
\verstab{14}{Schlachter}{dass du das Gebot unbefleckt und untadelig bewahrst bis zur Erscheinung unseres Herrn Jesus Christus,}
\verstab{15}{Schlachter}{welche zu seiner Zeit zeigen wird der Glückselige und allein Gewaltige, der König der Könige und der Herr der Herrschenden,}
\verstab{16}{Schlachter}{der allein Unsterblichkeit hat, der in einem unzugänglichen Licht wohnt, den kein Mensch gesehen hat noch sehen kann; ihm sei Ehre und ewige Macht! Amen.}
\verstab{17}{Schlachter}{Den Reichen in der jetzigen Weltzeit gebiete, nicht hochmütig zu sein, auch nicht ihre Hoffnung auf die Unbeständigkeit des Reichtums zu setzen, sondern auf den lebendigen Gott, der uns alles reichlich zum Genuss darreicht.}
\verstab{18}{Schlachter}{Sie sollen Gutes tun, reich werden an guten Werken, freigebig sein, bereit, mit anderen zu teilen,}
\verstab{19}{Schlachter}{damit sie das ewige Leben ergreifen und so für sich selbst eine gute Grundlage für die Zukunft sammeln.}
\verstab{20}{Schlachter}{O Timotheus, bewahre das anvertraute Gut, meide das unheilige, nichtige Geschwätz und die Widersprüche der fälschlich so genannten »Erkenntnis«!}
\verstab{21}{Schlachter}{Zu dieser haben sich etliche bekannt und haben darüber das Glaubensziel verfehlt. Die Gnade sei mit dir! Amen.}