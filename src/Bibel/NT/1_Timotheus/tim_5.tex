\subsection*{Kapitel 5}
\addcontentsline{toc}{subsection}{Kapitel 5}
\verstab{1}{Schlachter}{Einen älteren Mann fahre nicht hart an, sondern ermahne ihn wie einen Vater, jüngere wie Brüder,}
\verstab{2}{Schlachter}{ältere Frauen wie Mütter, jüngere wie Schwestern, in aller Keuschheit.}
\verstab{3}{Schlachter}{Ehre die Witwen, die wirklich Witwen sind.}
\verstab{4}{Schlachter}{Wenn aber eine Witwe Kinder oder Enkel hat, so sollen diese zuerst lernen, am eigenen Haus gottesfürchtig zu handeln und den Eltern Empfangenes zu vergelten; denn das ist gut und wohlgefällig vor Gott.}
\verstab{5}{Schlachter}{Eine wirkliche und vereinsamte Witwe aber hat ihre Hoffnung auf Gott gesetzt und bleibt beständig im Flehen und Gebet Tag und Nacht;}
\verstab{6}{Schlachter}{eine genusssüchtige jedoch ist lebendig tot.}
\verstab{7}{Schlachter}{Sprich das offen aus, damit sie untadelig sind!}
\verstab{8}{Schlachter}{Wenn aber jemand für die Seinen, besonders für seine Hausgenossen, nicht sorgt, so hat er den Glauben verleugnet und ist schlimmer als ein Ungläubiger.}
\verstab{9}{Schlachter}{Eine Witwe soll nur in die Liste eingetragen werden, wenn sie nicht weniger als 60 Jahre alt ist, die Frau eines Mannes war}
\verstab{10}{Schlachter}{und ein Zeugnis guter Werke hat; wenn sie Kinder aufgezogen, Gastfreundschaft geübt, die Füße der Heiligen gewaschen, Bedrängten geholfen hat, wenn sie sich jedem guten Werk gewidmet hat.}
\verstab{11}{Schlachter}{Jüngere Witwen aber weise ab; denn wenn sie gegen [den Willen des] Christus begehrlich geworden sind, wollen sie heiraten}
\verstab{12}{Schlachter}{und kommen [damit] unter das Urteil, dass sie die erste Treue gebrochen haben.}
\verstab{13}{Schlachter}{Zugleich lernen sie auch untätig zu sein, indem sie in den Häusern herumlaufen; und nicht nur untätig, sondern auch geschwätzig und neugierig zu sein; und sie reden, was sich nicht gehört.}
\verstab{14}{Schlachter}{So will ich nun, dass jüngere [Witwen] heiraten, Kinder gebären, den Haushalt führen und dem Widersacher keinen Anlass zur Lästerung geben;}
\verstab{15}{Schlachter}{denn etliche haben sich schon abgewandt, dem Satan nach.}
\verstab{16}{Schlachter}{Wenn ein Gläubiger oder eine Gläubige Witwen hat, so soll er sie versorgen, und die Gemeinde soll nicht belastet werden, damit diese für die wirklichen Witwen sorgen kann.}
\verstab{17}{Schlachter}{Die Ältesten, die gut vorstehen, sollen doppelter Ehre wertgeachtet werden, besonders die, welche im Wort und in der Lehre arbeiten.}
\verstab{18}{Schlachter}{Denn die Schrift sagt: »Du sollst dem Ochsen nicht das Maul verbinden, wenn er drischt!«, und »Der Arbeiter ist seines Lohnes wert«.}
\verstab{19}{Schlachter}{Gegen einen Ältesten nimm keine Klage an, außer aufgrund von zwei oder drei Zeugen.}
\verstab{20}{Schlachter}{Die, welche sündigen, weise zurecht vor allen, damit sich auch die anderen fürchten.}
\verstab{21}{Schlachter}{Ich ermahne dich ernstlich vor Gott und dem Herrn Jesus Christus und den auserwählten Engeln, dass du dies ohne Vorurteil befolgst und nichts aus Zuneigung tust!}
\verstab{22}{Schlachter}{Die Hände lege niemand schnell auf, mache dich auch nicht fremder Sünden teilhaftig; bewahre dich selbst rein!}
\verstab{23}{Schlachter}{Trinke nicht mehr nur Wasser, sondern gebrauche ein wenig Wein um deines Magens willen und wegen deines häufigen Unwohlseins.}
\verstab{24}{Schlachter}{Die Sünden mancher Menschen sind allen offenbar und kommen vorher ins Gericht; manchen aber folgen sie auch nach.}
\verstab{25}{Schlachter}{Gleicherweise sind auch die guten Werke allen offenbar; und die, mit welchen es sich anders verhält, können auch nicht verborgen bleiben.}