\author{OTS}
\documentclass[14pt]{../../inc/mybib}

\setincpath{../../inc/}

\usepackage{bible_style}
\graphicspath{{../../assets/images/}}
\usepackage{header}
\usepackage{changepage}
%%%%%%%%%%%%%%%%%%%%%%%%%%%%%%%%%%%%%%%%%%%%%%%%%%%%%%%%
%%%%%%%%%%%%%%%%%%%%%%%%%%%%%%%%%%%%%%%%%%%%%%%%%%%%%%%%
\usepackage{tabularx}
\usepackage{xparse}
\usepackage{calc}
\usepackage{expl3}

% Konfigurierbares Layout
\newcommand{\VersNumberBox}[1]{\makebox[0em][r]{\textsuperscript{#1}}}
\newcommand{\VersSourceFmt}[1]{[\textsc{#1}] }

\ExplSyntaxOn
\clist_new:N \g_lothar_versions_clist
\tl_new:N \g_lothar_default_tl

% Setze die Default-Bibel
\NewDocumentCommand{\setdefault}{m}
 {
   \tl_gset:Nn \g_lothar_default_tl { #1 }
 }

% Setze die Auswahl-Liste
\NewDocumentCommand{\setversions}{m}
 {
   \clist_gset:Nn \g_lothar_versions_clist { #1 }
 }
 \NewDocumentCommand{\getdefault}{}{\tl_use:N \g_lothar_default_tl}
 \NewDocumentCommand{\getversions}{}{\clist_use:Nn \g_lothar_versions_clist {, }}
% Prüfen, ob Version gesetzt werden soll
\prg_new_conditional:Npnn \lothar_version_selected_p:n #1 { p, T, F, TF }
 {
   % Wenn Liste leer -> nur Default anzeigen
   \clist_if_empty:NTF \g_lothar_versions_clist
     {
       \tl_if_eq:NnTF \g_lothar_default_tl { #1 }
         { \prg_return_true: }
         { \prg_return_false: }
     }
     {
       % Wenn "ALL" in Liste -> alle anzeigen
       \clist_if_in:NnTF \g_lothar_versions_clist { ALL }
         { \prg_return_true: }
         {
           % Sonst prüfen, ob Version in Liste ist
           \clist_if_in:NnTF \g_lothar_versions_clist { #1 }
             { \prg_return_true: } { \prg_return_false: }
         }
     }
 }
% Prüfen, ob #1 die Default-Version ist
\prg_new_conditional:Npnn \lothar_is_default_p:n #1 { p, T, F, TF }
  {
    \tl_if_eq:NnTF \g_lothar_default_tl { #1 }
      { \prg_return_true: } { \prg_return_false: }
  }
% Makro für die Ausgabe
\NewDocumentCommand{\verstab}{ m m +m }
 {
   \lothar_version_selected_p:nTF { #2 }
     {
       \noindent
       \begin{tabularx}{\linewidth}{@{} l X@{}}        
         \VersNumberBox{#1} & 
         \lothar_is_default_p:nTF{#2}
         {#3}         
         {\VersSourceFmt{#2}  #3}
       \end{tabularx}\par
     }
     {        
     }
 }
\bool_new:N \l_color_verbN_bool
\bool_new:N \l_color_verbI_bool
\bool_new:N \l_color_verbP_bool
\bool_new:N \l_color_person_bool
\bool_new:N \l_color_ort_bool
\bool_new:N \l_color_bindW_bool

\bool_set_true:N \l_color_verbN_bool
\bool_set_true:N \l_color_verbI_bool
\bool_set_true:N \l_color_verbP_bool
\bool_set_true:N \l_color_person_bool
\bool_set_true:N \l_color_ort_bool
\bool_set_true:N \l_color_bindW_bool

\NewDocumentCommand{\verbN}{m}{
  \bool_if:NTF \l_color_verbN_bool
    {\colorbox{green}{\textcolor{black}{#1}}}
    {#1}
}
\NewDocumentCommand{\verbI}{m}{
  \bool_if:NTF \l_color_verbI_bool
    {\colorbox{yellow}{\textcolor{black}{#1}}}
    {#1}
}
\NewDocumentCommand{\verbP}{m}{
  \bool_if:NTF \l_color_verbP_bool
    {\verbN{#1}\textsuperscript{\textbf{P}}}
    {#1}
}
\NewDocumentCommand{\ort}{m}{
  \bool_if:NTF \l_color_ort_bool
    {\colorbox{magenta}{\textcolor{white}{#1}}}
    {#1}
}
\NewDocumentCommand{\person}{m}{
  \bool_if:NTF \l_color_person_bool
    {\colorbox{blue}{\textcolor{white}{#1}}}
    {#1}
}
\NewDocumentCommand{\bindW}{m}{
  \bool_if:NTF \l_color_bindW_bool
    {\colorbox{red}{\textcolor{white}{#1}}}
    {#1}
}
\ExplSyntaxOff


% Optional: Absatzformatierung für schönere Blockoptik
\setlength{\parindent}{0pt}
\setlength{\parskip}{0.4em}
%%%%%%%%%%%%%%%%%%%%%%%%%%%%%%%%%%%%%%%%%%%%%%%%%%%%%%%%%%%%%%%%%
%%%%%%%%%%%%%%%%%%%%%%%%%%%%%%%%%%%%%%%%%%%%%%%%%%%%%%%%%%%%%%%%%
\newcommand{\hr}{\par\noindent\hrulefill\par}
\newcommand{\lineheight}[1]{\setlength{\baselineskip}{{#1}\baselineskip}}
\newcommand{\mySpacing}{1}
\newcommand{\myRight}{55mm}

\setlength{\footheight}{29pt} % oder etwas mehr, je nach Boxhöhe
\setlength{\footskip}{40pt}   % optional, falls Abstand zum Text zu klein
\author{Lothar Schmid}
\setlength{\baselineskip}{3.5\baselineskip}
\begin{document}
\tableofcontents
\newpage
\begin{adjustwidth*}{0mm}{\myRight} % [linke Einrückung]{rechte Einrückung}
    \leavevmode
    \begin{spacing}{\mySpacing} 
        %%%%%%%%%%%%%%%%%%%%%%%%%%%%%%%%%%%%%%%%%%%% 
        \section{Matthäus}
        \setdefault{Schlachter}
        \setversions{Schlachter}
        Standart-Bibel: \getdefault \\
        Ausdruch: \getversions \\        
        \subsection*{Kapitel 1}
\addcontentsline{toc}{subsection}{Kapitel 1}
\verstab{1}{Schlachter}{Geschlechtsregister Jesu Christi, des Sohnes Davids, des Sohnes Abrahams.}
\verstab{2}{Schlachter}{Abraham zeugte den Isaak; Isaak zeugte den Jakob; Jakob zeugte den Juda und seine Brüder;}
\verstab{3}{Schlachter}{Juda zeugte den Perez und den Serach mit der Tamar; Perez zeugte den Hezron; Hezron zeugte den Aram;}
\verstab{4}{Schlachter}{Aram zeugte den Amminadab; Amminadab zeugte den Nachschon; Nachschon zeugte den Salmon;}
\verstab{5}{Schlachter}{Salmon zeugte den Boas mit der Rahab; Boas zeugte den Obed mit der Ruth; Obed zeugte den Isai;}
\verstab{6}{Schlachter}{Isai zeugte den König David. Der König David zeugte den Salomo mit der Frau des Uria;}
\verstab{7}{Schlachter}{Salomo zeugte den Rehabeam; Rehabeam zeugte den Abija; Abija zeugte den Asa;}
\verstab{8}{Schlachter}{Asa zeugte den Josaphat; Josaphat zeugte den Joram; Joram zeugte den Usija;}
\verstab{9}{Schlachter}{Usija zeugte den Jotam; Jotam zeugte den Ahas; Ahas zeugte den Hiskia;}
\verstab{10}{Schlachter}{ Hiskia zeugte den Manasse; Manasse zeugte den Amon; Amon zeugte den Josia;}
\verstab{11}{Schlachter}{Josia zeugte den Jechonja und dessen Brüder zur Zeit der Wegführung nach Babylon.}
\verstab{12}{Schlachter}{Nach der Wegführung nach Babylon zeugte Jechonja den Schealtiel; Schealtiel zeugte den Serubbabel;}
\verstab{13}{Schlachter}{Serubbabel zeugte den Abihud; Abihud zeugte den Eljakim; Eljakim zeugte den Asor;}
\verstab{14}{Schlachter}{Asor zeugte den Zadok; Zadok zeugte den Achim; Achim zeugte den Eliud;}
\verstab{15}{Schlachter}{Eliud zeugte den Eleasar; Eleasar zeugte den Mattan; Mattan zeugte den Jakob;}
\verstab{16}{Schlachter}{Jakob zeugte den Joseph, den Mann der Maria, von welcher Jesus geboren ist, der Christus genannt wird.}
\verstab{17}{Schlachter}{So sind es nun von Abraham bis zu David insgesamt vierzehn Generationen und von David bis zur Wegführung nach Babylon vierzehn Generationen und von der Wegführung nach Babylon bis zu Christus vierzehn Generationen.}
\verstab{18}{Schlachter}{Die Geburt Jesu Christi aber geschah auf diese Weise: Als nämlich seine Mutter Maria mit Joseph verlobt war, noch ehe sie zusammengekommen waren, erwies es sich, dass sie vom Heiligen Geist schwanger geworden war.}
\verstab{19}{Schlachter}{Aber Joseph, ihr Mann, der gerecht war und sie doch nicht der öffentlichen Schande preisgeben wollte, gedachte sie heimlich zu entlassen.}
\verstab{20}{Schlachter}{Während er aber dies im Sinn hatte, siehe, da erschien ihm ein Engel des Herrn im Traum, der sprach: Joseph, Sohn Davids, scheue dich nicht, Maria, deine Frau, zu dir zu nehmen; denn was in ihr gezeugt ist, das ist vom Heiligen Geist.}
\verstab{21}{Schlachter}{Sie wird aber einen Sohn gebären, und du sollst ihm den Namen Jesus[3] geben, denn er wird sein Volk erretten von ihren Sünden.}
\verstab{22}{Schlachter}{Dies alles aber ist geschehen, damit erfüllt würde, was der Herr durch den Propheten geredet hat, der spricht:}
\verstab{23}{Schlachter}{»Siehe, die Jungfrau wird schwanger werden und einen Sohn gebären; und man wird ihm den Namen Immanuel geben«[4], das heißt übersetzt: »Gott mit uns«.}
\verstab{24}{Schlachter}{Als nun Joseph vom Schlaf erwachte, handelte er so, wie es ihm der Engel des Herrn befohlen hatte, und nahm seine Frau zu sich;}
\verstab{25}{Schlachter}{und er erkannte sie nicht, bis sie ihren erstgeborenen Sohn geboren hatte; und er gab ihm den Namen Jesus.}
        \subsection*{Kapitel 2}
\addcontentsline{toc}{subsection}{Kapitel 2}
\verstab{1}{Schlachter}{Als nun Jesus geboren war in Bethlehem in Judäa, in den Tagen des Königs Herodes, siehe, da kamen Weise aus dem Morgenland nach Jerusalem,}
\verstab{2}{Schlachter}{die sprachen: Wo ist der neugeborene König der Juden? Denn wir haben seinen Stern im Morgenland gesehen und sind gekommen, um ihn anzubeten!}
\verstab{3}{Schlachter}{Als das der König Herodes hörte, erschrak er, und ganz Jerusalem mit ihm.}
\verstab{4}{Schlachter}{Und er rief alle obersten Priester und Schriftgelehrten des Volkes zusammen und erfragte von ihnen, wo der Christus geboren werden sollte.}
\verstab{5}{Schlachter}{Sie aber sagten ihm: In Bethlehem in Judäa; denn so steht es geschrieben durch den Propheten:}
\verstab{6}{Schlachter}{»Und du, Bethlehem im Land Juda, bist keineswegs die geringste unter den Fürstenstädten Judas; denn aus dir wird ein Herrscher hervorgehen, der mein Volk Israel weiden soll«.}
\verstab{7}{Schlachter}{Da rief Herodes die Weisen heimlich zu sich und erkundigte sich bei ihnen genau nach der Zeit, wann der Stern erschienen war;}
\verstab{8}{Schlachter}{und er sandte sie nach Bethlehem und sprach: Zieht hin und forscht genau nach dem Kind. Und wenn ihr es gefunden habt, so lasst es mich wissen, damit auch ich komme und es anbete!}
\verstab{9}{Schlachter}{Und als sie den König gehört hatten, zogen sie hin. Und siehe, der Stern, den sie im Morgenland gesehen hatten, ging vor ihnen her, bis er ankam und über dem Ort stillstand, wo das Kind war.}
\verstab{10}{Schlachter}{Als sie nun den Stern sahen, wurden sie sehr hocherfreut;}
\verstab{11}{Schlachter}{und sie gingen in das Haus hinein und fanden das Kind samt Maria, seiner Mutter. Da fielen sie nieder und beteten es an; und sie öffneten ihre Schatzkästchen und brachten ihm Gaben: Gold, Weihrauch und Myrrhe.}
\verstab{12}{Schlachter}{Und da sie im Traum angewiesen wurden, nicht wieder zu Herodes zurückzukehren, zogen sie auf einem anderen Weg zurück in ihr Land.}
\verstab{13}{Schlachter}{Als sie aber weggezogen waren, siehe, da erscheint ein Engel des Herrn dem Joseph im Traum und spricht: Steh auf, nimm das Kind und seine Mutter mit dir und fliehe nach Ägypten und bleibe dort, bis ich es dir sage; denn Herodes will das Kind suchen, um es umzubringen!}
\verstab{14}{Schlachter}{Da stand er auf, nahm das Kind und seine Mutter bei Nacht mit sich und entfloh nach Ägypten.}
\verstab{15}{Schlachter}{Und er blieb dort bis zum Tod des Herodes, damit erfüllt würde, was der Herr durch den Propheten geredet hat, der spricht: »Aus Ägypten habe ich meinen Sohn gerufen«.}
\verstab{16}{Schlachter}{Als sich nun Herodes von den Weisen betrogen sah, wurde er sehr zornig, sandte hin und ließ alle Knaben töten, die in Bethlehem und in seinem ganzen Gebiet waren, von zwei Jahren und darunter, nach der Zeit, die er von den Weisen genau erforscht hatte.}
\verstab{17}{Schlachter}{Da wurde erfüllt, was durch den Propheten Jeremia gesagt ist, der spricht:}
\verstab{18}{Schlachter}{»Eine Stimme ist in Rama gehört worden, viel Jammern, Weinen und Klagen; Rahel beweint ihre Kinder und will sich nicht trösten lassen, weil sie nicht mehr sind«.}
\verstab{19}{Schlachter}{Als aber Herodes gestorben war, siehe, da erscheint ein Engel des Herrn dem Joseph in Ägypten im Traum}
\verstab{20}{Schlachter}{und spricht: Steh auf, nimm das Kind und seine Mutter zu dir und zieh in das Land Israel; denn die dem Kind nach dem Leben trachteten, sind gestorben!}
\verstab{21}{Schlachter}{Da stand er auf, nahm das Kind und seine Mutter zu sich und ging in das Land Israel.}
\verstab{22}{Schlachter}{Als er aber hörte, dass Archelaus anstatt seines Vaters Herodes über Judäa regierte, fürchtete er sich, dorthin zu gehen. Und auf eine Anweisung hin, die er im Traum erhielt, zog er weg in das Gebiet Galiläas.}
\verstab{23}{Schlachter}{Und dort angekommen, ließ er sich in einer Stadt namens Nazareth nieder, damit erfüllt würde, was durch die Propheten gesagt ist, dass er ein Nazarener genannt werden wird.}

        \subsection*{Kapitel 3}
\addcontentsline{toc}{subsection}{Kapitel 3}
\verstab{1}{Schlachter}{In jenen Tagen aber erscheint Johannes der Täufer und verkündigt in der Wüste von Judäa}
\verstab{2}{Schlachter}{und spricht: Tut Buße, denn das Reich der Himmel ist nahe herbeigekommen!}
\verstab{3}{Schlachter}{Das ist der, von welchem geredet wurde durch den Propheten Jesaja, der spricht: »Die Stimme eines Rufenden [ertönt] in der Wüste: Bereitet den Weg des Herrn, macht seine Pfade eben!«}
\verstab{4}{Schlachter}{Er aber, Johannes, hatte ein Gewand aus Kamelhaaren und einen ledernen Gürtel um seine Lenden, und seine Speise waren Heuschrecken und wilder Honig.}
\verstab{5}{Schlachter}{Da zog zu ihm hinaus Jerusalem und ganz Judäa und das ganze umliegende Gebiet des Jordan,}
\verstab{6}{Schlachter}{und es wurden von ihm im Jordan getauft, die ihre Sünden bekannten.}
\verstab{7}{Schlachter}{Als er aber viele von den Pharisäern und Sadduzäern zu seiner Taufe kommen sah, sprach er zu ihnen: Schlangenbrut! Wer hat euch eingeredet, ihr könntet dem zukünftigen Zorn entfliehen?}
\verstab{8}{Schlachter}{So bringt nun Früchte, die der Buße würdig sind!}
\verstab{9}{Schlachter}{Und denkt nicht, bei euch selbst sagen zu können: »Wir haben Abraham zum Vater«. Denn ich sage euch: Gott vermag dem Abraham aus diesen Steinen Kinder zu erwecken!}
\verstab{10}{Schlachter}{Es ist aber auch schon die Axt an die Wurzel der Bäume gelegt. Jeder Baum nun, der keine gute Frucht bringt, wird abgehauen und ins Feuer geworfen!}
\verstab{11}{Schlachter}{Ich taufe euch mit Wasser zur Buße; der aber nach mir kommt, ist stärker als ich, sodass ich nicht würdig bin, ihm die Schuhe zu tragen; der wird euch mit Heiligem Geist und Feuer taufen.}
\verstab{12}{Schlachter}{Er hat die Wurfschaufel in seiner Hand und wird seine Tenne gründlich reinigen und seinen Weizen in die Scheune sammeln; die Spreu aber wird er verbrennen mit unauslöschlichem Feuer.}
\verstab{13}{Schlachter}{Da kommt Jesus aus Galiläa an den Jordan zu Johannes, um sich von ihm taufen zu lassen.}
\verstab{14}{Schlachter}{Johannes aber wehrte ihm und sprach: Ich habe es nötig, von dir getauft zu werden, und du kommst zu mir?}
\verstab{15}{Schlachter}{Jesus aber antwortete und sprach zu ihm: Lass es jetzt so geschehen; denn so gebührt es uns, alle Gerechtigkeit zu erfüllen! Da gab er ihm nach.}
\verstab{16}{Schlachter}{Und als Jesus getauft war, stieg er sogleich aus dem Wasser; und siehe, da öffnete sich ihm der Himmel, und er sah den Geist Gottes wie eine Taube herabsteigen und auf ihn kommen.}
\verstab{17}{Schlachter}{Und siehe, eine Stimme [kam] vom Himmel, die sprach: Dies ist mein geliebter Sohn, an dem ich Wohlgefallen habe!}
        \subsection*{Kapitel 4}
\addcontentsline{toc}{subsection}{Kapitel 4}
\verstab{1}{Schlachter}{Darauf wurde Jesus vom Geist in die Wüste geführt, damit er vom Teufel versucht würde.}
\verstab{2}{Schlachter}{Und als er 40 Tage und 40 Nächte gefastet hatte, war er zuletzt hungrig.}
\verstab{3}{Schlachter}{Und der Versucher trat zu ihm und sprach: Wenn du Gottes Sohn bist, so sprich, dass diese Steine Brot werden!}
\verstab{4}{Schlachter}{Er aber antwortete und sprach: Es steht geschrieben: »Der Mensch lebt nicht vom Brot allein, sondern von einem jeden Wort, das aus dem Mund Gottes hervorgeht!«}
\verstab{5}{Schlachter}{Darauf nimmt ihn der Teufel mit sich in die heilige Stadt und stellt ihn auf die Zinne des Tempels}
\verstab{6}{Schlachter}{und spricht zu ihm: Wenn du Gottes Sohn bist, so stürze dich hinab; denn es steht geschrieben: »Er wird seinen Engeln deinetwegen Befehl geben, und sie werden dich auf den Händen tragen, damit du deinen Fuß nicht etwa an einen Stein stößt«.}
\verstab{7}{Schlachter}{Da sprach Jesus zu ihm: Wiederum steht geschrieben: »Du sollst den Herrn, deinen Gott, nicht versuchen!«}
\verstab{8}{Schlachter}{Wiederum nimmt ihn der Teufel mit auf einen sehr hohen Berg und zeigt ihm alle Reiche der Welt und ihre Herrlichkeit}
\verstab{9}{Schlachter}{ und spricht zu ihm: Dieses alles will ich dir geben, wenn du niederfällst und mich anbetest!}
\verstab{10}{Schlachter}{Da spricht Jesus zu ihm: Weiche, Satan! Denn es steht geschrieben: »Du sollst den Herrn, deinen Gott, anbeten und ihm allein dienen!«}
\verstab{11}{Schlachter}{Da verließ ihn der Teufel; und siehe, Engel traten hinzu und dienten ihm.}
\verstab{12}{Schlachter}{Als aber Jesus hörte, dass Johannes gefangen gesetzt worden war, zog er weg nach Galiläa.}
\verstab{13}{Schlachter}{Und er verließ Nazareth, kam und ließ sich in Kapernaum nieder, das am See liegt, im Gebiet von Sebulon und Naphtali,}
\verstab{14}{Schlachter}{damit erfüllt würde, was durch den Propheten Jesaja gesagt ist, der spricht:}
\verstab{15}{Schlachter}{»Das Land Sebulon und das Land Naphtali, am Weg des Sees, jenseits des Jordan, das Galiläa der Heiden}
\verstab{16}{Schlachter}{das Volk, das in der Finsternis wohnte, hat ein großes Licht gesehen, und denen, die im Land des Todesschattens wohnten, ist ein Licht aufgegangen«.}
\verstab{17}{Schlachter}{Von da an begann Jesus zu verkündigen und zu sprechen: Tut Buße, denn das Reich der Himmel ist nahe herbeigekommen!}
\verstab{18}{Schlachter}{Als Jesus aber am See von Galiläa entlangging, sah er zwei Brüder, Simon, genannt Petrus, und dessen Bruder Andreas; die warfen das Netz in den See, denn sie waren Fischer.}
\verstab{19}{Schlachter}{Und er spricht zu ihnen: Folgt mir nach, und ich will euch zu Menschenfischern machen!}
\verstab{20}{Schlachter}{Da verließen sie sogleich die Netze und folgten ihm nach.}
\verstab{21}{Schlachter}{Und als er von dort weiterging, sah er in einem Schiff zwei andere Brüder, Jakobus, den Sohn des Zebedäus, und dessen Bruder Johannes, mit ihrem Vater Zebedäus ihre Netze flicken; und er berief sie.}
\verstab{22}{Schlachter}{Da verließen sie sogleich das Schiff und ihren Vater und folgten ihm nach.}
\verstab{23}{Schlachter}{Und Jesus durchzog ganz Galiläa, lehrte in ihren Synagogen und verkündigte das Evangelium von dem Reich und heilte alle Krankheiten und alle Gebrechen im Volk.}
\verstab{24}{Schlachter}{Und sein Ruf verbreitete sich in ganz Syrien; und sie brachten alle Kranken zu ihm, die von mancherlei Krankheiten und Schmerzen geplagt waren, und Besessene und Mondsüchtige und Lahme; und er heilte sie.}
\verstab{25}{Schlachter}{Und es folgte ihm eine große Volksmenge nach aus Galiläa und aus dem Gebiet der Zehn Städte und aus Jerusalem und Judäa und von jenseits des Jordan.}

        \subsection*{Kapitel 5}
\addcontentsline{toc}{subsection}{Kapitel 5}
\verstab{1}{Schlachter}{Als er aber die Volksmenge sah, stieg er auf den Berg; und als er sich setzte, traten seine Jünger zu ihm.}
\verstab{2}{Schlachter}{Und er tat seinen Mund auf [zu einer Rede], lehrte sie und sprach:}
\verstab{3}{Schlachter}{Glückselig sind die geistlich Armen, denn ihrer ist das Reich der Himmel!}
\verstab{4}{Schlachter}{Glückselig sind die Trauernden, denn sie sollen getröstet werden!}
\verstab{5}{Schlachter}{Glückselig sind die Sanftmütigen, denn sie werden das Land erben!}
\verstab{6}{Schlachter}{Glückselig sind, die nach der Gerechtigkeit hungern und dürsten, denn sie sollen satt werden!}
\verstab{7}{Schlachter}{Glückselig sind die Barmherzigen, denn sie werden Barmherzigkeit erlangen!}
\verstab{8}{Schlachter}{Glückselig sind, die reinen Herzens sind, denn sie werden Gott schauen!}
\verstab{9}{Schlachter}{Glückselig sind die Friedfertigen, denn sie werden Söhne Gottes heißen!}
\verstab{10}{Schlachter}{Glückselig sind, die um der Gerechtigkeit willen verfolgt werden, denn ihrer ist das Reich der Himmel!}
\verstab{11}{Schlachter}{Glückselig seid ihr, wenn sie euch schmähen und verfolgen und lügnerisch jegliches böse Wort gegen euch reden um meinetwillen!}
\verstab{12}{Schlachter}{Freut euch und jubelt, denn euer Lohn ist groß im Himmel; denn ebenso haben sie die Propheten verfolgt, die vor euch gewesen sind.}
\verstab{13}{Schlachter}{Ihr seid das Salz der Erde. Wenn aber das Salz fade wird, womit soll es wieder salzig gemacht werden? Es taugt zu nichts mehr, als dass es hinausgeworfen und von den Leuten zertreten wird.}
\verstab{14}{Schlachter}{Ihr seid das Licht der Welt. Es kann eine Stadt, die auf einem Berg liegt, nicht verborgen bleiben.}
\verstab{15}{Schlachter}{Man zündet auch nicht ein Licht an und setzt es unter den Scheffel, sondern auf den Leuchter; so leuchtet es allen, die im Haus sind.}
\verstab{16}{Schlachter}{So soll euer Licht leuchten vor den Leuten, dass sie eure guten Werke sehen und euren Vater im Himmel preisen.}
\verstab{17}{Schlachter}{Ihr sollt nicht meinen, dass ich gekommen sei, um das Gesetz oder die Propheten aufzulösen. Ich bin nicht gekommen, um aufzulösen, sondern um zu erfüllen!}
\verstab{18}{Schlachter}{Denn wahrlich, ich sage euch: Bis Himmel und Erde vergangen sind, wird nicht ein Buchstabe noch ein einziges Strichlein vom Gesetz vergehen, bis alles geschehen ist.}
\verstab{19}{Schlachter}{Wer nun eines von diesen kleinsten Geboten auflöst und die Leute so lehrt, der wird der Kleinste genannt werden im Reich der Himmel; wer sie aber tut und lehrt, der wird groß genannt werden im Reich der Himmel.}
\verstab{20}{Schlachter}{Denn ich sage euch: Wenn eure Gerechtigkeit die der Schriftgelehrten und Pharisäer nicht weit übertrifft, so werdet ihr gar nicht in das Reich der Himmel eingehen!}
\verstab{21}{Schlachter}{Ihr habt gehört, dass zu den Alten gesagt ist: »Du sollst nicht töten!«, wer aber tötet, der wird dem Gericht verfallen sein.}
\verstab{22}{Schlachter}{Ich aber sage euch: Jeder, der seinem Bruder ohne Ursache zürnt, wird dem Gericht verfallen sein. Wer aber zu seinem Bruder sagt: Raka!, der wird dem Hohen Rat verfallen sein. Wer aber sagt: Du Narr!, der wird dem höllischen Feuer verfallen sein.}
\verstab{23}{Schlachter}{Wenn du nun deine Gabe zum Altar bringst und dich dort erinnerst, dass dein Bruder etwas gegen dich hat,}
\verstab{24}{Schlachter}{so lass deine Gabe dort vor dem Altar und geh zuvor hin und versöhne dich mit deinem Bruder, und dann komm und opfere deine Gabe!}
\verstab{25}{Schlachter}{Sei deinem Widersacher bald geneigt, während du noch mit ihm auf dem Weg bist, damit der Widersacher dich nicht etwa dem Richter ausliefert und der Richter dich dem Gerichtsdiener übergibt und du ins Gefängnis geworfen wirst.}
\verstab{26}{Schlachter}{Wahrlich, ich sage dir: Du wirst von dort nicht herauskommen, bis du den letzten Groschen bezahlt hast!}
\verstab{27}{Schlachter}{Ihr habt gehört, dass zu den Alten gesagt ist: »Du sollst nicht ehebrechen!«}
\verstab{28}{Schlachter}{Ich aber sage euch: Wer eine Frau ansieht, um sie zu begehren, der hat in seinem Herzen schon Ehebruch mit ihr begangen.}
\verstab{29}{Schlachter}{Wenn dir aber dein rechtes Auge ein Anstoß [zur Sünde] wird, so reiß es aus und wirf es von dir! Denn es ist besser für dich, dass eines deiner Glieder verlorengeht, als dass dein ganzer Leib in die Hölle geworfen wird.}
\verstab{30}{Schlachter}{Und wenn deine rechte Hand für dich ein Anstoß [zur Sünde] wird, so haue sie ab und wirf sie von dir! Denn es ist besser für dich, dass eines deiner Glieder verlorengeht, als dass dein ganzer Leib in die Hölle geworfen wird.}
\verstab{31}{Schlachter}{Es ist auch gesagt: »Wer sich von seiner Frau scheidet, der gebe ihr einen Scheidebrief«.}
\verstab{32}{Schlachter}{Ich aber sage euch: Wer sich von seiner Frau scheidet, ausgenommen wegen Unzucht, der macht, dass sie die Ehe bricht. Und wer eine Geschiedene heiratet, der bricht die Ehe.}
\verstab{33}{Schlachter}{Wiederum habt ihr gehört, dass zu den Alten gesagt ist: »Du sollst nicht falsch schwören; du sollst aber dem Herrn deine Schwüre halten«.}
\verstab{34}{Schlachter}{Ich aber sage euch, dass ihr überhaupt nicht schwören sollt, weder bei dem Himmel, denn er ist Gottes Thron,}
\verstab{35}{Schlachter}{noch bei der Erde, denn sie ist der Schemel seiner Füße, noch bei Jerusalem, denn sie ist die Stadt des großen Königs.}
\verstab{36}{Schlachter}{Auch bei deinem Haupt sollst du nicht schwören, denn du kannst kein einziges Haar weiß oder schwarz machen.}
\verstab{37}{Schlachter}{Es sei aber eure Rede: Ja, ja! Nein, nein! Was darüber ist, das ist vom Bösen.}
\verstab{38}{Schlachter}{Ihr habt gehört, dass gesagt ist: »Auge um Auge und Zahn um Zahn!«}
\verstab{39}{Schlachter}{Ich aber sage euch: Ihr sollt dem Bösen nicht widerstehen; sondern wenn dich jemand auf deine rechte Backe schlägt, so biete ihm auch die andere dar;}
\verstab{40}{Schlachter}{und dem, der mit dir vor Gericht gehen und dein Hemd nehmen will, dem lass auch den Mantel;}
\verstab{41}{Schlachter}{und wenn dich jemand nötigt, eine Meile weit zu gehen, so geh mit ihm zwei.}
\verstab{42}{Schlachter}{Gib dem, der dich bittet, und wende dich nicht ab von dem, der von dir borgen will!}
\verstab{43}{Schlachter}{Ihr habt gehört, dass gesagt ist: Du sollst deinen Nächsten lieben und deinen Feind hassen.}
\verstab{44}{Schlachter}{Ich aber sage euch: Liebt eure Feinde, segnet, die euch fluchen, tut wohl denen, die euch hassen, und bittet für die, welche euch beleidigen und verfolgen,}
\verstab{45}{Schlachter}{damit ihr Söhne eures Vaters im Himmel seid. Denn er lässt seine Sonne aufgehen über Böse und Gute und lässt es regnen über Gerechte und Ungerechte.}
\verstab{46}{Schlachter}{Denn wenn ihr die liebt, die euch lieben, was habt ihr für einen Lohn? Tun nicht auch die Zöllner dasselbe?}
\verstab{47}{Schlachter}{Und wenn ihr nur eure Brüder grüßt, was tut ihr Besonderes? Machen es nicht auch die Zöllner ebenso?}
\verstab{48}{Schlachter}{Darum sollt ihr vollkommen sein, gleichwie euer Vater im Himmel vollkommen ist!}
        \subsection*{Kapitel 6}
\addcontentsline{toc}{subsection}{Kapitel 6}
\verstab{1}{Schlachter}{Habt acht, dass ihr eure Almosen nicht vor den Leuten gebt, um von ihnen gesehen zu werden; sonst habt ihr keinen Lohn bei eurem Vater im Himmel.}
\verstab{2}{Schlachter}{Wenn du nun Almosen gibst, sollst du nicht vor dir her posaunen lassen, wie es die Heuchler in den Synagogen und auf den Gassen tun, um von den Leuten gepriesen zu werden. Wahrlich, ich sage euch: Sie haben ihren Lohn schon empfangen.}
\verstab{3}{Schlachter}{Wenn du aber Almosen gibst, so soll deine linke Hand nicht wissen, was deine rechte tut,}
\verstab{4}{Schlachter}{damit dein Almosen im Verborgenen ist. Und dein Vater, der ins Verborgene sieht, er wird es dir öffentlich vergelten.}
\verstab{5}{Schlachter}{Und wenn du betest, sollst du nicht sein wie die Heuchler; denn sie stellen sich gern in den Synagogen und an den Straßenecken auf und beten, um von den Leuten bemerkt zu werden. Wahrlich, ich sage euch: Sie haben ihren Lohn schon empfangen.}
\verstab{6}{Schlachter}{Du aber, wenn du betest, geh in dein Kämmerlein und schließe deine Türe zu und bete zu deinem Vater, der im Verborgenen ist; und dein Vater, der ins Verborgene sieht, wird es dir öffentlich vergelten.}
\verstab{7}{Schlachter}{Und wenn ihr betet, sollt ihr nicht plappern wie die Heiden; denn sie meinen, sie werden erhört um ihrer vielen Worte willen.}
\verstab{8}{Schlachter}{Darum sollt ihr ihnen nicht gleichen! Denn euer Vater weiß, was ihr benötigt, ehe ihr ihn bittet.}
\verstab{9}{Schlachter}{Deshalb sollt ihr auf diese Weise beten: Unser Vater, der du bist im Himmel! Geheiligt werde dein Name.}
\verstab{10}{Schlachter}{Dein Reich komme. Dein Wille geschehe, wie im Himmel, so auch auf Erden.}
\verstab{11}{Schlachter}{Gib uns heute unser tägliches Brot.}
\verstab{12}{Schlachter}{Und vergib uns unsere Schulden, wie auch wir vergeben unseren Schuldnern.}
\verstab{13}{Schlachter}{Und führe uns nicht in Versuchung, sondern errette uns von dem Bösen. Denn dein ist das Reich und die Kraft und die Herrlichkeit in Ewigkeit! Amen.}
\verstab{14}{Schlachter}{Denn wenn ihr den Menschen ihre Verfehlungen vergebt, so wird euer himmlischer Vater euch auch vergeben.}
\verstab{15}{Schlachter}{Wenn ihr aber den Menschen ihre Verfehlungen nicht vergebt, so wird euch euer Vater eure Verfehlungen auch nicht vergeben.}
\verstab{16}{Schlachter}{Wenn ihr aber fastet, sollt ihr nicht finster dreinsehen wie die Heuchler; denn sie verstellen ihr Angesicht, damit es von den Leuten bemerkt wird, dass sie fasten. Wahrlich, ich sage euch: Sie haben ihren Lohn schon empfangen.}
\verstab{17}{Schlachter}{Du aber, wenn du fastest, so salbe dein Haupt und wasche dein Angesicht,}
\verstab{18}{Schlachter}{damit es nicht von den Leuten bemerkt wird, dass du fastest, sondern von deinem Vater, der im Verborgenen ist; und dein Vater, der ins Verborgene sieht, wird es dir öffentlich vergelten.}
\verstab{19}{Schlachter}{Ihr sollt euch nicht Schätze sammeln auf Erden, wo die Motten und der Rost sie fressen und wo die Diebe nachgraben und stehlen.}
\verstab{20}{Schlachter}{Sammelt euch vielmehr Schätze im Himmel, wo weder die Motten noch der Rost sie fressen und wo die Diebe nicht nachgraben und stehlen!}
\verstab{21}{Schlachter}{Denn wo euer Schatz ist, da wird auch euer Herz sein.}
\verstab{22}{Schlachter}{Das Auge ist die Leuchte des Leibes. Wenn nun dein Auge lauter ist, so wird dein ganzer Leib licht sein.}
\verstab{23}{Schlachter}{Wenn aber dein Auge verdorben ist, so wird dein ganzer Leib finster sein. Wenn nun das Licht in dir Finsternis ist, wie groß wird dann die Finsternis sein!}
\verstab{24}{Schlachter}{Niemand kann zwei Herren dienen, denn entweder wird er den einen hassen und den anderen lieben, oder er wird dem einen anhängen und den anderen verachten. Ihr könnt nicht Gott dienen und dem Mammon!}
\verstab{25}{Schlachter}{Darum sage ich euch: Sorgt euch nicht um euer Leben, was ihr essen und was ihr trinken sollt, noch um euren Leib, was ihr anziehen sollt! Ist nicht das Leben mehr als die Speise und der Leib mehr als die Kleidung?}
\verstab{26}{Schlachter}{eht die Vögel des Himmels an: Sie säen nicht und ernten nicht, sie sammeln auch nicht in die Scheunen, und euer himmlischer Vater ernährt sie doch. Seid ihr nicht viel mehr wert als sie?}
\verstab{27}{Schlachter}{ Wer aber von euch kann durch sein Sorgen zu seiner Lebenslänge eine einzige Elle hinzusetzen?}
\verstab{28}{Schlachter}{Und warum sorgt ihr euch um die Kleidung? Betrachtet die Lilien des Feldes, wie sie wachsen! Sie mühen sich nicht und spinnen nicht;}
\verstab{29}{Schlachter}{ich sage euch aber, dass auch Salomo in all seiner Herrlichkeit nicht gekleidet gewesen ist wie eine von ihnen.}
\verstab{30}{Schlachter}{Wenn nun Gott das Gras des Feldes, das heute steht und morgen in den Ofen geworfen wird, so kleidet, wird er das nicht viel mehr euch tun, ihr Kleingläubigen?}
\verstab{31}{Schlachter}{Darum sollt ihr nicht sorgen und sagen: Was werden wir essen?, oder: Was werden wir trinken?, oder: Womit werden wir uns kleiden?}
\verstab{32}{Schlachter}{Denn nach allen diesen Dingen trachten die Heiden, aber euer himmlischer Vater weiß, dass ihr das alles benötigt.}
\verstab{33}{Schlachter}{Trachtet vielmehr zuerst nach dem Reich Gottes und nach seiner Gerechtigkeit, so wird euch dies alles hinzugefügt werden!}
\verstab{34}{Schlachter}{Darum sollt ihr euch nicht sorgen um den morgigen Tag; denn der morgige Tag wird für das Seine sorgen. Jedem Tag genügt seine eigene Plage.}

        \subsection*{Kapitel 7}
\addcontentsline{toc}{subsection}{Kapitel 7}
\verstab{1}{Schlachter}{Richtet nicht, damit ihr nicht gerichtet werdet!}
\verstab{2}{Schlachter}{Denn mit demselben Gericht, mit dem ihr richtet, werdet ihr gerichtet werden; und mit demselben Maß, mit dem ihr [anderen] zumesst, wird auch euch zugemessen werden.}
\verstab{3}{Schlachter}{Was siehst du aber den Splitter im Auge deines Bruders, und den Balken in deinem Auge bemerkst du nicht?}
\verstab{4}{Schlachter}{Oder wie kannst du zu deinem Bruder sagen: Halt, ich will den Splitter aus deinem Auge ziehen! — und siehe, der Balken ist in deinem Auge?}
\verstab{5}{Schlachter}{Du Heuchler, zieh zuerst den Balken aus deinem Auge, und dann wirst du klar sehen, um den Splitter aus dem Auge deines Bruders zu ziehen!}
\verstab{6}{Schlachter}{Gebt das Heilige nicht den Hunden und werft eure Perlen nicht vor die Säue, damit diese sie nicht mit ihren Füßen zertreten und [jene] sich nicht umwenden und euch zerreißen.}
\verstab{7}{Schlachter}{Bittet, so wird euch gegeben; sucht, so werdet ihr finden; klopft an, so wird euch aufgetan!}
\verstab{8}{Schlachter}{Denn jeder, der bittet, empfängt; und wer sucht, der findet; und wer anklopft, dem wird aufgetan.}
\verstab{9}{Schlachter}{Oder ist unter euch ein Mensch, der, wenn sein Sohn ihn um Brot bittet, ihm einen Stein gibt,}
\verstab{10}{Schlachter}{und, wenn er um einen Fisch bittet, ihm eine Schlange gibt?}
\verstab{11}{Schlachter}{Wenn nun ihr, die ihr böse seid, euren Kindern gute Gaben zu geben versteht, wie viel mehr wird euer Vater im Himmel denen Gutes geben, die ihn bitten!}
\verstab{12}{Schlachter}{Alles nun, was ihr wollt, dass die Leute euch tun sollen, das tut auch ihr ihnen ebenso; denn dies ist das Gesetz und die Propheten.}
\verstab{13}{Schlachter}{Geht ein durch die enge Pforte! Denn die Pforte ist weit und der Weg ist breit, der ins Verderben führt; und viele sind es, die da hineingehen.}
\verstab{14}{Schlachter}{Denn die Pforte ist eng und der Weg ist schmal, der zum Leben führt; und wenige sind es, die ihn finden.}
\verstab{15}{Schlachter}{Hütet euch aber vor den falschen Propheten, die in Schafskleidern zu euch kommen, inwendig aber reißende Wölfe sind!}
\verstab{16}{Schlachter}{An ihren Früchten werdet ihr sie erkennen. Sammelt man auch Trauben von Dornen, oder Feigen von Disteln?}
\verstab{17}{Schlachter}{So bringt jeder gute Baum gute Früchte, der schlechte Baum aber bringt schlechte Früchte.}
\verstab{18}{Schlachter}{Ein guter Baum kann keine schlechten Früchte bringen, und ein schlechter Baum kann keine guten Früchte bringen.}
\verstab{19}{Schlachter}{Jeder Baum, der keine gute Frucht bringt, wird abgehauen und ins Feuer geworfen.}
\verstab{20}{Schlachter}{Darum werdet ihr sie an ihren Früchten erkennen.}
\verstab{21}{Schlachter}{Nicht jeder, der zu mir sagt: Herr, Herr! wird in das Reich der Himmel eingehen, sondern wer den Willen meines Vaters im Himmel tut.}
\verstab{22}{Schlachter}{Viele werden an jenem Tag zu mir sagen: Herr, Herr, haben wir nicht in deinem Namen geweissagt und in deinem Namen Dämonen ausgetrieben und in deinem Namen viele Wundertaten vollbracht?}
\verstab{23}{Schlachter}{Und dann werde ich ihnen bezeugen: Ich habe euch nie gekannt; weicht von mir, ihr Gesetzlosen!}
\verstab{24}{Schlachter}{Ein jeder nun, der diese meine Worte hört und sie tut, den will ich mit einem klugen Mann vergleichen, der sein Haus auf den Felsen baute.}
\verstab{25}{Schlachter}{Als nun der Platzregen fiel und die Wasserströme kamen und die Winde stürmten und an dieses Haus stießen, fiel es nicht; denn es war auf den Felsen gegründet.}
\verstab{26}{Schlachter}{Und jeder, der diese meine Worte hört und sie nicht tut, wird einem törichten Mann gleich sein, der sein Haus auf den Sand baute.}
\verstab{27}{Schlachter}{Als nun der Platzregen fiel und die Wasserströme kamen und die Winde stürmten und an dieses Haus stießen, da stürzte es ein, und sein Einsturz war gewaltig.}
\verstab{28}{Schlachter}{Und es geschah, als Jesus diese Worte beendet hatte, erstaunte die Volksmenge über seine Lehre,}
\verstab{29}{Schlachter}{denn er lehrte sie wie einer, der Vollmacht hat, und nicht wie die Schriftgelehrten.}
        \subsection*{Kapitel 8}
\addcontentsline{toc}{subsection}{Kapitel 8}
\verstab{1}{Schlachter}{Als er aber von dem Berg herabstieg, folgte ihm eine große Volksmenge nach.}
\verstab{2}{Schlachter}{Und siehe, ein Aussätziger kam, fiel vor ihm nieder und sprach: Herr, wenn du willst, kannst du mich reinigen!}
\verstab{3}{Schlachter}{Und Jesus streckte die Hand aus, rührte ihn an und sprach: Ich will; sei gereinigt! Und sogleich wurde er von seinem Aussatz rein.}
\verstab{4}{Schlachter}{Und Jesus spricht zu ihm: Sieh zu, dass du es niemand sagst; sondern geh hin, zeige dich dem Priester und bringe das Opfer dar, das Mose befohlen hat, ihnen zum Zeugnis!}
\verstab{5}{Schlachter}{Als Jesus aber nach Kapernaum kam, trat ein Hauptmann zu ihm, bat ihn}
\verstab{6}{Schlachter}{und sprach: Herr, mein Knecht liegt daheim gelähmt danieder und ist furchtbar geplagt!}
\verstab{7}{Schlachter}{Und Jesus spricht zu ihm: Ich will kommen und ihn heilen!}
\verstab{8}{Schlachter}{Der Hauptmann antwortete und sprach: Herr, ich bin nicht wert, dass du unter mein Dach kommst, sondern sprich nur ein Wort, so wird mein Knecht gesund werden!}
\verstab{9}{Schlachter}{Denn auch ich bin ein Mensch, der unter Vorgesetzten steht, und habe Kriegsknechte unter mir; und wenn ich zu diesem sage: Geh hin!, so geht er; und zu einem anderen: Komm her!, so kommt er; und zu meinem Knecht: Tu das!, so tut er’s.}
\verstab{10}{Schlachter}{Als Jesus das hörte, verwunderte er sich und sprach zu denen, die nachfolgten: Wahrlich, ich sage euch: Einen so großen Glauben habe ich in Israel nicht gefunden!}
\verstab{11}{Schlachter}{Ich sage euch aber: Viele werden kommen vom Osten und vom Westen und werden im Reich der Himmel mit Abraham, Isaak und Jakob zu Tisch sitzen,}
\verstab{12}{Schlachter}{aber die Kinder des Reiches werden in die äußerste Finsternis hinausgeworfen werden; dort wird Heulen und Zähneknirschen sein.}
\verstab{13}{Schlachter}{Und Jesus sprach zu dem Hauptmann: Geh hin, und dir geschehe, wie du geglaubt hast! Und sein Knecht wurde in derselben Stunde gesund.}
\verstab{14}{Schlachter}{Und als Jesus in das Haus des Petrus kam, sah er, dass dessen Schwiegermutter daniederlag und Fieber hatte.}
\verstab{15}{Schlachter}{Und er rührte ihre Hand an; und das Fieber verließ sie, und sie stand auf und diente ihnen.}
\verstab{16}{Schlachter}{ls es aber Abend geworden war, brachten sie viele Besessene zu ihm, und er trieb die Geister aus mit einem Wort und heilte alle Kranken,}
\verstab{17}{Schlachter}{damit erfüllt würde, was durch den Propheten Jesaja gesagt ist, der spricht: »Er hat unsere Gebrechen weggenommen und unsere Krankheiten getragen«.}
\verstab{18}{Schlachter}{Als aber Jesus die große Volksmenge um sich sah, befahl er, ans jenseitige Ufer zu fahren.}
\verstab{19}{Schlachter}{Und ein Schriftgelehrter trat herzu und sprach zu ihm: Meister, ich will dir nachfolgen, wohin du auch gehst!}
\verstab{20}{Schlachter}{Und Jesus sprach zu ihm: Die Füchse haben Gruben, und die Vögel des Himmels haben Nester; aber der Sohn des Menschen hat nichts, wo er sein Haupt hinlegen kann.}
\verstab{21}{Schlachter}{Ein anderer seiner Jünger sprach zu ihm: Herr, erlaube mir, zuvor hinzugehen und meinen Vater zu begraben!}
\verstab{22}{Schlachter}{Jesus aber sprach zu ihm: Folge mir nach, und lass die Toten ihre Toten begraben!}
\verstab{23}{Schlachter}{Und er trat in das Schiff, und seine Jünger folgten ihm nach.}
\verstab{24}{Schlachter}{Und siehe, es erhob sich ein großer Sturm auf dem See, sodass das Schiff von den Wellen bedeckt wurde; er aber schlief.}
\verstab{25}{Schlachter}{Und seine Jünger traten zu ihm, weckten ihn auf und sprachen: Herr, rette uns! Wir kommen um!}
\verstab{26}{Schlachter}{Da sprach er zu ihnen: Was seid ihr so furchtsam, ihr Kleingläubigen? Dann stand er auf und befahl den Winden und dem See; und es entstand eine große Stille.}
\verstab{27}{Schlachter}{Die Menschen aber verwunderten sich und sprachen: Wer ist dieser, dass ihm selbst die Winde und der See gehorsam sind?}
\verstab{28}{Schlachter}{Und als er ans jenseitige Ufer in das Gebiet der Gergesener kam, liefen ihm zwei Besessene entgegen, die kamen aus den Gräbern heraus und waren sehr gefährlich, sodass niemand auf jener Straße wandern konnte.}
\verstab{29}{Schlachter}{Und siehe, sie schrien und sprachen: Was haben wir mit dir zu tun, Jesus, du Sohn Gottes? Bist du hierhergekommen, um uns vor der Zeit zu quälen?}
\verstab{30}{Schlachter}{Es war aber fern von ihnen eine große Herde Schweine auf der Weide.}
\verstab{31}{Schlachter}{Und die Dämonen baten ihn und sprachen: Wenn du uns austreibst, so erlaube uns, in die Schweineherde zu fahren!}
\verstab{32}{Schlachter}{Und er sprach zu ihnen: Geht hin! Da fuhren sie aus und fuhren in die Schweineherde. Und siehe, die ganze Schweineherde stürzte sich den Abhang hinunter in den See, und sie kamen im Wasser um.}
\verstab{33}{Schlachter}{Die Hirten aber flohen, gingen in die Stadt und verkündeten alles, auch was mit den Besessenen vorgegangen war.}
\verstab{34}{Schlachter}{Und siehe, die ganze Stadt kam heraus, Jesus entgegen. Und als sie ihn sahen, baten sie ihn, aus ihrem Gebiet wegzugehen.}

        \subsection*{Kapitel 9}
\addcontentsline{toc}{subsection}{Kapitel 9}
\verstab{1}{Schlachter}{Und er trat in das Schiff, fuhr hinüber und kam in seine Stadt.}
\verstab{2}{Schlachter}{Und siehe, da brachten sie einen Gelähmten zu ihm, der auf einer Liegematte lag. Und als Jesus ihren Glauben sah, sprach er zu dem Gelähmten: Sei getrost, mein Sohn, deine Sünden sind dir vergeben!}
\verstab{3}{Schlachter}{Und siehe, etliche der Schriftgelehrten sprachen bei sich selbst: Dieser lästert!}
\verstab{4}{Schlachter}{Und da Jesus ihre Gedanken sah, sprach er: Warum denkt ihr Böses in euren Herzen?}
\verstab{5}{Schlachter}{Was ist denn leichter, zu sagen: Deine Sünden sind dir vergeben!, oder zu sagen: Steh auf und geh umher?}
\verstab{6}{Schlachter}{Damit ihr aber wisst, dass der Sohn des Menschen Vollmacht hat, auf Erden Sünden zu vergeben — sprach er zu dem Gelähmten: Steh auf, nimm deine Liegematte und geh heim!}
\verstab{7}{Schlachter}{Und er stand auf und ging heim.}
\verstab{8}{Schlachter}{Als aber die Volksmenge das sah, verwunderte sie sich und pries Gott, der solche Vollmacht den Menschen gegeben hatte.}
\verstab{9}{Schlachter}{Und als Jesus von da weiterging, sah er einen Menschen an der Zollstätte sitzen, der hieß Matthäus; und er sprach zu ihm: Folge mir nach! Und er stand auf und folgte ihm nach.}
\verstab{10}{Schlachter}{Und es geschah, als er in dem Haus zu Tisch saß, siehe, da kamen viele Zöllner und Sünder und saßen mit Jesus und seinen Jüngern zu Tisch.}
\verstab{11}{Schlachter}{Und als die Pharisäer es sahen, sprachen sie zu seinen Jüngern: Warum isst euer Meister mit den Zöllnern und Sündern?}
\verstab{12}{Schlachter}{Jesus aber, als er es hörte, sprach zu ihnen: Nicht die Starken brauchen den Arzt, sondern die Kranken.}
\verstab{13}{Schlachter}{Geht aber hin und lernt, was das heißt: »Ich will Barmherzigkeit und nicht Opfer«. Denn ich bin nicht gekommen, Gerechte zu berufen, sondern Sünder zur Buße.}
\verstab{14}{Schlachter}{Da kamen die Jünger des Johannes zu ihm und sprachen: Warum fasten wir und die Pharisäer so viel, deine Jünger aber fasten nicht?}
\verstab{15}{Schlachter}{Und Jesus sprach zu ihnen: Können die Hochzeitsgäste trauern, solange der Bräutigam bei ihnen ist? Es werden aber Tage kommen, da der Bräutigam von ihnen genommen sein wird, und dann werden sie fasten.}
\verstab{16}{Schlachter}{Niemand aber setzt einen Lappen von neuem Tuch auf ein altes Kleid, denn der Flicken reißt von dem Kleid, und der Riss wird schlimmer.}
\verstab{17}{Schlachter}{Man füllt auch nicht neuen Wein in alte Schläuche, sonst zerreißen die Schläuche, und der Wein wird verschüttet, und die Schläuche verderben; sondern man füllt neuen Wein in neue Schläuche, so bleiben beide miteinander erhalten.}
\verstab{18}{Schlachter}{Und als er dies mit ihnen redete, siehe, da kam ein Vorsteher[3], fiel vor ihm nieder und sprach: Meine Tochter ist eben gestorben; aber komm und lege deine Hand auf sie, so wird sie leben!}
\verstab{19}{Schlachter}{Und Jesus stand auf und folgte ihm mit seinen Jüngern.}
\verstab{20}{Schlachter}{Und siehe, eine Frau, die zwölf Jahre blutflüssig war, trat von hinten herzu und rührte den Saum seines Gewandes an.}
\verstab{21}{Schlachter}{Denn sie sagte bei sich selbst: Wenn ich nur sein Gewand anrühre, so bin ich geheilt!}
\verstab{22}{Schlachter}{Jesus aber wandte sich um, sah sie und sprach: Sei getrost, meine Tochter! Dein Glaube hat dich gerettet! Und die Frau war geheilt von jener Stunde an.}
\verstab{23}{Schlachter}{Als nun Jesus in das Haus des Vorstehers kam und die Pfeifer und das Getümmel sah,}
\verstab{24}{Schlachter}{spricht er zu ihnen: Entfernt euch! Denn das Mädchen ist nicht gestorben, sondern es schläft. Und sie lachten ihn aus.}
\verstab{25}{Schlachter}{Als aber die Menge hinausgetrieben war, ging er hinein und ergriff ihre Hand; und das Mädchen stand auf.}
\verstab{26}{Schlachter}{Und die Nachricht hiervon verbreitete sich in jener ganzen Gegend.}
\verstab{27}{Schlachter}{Und als Jesus von dort weiterging, folgten ihm zwei Blinde nach, die schrien und sprachen: Du Sohn Davids, erbarme dich über uns!}
\verstab{28}{Schlachter}{Als er nun ins Haus kam, traten die Blinden zu ihm. Und Jesus fragte sie: Glaubt ihr, dass ich dies tun kann? Sie sprachen zu ihm: Ja, Herr!}
\verstab{29}{Schlachter}{Da rührte er ihre Augen an und sprach: Euch geschehe nach eurem Glauben!}
\verstab{30}{Schlachter}{Und ihre Augen wurden geöffnet. Und Jesus ermahnte sie ernstlich und sprach: Seht zu, dass es niemand erfährt!}
\verstab{31}{Schlachter}{Sie aber gingen hinaus und machten ihn in jener ganzen Gegend bekannt.}
\verstab{32}{Schlachter}{Als sie aber hinausgingen, siehe, da brachte man einen Menschen zu ihm, der stumm und besessen war.}
\verstab{33}{Schlachter}{Und nachdem der Dämon ausgetrieben war, redete der Stumme. Und die Volksmenge verwunderte sich und sprach: So etwas ist noch nie in Israel gesehen worden!}
\verstab{34}{Schlachter}{Die Pharisäer aber sagten: Durch den Obersten der Dämonen treibt er die Dämonen aus!}
\verstab{35}{Schlachter}{Und Jesus durchzog alle Städte und Dörfer, lehrte in ihren Synagogen, verkündigte das Evangelium von dem Reich und heilte jede Krankheit und jedes Gebrechen im Volk.}
\verstab{36}{Schlachter}{Als er aber die Volksmenge sah, empfand er Mitleid mit ihnen, weil sie ermattet und vernachlässigt waren wie Schafe, die keinen Hirten haben.}
\verstab{37}{Schlachter}{Da sprach er zu seinen Jüngern: Die Ernte ist groß, aber es sind wenige Arbeiter.}
\verstab{38}{Schlachter}{Darum bittet den Herrn der Ernte, dass er Arbeiter in seine Ernte aussende!}
        \subsection*{Kapitel 10}
\addcontentsline{toc}{subsection}{Kapitel 10}
\verstab{1}{Schlachter}{Da rief er seine zwölf Jünger zu sich und gab ihnen Vollmacht über die unreinen Geister, sie auszutreiben, und jede Krankheit und jedes Gebrechen zu heilen.}
\verstab{2}{Schlachter}{Die Namen der zwölf Apostel aber sind diese: der erste Simon, genannt Petrus, und sein Bruder Andreas; Jakobus, der Sohn des Zebedäus, und sein Bruder Johannes;}
\verstab{3}{Schlachter}{Philippus und Bartholomäus; Thomas und Matthäus der Zöllner; Jakobus, der Sohn des Alphäus, und Lebbäus, mit dem Beinamen Thaddäus;}
\verstab{4}{Schlachter}{Simon der Kananiter, und Judas Ischariot, der ihn auch verriet.}
\verstab{5}{Schlachter}{Diese zwölf sandte Jesus aus, gebot ihnen und sprach: Begebt euch nicht auf die Straße der Heiden und betretet keine Stadt der Samariter;}
\verstab{6}{Schlachter}{geht vielmehr zu den verlorenen Schafen des Hauses Israel.}
\verstab{7}{Schlachter}{Geht aber hin, verkündigt und sprecht: Das Reich der Himmel ist nahe herbeigekommen!}
\verstab{8}{Schlachter}{Heilt Kranke, reinigt Aussätzige, weckt Tote auf, treibt Dämonen aus! Umsonst habt ihr es empfangen, umsonst gebt es!}
\verstab{9}{Schlachter}{Nehmt weder Gold noch Silber noch Kupfer in eure Gürtel,}
\verstab{10}{Schlachter}{keine Tasche auf den Weg, auch nicht zwei Hemden, weder Schuhe noch Stab; denn der Arbeiter ist seiner Nahrung wert.}
\verstab{11}{Schlachter}{Wo ihr aber in eine Stadt oder in ein Dorf hineingeht, da erkundigt euch, wer es darin wert ist, und bleibt dort, bis ihr weiterzieht.}
\verstab{12}{Schlachter}{Wenn ihr aber in das Haus eintretet, so grüßt es [mit dem Friedensgruß]}
\verstab{13}{Schlachter}{Und wenn das Haus es wert ist, so komme euer Friede über dasselbe. Ist es aber dessen nicht wert, so soll euer Friede wieder zu euch zurückkehren.}
\verstab{14}{Schlachter}{Und wenn euch jemand nicht aufnehmen noch auf eure Worte hören wird, so geht fort aus diesem Haus oder dieser Stadt und schüttelt den Staub von euren Füßen!}
\verstab{15}{Schlachter}{Wahrlich, ich sage euch: Es wird dem Land Sodom und Gomorra erträglicher gehen am Tag des Gerichts als dieser Stadt.}
\verstab{16}{Schlachter}{Siehe, ich sende euch wie Schafe mitten unter die Wölfe. Darum seid klug wie die Schlangen und ohne Falsch wie die Tauben!}
\verstab{17}{Schlachter}{Hütet euch aber vor den Menschen! Denn sie werden euch den Gerichten ausliefern, und in ihren Synagogen werden sie euch geißeln;}
\verstab{18}{Schlachter}{auch vor Fürsten und Könige wird man euch führen um meinetwillen, ihnen und den Heiden zum Zeugnis.}
\verstab{19}{Schlachter}{Wenn sie euch aber ausliefern, so sorgt euch nicht darum, wie oder was ihr reden sollt; denn es wird euch in jener Stunde gegeben werden, was ihr reden sollt.}
\verstab{20}{Schlachter}{Denn nicht ihr seid es, die reden, sondern der Geist eures Vaters ist’s, der durch euch redet.}
\verstab{21}{Schlachter}{Es wird aber ein Bruder den anderen zum Tode ausliefern und ein Vater sein Kind; und Kinder werden sich gegen die Eltern erheben und werden sie töten helfen.}
\verstab{22}{Schlachter}{Und ihr werdet von jedermann gehasst sein um meines Namens willen. Wer aber ausharrt bis ans Ende, der wird gerettet werden.}
\verstab{23}{Schlachter}{Wenn sie euch aber in der einen Stadt verfolgen, so flieht in eine andere. Denn wahrlich, ich sage euch: Ihr werdet mit den Städten Israels nicht fertig sein, bis der Sohn des Menschen kommt.}
\verstab{24}{Schlachter}{Der Jünger ist nicht über dem Meister, noch der Knecht über seinem Herrn;}
\verstab{25}{Schlachter}{es ist für den Jünger genug, dass er sei wie sein Meister und der Knecht wie sein Herr. Haben sie den Hausherrn Beelzebul genannt, wie viel mehr seine Hausgenossen!}
\verstab{26}{Schlachter}{So fürchtet euch nun nicht vor ihnen! Denn es ist nichts verdeckt, das nicht aufgedeckt werden wird, und nichts verborgen, das man nicht erfahren wird.}
\verstab{27}{Schlachter}{Was ich euch im Finstern sage, das redet im Licht, und was ihr ins Ohr hört, das verkündigt auf den Dächern!}
\verstab{28}{Schlachter}{Und fürchtet euch nicht vor denen, die den Leib töten, die Seele aber nicht zu töten vermögen; fürchtet vielmehr den, der Seele und Leib verderben kann in der Hölle!}
\verstab{29}{Schlachter}{Verkauft man nicht zwei Sperlinge um einen Groschen? Und doch fällt keiner von ihnen auf die Erde ohne euren Vater.}
\verstab{30}{Schlachter}{Bei euch aber sind selbst die Haare des Hauptes alle gezählt.}
\verstab{31}{Schlachter}{Darum fürchtet euch nicht! Ihr seid mehr wert als viele Sperlinge.}
\verstab{32}{Schlachter}{Jeder nun, der sich zu mir bekennt vor den Menschen, zu dem werde auch ich mich bekennen vor meinem Vater im Himmel;}
\verstab{33}{Schlachter}{wer mich aber verleugnet vor den Menschen, den werde auch ich verleugnen vor meinem Vater im Himmel.}
\verstab{34}{Schlachter}{Ihr sollt nicht meinen, dass ich gekommen sei, Frieden auf die Erde zu bringen. Ich bin nicht gekommen, Frieden zu bringen, sondern das Schwert!}
\verstab{35}{Schlachter}{Denn ich bin gekommen, den Menschen zu entzweien mit seinem Vater und die Tochter mit ihrer Mutter und die Schwiegertochter mit ihrer Schwiegermutter;}
\verstab{36}{Schlachter}{und die Feinde des Menschen werden seine eigenen Hausgenossen sein.}
\verstab{37}{Schlachter}{Wer Vater oder Mutter mehr liebt als mich, der ist meiner nicht wert; und wer Sohn oder Tochter mehr liebt als mich, der ist meiner nicht wert.}
\verstab{38}{Schlachter}{Und wer nicht sein Kreuz auf sich nimmt[4] und mir nachfolgt, der ist meiner nicht wert.}
\verstab{39}{Schlachter}{Wer sein Leben[5] findet, der wird es verlieren; und wer sein Leben verliert um meinetwillen, der wird es finden!}
\verstab{40}{Schlachter}{Wer euch aufnimmt, der nimmt mich auf; und wer mich aufnimmt, der nimmt den auf, der mich gesandt hat.}
\verstab{41}{Schlachter}{Wer einen Propheten aufnimmt, weil er ein Prophet ist, der wird den Lohn eines Propheten empfangen; und wer einen Gerechten aufnimmt, weil er ein Gerechter ist, der wird den Lohn eines Gerechten empfangen;}
\verstab{42}{Schlachter}{und wer einem dieser Geringen auch nur einen Becher mit kaltem Wasser zu trinken gibt, weil er ein Jünger ist, wahrlich, ich sage euch, der wird seinen Lohn nicht verlieren!}
        %\subsection*{Kapitel 11}
\addcontentsline{toc}{subsection}{Kapitel 11}
\verstab{1}{Schlachter}{Und es geschah, als Jesus die Befehle an seine zwölf Jünger vollendet hatte, zog er von dort weg, um in ihren Städten zu lehren und zu verkündigen.}
\verstab{2}{Schlachter}{Als aber Johannes im Gefängnis von den Werken des Christus hörte, sandte er zwei seiner Jünger}
\verstab{3}{Schlachter}{und ließ ihm sagen: Bist du derjenige, der kommen soll, oder sollen wir auf einen anderen warten?}
\verstab{4}{Schlachter}{Und Jesus antwortete und sprach zu ihnen: Geht hin und berichtet dem Johannes, was ihr hört und seht:}
\verstab{5}{Schlachter}{Blinde werden sehend und Lahme gehen, Aussätzige werden rein und Taube hören, Tote werden auferweckt, und Armen wird das Evangelium verkündigt.}
\verstab{6}{Schlachter}{Und glückselig ist, wer nicht Anstoß nimmt an mir!}
\verstab{7}{Schlachter}{Als aber diese unterwegs waren, fing Jesus an, zu der Volksmenge über Johannes zu reden: Was seid ihr in die Wüste hinausgegangen zu sehen? Ein Rohr, das vom Wind bewegt wird?}
\verstab{8}{Schlachter}{Oder was seid ihr hinausgegangen zu sehen? Einen Menschen, mit weichen Kleidern bekleidet? Siehe, die, welche weiche Kleider tragen, sind in den Häusern der Könige!}
\verstab{9}{Schlachter}{Oder was seid ihr hinausgegangen zu sehen? Einen Propheten? Ja, ich sage euch: einen, der mehr ist als ein Prophet!}
\verstab{10}{Schlachter}{Denn dieser ist’s, von dem geschrieben steht: »Siehe, ich sende meinen Boten vor deinem Angesicht her, der deinen Weg vor dir bereiten soll«.}
\verstab{11}{Schlachter}{Wahrlich, ich sage euch: Unter denen, die von Frauen geboren sind, ist kein Größerer aufgetreten als Johannes der Täufer; doch der Kleinste im Reich der Himmel ist größer als er.}
\verstab{12}{Schlachter}{Aber von den Tagen Johannes des Täufers an bis jetzt leidet das Reich der Himmel Gewalt, und die, welche Gewalt anwenden, reißen es an sich.}
\verstab{13}{Schlachter}{Denn alle Propheten und das Gesetz haben geweissagt bis hin zu Johannes.}
\verstab{14}{Schlachter}{Und wenn ihr es annehmen wollt: Er ist der Elia, der kommen soll.}
\verstab{15}{Schlachter}{Wer Ohren hat zu hören, der höre!}
\verstab{16}{Schlachter}{Wem soll ich aber dieses Geschlecht vergleichen? Es ist Kindern gleich, die an den Marktplätzen sitzen und ihren Freunden zurufen}
\verstab{17}{Schlachter}{und sprechen: Wir haben euch aufgespielt, und ihr habt nicht getanzt; wir haben euch Klagelieder gesungen, und ihr habt nicht geweint!}
\verstab{18}{Schlachter}{Denn Johannes ist gekommen, der aß nicht und trank nicht; da sagen sie: Er hat einen Dämon!}
\verstab{19}{Schlachter}{Der Sohn des Menschen ist gekommen, der isst und trinkt; da sagen sie: Wie ist der Mensch ein Fresser und Weinsäufer, ein Freund der Zöllner und Sünder! Und doch ist die Weisheit gerechtfertigt worden von ihren Kindern.}
\verstab{20}{Schlachter}{Da fing er an, die Städte zu schelten, in denen die meisten seiner Wundertaten geschehen waren, weil sie nicht Buße getan hatten:}
\verstab{21}{Schlachter}{Wehe dir, Chorazin! Wehe dir, Bethsaida! Denn wenn in Tyrus und Zidon die Wundertaten geschehen wären, die bei euch geschehen sind, so hätten sie längst in Sack und Asche Buße getan.}
\verstab{22}{Schlachter}{Doch ich sage euch: Es wird Tyrus und Zidon erträglicher gehen am Tag des Gerichts als euch!}
\verstab{23}{Schlachter}{Und du, Kapernaum, die du bis zum Himmel erhöht worden bist, du wirst bis zum Totenreich hinabgeworfen werden! Denn wenn in Sodom die Wundertaten geschehen wären, die bei dir geschehen sind, es würde noch heutzutage stehen.}
\verstab{24}{Schlachter}{Doch ich sage euch: Es wird dem Land Sodom erträglicher gehen am Tag des Gerichts als dir!}
\verstab{25}{Schlachter}{Zu jener Zeit begann Jesus und sprach: Ich preise dich, Vater, Herr des Himmels und der Erde, dass du dies vor den Weisen und Klugen verborgen und es den Unmündigen geoffenbart hast!}
\verstab{26}{Schlachter}{Ja, Vater, denn so ist es wohlgefällig gewesen vor dir.}
\verstab{27}{Schlachter}{Alles ist mir von meinem Vater übergeben worden, und niemand erkennt den Sohn als nur der Vater; und niemand erkennt den Vater als nur der Sohn und der, welchem der Sohn es offenbaren will.}
\verstab{28}{Schlachter}{Kommt her zu mir alle, die ihr mühselig und beladen seid, so will ich euch erquicken!}
\verstab{29}{Schlachter}{Nehmt auf euch mein Joch und lernt von mir, denn ich bin sanftmütig und von Herzen demütig; so werdet ihr Ruhe finden für eure Seelen!}
\verstab{30}{Schlachter}{Denn mein Joch ist sanft und meine Last ist leicht.}
        %\subsection*{Kapitel 12}
\addcontentsline{toc}{subsection}{Kapitel 12}
\verstab{1}{Schlachter}{Zu jener Zeit ging Jesus am Sabbat durch die Kornfelder; seine Jünger aber waren hungrig und fingen an, Ähren abzustreifen und zu essen.}
\verstab{2}{Schlachter}{Als aber die Pharisäer das sahen, sprachen sie zu ihm: Siehe, deine Jünger tun, was am Sabbat zu tun nicht erlaubt ist!}
\verstab{3}{Schlachter}{Er aber sagte zu ihnen: Habt ihr nicht gelesen, was David tat, als er und seine Gefährten hungrig waren?}
\verstab{4}{Schlachter}{Wie er in das Haus Gottes hineinging und die Schaubrote aß, welche weder er noch seine Gefährten essen durften, sondern allein die Priester?}
\verstab{5}{Schlachter}{Oder habt ihr nicht im Gesetz gelesen, dass am Sabbat die Priester im Tempel den Sabbat entweihen und doch ohne Schuld sind?}
\verstab{6}{Schlachter}{Ich sage euch aber: Hier ist einer, der größer ist als der Tempel!}
\verstab{7}{Schlachter}{Wenn ihr aber wüsstet, was das heißt: »Ich will Barmherzigkeit und nicht Opfer«, so hättet ihr nicht die Unschuldigen verurteilt.}
\verstab{8}{Schlachter}{Denn der Sohn des Menschen ist Herr auch über den Sabbat.}
\verstab{9}{Schlachter}{Und er ging von dort weiter und kam in ihre Synagoge.}
\verstab{10}{Schlachter}{Und siehe, da war ein Mensch, der hatte eine verdorrte Hand. Und sie fragten ihn und sprachen: Darf man am Sabbat heilen?, damit sie ihn verklagen könnten.}
\verstab{11}{Schlachter}{Er aber sprach zu ihnen: Welcher Mensch ist unter euch, der ein Schaf hat und, wenn es am Sabbat in eine Grube fällt, es nicht ergreift und herauszieht?}
\verstab{12}{Schlachter}{Wie viel mehr ist nun ein Mensch wert als ein Schaf! Darum darf man am Sabbat wohl Gutes tun.}
\verstab{13}{Schlachter}{Dann sprach er zu dem Menschen: Strecke deine Hand aus! Und er streckte sie aus, und sie wurde gesund wie die andere.}
\verstab{14}{Schlachter}{Da gingen die Pharisäer hinaus und hielten Rat gegen ihn, wie sie ihn umbringen könnten.}
\verstab{15}{Schlachter}{Jesus aber zog sich von dort zurück, als er es bemerkte. Und es folgte ihm eine große Menge nach, und er heilte sie alle.}
\verstab{16}{Schlachter}{Und er befahl ihnen, dass sie ihn nicht offenbar machen sollten,}
\verstab{17}{Schlachter}{damit erfüllt würde, was durch den Propheten Jesaja geredet wurde, der spricht:}
\verstab{18}{Schlachter}{»Siehe, mein Knecht, den ich erwählt habe, mein Geliebter, an dem meine Seele Wohlgefallen hat! Ich will meinen Geist auf ihn legen, und er wird den Heiden das Recht verkündigen.}
\verstab{19}{Schlachter}{Er wird nicht streiten noch schreien, und niemand wird auf den Gassen seine Stimme hören.}
\verstab{20}{Schlachter}{Das geknickte Rohr wird er nicht zerbrechen, und den glimmenden Docht wird er nicht auslöschen, bis er das Recht zum Sieg hinausführt.}
\verstab{21}{Schlachter}{Und die Heiden werden auf seinen Namen hoffen.«}
\verstab{22}{Schlachter}{Da wurde ein Besessener zu ihm gebracht, der blind und stumm war, und er heilte ihn, sodass der Blinde und Stumme sowohl redete als auch sah.}
\verstab{23}{Schlachter}{Und die Volksmenge staunte und sprach: Ist dieser nicht etwa der Sohn Davids}
\verstab{24}{Schlachter}{Als aber die Pharisäer es hörten, sprachen sie: Dieser treibt die Dämonen nicht anders aus als durch Beelzebul, den Obersten der Dämonen!}
\verstab{25}{Schlachter}{Da aber Jesus ihre Gedanken kannte, sprach er zu ihnen: Jedes Reich, das mit sich selbst uneins ist, wird verwüstet, und keine Stadt, kein Haus, das mit sich selbst uneins ist, kann bestehen.}
\verstab{26}{Schlachter}{Wenn nun der Satan den Satan austreibt, so ist er mit sich selbst uneins. Wie kann dann sein Reich bestehen?}
\verstab{27}{Schlachter}{Und wenn ich die Dämonen durch Beelzebul austreibe, durch wen treiben eure Söhne sie aus? Darum werden sie eure Richter sein.}
\verstab{28}{Schlachter}{Wenn ich aber die Dämonen durch den Geist Gottes austreibe, so ist ja das Reich Gottes zu euch gekommen!}
\verstab{29}{Schlachter}{Oder wie kann jemand in das Haus des Starken hineingehen und seinen Hausrat rauben, wenn er nicht zuerst den Starken bindet? Erst dann kann er sein Haus berauben.}
\verstab{30}{Schlachter}{Wer nicht mit mir ist, der ist gegen mich, und wer nicht mit mir sammelt, der zerstreut!}
\verstab{31}{Schlachter}{Darum sage ich euch: Jede Sünde und Lästerung wird den Menschen vergeben werden; aber die Lästerung des Geistes wird den Menschen nicht vergeben werden.}
\verstab{32}{Schlachter}{Und wer ein Wort redet gegen den Sohn des Menschen, dem wird vergeben werden; wer aber gegen den Heiligen Geist redet, dem wird nicht vergeben werden, weder in dieser Weltzeit noch in der zukünftigen.}
\verstab{33}{Schlachter}{Entweder pflanzt einen guten Baum, so wird die Frucht gut, oder pflanzt einen schlechten Baum, so wird die Frucht schlecht! Denn an der Frucht erkennt man den Baum.}
\verstab{34}{Schlachter}{Schlangenbrut, wie könnt ihr Gutes reden, da ihr böse seid? Denn wovon das Herz voll ist, davon redet der Mund.}
\verstab{35}{Schlachter}{Der gute Mensch bringt aus dem guten Schatz des Herzens das Gute hervor, und der böse Mensch bringt aus seinem bösen Schatz Böses hervor.}
\verstab{36}{Schlachter}{Ich sage euch aber, dass die Menschen am Tag des Gerichts Rechenschaft geben müssen von jedem unnützen Wort, das sie geredet haben.}
\verstab{37}{Schlachter}{Denn nach deinen Worten wirst du gerechtfertigt, und nach deinen Worten wirst du verurteilt werden!}
\verstab{38}{Schlachter}{Da antworteten etliche der Schriftgelehrten und Pharisäer und sprachen: Meister, wir wollen von dir ein Zeichen sehen!}
\verstab{39}{Schlachter}{Er aber erwiderte und sprach zu ihnen: Ein böses und ehebrecherisches Geschlecht begehrt ein Zeichen; aber es wird ihm kein Zeichen gegeben werden als nur das Zeichen des Propheten Jona.}
\verstab{40}{Schlachter}{Denn gleichwie Jona drei Tage und drei Nächte im Bauch des Riesenfisches war, so wird der Sohn des Menschen drei Tage und drei Nächte im Schoß der Erde sein.}
\verstab{41}{Schlachter}{Die Männer von Ninive werden im Gericht auftreten gegen dieses Geschlecht und werden es verurteilen, denn sie taten Buße auf die Verkündigung des Jona hin; und siehe, hier ist einer, der größer ist als Jona!}
\verstab{42}{Schlachter}{Die Königin des Südens wird im Gericht auftreten gegen dieses Geschlecht und wird es verurteilen, denn sie kam vom Ende der Erde, um die Weisheit Salomos zu hören; und siehe, hier ist einer, der größer ist als Salomo!}
\verstab{43}{Schlachter}{Wenn aber der unreine Geist von dem Menschen ausgefahren ist, so durchzieht er wasserlose Stätten und sucht Ruhe und findet sie nicht.}
\verstab{44}{Schlachter}{Dann spricht er: Ich will in mein Haus zurückkehren, aus dem ich gegangen bin. Und wenn er kommt, findet er es leer, gesäubert und geschmückt.}
\verstab{45}{Schlachter}{Dann geht er hin und nimmt sieben andere Geister mit sich, die bösartiger sind als er; und sie ziehen ein und wohnen dort, und es wird zuletzt mit diesem Menschen schlimmer als zuerst. So wird es auch sein mit diesem bösen Geschlecht!}
\verstab{46}{Schlachter}{Während er aber noch zu dem Volk redete, siehe, da standen seine Mutter und seine Brüder draußen und wollten mit ihm reden.}
\verstab{47}{Schlachter}{Da sprach einer zu ihm: Siehe, deine Mutter und deine Brüder stehen draußen und wollen mit dir reden!}
\verstab{48}{Schlachter}{Er aber antwortete und sprach zu dem, der es ihm sagte: Wer ist meine Mutter, und wer sind meine Brüder?}
\verstab{49}{Schlachter}{Und er streckte seine Hand aus über seine Jünger und sprach: Seht da, meine Mutter und meine Brüder!}
\verstab{50}{Schlachter}{Denn wer den Willen meines Vaters im Himmel tut, der ist mir Bruder und Schwester und Mutter!}
        %\subsection*{Kapitel 13}
\addcontentsline{toc}{subsection}{Kapitel 13}
\verstab{1}{Schlachter}{An jenem Tag aber ging Jesus aus dem Haus hinaus und setzte sich an den See.}
\verstab{2}{Schlachter}{Und es versammelte sich eine große Volksmenge zu ihm, sodass er in das Schiff stieg und sich setzte; und alles Volk stand am Ufer.}
\verstab{3}{Schlachter}{Und er redete zu ihnen vieles in Gleichnissen und sprach: Siehe, der Sämann ging aus, um zu säen.}
\verstab{4}{Schlachter}{Und als er säte, fiel etliches an den Weg, und die Vögel kamen und fraßen es auf.}
\verstab{5}{Schlachter}{Anderes aber fiel auf den felsigen Boden, wo es nicht viel Erde hatte; und es ging sogleich auf, weil es keine tiefe Erde hatte.}
\verstab{6}{Schlachter}{Als aber die Sonne aufging, wurde es verbrannt, und weil es keine Wurzel hatte, verdorrte es.}
\verstab{7}{Schlachter}{Anderes aber fiel unter die Dornen; und die Dornen wuchsen auf und erstickten es.}
\verstab{8}{Schlachter}{Anderes aber fiel auf das gute Erdreich und brachte Frucht, etliches hundertfältig, etliches sechzigfältig und etliches dreißigfältig.}
\verstab{9}{Schlachter}{Wer Ohren hat zu hören, der höre!}
\verstab{10}{Schlachter}{Da traten die Jünger herzu und sprachen zu ihm: Warum redest du in Gleichnissen mit ihnen?}
\verstab{11}{Schlachter}{Er aber antwortete und sprach zu ihnen: Weil es euch gegeben ist, die Geheimnisse des Reiches der Himmel zu verstehen; jenen aber ist es nicht gegeben.}
\verstab{12}{Schlachter}{Denn wer hat, dem wird gegeben werden, und er wird Überfluss haben; wer aber nicht hat, von dem wird auch das genommen werden, was er hat.}
\verstab{13}{Schlachter}{Darum rede ich in Gleichnissen zu ihnen, weil sie sehen und doch nicht sehen und hören und doch nicht hören und nicht verstehen;}
\verstab{14}{Schlachter}{und es wird an ihnen die Weissagung des Jesaja erfüllt, welche lautet: »Mit den Ohren werdet ihr hören und nicht verstehen, und mit den Augen werdet ihr sehen und nicht erkennen!}
\verstab{15}{Schlachter}{Denn das Herz dieses Volkes ist verstockt, und mit den Ohren hören sie schwer, und ihre Augen haben sie verschlossen, dass sie nicht etwa mit den Augen sehen und mit den Ohren hören und mit dem Herzen verstehen und sich bekehren und ich sie heile.«}
\verstab{16}{Schlachter}{Aber glückselig sind eure Augen, dass sie sehen, und eure Ohren, dass sie hören!}
\verstab{17}{Schlachter}{Denn wahrlich, ich sage euch: Viele Propheten und Gerechte haben zu sehen begehrt, was ihr seht, und haben es nicht gesehen, und zu hören, was ihr hört, und haben es nicht gehört.}
\verstab{18}{Schlachter}{So hört nun ihr das Gleichnis vom Sämann:}
\verstab{19}{Schlachter}{Sooft jemand das Wort vom Reich hört und nicht versteht, kommt der Böse und raubt das, was in sein Herz gesät ist. Das ist der, bei dem es an den Weg gestreut war.}
\verstab{20}{Schlachter}{Auf den felsigen Boden gestreut aber ist es bei dem, der das Wort hört und sogleich mit Freuden aufnimmt;}
\verstab{21}{Schlachter}{er hat aber keine Wurzel in sich, sondern ist wetterwendisch. Wenn nun Bedrängnis oder Verfolgung entsteht um des Wortes willen, so nimmt er sogleich Anstoß.}
\verstab{22}{Schlachter}{Unter die Dornen gesät aber ist es bei dem, der das Wort hört, aber die Sorge dieser Weltzeit und der Betrug des Reichtums ersticken das Wort, und es wird unfruchtbar.}
\verstab{23}{Schlachter}{Auf das gute Erdreich gesät aber ist es bei dem, der das Wort hört und versteht; der bringt dann auch Frucht, und der eine trägt hundertfältig, ein anderer sechzigfältig, ein dritter dreißigfältig.}
\verstab{24}{Schlachter}{Ein anderes Gleichnis legte er ihnen vor und sprach: Das Reich der Himmel gleicht einem Menschen, der guten Samen auf seinen Acker säte.}
\verstab{25}{Schlachter}{Während aber die Leute schliefen, kam sein Feind und säte Unkraut mitten unter den Weizen und ging davon.}
\verstab{26}{Schlachter}{Als nun die Saat wuchs und Frucht ansetzte, da zeigte sich auch das Unkraut.}
\verstab{27}{Schlachter}{Und die Knechte des Hausherrn traten herzu und sprachen zu ihm: Herr, hast du nicht guten Samen in deinen Acker gesät? Woher hat er denn das Unkraut?}
\verstab{28}{Schlachter}{Er aber sprach zu ihnen: Das hat der Feind getan! Da sagten die Knechte zu ihm: Willst du nun, dass wir hingehen und es zusammenlesen?}
\verstab{29}{Schlachter}{Er aber sprach: Nein!, damit ihr nicht beim Zusammenlesen des Unkrauts zugleich mit ihm den Weizen ausreißt.}
\verstab{30}{Schlachter}{Lasst beides miteinander wachsen bis zur Ernte, und zur Zeit der Ernte will ich den Schnittern sagen: Lest zuerst das Unkraut zusammen und bindet es in Bündel, dass man es verbrenne; den Weizen aber sammelt in meine Scheune!}
\verstab{31}{Schlachter}{Ein anderes Gleichnis legte er ihnen vor und sprach: Das Reich der Himmel gleicht einem Senfkorn, das ein Mensch nahm und auf seinen Acker säte.}
\verstab{32}{Schlachter}{Dieses ist zwar von allen Samenkörnern das kleinste; wenn es aber wächst, so wird es größer als die Gartengewächse und wird ein Baum, sodass die Vögel des Himmels kommen und in seinen Zweigen nisten.}
\verstab{33}{Schlachter}{Ein anderes Gleichnis sagte er ihnen: Das Reich der Himmel gleicht einem Sauerteig, den eine Frau nahm und heimlich in drei Scheffel Mehl hineinmischte, bis das Ganze durchsäuert war.}
\verstab{34}{Schlachter}{Dies alles redete Jesus in Gleichnissen zu der Volksmenge, und ohne Gleichnis redete er nicht zu ihnen,}
\verstab{35}{Schlachter}{damit erfüllt würde, was durch den Propheten gesagt ist, der spricht: »Ich will meinen Mund zu Gleichnisreden öffnen; ich will verkündigen, was von Grundlegung der Welt an verborgen war«.}
\verstab{36}{Schlachter}{Da entließ Jesus die Volksmenge und ging in das Haus. Und seine Jünger traten zu ihm und sprachen: Erkläre uns das Gleichnis vom Unkraut auf dem Acker!}
\verstab{37}{Schlachter}{Und er antwortete und sprach zu ihnen: Der den guten Samen sät, ist der Sohn des Menschen.}
\verstab{38}{Schlachter}{Der Acker ist die Welt; der gute Same sind die Kinder des Reichs; das Unkraut aber sind die Kinder des Bösen.}
\verstab{39}{Schlachter}{Der Feind, der es sät, ist der Teufel; die Ernte ist das Ende der Weltzeit; die Schnitter sind die Engel.}
\verstab{40}{Schlachter}{Gleichwie man nun das Unkraut sammelt und mit Feuer verbrennt, so wird es sein am Ende dieser Weltzeit.}
\verstab{41}{Schlachter}{Der Sohn des Menschen wird seine Engel aussenden, und sie werden alle Ärgernisse und die Gesetzlosigkeit verüben aus seinem Reich sammeln}
\verstab{42}{Schlachter}{und werden sie in den Feuerofen werfen; dort wird das Heulen und das Zähneknirschen sein.}
\verstab{43}{Schlachter}{Dann werden die Gerechten leuchten wie die Sonne im Reich ihres Vaters. Wer Ohren hat zu hören, der höre!}
\verstab{44}{Schlachter}{Wiederum gleicht das Reich der Himmel einem verborgenen Schatz im Acker, den ein Mensch fand und verbarg. Und vor Freude darüber geht er hin und verkauft alles, was er hat, und kauft jenen Acker.}
\verstab{45}{Schlachter}{Wiederum gleicht das Reich der Himmel einem Kaufmann, der schöne Perlen suchte.}
\verstab{46}{Schlachter}{Als er eine kostbare Perle fand, ging er hin, verkaufte alles, was er hatte, und kaufte sie.}
\verstab{47}{Schlachter}{Wiederum gleicht das Reich der Himmel einem Netz, das ins Meer geworfen wurde und alle Arten [von Fischen] zusammenbrachte.}
\verstab{48}{Schlachter}{Als es voll war, zogen sie es ans Ufer, setzten sich und sammelten die guten in Gefäße, die faulen aber warfen sie weg.}
\verstab{49}{Schlachter}{So wird es am Ende der Weltzeit sein: Die Engel werden ausgehen und die Bösen aus der Mitte der Gerechten aussondern}
\verstab{50}{Schlachter}{und sie in den Feuerofen werfen. Dort wird das Heulen und Zähneknirschen sein.}
\verstab{51}{Schlachter}{Jesus sprach zu ihnen: Habt ihr das alles verstanden? Sie sprachen zu ihm: Ja, Herr!}
\verstab{52}{Schlachter}{Da sagte er zu ihnen: Darum gleicht jeder Schriftgelehrte, der für das Reich der Himmel unterrichtet ist, einem Hausvater, der aus seinem Schatz Neues und Altes hervorholt.}
\verstab{53}{Schlachter}{Und es geschah, als Jesus diese Gleichnisse beendet hatte, zog er von dort weg.}
\verstab{54}{Schlachter}{Und als er in seine Vaterstadt kam, lehrte er sie in ihrer Synagoge, sodass sie staunten und sprachen: Woher hat dieser solche Weisheit und solche Wunderkräfte?}
\verstab{55}{Schlachter}{Ist dieser nicht der Sohn des Zimmermanns? Heißt nicht seine Mutter Maria, und seine Brüder [heißen] Jakobus und Joses und Simon und Judas?}
\verstab{56}{Schlachter}{Und sind nicht seine Schwestern alle bei uns? Woher hat dieser denn das alles?}
\verstab{57}{Schlachter}{Und sie nahmen Anstoß an ihm. Jesus aber sprach zu ihnen: Ein Prophet ist nirgends verachtet außer in seinem Vaterland und in seinem Haus!}
\verstab{58}{Schlachter}{Und er tat dort nicht viele Wunder um ihres Unglaubens willen.}
        %\subsection*{Kapitel 14}
\addcontentsline{toc}{subsection}{Kapitel 14}
\verstab{1}{Schlachter}{Zu jener Zeit hörte der Vierfürst Herodes das Gerücht von Jesus.}
\verstab{2}{Schlachter}{Und er sprach zu seinen Dienern: Das ist Johannes der Täufer, der ist aus den Toten auferstanden; darum wirken auch die Wunderkräfte in ihm!}
\verstab{3}{Schlachter}{Denn Herodes hatte den Johannes ergreifen lassen und ihn binden und ins Gefängnis bringen lassen wegen Herodias, der Frau seines Bruders Philippus.}
\verstab{4}{Schlachter}{Denn Johannes hatte zu ihm gesagt: Es ist dir nicht erlaubt, sie zu haben!}
\verstab{5}{Schlachter}{Und er wollte ihn töten, fürchtete aber die Volksmenge, denn sie hielten ihn für einen Propheten.}
\verstab{6}{Schlachter}{Als nun Herodes seinen Geburtstag beging, tanzte die Tochter der Herodias vor den Gästen und gefiel dem Herodes.}
\verstab{7}{Schlachter}{Darum versprach er ihr mit einem Eid, ihr zu geben, was sie auch fordern würde.}
\verstab{8}{Schlachter}{Da sie aber von ihrer Mutter angeleitet war, sprach sie: Gib mir hier auf einer Schüssel das Haupt Johannes des Täufers!}
\verstab{9}{Schlachter}{Und der König wurde betrübt; doch um des Eides willen und derer, die mit ihm zu Tisch saßen, befahl er, es zu geben.}
\verstab{10}{Schlachter}{Und er sandte hin und ließ Johannes im Gefängnis enthaupten.}
\verstab{11}{Schlachter}{Und sein Haupt wurde auf einer Schüssel gebracht und dem Mädchen gegeben, und sie brachte es ihrer Mutter.}
\verstab{12}{Schlachter}{Und seine Jünger kamen herbei, nahmen den Leib und begruben ihn und gingen hin und verkündeten es Jesus.}
\verstab{13}{Schlachter}{Und als Jesus das hörte, zog er sich von dort in einem Schiff abseits an einen einsamen Ort zurück. Und als die Volksmenge es vernahm, folgte sie ihm aus den Städten zu Fuß nach.}
\verstab{14}{Schlachter}{Als nun Jesus ausstieg, sah er eine große Menge; und er erbarmte sich über sie und heilte ihre Kranken.}
\verstab{15}{Schlachter}{Und als es Abend geworden war, traten seine Jünger zu ihm und sprachen: Der Ort ist einsam, und die Stunde ist schon vorgeschritten; entlasse das Volk, damit sie in die Dörfer gehen und sich Speise kaufen!}
\verstab{16}{Schlachter}{Jesus aber sprach zu ihnen: Sie haben es nicht nötig, wegzugehen. Gebt ihr ihnen zu essen!}
\verstab{17}{Schlachter}{Sie sprachen zu ihm: Wir haben nichts hier als fünf Brote und zwei Fische.}
\verstab{18}{Schlachter}{Da sprach er: Bringt sie mir hierher!}
\verstab{19}{Schlachter}{Und er befahl der Volksmenge, sich in das Gras zu lagern, und nahm die fünf Brote und die zwei Fische, sah zum Himmel auf, dankte, brach die Brote und gab sie den Jüngern; die Jünger aber gaben sie dem Volk.}
\verstab{20}{Schlachter}{Und sie aßen alle und wurden satt; und sie hoben auf, was an Brocken übrig blieb, zwölf Körbe voll.}
\verstab{21}{Schlachter}{Die aber gegessen hatten, waren etwa 5 000 Männer, ohne Frauen und Kinder.}
\verstab{22}{Schlachter}{Und sogleich nötigte Jesus seine Jünger, in das Schiff zu steigen und vor ihm ans jenseitige Ufer zu fahren, bis er die Volksmenge entlassen hätte.}
\verstab{23}{Schlachter}{Und nachdem er die Menge entlassen hatte, stieg er auf den Berg, um abseits zu beten; und als es Abend geworden war, war er dort allein.}
\verstab{24}{Schlachter}{Das Schiff aber war schon mitten auf dem See und litt Not von den Wellen; denn der Wind stand ihnen entgegen.}
\verstab{25}{Schlachter}{Aber um die vierte Nachtwache kam Jesus zu ihnen und ging auf dem See.}
\verstab{26}{Schlachter}{Und als ihn die Jünger auf dem See gehen sahen, erschraken sie und sprachen: Es ist ein Gespenst!, und schrien vor Furcht.}
\verstab{27}{Schlachter}{Jesus aber redete sogleich mit ihnen und sprach: Seid getrost, ich bin’s; fürchtet euch nicht!}
\verstab{28}{Schlachter}{Petrus aber antwortete ihm und sprach: Herr, wenn du es bist, so befiehl mir, zu dir auf das Wasser zu kommen!}
\verstab{29}{Schlachter}{Da sprach er: Komm! Und Petrus stieg aus dem Schiff und ging auf dem Wasser, um zu Jesus zu kommen.}
\verstab{30}{Schlachter}{Als er aber den starken Wind sah, fürchtete er sich, und da er zu sinken anfing, schrie er und sprach: Herr, rette mich!}
\verstab{31}{Schlachter}{Jesus aber streckte sogleich die Hand aus, ergriff ihn und sprach zu ihm: Du Kleingläubiger, warum hast du gezweifelt?}
\verstab{32}{Schlachter}{Und als sie in das Schiff stiegen, legte sich der Wind.}
\verstab{33}{Schlachter}{Da kamen die in dem Schiff waren, warfen sich anbetend vor ihm nieder und sprachen: Wahrhaftig, du bist Gottes Sohn!}
\verstab{34}{Schlachter}{Und sie fuhren hinüber und kamen in das Land Genezareth.}
\verstab{35}{Schlachter}{Und als ihn die Männer dieser Gegend erkannten, sandten sie in die ganze Umgebung und brachten alle Kranken zu ihm.}
\verstab{36}{Schlachter}{Und sie baten ihn, dass sie nur den Saum seines Gewandes anrühren dürften; und alle, die ihn anrührten, wurden ganz gesund.}

        %\subsection*{Kapitel 15}
\addcontentsline{toc}{subsection}{Kapitel 15}
\verstab{1}{Schlachter}{Da bekamen die Schriftgelehrten und Pharisäer von Jerusalem zu Jesus und sprachen:}
\verstab{2}{Schlachter}{Warum übertreten deine Jünger die Überlieferung der Alten?Denn sie waschen ihre Hände nicht, wenn sie Brot essen.}
\verstab{3}{Schlachter}{Er aber antwortete und sprach zu ihnen: Und warum übertretet ihr das Gebot Gottes um eurer Überlieferung willen?}
\verstab{4}{Schlachter}{Denn Gott hat geboten und gesagt: "Du sollst deinen Vater und deine Mutter ehren!" und "Wer Vater oder Mutter flucht, der soll des Todes sterben!"}
\verstab{5}{Schlachter}{Ihr aber sagt: Wer zum Vater oder Mutter spricht: Ich habe zur Weihegabe bestimmt, was dir von mir zugutekommen sollte!, der braucht auch seinen Vater oder seine Mutter nicht mehr ehren.}
\verstab{6}{Schlachter}{Und so habt ihr das Gebot Gottes um eurer Überlieferung willen aufgehoben.}
\verstab{7}{Schlachter}{hr Heuchler! Treffend hat Jesaja von euch geweissagt, wenn er spricht:}
\verstab{8}{Schlachter}{»Dieses Volk naht sich zu mir mit seinem Mund und ehrt mich mit den Lippen, aber ihr Herz ist fern von mir.}
\verstab{9}{Schlachter}{Vergeblich aber verehren sie mich, weil sie Lehren vortragen, die Menschengebote sind.«}
\verstab{10}{Schlachter}{Und er rief die Volksmenge zu sich und sprach zu ihnen: Hört und versteht!}
\verstab{11}{Schlachter}{Nicht das, was zum Mund hineinkommt, verunreinigt den Menschen, sondern was aus dem Mund herauskommt, das verunreinigt den Menschen.}
\verstab{12}{Schlachter}{Da traten seine Jünger herzu und sprachen zu ihm: Weißt du, dass die Pharisäer Anstoß nahmen, als sie das Wort hörten?}
\verstab{13}{Schlachter}{Er aber antwortete und sprach: Jede Pflanze, die nicht mein himmlischer Vater gepflanzt hat, wird ausgerissen werden.}
\verstab{14}{Schlachter}{Lasst sie; sie sind blinde Blindenleiter! Wenn aber ein Blinder den anderen leitet, werden beide in die Grube fallen.}
\verstab{15}{Schlachter}{Petrus aber antwortete und sprach zu ihm: Erkläre uns dieses Gleichnis!}
\verstab{16}{Schlachter}{Jesus aber sprach: Seid denn auch ihr noch unverständig?}
\verstab{17}{Schlachter}{Begreift ihr noch nicht, dass alles, was zum Mund hineinkommt, in den Bauch kommt und in den Abort geworfen wird?}
\verstab{18}{Schlachter}{Was aber aus dem Mund herauskommt, das kommt aus dem Herzen, und das verunreinigt den Menschen.}
\verstab{19}{Schlachter}{Denn aus dem Herzen kommen böse Gedanken, Mord, Ehebruch, Unzucht, Diebstahl, falsche Zeugnisse, Lästerungen.}
\verstab{20}{Schlachter}{Das ist’s, was den Menschen verunreinigt! Aber mit ungewaschenen Händen essen, das verunreinigt den Menschen nicht.}
\verstab{21}{Schlachter}{Und Jesus ging von dort weg und zog sich in die Gegend von Tyrus und Zidon zurück.}
\verstab{22}{Schlachter}{Und siehe, eine kanaanäische Frau kam aus jener Gegend, rief ihn an und sprach: Erbarme dich über mich, Herr, du Sohn Davids! Meine Tochter ist schlimm besessen!}
\verstab{23}{Schlachter}{Er aber antwortete ihr nicht ein Wort. Da traten seine Jünger herzu, baten ihn und sprachen: Fertige sie ab, denn sie schreit uns nach!}
\verstab{24}{Schlachter}{Er aber antwortete und sprach: Ich bin nur gesandt zu den verlorenen Schafen des Hauses Israel.}
\verstab{25}{Schlachter}{Da kam sie, fiel vor ihm nieder und sprach: Herr, hilf mir!}
\verstab{26}{Schlachter}{Er aber antwortete und sprach: Es ist nicht recht, dass man das Brot der Kinder nimmt und es den Hunden vorwirft.}
\verstab{27}{Schlachter}{Sie aber sprach: Ja, Herr; und doch essen die Hunde von den Brosamen, die vom Tisch ihrer Herren fallen!}
\verstab{28}{Schlachter}{Da antwortete Jesus und sprach zu ihr: O Frau, dein Glaube ist groß; dir geschehe, wie du willst! Und ihre Tochter war geheilt von jener Stunde an.}
\verstab{29}{Schlachter}{Und Jesus zog von dort weiter und kam an den See von Galiläa; und er stieg auf den Berg und setzte sich dort.}
\verstab{30}{Schlachter}{Und es kamen große Volksmengen zu ihm, die hatten Lahme, Blinde, Stumme, Krüppel und viele andere bei sich. Und sie legten sie zu Jesu Füßen, und er heilte sie,}
\verstab{31}{Schlachter}{sodass sich die Menge verwunderte, als sie sah, dass Stumme redeten, Krüppel gesund wurden, Lahme gingen und Blinde sehend wurden; und sie priesen den Gott Israels.}
\verstab{32}{Schlachter}{Da rief Jesus seine Jünger zu sich und sprach: Ich bin voll Mitleid mit der Menge; denn sie verharren nun schon drei Tage bei mir und haben nichts zu essen, und ich will sie nicht ohne Speise entlassen, damit sie nicht auf dem Weg verschmachten.}
\verstab{33}{Schlachter}{Und seine Jünger sprachen zu ihm: Woher sollen wir in der Einöde so viele Brote nehmen, um eine so große Menge zu sättigen?}
\verstab{34}{Schlachter}{Und Jesus sprach zu ihnen: Wie viele Brote habt ihr? Sie sprachen: Sieben, und ein paar Fische.}
\verstab{35}{Schlachter}{Da gebot er dem Volk, sich auf die Erde zu lagern,}
\verstab{36}{Schlachter}{und nahm die sieben Brote und die Fische, dankte, brach sie und gab sie seinen Jüngern; die Jünger aber gaben sie dem Volk.}
\verstab{37}{Schlachter}{Und sie aßen alle und wurden satt und hoben auf, was an Brocken übrig blieb, sieben Körbe voll.}
\verstab{38}{Schlachter}{Es waren aber etwa 4 000 Männer, die gegessen hatten, ohne Frauen und Kinder.}
\verstab{39}{Schlachter}{Und nachdem er die Volksmenge entlassen hatte, stieg er in das Schiff und kam in die Gegend von Magdala.}
        %\subsection*{Kapitel 16}
\addcontentsline{toc}{subsection}{Kapitel 16}
\verstab{1}{Schlachter}{Und die Pharisäer und Sadduzäer traten herzu, versuchten ihn und verlangten, dass er ihnen ein Zeichen aus dem Himmel zeigen möge.}
\verstab{2}{Schlachter}{Er aber antwortete und sprach zu ihnen: Am Abend sagt ihr: Es wird schön, denn der Himmel ist rot!,}
\verstab{3}{Schlachter}{und am Morgen: Heute kommt ein Ungewitter, denn der Himmel ist rot und trübe! Ihr Heuchler, das Aussehen des Himmels versteht ihr zu beurteilen, die Zeichen der Zeit aber nicht!}
\verstab{4}{Schlachter}{Ein böses und ehebrecherisches Geschlecht begehrt ein Zeichen, aber es wird ihm kein Zeichen gegeben werden als nur das Zeichen des Propheten Jona! Und er verließ sie und ging davon.}
\verstab{5}{Schlachter}{Als seine Jünger ans jenseitige Ufer kamen, hatten sie vergessen, Brot mitzunehmen.}
\verstab{6}{Schlachter}{Jesus aber sprach zu ihnen: Habt acht und hütet euch vor dem Sauerteig der Pharisäer und Sadduzäer!}
\verstab{7}{Schlachter}{Da machten sie sich untereinander Gedanken und sagten: Weil wir kein Brot mitgenommen haben!}
\verstab{8}{Schlachter}{Als es aber Jesus merkte, sprach er zu ihnen: Ihr Kleingläubigen, was macht ihr euch Gedanken darüber, dass ihr kein Brot mitgenommen habt?}
\verstab{9}{Schlachter}{Versteht ihr noch nicht, und denkt ihr nicht an die fünf Brote für die Fünftausend, und wie viele Körbe ihr da aufgehoben habt?}
\verstab{10}{Schlachter}{Auch nicht an die sieben Brote für die Viertausend, und wie viele Körbe ihr da aufgehoben habt?}
\verstab{11}{Schlachter}{Warum versteht ihr denn nicht, dass ich euch nicht wegen des Brotes gesagt habe, dass ihr euch vor dem Sauerteig der Pharisäer und Sadduzäer hüten solltet?}
\verstab{12}{Schlachter}{Da sahen sie ein, dass er nicht gesagt hatte, sie sollten sich hüten vor dem Sauerteig des Brotes, sondern vor der Lehre der Pharisäer und Sadduzäer.}
\verstab{13}{Schlachter}{Als aber Jesus in die Gegend von Cäsarea Philippi gekommen war, fragte er seine Jünger und sprach: Für wen halten die Leute mich, den Sohn des Menschen?}
\verstab{14}{Schlachter}{Sie sprachen: Etliche für Johannes den Täufer; andere aber für Elia; noch andere für Jeremia oder einen der Propheten.}
\verstab{15}{Schlachter}{Da spricht er zu ihnen: Ihr aber, für wen haltet ihr mich?}
\verstab{16}{Schlachter}{Da antwortete Simon Petrus und sprach: Du bist der Christus, der Sohn des lebendigen Gottes!}
\verstab{17}{Schlachter}{Und Jesus antwortete und sprach zu ihm: Glückselig bist du, Simon, Sohn des Jona; denn Fleisch und Blut hat dir das nicht geoffenbart, sondern mein Vater im Himmel!}
\verstab{18}{Schlachter}{Und ich sage dir auch: Du bist Petrus, und auf diesen Felsen will ich meine Gemeinde bauen,und die Pforten des Totenreiches sollen sie nicht überwältigen.}
\verstab{19}{Schlachter}{Und ich will dir die Schlüssel des Reiches der Himmel geben; und was du auf Erden binden wirst, das wird im Himmel gebunden sein; und was du auf Erden lösen wirst, das wird im Himmel gelöst sein.}
\verstab{20}{Schlachter}{Da gebot er seinen Jüngern, dass sie niemand sagen sollten, dass er Jesus der Christus sei.}
\verstab{21}{Schlachter}{Von da an begann Jesus seinen Jüngern zu zeigen, dass er nach Jerusalem gehen und viel leiden müsse von den Ältesten, den obersten Priestern und Schriftgelehrten, und getötet werden und am dritten Tag auferweckt werden müsse.}
\verstab{22}{Schlachter}{Da nahm Petrus ihn beiseite und fing an, ihm zu wehren und sprach: Herr, schone dich selbst! Das widerfahre dir nur nicht!}
\verstab{23}{Schlachter}{Er aber wandte sich um und sprach zu Petrus: Weiche von mir, Satan! Du bist mir ein Ärgernis; denn du denkst nicht göttlich, sondern menschlich!}
\verstab{24}{Schlachter}{Da sprach Jesus zu seinen Jüngern: Wenn jemand mir nachkommen will, so verleugne er sich selbst und nehme sein Kreuz auf sich und folge mir nach!}
\verstab{25}{Schlachter}{Denn wer sein Leben retten will, der wird es verlieren; wer aber sein Leben verliert um meinetwillen, der wird es finden.}
\verstab{26}{Schlachter}{Denn was hilft es dem Menschen, wenn er die ganze Welt gewinnt, aber sein Leben verliert? Oder was kann der Mensch als Lösegeld für sein Leben geben?}
\verstab{27}{Schlachter}{Denn der Sohn des Menschen wird in der Herrlichkeit seines Vaters mit seinen Engeln kommen, und dann wird er jedem Einzelnen vergelten nach seinem Tun.}
\verstab{28}{Schlachter}{Wahrlich, ich sage euch: Es stehen einige hier, die den Tod nicht schmecken werden, bis sie den Sohn des Menschen haben kommen sehen in seinem Reich!}
        %\input{Matthäus/matth_17}
        %\subsection*{Kapitel 18}
\addcontentsline{toc}{subsection}{Kapitel 18}
\verstab{1}{Schlachter}{Zu jener Stunde traten die Jünger zu Jesus und sprachen: Wer ist wohl der Größte im Reich der Himmel?}
\verstab{2}{Schlachter}{Und Jesus rief ein Kind herbei, stellte es in ihre Mitte}
\verstab{3}{Schlachter}{und sprach: Wahrlich, ich sage euch: Wenn ihr nicht umkehrt und werdet wie die Kinder, so werdet ihr nicht in das Reich der Himmel kommen!}
\verstab{4}{Schlachter}{Wer nun sich selbst erniedrigt wie dieses Kind, der ist der Größte im Reich der Himmel.}
\verstab{5}{Schlachter}{Und wer ein solches Kind in meinem Namen aufnimmt, der nimmt mich auf.}
\verstab{6}{Schlachter}{Wer aber einem von diesen Kleinen, die an mich glauben, Anstoß [zur Sünde] gibt, für den wäre es besser, dass ein großer Mühlstein an seinen Hals gehängt und er in die Tiefe des Meeres versenkt würde.}
\verstab{7}{Schlachter}{Wehe der Welt wegen der Anstöße [zur Sünde]! Denn es ist zwar notwendig, dass die Anstöße [zur Sünde] kommen, aber wehe jenem Menschen, durch den der Anstoß [zur Sünde] kommt!}
\verstab{8}{Schlachter}{Wenn aber deine Hand oder dein Fuß für dich ein Anstoß [zur Sünde] wird, so haue sie ab und wirf sie von dir! Es ist besser für dich, dass du lahm oder verstümmelt in das Leben eingehst, als dass du zwei Hände oder zwei Füße hast und in das ewige Feuer geworfen wirst.}
\verstab{9}{Schlachter}{Und wenn dein Auge für dich ein Anstoß [zur Sünde] wird, so reiß es aus und wirf es von dir! Es ist besser für dich, dass du einäugig in das Leben eingehst, als dass du zwei Augen hast und in das höllische Feuer geworfen wirst.}
\verstab{10}{Schlachter}{Seht zu, dass ihr keinen dieser Kleinen verachtet! Denn ich sage euch: Ihre Engel im Himmel schauen allezeit das Angesicht meines Vaters im Himmel.}
\verstab{11}{Schlachter}{Denn der Sohn des Menschen ist gekommen, um das Verlorene zu retten.}
\verstab{12}{Schlachter}{Was meint ihr? Wenn ein Mensch hundert Schafe hat, und es verirrt sich eines von ihnen, lässt er nicht die neunundneunzig auf den Bergen, geht hin und sucht das verirrte?}
\verstab{13}{Schlachter}{Und wenn es geschieht, dass er es findet, wahrlich, ich sage euch: Er freut sich darüber mehr als über die neunundneunzig, die nicht verirrt waren.}
\verstab{14}{Schlachter}{So ist es auch nicht der Wille eures Vaters im Himmel, dass eines dieser Kleinen verlorengeht.}
\verstab{15}{Schlachter}{Wenn aber dein Bruder an dir gesündigt hat, so geh hin und weise ihn zurecht unter vier Augen. Hört er auf dich, so hast du deinen Bruder gewonnen.}
\verstab{16}{Schlachter}{Hört er aber nicht, so nimm noch einen oder zwei mit dir, damit jede Sache auf der Aussage von zwei oder drei Zeugen beruht.}
\verstab{17}{Schlachter}{Hört er aber auf diese nicht, so sage es der Gemeinde. Hört er aber auch auf die Gemeinde nicht, so sei er für dich wie ein Heide und ein Zöllner.}
\verstab{18}{Schlachter}{Wahrlich, ich sage euch: Was ihr auf Erden binden werdet, das wird im Himmel gebunden sein, und was ihr auf Erden lösen werdet, das wird im Himmel gelöst sein.}
\verstab{19}{Schlachter}{Weiter sage ich euch: Wenn zwei von euch auf Erden übereinkommen über irgendeine Sache, für die sie bitten wollen, so soll sie ihnen zuteilwerden von meinem Vater im Himmel.}
\verstab{20}{Schlachter}{Denn wo zwei oder drei in meinem Namen versammelt sind, da bin ich in ihrer Mitte.}
\verstab{21}{Schlachter}{Da trat Petrus zu ihm und sprach: Herr, wie oft soll ich meinem Bruder vergeben, der gegen mich sündigt? Bis siebenmal?}
\verstab{22}{Schlachter}{Jesus antwortete ihm: Ich sage dir, nicht bis siebenmal, sondern bis siebzigmalsiebenmal!}
\verstab{23}{Schlachter}{Darum gleicht das Reich der Himmel einem König, der mit seinen Knechten abrechnen wollte.}
\verstab{24}{Schlachter}{Und als er anfing abzurechnen, wurde einer vor ihn gebracht, der war 10 000 Talente schuldig.}
\verstab{25}{Schlachter}{Weil er aber nicht bezahlen konnte, befahl sein Herr, ihn und seine Frau und seine Kinder und alles, was er hatte, zu verkaufen und so zu bezahlen.}
\verstab{26}{Schlachter}{Da warf sich der Knecht nieder, huldigte ihm und sprach: Herr, habe Geduld mit mir, so will ich dir alles bezahlen!}
\verstab{27}{Schlachter}{Da erbarmte sich der Herr über diesen Knecht, gab ihn frei und erließ ihm die Schuld.}
\verstab{28}{Schlachter}{Als aber dieser Knecht hinausging, fand er einen Mitknecht, der war ihm 100 Denare schuldig; den ergriff er, würgte ihn und sprach: Bezahle mir, was du schuldig bist!}
\verstab{29}{Schlachter}{Da warf sich ihm sein Mitknecht zu Füßen, bat ihn und sprach: Habe Geduld mit mir, so will ich dir alles bezahlen!}
\verstab{30}{Schlachter}{Er aber wollte nicht, sondern ging hin und warf ihn ins Gefängnis, bis er bezahlt hätte, was er schuldig war.}
\verstab{31}{Schlachter}{Als aber seine Mitknechte sahen, was geschehen war, wurden sie sehr betrübt, kamen und berichteten ihrem Herrn den ganzen Vorfall.}
\verstab{32}{Schlachter}{Da ließ sein Herr ihn kommen und sprach zu ihm: Du böser Knecht! Jene ganze Schuld habe ich dir erlassen, weil du mich batest;}
\verstab{33}{Schlachter}{solltest denn nicht auch du dich über deinen Mitknecht erbarmen, wie ich mich über dich erbarmt habe?}
\verstab{34}{Schlachter}{Und voll Zorn übergab ihn sein Herr den Folterknechten, bis er alles bezahlt hätte, was er ihm schuldig war.}
\verstab{35}{Schlachter}{So wird auch mein himmlischer Vater euch behandeln, wenn ihr nicht jeder seinem Bruder von Herzen seine Verfehlungen vergebt.}
        %\subsection*{Kapitel 19}
\addcontentsline{toc}{subsection}{Kapitel 19}
\verstab{1}{Schlachter}{Und es geschah, als Jesus diese Worte beendet hatte, verließ er Galiläa und kam in das Gebiet von Judäa jenseits des Jordan.}
\verstab{2}{Schlachter}{Und es folgte ihm eine große Volksmenge nach, und er heilte sie dort.}
\verstab{3}{Schlachter}{Da traten die Pharisäer zu ihm, versuchten ihn und fragten ihn: Ist es einem Mann erlaubt, aus irgendeinem Grund seine Frau zu entlassen?}
\verstab{4}{Schlachter}{Er aber antwortete und sprach zu ihnen: Habt ihr nicht gelesen, dass der Schöpfer sie am Anfang als Mann und Frau erschuf}
\verstab{5}{Schlachter}{und sprach: »Darum wird ein Mann Vater und Mutter verlassen und seiner Frau anhängen; und die zwei werden ein Fleisch sein«?}
\verstab{6}{Schlachter}{So sind sie nicht mehr zwei, sondern ein Fleisch. Was nun Gott zusammengefügt hat, das soll der Mensch nicht scheiden!}
\verstab{7}{Schlachter}{Da sprachen sie zu ihm: Warum hat denn Mose befohlen, ihr einen Scheidebrief zu geben und sie so zu entlassen?}
\verstab{8}{Schlachter}{Er sprach zu ihnen: Mose hat euch wegen der Härtigkeit eures Herzens erlaubt, eure Frauen zu entlassen; von Anfang an aber ist es nicht so gewesen.}
\verstab{9}{Schlachter}{Ich sage euch aber: Wer seine Frau entlässt, es sei denn wegen Unzucht, und eine andere heiratet, der bricht die Ehe; und wer eine Geschiedene heiratet, der bricht die Ehe.}
\verstab{10}{Schlachter}{Da sprechen seine Jünger zu ihm: Wenn ein Mann solche Pflichten gegen seine Frau hat, so ist es nicht gut, zu heiraten!}
\verstab{11}{Schlachter}{Er aber sprach zu ihnen: Nicht alle fassen dieses Wort, sondern nur die, denen es gegeben ist.}
\verstab{12}{Schlachter}{Denn es gibt Verschnittene, die von Mutterleib so geboren sind; und es gibt Verschnittene, die von Menschen verschnitten sind; und es gibt Verschnittene, die sich selbst verschnitten haben um des Reiches der Himmel willen. Wer es fassen kann, der fasse es!}
\verstab{13}{Schlachter}{Da wurden Kinder zu ihm gebracht, damit er die Hände auf sie lege und bete. Die Jünger aber tadelten sie.}
\verstab{14}{Schlachter}{Aber Jesus sprach: Lasst die Kinder und wehrt ihnen nicht, zu mir zu kommen; denn solcher ist das Reich der Himmel!}
\verstab{15}{Schlachter}{Und nachdem er ihnen die Hände aufgelegt hatte, zog er von dort weg.}
\verstab{16}{Schlachter}{Und siehe, einer trat herzu und fragte ihn: Guter Meister, was soll ich Gutes tun, um das ewige Leben zu erlangen?}
\verstab{17}{Schlachter}{Er aber sprach zu ihm: Was nennst du mich gut? Niemand ist gut als Gott allein! Willst du aber in das Leben eingehen, so halte die Gebote!}
\verstab{18}{Schlachter}{Er sagt zu ihm: Welche? Jesus aber sprach: Das »Du sollst nicht töten! Du sollst nicht ehebrechen! Du sollst nicht stehlen! Du sollst nicht falsches Zeugnis reden!}
\verstab{19}{Schlachter}{Ehre deinen Vater und deine Mutter!« und »Du sollst deinen Nächsten lieben wie dich selbst!«}
\verstab{20}{Schlachter}{Der junge Mann spricht zu ihm: Das habe ich alles gehalten von meiner Jugend an; was fehlt mir noch?}
\verstab{21}{Schlachter}{Jesus sprach zu ihm: Willst du vollkommen sein, so geh hin, verkaufe, was du hast, und gib es den Armen, so wirst du einen Schatz im Himmel haben; und komm, folge mir nach!}
\verstab{22}{Schlachter}{Als aber der junge Mann das Wort hörte, ging er betrübt davon; denn er hatte viele Güter.}
\verstab{23}{Schlachter}{Da sprach Jesus zu seinen Jüngern: Wahrlich, ich sage euch: Ein Reicher hat es schwer, in das Reich der Himmel hineinzukommen!}
\verstab{24}{Schlachter}{Und wiederum sage ich euch: Es ist leichter, dass ein Kamel durch ein Nadelöhr geht, als dass ein Reicher in das Reich Gottes hineinkommt!}
\verstab{25}{Schlachter}{Als seine Jünger das hörten, entsetzten sie sich sehr und sprachen: Wer kann dann überhaupt gerettet werden?}
\verstab{26}{Schlachter}{Jesus aber sah sie an und sprach zu ihnen: Bei den Menschen ist dies unmöglich; aber bei Gott sind alle Dinge möglich.}
\verstab{27}{Schlachter}{Da antwortete Petrus und sprach zu ihm: Siehe, wir haben alles verlassen und sind dir nachgefolgt; was wird uns dafür zuteil?}
\verstab{28}{Schlachter}{Jesus aber sprach zu ihnen: Wahrlich, ich sage euch: Ihr, die ihr mir nachgefolgt seid, werdet in der Wiedergeburt, wenn der Sohn des Menschen auf dem Thron seiner Herrlichkeit sitzen wird, auch auf zwölf Thronen sitzen und die zwölf Stämme Israels richten.}
\verstab{29}{Schlachter}{Und jeder, der Häuser oder Brüder oder Schwestern oder Vater oder Mutter oder Frau oder Kinder oder Äcker verlassen hat um meines Namens willen, der wird es hundertfältig empfangen und das ewige Leben erben.}
\verstab{30}{Schlachter}{Aber viele von den Ersten werden Letzte, und Letzte werden Erste sein.}
        %\subsection*{Kapitel 20}
\addcontentsline{toc}{subsection}{Kapitel 20}
\verstab{1}{Schlachter}{Denn das Reich der Himmel gleicht einem Hausherrn, der am Morgen früh ausging, um Arbeiter in seinen Weinberg einzustellen.}
\verstab{2}{Schlachter}{Und nachdem er mit den Arbeitern um einen Denar für den Tag übereingekommen war, sandte er sie in seinen Weinberg.}
\verstab{3}{Schlachter}{Als er um die dritte Stunde ausging, sah er andere auf dem Markt untätig stehen}
\verstab{4}{Schlachter}{und sprach zu diesen: Geht auch ihr in den Weinberg, und was recht ist, will ich euch geben!}
\verstab{5}{Schlachter}{Und sie gingen hin. Wiederum ging er aus um die sechste und um die neunte Stunde und tat dasselbe.}
\verstab{6}{Schlachter}{Als er aber um die elfte Stunde ausging, fand er andere untätig dastehen und sprach zu ihnen: Warum steht ihr hier den ganzen Tag untätig?}
\verstab{7}{Schlachter}{Sie sprachen zu ihm: Es hat uns niemand eingestellt! Er spricht zu ihnen: Geht auch ihr in den Weinberg, und was recht ist, das werdet ihr empfangen!}
\verstab{8}{Schlachter}{Als es aber Abend geworden war, sprach der Herr des Weinbergs zu seinem Verwalter: Rufe die Arbeiter und bezahle ihnen den Lohn, indem du bei den Letzten anfängst, bis zu den Ersten.}
\verstab{9}{Schlachter}{Und es kamen die, welche um die elfte Stunde [eingestellt worden waren], und empfingen jeder einen Denar.}
\verstab{10}{Schlachter}{Als aber die Ersten kamen, meinten sie, sie würden mehr empfangen; da empfingen auch sie jeder einen Denar.}
\verstab{11}{Schlachter}{Und als sie ihn empfangen hatten, murrten sie gegen den Hausherrn}
\verstab{12}{Schlachter}{und sprachen: Diese Letzten haben nur eine Stunde gearbeitet, und du hast sie uns gleichgemacht, die wir die Last und Hitze des Tages getragen haben!}
\verstab{13}{Schlachter}{Er aber antwortete und sprach zu einem unter ihnen: Freund, ich tue dir nicht unrecht. Bist du nicht um einen Denar mit mir übereingekommen?}
\verstab{14}{Schlachter}{Nimm das Deine und geh hin! Ich will aber diesem Letzten so viel geben wie dir.}
\verstab{15}{Schlachter}{Oder habe ich nicht Macht, mit dem Meinen zu tun, was ich will? Blickst du darum neidisch, weil ich gütig bin?}
\verstab{16}{Schlachter}{So werden die Letzten die Ersten und die Ersten die Letzten sein. Denn viele sind berufen, aber wenige auserwählt.}
\verstab{17}{Schlachter}{Und als Jesus nach Jerusalem hinaufzog, nahm er die zwölf Jünger auf dem Weg beiseite und sprach zu ihnen:}
\verstab{18}{Schlachter}{Siehe, wir ziehen hinauf nach Jerusalem, und der Sohn des Menschen wird den obersten Priestern und Schriftgelehrten ausgeliefert werden, und sie werden ihn zum Tode verurteilen}
\verstab{19}{Schlachter}{und werden ihn den Heiden ausliefern, damit diese ihn verspotten und geißeln und kreuzigen; und am dritten Tag wird er auferstehen.}
\verstab{20}{Schlachter}{Da trat die Mutter der Söhne des Zebedäus mit ihren Söhnen zu ihm und warf sich vor ihm nieder, um etwas von ihm zu erbitten.}
\verstab{21}{Schlachter}{Er aber sprach zu ihr: Was willst du? Sie sagt zu ihm: Sprich, dass diese meine beiden Söhne einer zu deiner Rechten, der andere zur Linken sitzen sollen in deinem Reich!}
\verstab{22}{Schlachter}{Aber Jesus antwortete und sprach: Ihr wisst nicht, um was ihr bittet! Könnt ihr den Kelch trinken, den ich trinke, und getauft werden mit der Taufe, womit ich getauft werde? Sie sprechen zu ihm: Wir können es!}
\verstab{23}{Schlachter}{Und er spricht zu ihnen: Ihr werdet zwar meinen Kelch trinken und getauft werden mit der Taufe, womit ich getauft werde. Aber das Sitzen zu meiner Rechten und zu meiner Linken zu verleihen, steht nicht mir zu, sondern es wird denen zuteil, denen es von meinem Vater bereitet ist.}
\verstab{24}{Schlachter}{Und als die Zehn es hörten, wurden sie unwillig über die beiden Brüder.}
\verstab{25}{Schlachter}{Aber Jesus rief sie zu sich und sprach: Ihr wisst, dass die Fürsten der Heidenvölker sie unterdrücken und dass die Großen Gewalt über sie ausüben.}
\verstab{26}{Schlachter}{Unter euch aber soll es nicht so sein; sondern wer unter euch groß werden will, der sei euer Diener,}
\verstab{27}{Schlachter}{und wer unter euch der Erste sein will, der sei euer Knecht,}
\verstab{28}{Schlachter}{gleichwie der Sohn des Menschen nicht gekommen ist, um sich dienen zu lassen, sondern um zu dienen und sein Leben zu geben als Lösegeld für viele.}
\verstab{29}{Schlachter}{Und als sie von Jericho auszogen, folgte ihm eine große Volksmenge nach.}
\verstab{30}{Schlachter}{Und siehe, zwei Blinde saßen am Weg. Als sie hörten, dass Jesus vorüberziehe, riefen sie und sprachen: Herr, du Sohn Davids, erbarme dich über uns!}
\verstab{31}{Schlachter}{Aber das Volk gebot ihnen, sie sollten schweigen. Sie aber riefen nur noch mehr und sprachen: Herr, du Sohn Davids, erbarme dich über uns!}
\verstab{32}{Schlachter}{Und Jesus stand still, rief sie und sprach: Was wollt ihr, dass ich euch tun soll?}
\verstab{33}{Schlachter}{Sie sagten zu ihm: Herr, dass unsere Augen geöffnet werden!}
\verstab{34}{Schlachter}{Da erbarmte sich Jesus über sie und rührte ihre Augen an, und sogleich wurden ihre Augen wieder sehend, und sie folgten ihm nach.}
        %\subsection*{Kapitel 21}
\addcontentsline{toc}{subsection}{Kapitel 21}
\verstab{1}{Schlachter}{Als sie sich nun Jerusalem näherten und nach Bethphage an den Ölberg kamen, sandte Jesus zwei Jünger}
\verstab{2}{Schlachter}{und sprach zu ihnen: Geht in das Dorf, das vor euch liegt, und sogleich werdet ihr eine Eselin angebunden finden und ein Füllen bei ihr; die bindet los und führt sie zu mir!}
\verstab{3}{Schlachter}{Und wenn euch jemand etwas sagt, so sprecht: Der Herr braucht sie!, dann wird er sie sogleich senden.}
\verstab{4}{Schlachter}{Das ist aber alles geschehen, damit erfüllt würde, was durch den Propheten gesagt ist, der spricht:}
\verstab{5}{Schlachter}{»Sagt der Tochter Zion: Siehe, dein König kommt zu dir demütig und reitend auf einem Esel, und zwar auf einem Füllen, dem Jungen des Lasttiers«.}
\verstab{6}{Schlachter}{Die Jünger aber gingen hin und taten, wie Jesus ihnen befohlen hatte,}
\verstab{7}{Schlachter}{und brachten die Eselin und das Füllen und legten ihre Kleider auf sie und setzten ihn darauf.}
\verstab{8}{Schlachter}{Aber die meisten aus der Menge breiteten ihre Kleider aus auf dem Weg; andere hieben Zweige von den Bäumen und streuten sie auf den Weg.}
\verstab{9}{Schlachter}{Und die Volksmenge, die vorausging, und die, welche nachfolgten, riefen und sprachen: Hosianna dem Sohn Davids! Gepriesen sei der, welcher kommt im Namen des Herrn! Hosianna in der Höhe!}
\verstab{10}{Schlachter}{Und als er in Jerusalem einzog, kam die ganze Stadt in Bewegung und sprach: Wer ist dieser?}
\verstab{11}{Schlachter}{Die Menge aber sagte: Das ist Jesus, der Prophet von Nazareth in Galiläa!}
\verstab{12}{Schlachter}{Und Jesus ging in den Tempel Gottes hinein und trieb alle hinaus, die im Tempel verkauften und kauften, und stieß die Tische der Wechsler um und die Stühle der Taubenverkäufer.}
\verstab{13}{Schlachter}{Und er sprach zu ihnen: Es steht geschrieben: »Mein Haus soll ein Bethaus genannt werden! «Ihr aber habt eine Räuberhöhle daraus gemacht!}
\verstab{14}{Schlachter}{Und es kamen Blinde und Lahme im Tempel zu ihm, und er heilte sie.}
\verstab{15}{Schlachter}{Als aber die obersten Priester und die Schriftgelehrten die Wunder sahen, die er tat, und die Kinder, die im Tempel riefen und sprachen: Hosianna dem Sohn Davids!, da wurden sie entrüstet}
\verstab{16}{Schlachter}{und sprachen zu ihm: Hörst du, was diese sagen? Jesus aber sprach zu ihnen: Ja! Habt ihr noch nie gelesen: »Aus dem Mund der Unmündigen und Säuglinge hast du ein Lob bereitet«?}
\verstab{17}{Schlachter}{Und er verließ sie, ging zur Stadt hinaus nach Bethanien und übernachtete dort.}
\verstab{18}{Schlachter}{Als er aber früh am Morgen in die Stadt zurückkehrte, hatte er Hunger.}
\verstab{19}{Schlachter}{Und als er einen einzelnen Feigenbaum am Weg sah, ging er zu ihm hin und fand nichts daran als nur Blätter. Da sprach er zu ihm: Nun soll von dir keine Frucht mehr kommen in Ewigkeit! Und auf der Stelle verdorrte der Feigenbaum.}
\verstab{20}{Schlachter}{Und als die Jünger es sahen, verwunderten sie sich und sprachen: Wie ist der Feigenbaum so plötzlich verdorrt?}
\verstab{21}{Schlachter}{Jesus aber antwortete und sprach zu ihnen: Wahrlich, ich sage euch: Wenn ihr Glauben habt und nicht zweifelt, so werdet ihr nicht nur tun, was mit dem Feigenbaum geschah, sondern auch, wenn ihr zu diesem Berg sagt: Hebe dich und wirf dich ins Meer!, so wird es geschehen.}
\verstab{22}{Schlachter}{Und alles, was ihr glaubend erbittet im Gebet, das werdet ihr empfangen!}
\verstab{23}{Schlachter}{Und als er in den Tempel kam, traten die obersten Priester und die Ältesten des Volkes zu ihm, während er lehrte, und sprachen: In welcher Vollmacht tust du dies, und wer hat dir diese Vollmacht gegeben?}
\verstab{24}{Schlachter}{Und Jesus antwortete und sprach zu ihnen: Auch ich will euch ein Wort fragen; wenn ihr mir darauf antwortet, will ich euch auch sagen, in welcher Vollmacht ich dies tue.}
\verstab{25}{Schlachter}{Woher war die Taufe des Johannes? Vom Himmel oder von Menschen? Da überlegten sie bei sich selbst und sprachen: Wenn wir sagen: Vom Himmel, so wird er uns fragen: Warum habt ihr ihm dann nicht geglaubt?}
\verstab{26}{Schlachter}{Wenn wir aber sagen: Von Menschen, so müssen wir die Volksmenge fürchten, denn alle halten Johannes für einen Propheten.}
\verstab{27}{Schlachter}{Und sie antworteten Jesus und sprachen: Wir wissen es nicht! Da sprach er zu ihnen: So sage ich euch auch nicht, in welcher Vollmacht ich dies tue.}
\verstab{28}{Schlachter}{Was meint ihr aber? Ein Mensch hatte zwei Söhne. Und er ging zu dem ersten und sprach: Sohn, mache dich auf und arbeite heute in meinem Weinberg!}
\verstab{29}{Schlachter}{Der aber antwortete und sprach: Ich will nicht! Danach aber reute es ihn, und er ging.}
\verstab{30}{Schlachter}{Und er ging zu dem zweiten und sagte dasselbe. Da antwortete dieser und sprach: Ich [gehe], Herr! und ging nicht.}
\verstab{31}{Schlachter}{Wer von diesen beiden hat den Willen des Vaters getan? Sie sprachen zu ihm: Der erste. Da spricht Jesus zu ihnen: Wahrlich, ich sage euch: Die Zöllner und die Huren kommen eher in das Reich Gottes als ihr!}
\verstab{32}{Schlachter}{Denn Johannes ist zu euch gekommen mit dem Weg der Gerechtigkeit, und ihr habt ihm nicht geglaubt. Die Zöllner und die Huren aber glaubten ihm; und obwohl ihr es gesehen habt, reute es euch nicht nachträglich, sodass ihr ihm geglaubt hättet.}
\verstab{33}{Schlachter}{Hört ein anderes Gleichnis: Es war ein gewisser Hausherr, der pflanzte einen Weinberg, zog einen Zaun darum, grub eine Kelter darin, baute einen Wachtturm, verpachtete ihn an Weingärtner und reiste außer Landes.}
\verstab{34}{Schlachter}{Als nun die Zeit der Früchte nahte, sandte er seine Knechte zu den Weingärtnern, um seine Früchte in Empfang zu nehmen.}
\verstab{35}{Schlachter}{Aber die Weingärtner ergriffen seine Knechte und schlugen den einen, den anderen töteten sie, den dritten steinigten sie.}
\verstab{36}{Schlachter}{Da sandte er wieder andere Knechte, mehr als zuvor; und sie behandelten sie ebenso.}
\verstab{37}{Schlachter}{Zuletzt sandte er seinen Sohn zu ihnen und sprach: Sie werden sich vor meinem Sohn scheuen!}
\verstab{38}{Schlachter}{Als aber die Weingärtner den Sohn sahen, sprachen sie untereinander: Das ist der Erbe! Kommt, lasst uns ihn töten und sein Erbgut in Besitz nehmen!}
\verstab{39}{Schlachter}{Und sie ergriffen ihn, stießen ihn zum Weinberg hinaus und töteten ihn.}
\verstab{40}{Schlachter}{Wenn nun der Herr des Weinbergs kommt, was wird er mit diesen Weingärtnern tun?}
\verstab{41}{Schlachter}{Sie sprachen zu ihm: Er wird die Übeltäter auf üble Weise umbringen und den Weinberg anderen Weingärtnern verpachten, welche ihm die Früchte zu ihrer Zeit abliefern werden.}
\verstab{42}{Schlachter}{Jesus spricht zu ihnen: Habt ihr noch nie in den Schriften gelesen: »Der Stein, den die Bauleute verworfen haben, der ist zum Eckstein geworden. Vom Herrn ist das geschehen, und es ist wunderbar in unseren Augen«?}
\verstab{43}{Schlachter}{Darum sage ich euch: Das Reich Gottes wird von euch genommen und einem Volk gegeben werden, das dessen Früchte bringt}
\verstab{44}{Schlachter}{Und wer auf diesen Stein fällt, der wird zerschmettert werden; auf wen er aber fällt, den wird er zermalmen.}
\verstab{45}{Schlachter}{Und als die obersten Priester und die Pharisäer seine Gleichnisse hörten, erkannten sie, dass er von ihnen redete.}
\verstab{46}{Schlachter}{Und sie suchten ihn zu ergreifen, fürchteten aber die Volksmenge, weil sie ihn für einen Propheten hielt.}
        %\subsection*{Kapitel 22}
\addcontentsline{toc}{subsection}{Kapitel 22}
\verstab{1}{Schlachter}{Da begann Jesus und redete wieder in Gleichnissen zu ihnen und sprach:}
\verstab{2}{Schlachter}{Das Reich der Himmel gleicht einem König, der für seinen Sohn das Hochzeitsfest veranstaltete.}
\verstab{3}{Schlachter}{Und er sandte seine Knechte aus, um die Geladenen zur Hochzeit zu rufen; aber sie wollten nicht kommen.}
\verstab{4}{Schlachter}{Da sandte er nochmals andere Knechte und sprach: Sagt den Geladenen: Siehe, meine Mahlzeit habe ich bereitet; meine Ochsen und das Mastvieh sind geschlachtet, und alles ist bereit; kommt zur Hochzeit!}
\verstab{5}{Schlachter}{Sie aber achteten nicht darauf, sondern gingen hin, der eine auf seinen Acker, der andere zu seinem Gewerbe;}
\verstab{6}{Schlachter}{die Übrigen aber ergriffen seine Knechte, misshandelten und töteten sie.}
\verstab{7}{Schlachter}{Als der König das hörte, wurde er zornig, sandte seine Heere aus und brachte diese Mörder um und zündete ihre Stadt an.}
\verstab{8}{Schlachter}{Dann sprach er zu seinen Knechten: Die Hochzeit ist zwar bereit, aber die Geladenen waren nicht würdig.}
\verstab{9}{Schlachter}{Darum geht hin an die Kreuzungen der Straßen und ladet zur Hochzeit ein, so viele ihr findet!}
\verstab{10}{Schlachter}{Und jene Knechte gingen hinaus auf die Straßen und brachten alle zusammen, so viele sie fanden, Böse und Gute, und der Hochzeitssaal wurde voll von Gästen.}
\verstab{11}{Schlachter}{Als aber der König hineinging, um sich die Gäste anzusehen, sah er dort einen Menschen, der kein hochzeitliches Gewand anhatte;}
\verstab{12}{Schlachter}{und er sprach zu ihm: Freund, wie bist du hier hereingekommen und hast doch kein hochzeitliches Gewand an? Er aber verstummte.}
\verstab{13}{Schlachter}{Da sprach der König zu den Dienern: Bindet ihm Hände und Füße, führt ihn weg und werft ihn hinaus in die äußerste Finsternis! Da wird das Heulen und Zähneknirschen sein.}
\verstab{14}{Schlachter}{Denn viele sind berufen, aber wenige sind auserwählt!}
\verstab{15}{Schlachter}{Da gingen die Pharisäer und hielten Rat, wie sie ihn in der Rede fangen könnten.}
\verstab{16}{Schlachter}{Und sie sandten ihre Jünger samt den Herodianern zu ihm, die sprachen: Meister, wir wissen, dass du wahrhaftig bist und den Weg Gottes in Wahrheit lehrst und auf niemand Rücksicht nimmst; denn du siehst die Person der Menschen nicht an.}
\verstab{17}{Schlachter}{Darum sage uns, was meinst du: Ist es erlaubt, dem Kaiser die Steuer zu geben, oder nicht?}
\verstab{18}{Schlachter}{Da aber Jesus ihre Bosheit erkannte, sprach er: Ihr Heuchler, was versucht ihr mich?}
\verstab{19}{Schlachter}{Zeigt mir die Steuermünze! Da reichten sie ihm einen Denar.}
\verstab{20}{Schlachter}{Und er spricht zu ihnen: Wessen ist dieses Bild und die Aufschrift?}
\verstab{21}{Schlachter}{Sie antworteten ihm: Des Kaisers. Da spricht er zu ihnen: So gebt dem Kaiser, was des Kaisers ist, und Gott, was Gottes ist!}
\verstab{22}{Schlachter}{Als sie das hörten, verwunderten sie sich, und sie ließen ab von ihm und gingen davon.}
\verstab{23}{Schlachter}{An jenem Tag traten Sadduzäer zu ihm, die sagen, es gebe keine Auferstehung, und sie fragten ihn}
\verstab{24}{Schlachter}{und sprachen: Meister, Mose hat gesagt: Wenn jemand ohne Kinder stirbt, so soll sein Bruder dessen Frau zur Ehe nehmen und seinem Bruder Nachkommen erwecken.}
\verstab{25}{Schlachter}{Nun waren bei uns sieben Brüder. Der erste heiratete und starb; und weil er keine Nachkommen hatte, hinterließ er seine Frau seinem Bruder.}
\verstab{26}{Schlachter}{Gleicherweise auch der andere und der dritte, bis zum siebten.}
\verstab{27}{Schlachter}{Zuletzt, nach allen, starb auch die Frau.}
\verstab{28}{Schlachter}{Wem von den Sieben wird sie nun in der Auferstehung als Frau angehören? Denn alle haben sie zur Frau gehabt.}
\verstab{29}{Schlachter}{Aber Jesus antwortete und sprach zu ihnen: Ihr irrt, weil ihr weder die Schriften noch die Kraft Gottes kennt.}
\verstab{30}{Schlachter}{Denn in der Auferstehung heiraten sie nicht, noch werden sie verheiratet, sondern sie sind wie die Engel Gottes im Himmel.}
\verstab{31}{Schlachter}{Was aber die Auferstehung der Toten betrifft, habt ihr nicht gelesen, was euch von Gott gesagt ist, der spricht:}
\verstab{32}{Schlachter}{»Ich bin der Gott Abrahams und der Gott Isaaks und der Gott Jakobs«? Gott ist aber nicht ein Gott der Toten, sondern der Lebendigen.}
\verstab{33}{Schlachter}{Und als die Menge dies hörte, erstaunte sie über seine Lehre.}
\verstab{34}{Schlachter}{Als nun die Pharisäer hörten, dass er den Sadduzäern den Mund gestopft hatte, versammelten sie sich;}
\verstab{35}{Schlachter}{und einer von ihnen, ein Gesetzesgelehrter, stellte ihm eine Frage, um ihn zu versuchen, und sprach:}
\verstab{36}{Schlachter}{Meister, welches ist das größte Gebot im Gesetz?}
\verstab{37}{Schlachter}{Und Jesus sprach zu ihm: »Du sollst den Herrn, deinen Gott, lieben mit deinem ganzen Herzen und mit deiner ganzen Seele und mit deinem ganzen Denken«.}
\verstab{38}{Schlachter}{Das ist das erste und größte Gebot.}
\verstab{39}{Schlachter}{Und das zweite ist ihm vergleichbar: »Du sollst deinen Nächsten lieben wie dich selbst«.}
\verstab{40}{Schlachter}{An diesen zwei Geboten hängen das ganze Gesetz und die Propheten.}
\verstab{41}{Schlachter}{Als nun die Pharisäer versammelt waren, fragte sie Jesus}
\verstab{42}{Schlachter}{und sprach: Was denkt ihr von dem Christus? Wessen Sohn ist er? Sie sagten zu ihm: Davids.}
\verstab{43}{Schlachter}{Er spricht zu ihnen: Wieso nennt ihn denn David im Geist »Herr«, indem er spricht:}
\verstab{44}{Schlachter}{»Der Herr hat zu meinem Herrn gesagt: Setze dich zu meiner Rechten, bis ich deine Feinde hinlege als Schemel für deine Füße«?}
\verstab{45}{Schlachter}{Wenn also David ihn Herr nennt, wie kann er dann sein Sohn sein?}
\verstab{46}{Schlachter}{Und niemand konnte ihm ein Wort erwidern. Auch getraute sich von jenem Tag an niemand mehr, ihn zu fragen.}
        %\subsection*{Kapitel 23}
\addcontentsline{toc}{subsection}{Kapitel 23}
\verstab{1}{Schlachter}{Da redete Jesus zu der Volksmenge und zu seinen Jüngern}
\verstab{2}{Schlachter}{und sprach: Die Schriftgelehrten und Pharisäer haben sich auf Moses Stuhl gesetzt.}
\verstab{3}{Schlachter}{Alles nun, was sie euch sagen, dass ihr halten sollt, das haltet und tut; aber nach ihren Werken tut nicht, denn sie sagen es wohl, tun es aber nicht.}
\verstab{4}{Schlachter}{Sie binden nämlich schwere und kaum erträgliche Bürden und legen sie den Menschen auf die Schultern; sie aber wollen sie nicht mit einem Finger anrühren.}
\verstab{5}{Schlachter}{Alle ihre Werke tun sie aber, um von den Leuten gesehen zu werden. Sie machen nämlich ihre Gebetsriemen breit und die Säume an ihren Gewändern groß,}
\verstab{6}{Schlachter}{und sie lieben den obersten Platz bei den Mahlzeiten und die ersten Sitze in den Synagogen}
\verstab{7}{Schlachter}{und die Begrüßungen auf den Märkten, und wenn sie von den Leuten »Rabbi, Rabbi« genannt werden.}
\verstab{8}{Schlachter}{Ihr aber sollt euch nicht Rabbi nennen lassen, denn einer ist euer Meister, der Christus; ihr aber seid alle Brüder.}
\verstab{9}{Schlachter}{Nennt auch niemand auf Erden euren Vater; denn einer ist euer Vater, der im Himmel ist.}
\verstab{10}{Schlachter}{Auch sollt ihr euch nicht Meister nennen lassen; denn einer ist euer Meister, der Christus.}
\verstab{11}{Schlachter}{Der Größte aber unter euch soll euer Diener sein.}
\verstab{12}{Schlachter}{Wer sich aber selbst erhöht, der wird erniedrigt werden; und wer sich selbst erniedrigt, der wird erhöht werden.}
\verstab{13}{Schlachter}{Aber wehe euch, ihr Schriftgelehrten und Pharisäer, ihr Heuchler, dass ihr das Reich der Himmel vor den Menschen zuschließt! Ihr selbst geht nicht hinein, und die hinein wollen, die lasst ihr nicht hinein.}
\verstab{14}{Schlachter}{Wehe euch, ihr Schriftgelehrten und Pharisäer, ihr Heuchler, dass ihr die Häuser der Witwen fresst und zum Schein lange betet. Darum werdet ihr ein schwereres Gericht empfangen!}
\verstab{15}{Schlachter}{Wehe euch, ihr Schriftgelehrten und Pharisäer, ihr Heuchler, dass ihr Meer und Land durchzieht, um einen einzigen Proselyten[2] zu machen, und wenn er es geworden ist, macht ihr einen Sohn der Hölle aus ihm, zweimal mehr, als ihr es seid!}
\verstab{16}{Schlachter}{Wehe euch, ihr blinden Führer, die ihr sagt: Wer beim Tempel schwört, das gilt nichts; wer aber beim Gold des Tempels schwört, der ist gebunden.}
\verstab{17}{Schlachter}{Ihr Narren und Blinden, was ist denn größer, das Gold oder der Tempel, der das Gold heiligt?}
\verstab{18}{Schlachter}{Und: Wer beim Brandopferaltar schwört, das gilt nichts; wer aber beim Opfer schwört, das darauf liegt, der ist gebunden.}
\verstab{19}{Schlachter}{Ihr Narren und Blinden! Was ist denn größer, das Opfer oder der Brandopferaltar, der das Opfer heiligt?}
\verstab{20}{Schlachter}{Darum, wer beim Altar schwört, der schwört bei ihm und bei allem, was darauf ist.}
\verstab{21}{Schlachter}{Und wer beim Tempel schwört, der schwört bei ihm und bei dem, der darin wohnt.}
\verstab{22}{Schlachter}{Und wer beim Himmel schwört, der schwört bei dem Thron Gottes und bei dem, der darauf sitzt.}
\verstab{23}{Schlachter}{Wehe euch, ihr Schriftgelehrten und Pharisäer, ihr Heuchler, dass ihr die Minze und den Anis und den Kümmel verzehntet und das Wichtigere im Gesetz vernachlässigt, nämlich das Recht und das Erbarmen und den Glauben! Dieses sollte man tun und jenes nicht lassen.}
\verstab{24}{Schlachter}{Ihr blinden Führer, die ihr die Mücke aussiebt, das Kamel aber verschluckt!}
\verstab{25}{Schlachter}{Wehe euch, ihr Schriftgelehrten und Pharisäer, ihr Heuchler, dass ihr das Äußere des Bechers und der Schüssel reinigt, inwendig aber sind sie voller Raub und Unmäßigkeit!}
\verstab{26}{Schlachter}{Du blinder Pharisäer, reinige zuerst das Inwendige des Bechers und der Schüssel, damit auch ihr Äußeres rein werde!}
\verstab{27}{Schlachter}{Wehe euch, ihr Schriftgelehrten und Pharisäer, ihr Heuchler, dass ihr getünchten Gräbern gleicht, die äußerlich zwar schön scheinen, inwendig aber voller Totengebeine und aller Unreinheit sind!}
\verstab{28}{Schlachter}{So erscheint auch ihr äußerlich vor den Menschen als gerecht, inwendig aber seid ihr voller Heuchelei und Gesetzlosigkeit.}
\verstab{29}{Schlachter}{Wehe euch, ihr Schriftgelehrten und Pharisäer, ihr Heuchler, dass ihr die Gräber der Propheten baut und die Denkmäler der Gerechten schmückt}
\verstab{30}{Schlachter}{und sagt: Hätten wir in den Tagen unserer Väter gelebt, wir hätten uns nicht mit ihnen des Blutes der Propheten schuldig gemacht.}
\verstab{31}{Schlachter}{So gebt ihr ja euch selbst das Zeugnis, dass ihr Söhne der Prophetenmörder seid.}
\verstab{32}{Schlachter}{Ja, macht ihr nur das Maß eurer Väter voll!}
\verstab{33}{Schlachter}{Ihr Schlangen! Ihr Otterngezücht! Wie wollt ihr dem Gericht der Hölle entgehen?}
\verstab{34}{Schlachter}{Siehe, darum sende ich zu euch Propheten und Weise und Schriftgelehrte; und etliche von ihnen werdet ihr töten und kreuzigen, und etliche werdet ihr in euren Synagogen geißeln und sie verfolgen von einer Stadt zur anderen,}
\verstab{35}{Schlachter}{damit über euch alles gerechte Blut kommt, das auf Erden vergossen worden ist, vom Blut Abels, des Gerechten, bis zum Blut des Zacharias, des Sohnes Barachias, den ihr zwischen dem Tempel und dem Altar getötet habt.}
\verstab{36}{Schlachter}{Wahrlich, ich sage euch: Dies alles wird über dieses Geschlecht kommen!}
\verstab{37}{Schlachter}{Jerusalem, Jerusalem, die du die Propheten tötest und steinigst, die zu dir gesandt sind! Wie oft habe ich deine Kinder sammeln wollen, wie eine Henne ihre Küken unter die Flügel sammelt, aber ihr habt nicht gewollt!}
\verstab{38}{Schlachter}{Siehe, euer Haus wird euch verwüstet gelassen werden;}
\verstab{39}{Schlachter}{denn ich sage euch: Ihr werdet mich von jetzt an nicht mehr sehen, bis ihr sprechen werdet: »Gepriesen sei der, welcher kommt im Namen des Herrn!«}
        %\subsection*{Kapitel 24}
\addcontentsline{toc}{subsection}{Kapitel 24}
\verstab{1}{Schlachter}{Und Jesus trat hinaus und ging vom Tempel hinweg. Und seine Jünger kamen herzu, um ihm die Gebäude des Tempels zu zeigen.}
\verstab{2}{Schlachter}{Jesus aber sprach zu ihnen: Seht ihr nicht dies alles? Wahrlich, ich sage euch: Hier wird kein Stein auf dem anderen bleiben, der nicht abgebrochen wird!}
\verstab{3}{Schlachter}{Als er aber auf dem Ölberg saß, traten die Jünger allein zu ihm und sprachen: Sage uns, wann wird dies geschehen, und was wird das Zeichen deiner Wiederkunft und des Endes der Weltzeit sein?}
\verstab{4}{Schlachter}{Und Jesus antwortete und sprach zu ihnen: Habt acht, dass euch niemand verführt!}
\verstab{5}{Schlachter}{Denn viele werden unter meinem Namen kommen und sagen: Ich bin der Christus! Und sie werden viele verführen.}
\verstab{6}{Schlachter}{Ihr werdet aber von Kriegen und Kriegsgerüchten hören; habt acht, erschreckt nicht; denn dies alles muss geschehen; aber es ist noch nicht das Ende.}
\verstab{7}{Schlachter}{Denn ein Heidenvolk wird sich gegen das andere erheben und ein Königreich gegen das andere; und es werden hier und dort Hungersnöte, Seuchen und Erdbeben geschehen.}
\verstab{8}{Schlachter}{Dies alles ist der Anfang der Wehen.}
\verstab{9}{Schlachter}{Dann wird man euch der Drangsal preisgeben und euch töten; und ihr werdet gehasst sein von allen Heidenvölkern um meines Namens willen.}
\verstab{10}{Schlachter}{Und dann werden viele Anstoß nehmen, einander verraten und einander hassen.}
\verstab{11}{Schlachter}{Und es werden viele falsche Propheten auftreten und werden viele verführen.}
\verstab{12}{Schlachter}{Und weil die Gesetzlosigkeit überhandnimmt, wird die Liebe in vielen erkalten.}
\verstab{13}{Schlachter}{Wer aber ausharrt bis ans Ende, der wird gerettet werden.}
\verstab{14}{Schlachter}{Und dieses Evangelium vom Reich wird in der ganzen Welt verkündigt werden, zum Zeugnis für alle Heidenvölker, und dann wird das Ende kommen.}
\verstab{15}{Schlachter}{Wenn ihr nun den Gräuel der Verwüstung, von dem durch den Propheten Daniel geredet wurde, an heiliger Stätte stehen seht (wer es liest, der achte darauf!),}
\verstab{16}{Schlachter}{dann fliehe auf die Berge, wer in Judäa ist;}
\verstab{17}{Schlachter}{wer auf dem Dach ist, der steige nicht hinab, um etwas aus seinem Haus zu holen,}
\verstab{18}{Schlachter}{und wer auf dem Feld ist, der kehre nicht zurück, um seine Kleider zu holen.}
\verstab{19}{Schlachter}{Wehe aber den Schwangeren und den Stillenden in jenen Tagen!}
\verstab{20}{Schlachter}{Bittet aber, dass eure Flucht nicht im Winter noch am Sabbat geschieht.}
\verstab{21}{Schlachter}{Denn dann wird eine große Drangsal sein, wie von Anfang der Welt an bis jetzt keine gewesen ist und auch keine mehr kommen wird.}
\verstab{22}{Schlachter}{Und wenn jene Tage nicht verkürzt würden, so würde kein Fleisch gerettet werden; aber um der Auserwählten willen sollen jene Tage verkürzt werden.}
\verstab{23}{Schlachter}{Wenn dann jemand zu euch sagen wird: Siehe, hier ist der Christus, oder dort, so glaubt es nicht!}
\verstab{24}{Schlachter}{Denn es werden falsche Christusse und falsche Propheten auftreten und werden große Zeichen und Wunder tun, um, wenn möglich, auch die Auserwählten zu verführen.}
\verstab{25}{Schlachter}{Siehe, ich habe es euch vorhergesagt.}
\verstab{26}{Schlachter}{Wenn sie nun zu euch sagen werden: »Siehe, er ist in der Wüste!«, so geht nicht hinaus; »Siehe, er ist in den Kammern!«, so glaubt es nicht!}
\verstab{27}{Schlachter}{Denn wie der Blitz vom Osten ausfährt und bis zum Westen scheint, so wird auch die Wiederkunft des Menschensohnes sein.}
\verstab{28}{Schlachter}{Denn wo das Aas ist, da sammeln sich die Geier.}
\verstab{29}{Schlachter}{Bald aber nach der Drangsal jener Tage wird die Sonne verfinstert werden, und der Mond wird seinen Schein nicht geben, und die Sterne werden vom Himmel fallen und die Kräfte des Himmels erschüttert werden.}
\verstab{30}{Schlachter}{Und dann wird das Zeichen des Menschensohnes am Himmel erscheinen, und dann werden sich alle Geschlechter der Erde an die Brust schlagen, und sie werden den Sohn des Menschen kommen sehen auf den Wolken des Himmels mit großer Kraft und Herrlichkeit.}
\verstab{31}{Schlachter}{Und er wird seine Engel aussenden mit starkem Posaunenschall, und sie werden seine Auserwählten versammeln von den vier Windrichtungen her, von einem Ende des Himmels bis zum anderen.}
\verstab{32}{Schlachter}{Von dem Feigenbaum aber lernt das Gleichnis: Wenn sein Zweig schon saftig wird und Blätter treibt, so erkennt ihr, dass der Sommer nahe ist.}
\verstab{33}{Schlachter}{Also auch ihr, wenn ihr dies alles seht, so erkennt, dass er nahe vor der Türe ist.}
\verstab{34}{Schlachter}{Wahrlich, ich sage euch: Dieses Geschlecht wird nicht vergehen, bis dies alles geschehen ist.}
\verstab{35}{Schlachter}{Himmel und Erde werden vergehen, aber meine Worte werden nicht vergehen.}
\verstab{36}{Schlachter}{Um jenen Tag aber und die Stunde weiß niemand, auch die Engel im Himmel nicht, sondern allein mein Vater.}
\verstab{37}{Schlachter}{Wie es aber in den Tagen Noahs war, so wird es auch bei der Wiederkunft des Menschensohnes sein.}
\verstab{38}{Schlachter}{Denn wie sie in den Tagen vor der Sintflut aßen und tranken, heirateten und verheirateten bis zu dem Tag, als Noah in die Arche ging,}
\verstab{39}{Schlachter}{und nichts merkten, bis die Sintflut kam und sie alle dahinraffte, so wird auch die Wiederkunft des Menschensohnes sein.}
\verstab{40}{Schlachter}{Dann werden zwei auf dem Feld sein; der eine wird genommen, und der andere wird zurückgelassen.}
\verstab{41}{Schlachter}{Zwei werden auf der Mühle mahlen; die eine wird genommen, und die andere wird zurückgelassen.}
\verstab{42}{Schlachter}{So wacht nun, da ihr nicht wisst, in welcher Stunde euer Herr kommt!}
\verstab{43}{Schlachter}{Das aber erkennt: Wenn der Hausherr wüsste, in welcher Nachtstunde der Dieb käme, so würde er wohl wachen und nicht in sein Haus einbrechen lassen.}
\verstab{44}{Schlachter}{Darum seid auch ihr bereit! Denn der Sohn des Menschen kommt zu einer Stunde, da ihr es nicht meint.}
\verstab{45}{Schlachter}{Wer ist nun der treue und kluge Knecht, den sein Herr über seine Dienerschaft gesetzt hat, damit er ihnen die Speise gibt zur rechten Zeit?}
\verstab{46}{Schlachter}{Glückselig ist jener Knecht, den sein Herr, wenn er kommt, bei solchem Tun finden wird.}
\verstab{47}{Schlachter}{Wahrlich, ich sage euch: Er wird ihn über alle seine Güter setzen.}
\verstab{48}{Schlachter}{Wenn aber jener böse Knecht in seinem Herzen spricht: Mein Herr säumt zu kommen!,}
\verstab{49}{Schlachter}{und anfängt, die Mitknechte zu schlagen und mit den Schlemmern zu essen und zu trinken,}
\verstab{50}{Schlachter}{so wird der Herr jenes Knechtes an einem Tag kommen, da er es nicht erwartet, und zu einer Stunde, die er nicht kennt,}
\verstab{51}{Schlachter}{und wird ihn entzweihauen und ihm seinen Teil mit den Heuchlern geben. Da wird das Heulen und Zähneknirschen sein.}
        %\subsection*{Kapitel 25}
\addcontentsline{toc}{subsection}{Kapitel 25}
\verstab{1}{Schlachter}{Dann wird das Reich der Himmel zehn Jungfrauen gleichen, die ihre Lampen nahmen und dem Bräutigam entgegengingen.}
\verstab{2}{Schlachter}{Fünf von ihnen aber waren klug und fünf töricht.}
\verstab{3}{Schlachter}{ie törichten nahmen zwar ihre Lampen, aber sie nahmen kein Öl mit sich.}
\verstab{4}{Schlachter}{Die klugen aber nahmen Öl in ihren Gefäßen mitsamt ihren Lampen.}
\verstab{5}{Schlachter}{Als nun der Bräutigam auf sich warten ließ, wurden sie alle schläfrig und schliefen ein.}
\verstab{6}{Schlachter}{Um Mitternacht aber entstand ein Geschrei: Siehe, der Bräutigam kommt! Geht aus, ihm entgegen!}
\verstab{7}{Schlachter}{Da erwachten alle jene Jungfrauen und machten ihre Lampen bereit.}
\verstab{8}{Schlachter}{Die törichten aber sprachen zu den klugen: Gebt uns von eurem Öl, denn unsere Lampen erlöschen!}
\verstab{9}{Schlachter}{Aber die klugen antworteten und sprachen: Nein, es würde nicht reichen für uns und für euch. Geht doch vielmehr hin zu den Händlern und kauft für euch selbst!}
\verstab{10}{Schlachter}{Während sie aber hingingen, um zu kaufen, kam der Bräutigam; und die bereit waren, gingen mit ihm hinein zur Hochzeit; und die Tür wurde verschlossen.}
\verstab{11}{Schlachter}{Danach kommen auch die übrigen Jungfrauen und sagen: Herr, Herr, tue uns auf!}
\verstab{12}{Schlachter}{Er aber antwortete und sprach: Wahrlich, ich sage euch: Ich kenne euch nicht!}
\verstab{13}{Schlachter}{Darum wacht! Denn ihr wisst weder den Tag noch die Stunde, in welcher der Sohn des Menschen kommen wird.}
\verstab{14}{Schlachter}{Denn es ist wie bei einem Menschen, der außer Landes reisen wollte, seine Knechte rief und ihnen seine Güter übergab.}
\verstab{15}{Schlachter}{Dem einen gab er fünf Talente, dem anderen zwei, dem dritten eins, jedem nach seiner Kraft, und er reiste sogleich ab.}
\verstab{16}{Schlachter}{Da ging der hin, welcher die fünf Talente empfangen hatte, handelte mit ihnen und gewann fünf weitere Talente.}
\verstab{17}{Schlachter}{nd ebenso der, welcher die zwei Talente [empfangen hatte], auch er gewann zwei weitere.}
\verstab{18}{Schlachter}{Aber der, welcher das eine empfangen hatte, ging hin, grub die Erde auf und verbarg das Geld seines Herrn.}
\verstab{19}{Schlachter}{Nach langer Zeit aber kommt der Herr dieser Knechte und hält Abrechnung mit ihnen.}
\verstab{20}{Schlachter}{Und es trat der hinzu, der die fünf Talente empfangen hatte, brachte noch fünf weitere Talente herzu und sprach: Herr, du hast mir fünf Talente übergeben; siehe, ich habe mit ihnen fünf weitere Talente gewonnen.}
\verstab{21}{Schlachter}{Da sagte sein Herr zu ihm: Recht so, du guter und treuer Knecht! Du bist über wenigem treu gewesen, ich will dich über vieles setzen; geh ein zur Freude deines Herrn!}
\verstab{22}{Schlachter}{Und es trat auch der hinzu, der die zwei Talente empfangen hatte, und sprach: Herr, du hast mir zwei Talente übergeben; siehe, ich habe mit ihnen zwei andere Talente gewonnen.}
\verstab{23}{Schlachter}{Sein Herr sagte zu ihm: Recht so, du guter und treuer Knecht! Du bist über wenigem treu gewesen, ich will dich über vieles setzen; geh ein zur Freude deines Herrn!}
\verstab{24}{Schlachter}{Da trat auch der hinzu, der das eine Talent empfangen hatte, und sprach: Herr, ich kannte dich, dass du ein harter Mann bist. Du erntest, wo du nicht gesät, und sammelst, wo du nicht ausgestreut hast;}
\verstab{25}{Schlachter}{und ich fürchtete mich, ging hin und verbarg dein Talent in der Erde. Siehe, da hast du das Deine!}
\verstab{26}{Schlachter}{Aber sein Herr antwortete und sprach zu ihm: Du böser und fauler Knecht! Wusstest du, dass ich ernte, wo ich nicht gesät, und sammle, wo ich nicht ausgestreut habe?}
\verstab{27}{Schlachter}{Dann hättest du mein Geld den Wechslern bringen sollen, so hätte ich bei meinem Kommen das Meine mit Zinsen zurückerhalten.}
\verstab{28}{Schlachter}{Darum nehmt ihm das Talent weg und gebt es dem, der die zehn Talente hat!}
\verstab{29}{Schlachter}{Denn wer hat, dem wird gegeben werden, damit er Überfluss hat; von dem aber, der nicht hat, wird auch das genommen werden, was er hat.}
\verstab{30}{Schlachter}{Und den unnützen Knecht werft hinaus in die äußerste Finsternis! Dort wird das Heulen und Zähneknirschen sein.}
\verstab{31}{Schlachter}{Wenn aber der Sohn des Menschen in seiner Herrlichkeit kommen wird und alle heiligen Engel mit ihm, dann wird er auf dem Thron seiner Herrlichkeit sitzen,}
\verstab{32}{Schlachter}{und vor ihm werden alle Heidenvölker versammelt werden. Und er wird sie voneinander scheiden, wie ein Hirte die Schafe von den Böcken scheidet,}
\verstab{33}{Schlachter}{und er wird die Schafe zu seiner Rechten stellen, die Böcke aber zu seiner Linken.}
\verstab{34}{Schlachter}{Dann wird der König denen zu seiner Rechten sagen: Kommt her, ihr Gesegneten meines Vaters, und erbt das Reich, das euch bereitet ist seit Grundlegung der Welt!}
\verstab{35}{Schlachter}{Denn ich bin hungrig gewesen, und ihr habt mich gespeist; ich bin durstig gewesen, und ihr habt mir zu trinken gegeben; ich bin ein Fremdling gewesen, und ihr habt mich beherbergt;}
\verstab{36}{Schlachter}{ich bin ohne Kleidung gewesen, und ihr habt mich bekleidet; ich bin krank gewesen, und ihr habt mich besucht; ich bin gefangen gewesen, und ihr seid zu mir gekommen.}
\verstab{37}{Schlachter}{Dann werden ihm die Gerechten antworten und sagen: Herr, wann haben wir dich hungrig gesehen und haben dich gespeist, oder durstig, und haben dir zu trinken gegeben?}
\verstab{38}{Schlachter}{Wann haben wir dich als Fremdling gesehen und haben dich beherbergt, oder ohne Kleidung, und haben dich bekleidet?}
\verstab{39}{Schlachter}{Wann haben wir dich krank gesehen oder im Gefängnis, und sind zu dir gekommen?}
\verstab{40}{Schlachter}{Und der König wird ihnen antworten und sagen: Wahrlich, ich sage euch: Was ihr einem dieser meiner geringsten Brüder getan habt, das habt ihr mir getan!}
\verstab{41}{Schlachter}{Dann wird er auch denen zur Linken sagen: Geht hinweg von mir, ihr Verfluchten, in das ewige Feuer, das dem Teufel und seinen Engeln bereitet ist!}
\verstab{42}{Schlachter}{Denn ich bin hungrig gewesen, und ihr habt mich nicht gespeist; ich bin durstig gewesen, und ihr habt mir nicht zu trinken gegeben;}
\verstab{43}{Schlachter}{ich bin ein Fremdling gewesen, und ihr habt mich nicht beherbergt; ohne Kleidung, und ihr habt mich nicht bekleidet; krank und gefangen, und ihr habt mich nicht besucht!}
\verstab{44}{Schlachter}{Dann werden auch sie ihm antworten und sagen: Herr, wann haben wir dich hungrig oder durstig oder als Fremdling oder ohne Kleidung oder krank oder gefangen gesehen und haben dir nicht gedient?}
\verstab{45}{Schlachter}{Dann wird er ihnen antworten: Wahrlich, ich sage euch: Was ihr einem dieser Geringsten nicht getan habt, das habt ihr mir auch nicht getan!}
\verstab{46}{Schlachter}{Und sie werden in die ewige Strafe hingehen, die Gerechten aber in das ewige Leben.}
        %\subsection*{Kapitel 26}
\addcontentsline{toc}{subsection}{Kapitel 26}
\verstab{1}{Schlachter}{Und es geschah, als Jesus alle diese Worte beendet hatte, sprach er zu seinen Jüngern:}
\verstab{2}{Schlachter}{Ihr wisst, dass in zwei Tagen das Passah ist; dann wird der Sohn des Menschen ausgeliefert, damit er gekreuzigt werde.}
\verstab{3}{Schlachter}{Da versammelten sich die obersten Priester und die Schriftgelehrten und die Ältesten des Volkes im Hof des Hohenpriesters, der Kajaphas hieß.}
\verstab{4}{Schlachter}{Und sie hielten miteinander Rat, wie sie Jesus mit List ergreifen und töten könnten.}
\verstab{5}{Schlachter}{Sie sprachen aber: Nicht während des Festes, damit kein Aufruhr unter dem Volk entsteht!}
\verstab{6}{Schlachter}{Als nun Jesus in Bethanien im Haus Simons des Aussätzigen war,}
\verstab{7}{Schlachter}{da trat eine Frau zu ihm mit einer alabasternen Flasche voll kostbaren Salböls und goss es auf sein Haupt, während er zu Tisch saß.}
\verstab{8}{Schlachter}{Als das seine Jünger sahen, wurden sie unwillig und sprachen: Wozu diese Verschwendung?}
\verstab{9}{Schlachter}{Man hätte dieses Salböl doch teuer verkaufen und den Armen geben können!}
\verstab{10}{Schlachter}{Als es aber Jesus bemerkte, sprach er zu ihnen: Warum bekümmert ihr diese Frau? Sie hat doch ein gutes Werk an mir getan!}
\verstab{11}{Schlachter}{Denn die Armen habt ihr allezeit bei euch, mich aber habt ihr nicht allezeit.}
\verstab{12}{Schlachter}{Damit, dass sie dieses Salböl auf meinen Leib goss, hat sie mich zum Begräbnis bereitet.}
\verstab{13}{Schlachter}{Wahrlich, ich sage euch: Wo immer dieses Evangelium verkündigt wird in der ganzen Welt, da wird man auch von dem sprechen, was diese getan hat, zu ihrem Gedenken!}
\verstab{14}{Schlachter}{Da ging einer der Zwölf namens Judas Ischariot hin zu den obersten Priestern}
\verstab{15}{Schlachter}{und sprach: Was wollt ihr mir geben, wenn ich ihn euch verrate? Und sie setzten ihm 30 Silberlinge fest.}
\verstab{16}{Schlachter}{Und von da an suchte er eine gute Gelegenheit, ihn zu verraten.}
\verstab{17}{Schlachter}{Am ersten Tag der ungesäuerten Brote traten die Jünger nun zu Jesus und sprachen zu ihm: Wo willst du, dass wir dir das Passahmahl zu essen bereiten?}
\verstab{18}{Schlachter}{Und er sprach: Geht hin in die Stadt zu dem und dem und sprecht zu ihm: Der Meister lässt dir sagen: Meine Zeit ist nahe; bei dir will ich mit meinen Jüngern das Passah halten!}
\verstab{19}{Schlachter}{Und die Jünger machten es, wie Jesus ihnen befohlen hatte, und bereiteten das Passah.}
\verstab{20}{Schlachter}{Als es nun Abend geworden war, setzte er sich mit den Zwölfen zu Tisch.}
\verstab{21}{Schlachter}{Und während sie aßen, sprach er: Wahrlich, ich sage euch: Einer von euch wird mich verraten!}
\verstab{22}{Schlachter}{Da wurden sie sehr betrübt, und jeder von ihnen fing an, ihn zu fragen: Herr, doch nicht ich?}
\verstab{23}{Schlachter}{Er antwortete aber und sprach: Der mit mir die Hand in die Schüssel taucht, der wird mich verraten.}
\verstab{24}{Schlachter}{Der Sohn des Menschen geht zwar dahin, wie von ihm geschrieben steht; aber wehe jenem Menschen, durch den der Sohn des Menschen verraten wird! Es wäre für jenen Menschen besser, wenn er nicht geboren wäre.}
\verstab{25}{Schlachter}{Da antwortete Judas, der ihn verriet, und sprach: Rabbi, doch nicht ich? Er spricht zu ihm: Du hast es gesagt!}
\verstab{26}{Schlachter}{Als sie nun aßen, nahm Jesus das Brot und sprach den Segen, brach es, gab es den Jüngern und sprach: Nehmt, esst! Das ist mein Leib.}
\verstab{27}{Schlachter}{Und er nahm den Kelch und dankte, gab ihnen denselben und sprach: Trinkt alle daraus!}
\verstab{28}{Schlachter}{Denn das ist mein Blut, das des neuen Bundes, das für viele vergossen wird zur Vergebung der Sünden.}
\verstab{29}{Schlachter}{Ich sage euch aber: Ich werde von jetzt an von diesem Gewächs des Weinstocks nicht mehr trinken bis zu jenem Tag, da ich es neu mit euch trinken werde im Reich meines Vaters!}
\verstab{30}{Schlachter}{Und nachdem sie den Lobgesang gesungen hatten, gingen sie hinaus an den Ölberg.}
\verstab{31}{Schlachter}{Da spricht Jesus zu ihnen: Ihr werdet in dieser Nacht alle an mir Anstoß nehmen; denn es steht geschrieben: »Ich werde den Hirten schlagen, und die Schafe der Herde werden sich zerstreuen«.}
\verstab{32}{Schlachter}{Aber nachdem ich auferweckt worden bin, will ich euch nach Galiläa vorangehen.}
\verstab{33}{Schlachter}{Da antwortete Petrus und sprach zu ihm: Wenn auch alle an dir Anstoß nehmen, so werde doch ich niemals Anstoß nehmen!}
\verstab{34}{Schlachter}{Jesus spricht zu ihm: Wahrlich, ich sage dir: In dieser Nacht, ehe der Hahn kräht, wirst du mich dreimal verleugnen!}
\verstab{35}{Schlachter}{Petrus spricht zu ihm: Und wenn ich auch mit dir sterben müsste, werde ich dich nicht verleugnen! Ebenso sprachen auch alle Jünger.}
\verstab{36}{Schlachter}{Da kommt Jesus mit ihnen zu einem Grundstück, das Gethsemane genannt wird. Und er spricht zu den Jüngern: Setzt euch hier hin, während ich weggehe und dort bete!}
\verstab{37}{Schlachter}{Und er nahm Petrus und die zwei Söhne des Zebedäus mit sich; und er fing an, betrübt zu werden, und ihm graute sehr.}
\verstab{38}{Schlachter}{Da spricht er zu ihnen: Meine Seele ist tief betrübt bis zum Tod. Bleibt hier und wacht mit mir!}
\verstab{39}{Schlachter}{Und er ging ein wenig weiter, warf sich auf sein Angesicht, betete und sprach: Mein Vater! Ist es möglich, so gehe dieser Kelch an mir vorüber; doch nicht wie ich will, sondern wie du willst!}
\verstab{40}{Schlachter}{Und er kommt zu den Jüngern und findet sie schlafend und spricht zu Petrus: Könnt ihr also nicht eine Stunde mit mir wachen?}
\verstab{41}{Schlachter}{Wacht und betet, damit ihr nicht in Versuchung kommt! Der Geist ist willig, aber das Fleisch ist schwach.}
\verstab{42}{Schlachter}{Wiederum ging er zum zweiten Mal hin, betete und sprach: Mein Vater, wenn dieser Kelch nicht an mir vorübergehen kann, ohne dass ich ihn trinke, so geschehe dein Wille!}
\verstab{43}{Schlachter}{Und er kommt und findet sie wieder schlafend; denn die Augen waren ihnen schwer geworden.}
\verstab{44}{Schlachter}{Und er ließ sie, ging wieder hin, betete zum dritten Mal und sprach dieselben Worte.}
\verstab{45}{Schlachter}{Dann kommt er zu seinen Jüngern und spricht zu ihnen: Schlaft ihr noch immer und ruht? Siehe, die Stunde ist nahe, und der Sohn des Menschen wird in die Hände der Sünder ausgeliefert.}
\verstab{46}{Schlachter}{Steht auf, lasst uns gehen! Siehe, der mich verrät, ist nahe.}
\verstab{47}{Schlachter}{Und während er noch redete, siehe, da kam Judas, einer der Zwölf, und mit ihm eine große Schar mit Schwertern und Stöcken, [gesandt] von den obersten Priestern und Ältesten des Volkes.}
\verstab{48}{Schlachter}{Der ihn aber verriet, hatte ihnen ein Zeichen gegeben und gesagt: Der, den ich küssen werde, der ist’s, den ergreift!}
\verstab{49}{Schlachter}{Und sogleich trat er zu Jesus und sprach: Sei gegrüßt, Rabbi!, und küsste ihn.}
\verstab{50}{Schlachter}{Jesus aber sprach zu ihm: Freund, wozu bist du hier? Da traten sie hinzu, legten Hand an Jesus und nahmen ihn fest.}
\verstab{51}{Schlachter}{Und siehe, einer von denen, die bei Jesus waren, streckte die Hand aus, zog sein Schwert, schlug den Knecht des Hohenpriesters und hieb ihm ein Ohr ab.}
\verstab{52}{Schlachter}{Da sprach Jesus zu ihm: Stecke dein Schwert an seinen Platz! Denn alle, die zum Schwert greifen, werden durch das Schwert umkommen!}
\verstab{53}{Schlachter}{Oder meinst du, ich könnte nicht jetzt meinen Vater bitten, und er würde mir mehr als zwölf Legionen Engel schicken?}
\verstab{54}{Schlachter}{Wie würden dann aber die Schriften erfüllt, dass es so kommen muss?}
\verstab{55}{Schlachter}{In jener Stunde sprach Jesus zu der Volksmenge: Wie gegen einen Räuber seid ihr ausgezogen mit Schwertern und Stöcken, um mich zu fangen! Täglich bin ich bei euch im Tempel gesessen und habe gelehrt, und ihr habt mich nicht ergriffen.}
\verstab{56}{Schlachter}{Das alles aber ist geschehen, damit die Schriften der Propheten erfüllt würden. — Da verließen ihn alle Jünger und flohen.}
\verstab{57}{Schlachter}{Die aber Jesus festgenommen hatten, führten ihn ab zu dem Hohenpriester Kajaphas, wo die Schriftgelehrten und die Ältesten versammelt waren.}
\verstab{58}{Schlachter}{Petrus aber folgte ihnen von ferne bis zum Hof des Hohenpriesters. Und er ging hinein und setzte sich zu den Dienern, um den Ausgang [der Sache] zu sehen.}
\verstab{59}{Schlachter}{Aber die obersten Priester und die Ältesten und der ganze Hohe Rat suchten ein falsches Zeugnis gegen Jesus, um ihn zu töten.}
\verstab{60}{Schlachter}{Aber sie fanden keines; und obgleich viele falsche Zeugen herzukamen, fanden sie doch keines.}
\verstab{61}{Schlachter}{Zuletzt aber kamen zwei falsche Zeugen und sprachen: Dieser hat gesagt: Ich kann den Tempel Gottes zerstören und ihn in drei Tagen aufbauen!}
\verstab{62}{Schlachter}{Und der Hohepriester stand auf und sprach zu ihm: Antwortest du nichts auf das, was diese gegen dich aussagen?}
\verstab{63}{Schlachter}{Jesus aber schwieg. Und der Hohepriester begann und sprach zu ihm: Ich beschwöre dich bei dem lebendigen Gott, dass du uns sagst, ob du der Christus bist, der Sohn Gottes!}
\verstab{64}{Schlachter}{Jesus spricht zu ihm: Du hast es gesagt! Überdies sage ich euch: Künftig werdet ihr den Sohn des Menschen sitzen sehen zur Rechten der Macht und kommen auf den Wolken des Himmels!}
\verstab{65}{Schlachter}{Da zerriss der Hohepriester seine Kleider und sprach: Er hat gelästert! Was brauchen wir weitere Zeugen? Siehe, nun habt ihr seine Lästerung gehört.}
\verstab{66}{Schlachter}{Was meint ihr? Sie antworteten und sprachen: Er ist des Todes schuldig!}
\verstab{67}{Schlachter}{Da spuckten sie ihm ins Angesicht und schlugen ihn mit Fäusten; andere gaben ihm Backenstreiche}
\verstab{68}{Schlachter}{und sprachen: Christus, weissage uns! Wer ist’s, der dich geschlagen hat?}
\verstab{69}{Schlachter}{Petrus aber saß draußen im Hof. Und eine Magd trat zu ihm und sprach: Auch du warst mit Jesus, dem Galiläer!}
\verstab{70}{Schlachter}{Er aber leugnete vor allen und sprach: Ich weiß nicht, was du sagst!}
\verstab{71}{Schlachter}{Als er dann in den Vorhof hinausging, sah ihn eine andere und sprach zu denen, die dort waren: Auch dieser war mit Jesus, dem Nazarener!}
\verstab{72}{Schlachter}{Und er leugnete nochmals mit einem Schwur: Ich kenne den Menschen nicht!}
\verstab{73}{Schlachter}{Bald darauf aber traten die Umstehenden herzu und sagten zu Petrus: Wahrhaftig, du bist auch einer von ihnen; denn auch deine Sprache verrät dich.}
\verstab{74}{Schlachter}{Da fing er an, [sich] zu verfluchen[3] und zu schwören: Ich kenne den Menschen nicht! Und sogleich krähte der Hahn.}
\verstab{75}{Schlachter}{Und Petrus erinnerte sich an das Wort Jesu, der zu ihm gesagt hatte: Ehe der Hahn kräht, wirst du mich dreimal verleugnen. Und er ging hinaus und weinte bitterlich.}
        %\subsection*{Kapitel 27}
\addcontentsline{toc}{subsection}{Kapitel 27}
\verstab{1}{Schlachter}{Als es aber Morgen geworden war, hielten alle obersten Priester und die Ältesten des Volkes einen Rat gegen Jesus, um ihn zu töten.}
\verstab{2}{Schlachter}{Und sie banden ihn, führten ihn ab und lieferten ihn dem Statthalter Pontius Pilatus aus.}
\verstab{3}{Schlachter}{Als nun Judas, der ihn verraten hatte, sah, dass er verurteilt war, reute es ihn; und er brachte die 30 Silberlinge den obersten Priestern und den Ältesten zurück}
\verstab{4}{Schlachter}{und sprach: Ich habe gesündigt, dass ich unschuldiges Blut verraten habe! Sie aber sprachen: Was geht das uns an? Da sieh du zu!}
\verstab{5}{Schlachter}{Da warf er die Silberlinge im Tempel hin und machte sich davon, ging hin und erhängte sich.}
\verstab{6}{Schlachter}{Die obersten Priester aber nahmen die Silberlinge und sprachen: Wir dürfen sie nicht in den Opferkasten legen, weil es Blutgeld ist!}
\verstab{7}{Schlachter}{Nachdem sie aber Rat gehalten hatten, kauften sie dafür den Acker des Töpfers als Begräbnisstätte für die Fremdlinge.}
\verstab{8}{Schlachter}{Daher wird jener Acker »Blutacker« genannt bis zum heutigen Tag.}
\verstab{9}{Schlachter}{Da wurde erfüllt, was durch den Propheten Jeremia gesagt ist, der spricht: »Und sie nahmen die 30 Silberlinge, den Wert dessen, der geschätzt wurde, den die Kinder Israels geschätzt hatten,}
\verstab{10}{Schlachter}{und gaben sie für den Acker des Töpfers, wie der Herr mir befohlen hatte.«}
\verstab{11}{Schlachter}{Jesus aber stand vor dem Statthalter; und der Statthalter fragte ihn und sprach: Bist du der König der Juden? Jesus sprach zu ihm: Du sagst es!}
\verstab{12}{Schlachter}{Und als er von den obersten Priestern und den Ältesten verklagt wurde, antwortete er nichts.}
\verstab{13}{Schlachter}{Da sprach Pilatus zu ihm: Hörst du nicht, was sie alles gegen dich aussagen?}
\verstab{14}{Schlachter}{Und er antwortete ihm auch nicht auf ein einziges Wort, sodass der Statthalter sich sehr verwunderte.}
\verstab{15}{Schlachter}{Aber anlässlich des Festes pflegte der Statthalter der Volksmenge einen Gefangenen freizugeben, welchen sie wollten.}
\verstab{16}{Schlachter}{Sie hatten aber damals einen berüchtigten Gefangenen namens Barabbas.}
\verstab{17}{Schlachter}{Als sie nun versammelt waren, sprach Pilatus zu ihnen: Welchen wollt ihr, dass ich euch freilasse, Barabbas oder Jesus, den man Christus nennt?}
\verstab{18}{Schlachter}{Denn er wusste, dass sie ihn aus Neid ausgeliefert hatten.}
\verstab{19}{Schlachter}{Als er aber auf dem Richterstuhl saß, sandte seine Frau zu ihm und ließ ihm sagen: Habe du nichts zu schaffen mit diesem Gerechten; denn ich habe heute im Traum seinetwegen viel gelitten!}
\verstab{20}{Schlachter}{Aber die obersten Priester und die Ältesten überredeten die Volksmenge, den Barabbas zu erbitten, Jesus aber umbringen zu lassen.}
\verstab{21}{Schlachter}{Der Statthalter aber antwortete und sprach zu ihnen: Welchen von diesen beiden wollt ihr, dass ich euch freilasse? Sie sprachen: Den Barabbas!}
\verstab{22}{Schlachter}{Pilatus spricht zu ihnen: Was soll ich denn mit Jesus tun, den man Christus nennt? Sie sprachen alle zu ihm: Kreuzige ihn!}
\verstab{23}{Schlachter}{Da sagte der Statthalter: Was hat er denn Böses getan? Sie aber schrien noch viel mehr und sprachen: Kreuzige ihn!}
\verstab{24}{Schlachter}{Als nun Pilatus sah, dass er nichts ausrichtete, sondern dass vielmehr ein Aufruhr entstand, nahm er Wasser und wusch sich vor der Volksmenge die Hände und sprach: Ich bin unschuldig an dem Blut dieses Gerechten; seht ihr zu!}
\verstab{25}{Schlachter}{Und das ganze Volk antwortete und sprach: Sein Blut komme über uns und über unsere Kinder!}
\verstab{26}{Schlachter}{Da gab er ihnen den Barabbas frei; Jesus aber ließ er geißeln und übergab ihn zur Kreuzigung.}
\verstab{27}{Schlachter}{Da nahmen die Kriegsknechte des Statthalters Jesus in das Prätorium und versammelten die ganze Schar um ihn.}
\verstab{28}{Schlachter}{Und sie zogen ihn aus und legten ihm einen Purpurmantel um}
\verstab{29}{Schlachter}{und flochten eine Krone aus Dornen, setzten sie auf sein Haupt, gaben ihm ein Rohr in die rechte Hand und beugten vor ihm die Knie, verspotteten ihn und sprachen: Sei gegrüßt, König der Juden!}
\verstab{30}{Schlachter}{Dann spuckten sie ihn an und nahmen das Rohr und schlugen ihn auf das Haupt.}
\verstab{31}{Schlachter}{Und nachdem sie ihn verspottet hatten, zogen sie ihm den Mantel aus und legten ihm seine Kleider an. Und sie führten ihn ab, um ihn zu kreuzigen.}
\verstab{32}{Schlachter}{Als sie aber hinauszogen, fanden sie einen Mann von Kyrene namens Simon; den zwangen sie, ihm das Kreuz zu tragen.}
\verstab{33}{Schlachter}{Und als sie an den Platz kamen, den man Golgatha nennt, das heißt »Schädelstätte«,}
\verstab{34}{Schlachter}{gaben sie ihm Essig mit Galle vermischt zu trinken; und als er es gekostet hatte, wollte er nicht trinken.}
\verstab{35}{Schlachter}{Nachdem sie ihn nun gekreuzigt hatten, teilten sie seine Kleider unter sich und warfen das Los, damit erfüllt würde, was durch den Propheten gesagt ist: »Sie haben meine Kleider unter sich geteilt, und das Los über mein Gewand geworfen«.}
\verstab{36}{Schlachter}{Und sie saßen dort und bewachten ihn.}
\verstab{37}{Schlachter}{Und sie befestigten über seinem Haupt die Inschrift seiner Schuld: »Dies ist Jesus, der König der Juden«.}
\verstab{38}{Schlachter}{Dann wurden mit ihm zwei Räuber gekreuzigt, einer zur Rechten, der andere zur Linken.}
\verstab{39}{Schlachter}{Aber die Vorübergehenden lästerten ihn, schüttelten den Kopf}
\verstab{40}{Schlachter}{und sprachen: Der du den Tempel zerstörst und in drei Tagen aufbaust, rette dich selbst! Wenn du Gottes Sohn bist, so steige vom Kreuz herab!}
\verstab{41}{Schlachter}{Gleicherweise spotteten aber auch die obersten Priester samt den Schriftgelehrten und Ältesten und sprachen:}
\verstab{42}{Schlachter}{Andere hat er gerettet, sich selbst kann er nicht retten! Ist er der König Israels, so steige er nun vom Kreuz herab, und wir wollen ihm glauben!}
\verstab{43}{Schlachter}{r hat auf Gott vertraut; der befreie ihn jetzt, wenn er Lust an ihm hat; denn er hat ja gesagt: Ich bin Gottes Sohn!}
\verstab{44}{Schlachter}{Ebenso schmähten ihn auch die Räuber, die mit ihm gekreuzigt waren.}
\verstab{45}{Schlachter}{Aber von der sechsten Stunde an kam eine Finsternis über das ganze Land bis zur neunten Stunde.}
\verstab{46}{Schlachter}{Und um die neunte Stunde rief Jesus mit lauter Stimme: Eli, Eli, lama sabachthani, das heißt: »Mein Gott, mein Gott, warum hast du mich verlassen?«}
\verstab{47}{Schlachter}{Etliche der Anwesenden sprachen, als sie es hörten: Der ruft den Elia!}
\verstab{48}{Schlachter}{Und sogleich lief einer von ihnen, nahm einen Schwamm, füllte ihn mit Essig, steckte ihn auf ein Rohr und gab ihm zu trinken.}
\verstab{49}{Schlachter}{Die Übrigen aber sprachen: Halt, lasst uns sehen, ob Elia kommt, um ihn zu retten!}
\verstab{50}{Schlachter}{Jesus aber schrie nochmals mit lauter Stimme und gab den Geist auf.}
\verstab{51}{Schlachter}{Und siehe, der Vorhang im Tempel riss von oben bis unten entzwei, und die Erde erbebte, und die Felsen spalteten sich.}
\verstab{52}{Schlachter}{Und die Gräber öffneten sich, und viele Leiber der entschlafenen Heiligen wurden auferweckt}
\verstab{53}{Schlachter}{und gingen aus den Gräbern hervor nach seiner Auferstehung und kamen in die heilige Stadt und erschienen vielen.}
\verstab{54}{Schlachter}{Als aber der Hauptmann und die, welche mit ihm Jesus bewachten, das Erdbeben sahen und was da geschah, fürchteten sie sich sehr und sprachen: Wahrhaftig, dieser war Gottes Sohn!}
\verstab{55}{Schlachter}{Es waren aber dort viele Frauen, die von ferne zusahen, welche Jesus von Galiläa her gefolgt waren und ihm gedient hatten;}
\verstab{56}{Schlachter}{unter ihnen waren Maria Magdalena und Maria, die Mutter des Jakobus und Joses, und die Mutter der Söhne des Zebedäus.}
\verstab{57}{Schlachter}{Als es nun Abend geworden war, kam ein reicher Mann von Arimathia namens Joseph, der auch ein Jünger Jesu geworden war.}
\verstab{58}{Schlachter}{Dieser ging zu Pilatus und bat um den Leib Jesu. Da befahl Pilatus, dass ihm der Leib gegeben werde.}
\verstab{59}{Schlachter}{Und Joseph nahm den Leib, wickelte ihn in reine Leinwand}
\verstab{60}{Schlachter}{und legte ihn in sein neues Grab, das er im Felsen hatte aushauen lassen; und er wälzte einen großen Stein vor den Eingang des Grabes und ging davon.}
\verstab{61}{Schlachter}{Es waren aber dort Maria Magdalena und die andere Maria, die saßen dem Grab gegenüber.}
\verstab{62}{Schlachter}{Am anderen Tag nun, der auf den Rüsttag folgt, versammelten sich die obersten Priester und die Pharisäer bei Pilatus}
\verstab{63}{Schlachter}{und sprachen: Herr, wir erinnern uns, dass dieser Verführer sprach, als er noch lebte: Nach drei Tagen werde ich auferstehen.}
\verstab{64}{Schlachter}{So befiehl nun, dass das Grab sicher bewacht wird bis zum dritten Tag, damit nicht etwa seine Jünger in der Nacht kommen, ihn stehlen und zum Volk sagen: Er ist aus den Toten auferstanden!, und der letzte Betrug schlimmer wird als der erste.}
\verstab{65}{Schlachter}{Pilatus aber sprach zu ihnen: Ihr sollt eine Wache haben! Geht hin und bewacht es, so gut ihr könnt!}
\verstab{66}{Schlachter}{Da gingen sie hin, versiegelten den Stein und bewachten das Grab mit der Wache.}
        %\subsection*{Kapitel 28}
\addcontentsline{toc}{subsection}{Kapitel 28}
\verstab{1}{Schlachter}{Nach dem Sabbat aber, als der erste Tag der Woche anbrach, kamen Maria Magdalena und die andere Maria, um das Grab zu besehen.}
\verstab{2}{Schlachter}{Und siehe, es geschah ein großes Erdbeben, denn ein Engel des Herrn stieg vom Himmel herab, trat herzu, wälzte den Stein von dem Eingang hinweg und setzte sich darauf.}
\verstab{3}{Schlachter}{Sein Aussehen war wie der Blitz und sein Gewand weiß wie der Schnee.}
\verstab{4}{Schlachter}{Vor seinem furchtbaren Anblick aber erbebten die Wächter und wurden wie tot.}
\verstab{5}{Schlachter}{Der Engel aber wandte sich zu den Frauen und sprach: Fürchtet ihr euch nicht! Ich weiß wohl, dass ihr Jesus, den Gekreuzigten, sucht.}
\verstab{6}{Schlachter}{Er ist nicht hier, denn er ist auferstanden, wie er gesagt hat. Kommt her, seht den Ort, wo der Herr gelegen hat!}
\verstab{7}{Schlachter}{Und geht schnell hin und sagt seinen Jüngern, dass er aus den Toten auferstanden ist. Und siehe, er geht euch voran nach Galiläa; dort werdet ihr ihn sehen. Siehe, ich habe es euch gesagt!}
\verstab{8}{Schlachter}{Und sie gingen schnell zum Grab hinaus mit Furcht und großer Freude und liefen, um es seinen Jüngern zu verkünden.}
\verstab{9}{Schlachter}{Und als sie gingen, um es seinen Jüngern zu verkünden, siehe, da begegnete ihnen Jesus und sprach: Seid gegrüßt! Sie aber traten herzu und umfassten seine Füße und beteten ihn an.}
\verstab{10}{Schlachter}{Da sprach Jesus zu ihnen: Fürchtet euch nicht! Geht hin, verkündet meinen Brüdern, dass sie nach Galiläa gehen sollen; dort werden sie mich sehen!}
\verstab{11}{Schlachter}{Während sie aber hingingen, siehe, da kamen etliche von der Wache in die Stadt und verkündeten den obersten Priestern alles, was geschehen war.}
\verstab{12}{Schlachter}{Diese versammelten sich samt den Ältesten, und nachdem sie Rat gehalten hatten, gaben sie den Kriegsknechten Geld genug}
\verstab{13}{Schlachter}{und sprachen: Sagt, seine Jünger sind bei Nacht gekommen und haben ihn gestohlen, während wir schliefen.}
\verstab{14}{Schlachter}{Und wenn dies vor den Statthalter kommt, so wollen wir ihn besänftigen und machen, dass ihr ohne Sorge sein könnt.}
\verstab{15}{Schlachter}{Sie aber nahmen das Geld und machten es so, wie sie belehrt worden waren. Und so wurde dieses Wort unter den Juden verbreitet bis zum heutigen Tag.}
\verstab{16}{Schlachter}{Die elf Jünger aber gingen nach Galiläa auf den Berg, wohin Jesus sie bestellt hatte.}
\verstab{17}{Schlachter}{Und als sie ihn sahen, warfen sie sich anbetend vor ihm nieder; etliche aber zweifelten.}
\verstab{18}{Schlachter}{Und Jesus trat herzu, redete mit ihnen und sprach: Mir ist gegeben alle Macht im Himmel und auf Erden.}
\verstab{19}{Schlachter}{So geht nun hin und macht zu Jüngern alle Völker, und tauft sie auf den Namen des Vaters und des Sohnes und des Heiligen Geistes}
\verstab{20}{Schlachter}{und lehrt sie alles halten, was ich euch befohlen habe. Und siehe, ich bin bei euch alle Tage bis an das Ende der Weltzeit! Amen.}  
        
\section{Philipper}
        \setdefault{ESRA}
        \setversions{ESRA}
        Standart-Bibel: \getdefault \\
        Ausdruch: \getversions \\        
        
\subsection*{Kapitel 1}
\addcontentsline{toc}{subsection}{Kapitel 1}
\verstab{1}{ESRA}{
    \person{Paulus} \bindV{und} \person{Timotheus}, \person{Knechte Jesu Christi}, alle \person{Heiligen} in \person{Jesus}, der, \person{\person{Christus}}, die in \ort{Philippi} sind, zusammen mit \person{Aufsehern} \bindV{und} \person{Dienern}:} 

\verstab{2}{ESRA}{
    Gnade euch \bindV{und} Friede von \person{Gott}, unserem \person{Vater}, \bindV{und} dem \person{Herrn Jesus, dem Gesalbten}!} 
\verstab{3}{ESRA}{Ich \verbN{danke} meinem \person{Gott} bei jedem Gedenken an euch,}
\verstab{4}{ESRA}{
    allezeit in all meinem Beten für euch alle, \bindA{dabei} das Gebet mit Freuden verrichtend
    }
\verstab{5}{ESRA}{
    \bindB{wegen} eurer Teilnahme am Evangelium vom ersten Tag an bis jetzt,
    }
\verstab{6}{ESRA}{ 
    \bindA{weil} ich davon \verbP{überzeugt bin}, \bindA{dass} der, der ein gutes Werk in euch \verbN{angefangen hat}, \es{es} zu Ende \verbN{führen wird} bis zum Tag Jesu Christi;}
\verstab{7}{ESRA}{
    \bindA{so} wie es für mich \verbN{recht ist}, dies über euch alle zu \verbN{denken}, weil ich euch im Herzen \verbN{habe}, da ihr alle sowohl in meinen Fesseln als auch in der Verteidigung \bindV{und} Bekräftigung des Evangeliums zusammen mit mir Teilhaber \verbP{seid} an der Gnade.}
\verstab{8}{ESRA}{
    Denn \person{Gott} \verbN{ist} mein Zeuge, wie ich mich nach euch allen \verbN{sehne} mit dem herzlichen Empfinden \person{Jesu, des Gesalbten}.}
\verstab{9}{ESRA}{
    \bindV{Und} dieses \verbN{erbete} ich, \bindA{dass} eure Liebe noch mehr \bindV{und} mehr \verbN{zunehme} in der Erkenntnis \bindV{und} allem Empfinden,}
\verstab{10}{ESRA}{
    \bindA{sodass} ihr \verbN{prüfen könnt}, was das Vorzuziehende \verbN{sei} \bindB{damit} ihr lauter \bindV{und} ohne Anstoss \verbN{seid} am Tag Christi,}
\verstab{11}{ESRA}{
\verbP{erfüllt} mit der Frucht der Gerechtigkeit, die durch \person{Jesus, den Gesalbten}, \es{\verbN{ist}}, zur Herrlichkeit \bindV{und} zum Lob \person{Gottes}.}
\verstab{12}{ESRA}{
    Ich \verbN{will} aber, \bindA{dass} ihr \verbN{wisst}, \person{Brüder}, \bindA{dass} meine Umstände mehr zum Fortschreiten des Evangeliums \verbN{geführt haben},}
\verstab{13}{ESRA}{
    \bindA{sodass} meine Fesseln \es{als Fesseln} in \person{Christus} offenbar \verbN{geworden sind} im ganzen \ort{Prätorium} \bindV{und} den übrigen allen,}
\verstab{14}{ESRA}{
    \bindV{und} \bindA{dass} die meisten der \person{Brüder}, \bindB{da} sie im Herrn \verbP{Vertrauen haben} durch meine Fesseln, \bindA{umso} mehr \verbN{wagen}, das Wort Gottes zu \verbN{sagen} ohne Furcht.}
\verstab{15}{ESRA}{Zwar \verbN{verkündigen} einige den \person{Christus} gar aus Neid \bindV{und} Streit, andere \bindB{dagegen} aus gutem Willen.}
\verstab{16}{ESRA}{
    Die einen aus Liebe, \bindB{da} sie \verbP{wissen}, \bindA{dass} ich zur Verteidigung des Evangeliums \verbP{bestimmt bin};}
\verstab{17}{ESRA}{
    die anderen \verbN{verkünden} \person{Christus} aus Eigennutz, nicht lauter, \bindB{da} sie \verbN{meinen}, \es{mir} in meinen Fesseln Begrängnis zu \verbP{erwecken}.}
\verstab{18}{ESRA}{
    Doch was \es{\verbN{tut}\textquotesingle s}? Jedenfalls \verbN{wird} auf alle Weise, \verbN{sei} es zum Vorwand \bindV{oder} in Wahrheit, \person{Christus} \verbP{verkündet}, \bindV{und} darüber \verbN{freue} ich mich, ich \verbN{werde} mich \bindB{auch} \es{weiterhin} \verbN{freuen}.}
\verstab{19}{ESRA}{
    Ich \verbN{weiss} nähmlich: \enquote{Dies \verbN{wird} mir zum Heil ausgehen} durch euer Bitten \bindV{und} durch die Unterstützung des Geistes \person{Jesu, des Gesalbten},}
\verstab{20}{ESRA}{
    \bindA{gemäss} meinem erwartungsvollen Harren \bindV{und} der Hoffnung, \bindA{dass} ich in nichts \verbP{werde beschämt werden}, \bindA{sondern} mit allem Freimut, \bindA{wie} allezeit, \bindA{so auch} jetzt \person{Christus} \verbN{gross gemacht wird} an meinem Leib, \bind{ob} durch Leben \bindV{oder} Tod.}
\verstab{21}{ESRA}{
    \bindA{Denn} zu \verbN{leben ist} für mich \person{Christus} \bindV{und} zu \verbN{sterben} Gewinn.}
\verstab{22}{ESRA}{\bindA{Wenn} \bindV{aber} im Fleisch zu \verbN{leben} -- das \es{\verbN{hiesse}} für mich Frucht aus \es{weiterem} Wirken. \bindV{Und} was ich \verbN{wählen soll}, \verbN{weiss} ich nicht.}
\verstab{23}{ESRA}{
    Ich \verbP{werde bedrängt} von beidem, \bindB{da} ich Lust \verbN{habe}, aufzubrechen \bindV{und} bei \person{Christus} zu \verbN{sein}, \bindA{denn} \es{das \verbN{wäre}} um vieles \verbN{besser};}
\verstab{24}{ESRA}{
    \bindB{doch} das Verbleiben im Fleisch \verbN{ist nötiger} euretwegen.}
\verstab{25}{ESRA}{
    \bindA{Weil} ich von diesem \verbN{überzeugt bin}, \verbN{weiss} ich: Ich \verbN{werde bleiben} \bindV{und} bei euch allen \verbN{verbleiben} zu eurem Fortschreiten \bindV{und} eurer Freude im Glauben,}
\verstab{26}{ESRA}{
    \bindA{damit} euer Rühmen an mir in Jesus, dem Gesalbten, \verbN{zunehme} durch meine erneute Ankunft bei euch.}
\verstab{27}{ESRA}{
    Nur: \verbI{Führt} euer Leben \es{im Gemeinwesen} würdig des Evangeliums des \person{Christus}, \bindA{damit}, \bindB{ob} ich \verbN{ankomme} \bindV{und} euch \verbN{erblicke} oder \verbN{abwesend bin}, ich von euren Umständen \verbN{höre}, \bindA{dass} ihr \es{fest}steht in einem Geist, \bindV{mit} einer Seele zusammen \verbN{kämpfend} für den Glauben des Evangeliums}
\verstab{28}{ESRA}{
    \bindV{und} durch nichts \verbN{eingeschüchtert} von den \person{Widerstreitenden}, was für sie ein Anzeichen des Verderbens \verbN{ist}, aber eures Heils -- \bindV{und} das von Gott her;} 
\verstab{29}{ESRA}{
    \bindA{denn} euch \verbN{ist} es hinsichtlich \person{Christi} \verbP{geschenkt worden}, nicht allein an ihn zu verbN{glauben}, \bindA{sondern auch} für ihn zu \verbN{leiden},}
\verstab{30}{ESRA}{
    die ihr ja den gleichen Kampf \verbN{habt}, so beschaffen, wie ihr \es{ihn} an mir \verbN{gesehen habt} \bindV{und} von mir \verbN{hört}.}


        \section{2}
\hh{1}Wenn es also \es{so ist, dass es} Ermunterung gibt in Christus, wenn Zuspruch der Liebe, wenn Gemeinschaft des Geistes, wenn inniges Mitgefühl und Erbarmungen, \hh{2}dann macht meine Freude \es{damit} voll, dass ihr \es[auf] das Gleiche sinnt, indem ihr dieselbe Liebe habt, in einer Seele verbunden seid und indem ihr auf eines sinnt, \hh{3}indem ihr nichts aus Eigennutz oder leerer Ruhmsucht \es[tut], sondern in der Demut einer den anderen für höher hält als sich selbst, \hh{4}indem ein jeder auch auf das der anderen. \hh{5}Unter euch sei diese Gesinnung, die auch in Jesus, dem Gesalbten, war, \hh{6}der, obwohl in Gestalt Gottes seiend, das Gott Gleichsein nicht wie eine Beute ansah, \hh{7}sondern sich selbst entäusserte, indem er die Gestalt eines Knechtes annahm. Den Menschen gleich geworden und in der äusseren Erscheinung wie ein Mensch erfunden, \hh{8}erniedrigte er sich selbst, indem er gehorsahm wurde bis zum Tod, zum Tod an einem Kreuz. \hh{9}Darum erhöht Gott ihn auch über \es{alles} und gab ihm den Namen, der über jeden Namen ist. \hh{10}damit im Namen Jesu sich beuge jedes Knie, \es{der} Himmlischen der Irdischen und Unterirdischen, \hh{11}und jede Zunge bekenne, dass Jesus, der Gesalbte, Herr ist, zur Verherrlichung Gottes, des Vaters.

\hh{12}So denn, meine Geliebte, wie ihr allezeit gehorcht habt, nicht nur wie in meiner Anwesenheit, sondern jetzt vielmehr in meiner Abwesenheit, bringt euer eigenes Hiel hervor mit Frucht und Zittern; \hh{13}denn Gott ist der in euch Wirkende -- sowohl das Wollen als auch das Wirken -- wegen \es{seines} Wohlgefallens.

\hh{14}Tut alles ohne Murren und Bedenken, \hh{15}damit ihr untadelig und unverfälscht werdet, Kinder Gottes ohne Makel inmitten eines krummen und verdrehten Geschlechts, unter dem ihr aufscheint wie Lichter in der Welt, \hh{16}indem ihr festhaltet das Wort des Lebens, mir zum \es{Gegenstand des} Rühmens auf den Tag Christi, weil ich \es{dann} nicht vergeblich gelaufen bin, noch auch vergeblich gearbeitethabe. \hh{17}Wenn ich aber auch \es{als Gussopfer} ausgegossen werde über das Opfer und den Priesterdienst für euren Glauben, freue ich mich und freue mich mit euch allen. \hh{18}Ebenso freut auch ihr euch und freut euch zusammen mit mir.

\hh{19}Ichhoffe aber in dem Herrn Jesus, Timotheus bald zu euch zu senden, damit auch ich frohgemut sei, wenn ich eure Umstände erfahre.
 \hh{20} Ich habe nämlich niemand gleichgesinnt, der in echter Weise für das eure besorgt sein wird; \hh{21}denn alle suchen das Eigene, nicht das, \es{was} Jesu Christi \es[ist]. \hh{22} Aber seine Bewährtheit kennt ihr, dass er wie ein Kind dem Vater zusammen mit mir gedient hat im Evangelium. \hh{23}Diesen also hoffe ich, sofort zu schicken, sobald ich absehe wie es um mich steht. \hh{24}Doch ich bin zuversichtlich im Herrn, dass auch ich selbst bald kommen werden. \hh{25}Ich hielt es aber für notwendig, Epaphroditus meinen Bruder und Mitarbeiter und Mitkämpfer, aber euren Abgesandten und Diener meines Bedarfs, zu euch zu schicken, \hh{26}da er sich nach euch allen sehnte und in Unruhe war, weil ihr gehört hattet, dass er erkrankt, dem Tod nahe. Doch Gott erbarmte sich über ihn, und nicht nur über ihn, sondern auch über mich, damit ich nicht Kummer über Kummer bekäme. \hh{28}Also habe ich ihn \es{unso} eiliger geschickt, damit ihr, wenn ihr ihn seht, wieder froh werdet und ich weniger bekümmert sei. \hh{29}Nehmt ihn also auf im Herrn mit aller Freude, und haltet solche in Ehren; \hh{30}denn wegen des Werkes Christi kam er dem Tod nahe, indem er sein Leben gering achtete, um euren Mangel im Dienst für mich aufzufüllen.
        \subsection*{Kapitel 3}
\addcontentsline{toc}{subsection}{Kapitel 3}
\verstab{1}{ESRA}{Des Weiteren, meine Brüder, freut euch \es{stets} im Herrn! Euch das Gleiche \es{wiederholt} zu schreiben, macht mir keine Bedenken, auch aber \es{gibt es} Festigkeit.}

\verstab{2}{ESRA}{Habt ein Auge auf die Hunde, habt ein Auge auf die bösen Arbeiter, habt ein Auge auf die Zerschneidung.}
\verstab{3}{ESRA}{Denn wir sind die Beschneidung, die im Geist Gottes \es{Gott} dienen und uns in Jesus, dem Gesalbten, rühmen und nicht auf Fleisch vertrauen,}
\verstab{4}{ESRA}{obwohl auch ich \es{Grund hätte}, auf Fleisch zu vertrauen. Wenn irgendein anderer meint, er \es{habe Grund}, auf Fleisch zu vertrauen, ich mehr:}
\verstab{5}{ESRA}{Beschneidung als Achtjähriger, aus dem Geschlecht Israel, dem Stamm Benjamin, Hebräer von Hebräern; dem Gesetz nach Pharisäer;}
\verstab{6}{ESRA}{dem Eifer nach Verfolger der Gemeinde; der Gerechtigkeit nach, die im Gesetz \es{ist}, untadelig geworden.}

\verstab{7}{ESRA}{Jedoch, was irgend mit Gewinn war, das habe ich des Gesalbten wegen für Verlust geachtet;}
\verstab{8}{ESRA}{ja, vielmehr, ich achte noch alles für Verlust aufgrund des überragenden \es{Wertes} der Erkenntnis Christi Jesu, meines Herrn, dessentwegen ich alles verloren habe, und ich halte es für Unrat, damit ich Christus gewinne}
\verstab{9}{ESRA}{und in ihm erfunden werde, wobei ich nicht meine Gerechtigkeit habe -- die aus dem Gesetz --, sondern die durch den Glauben an den Gesalbten, die Gerechtigkeit aus Gott aufgrund des Glaubens,}
\verstab{10}{ESRA}{um die Erkenntnis zu erlangen von ihm und von der Kraft seiner Auferstehung und die Gemeinschaft mit seinen Leiden, womit ich seinen Tod gleichgestaltet werde,}
\verstab{11}{ESRA}{ob ich \es{vielleicht} hingelange zur Auferstehung aus den Toten.}

\verstab{12}{ESRA}{Nicht das ich \es{es} schon ergriffen habe oder schon vollendet bin; ich jage \es{ihm} aber nach, ob ich es auch ergreigen möge, weil ich \es{ja} auch ergriffen wurde von Jesus, dem Gesalbten.}
\verstab{13}{ESRA}{Brüder, ich selbst halte mich nicht dafür, \es{es} ergriffen zu haben; eines aber: Indem ich vergesse, was dahinten ist, und indem ich mich ausstrecke nach dem, was vorn ist,}
\verstab{14}{ESRA}{jage ich nach dem Ziel, hin zum Siegespreis, dem Ruf Gottes nach oben in Jesus, dem Gesalbten.}
\verstab{15}{ESRA}{So viele also vollkommen \es{sind}, lasst uns so gesinnt sein! Und wenn ihr anders gesinnt seid, auch das wird Gott euch aufdecken.}
\verstab{16}{ESRA}{Doch wozu wir gelangt sind; Richten wir uns nach derseleben \es{Ordnung} aus!}

\verstab{17}{ESRA}{Seid zusammen meine Nachahmer, Brüder, und achtet \es{stets} auf jene, die so wandeln, wie ihr uns zum Vorbild habt!}
\verstab{18}{ESRA}{Denn viele wandeln, von denen ich euch oft gesagt habe, jetzt aber sogar weinend sage: Sie \es{sind} die Feinde des Kreuzes Christi,}
\verstab{19}{ESRA}{deren Ende Verderben, deren Gott der Bauch und die Herrlichkeit in ihrer Schande ist, die auf das Irdische sinnen.}
\verstab{20}{ESRA}{Aber unser Gemeinwesen ist in den Himmeln, von woher wir auch als Retter den Herrn Jesus, den Gesalbten, erwarten,} 
\verstab{21}{ESRA}{der unseren Leib der Niedrigkeit umwandeln wird, sodass er gleichgestaltet wird seinem Leib der Herrlichkeit, nach der Wirkkraft, mit der er sich auch Alles zu unterwerfen vermag.}
        \subsection*{Kapitel 4}
\addcontentsline{toc}{subsection}{Kapitel 4}
\verstab{1}{ESRA}{Daher meine geliebten und ersehnten Brüder, meine Freude und mein Siegeskranz: Auf diese Weise steht \es{fest} im Herrn, Geliebte!}

\verstab{2}{ESRA}{Evodia rufe ich auf, und Syntyche rufe ich auf, das Gleiche zu sinnen im Herrn.}
\verstab{3}{ESRA}{Ja, ich bitte auch dich, echter Jochgenosse, stehe ihnen bei, die im Evangelium mit mir gekämpft haben, samt Clemens und meinen übrigen Mitarbeitern, deren Namen im Buch des Lebens \es{stehen}.}

\verstab{4}{ESRA}{Freut euch im Herrn allezeit! Nochmals will ich sagen: Freut euch!}
\verstab{5}{ESRA}{Eure Milde werde allen Menschen bekannt! Der Herr ist nahe.}
\verstab{6}{ESRA}{Macht euch um nichts Sorgen, sondern in allem sollen eure Bitten durch Gebet und Flehen mit Danksagung vor Gott kundwerden,}
\verstab{7}{ESRA}{und der alles Denken überragende Friede Gottes wird eure Herzen und eure Gedanken in Gewahrsam halten in Jesus, dem Gesalbten.}

\verstab{8}{ESRA}{Des Weiteren, Brüder, alles, was wahr, was ehrbar, was gerecht, was rein, was liebenswert ist, was wohltuend ist, ob eine Tugend, ob ein Lob -- diese Dinge bedenkt.}
\verstab{9}{ESRA}{ Was ihr auch gelernt und übernommen und gehört und an mir gesehen habt, das tut, und der Gott des Friedens wird mit euch sein.}

\verstab{10}{ESRA}{Ich habe mich im Herrn hoch gefreut, dass ihr endlich wieder aufgeblüht seid, an mich zu denken; woran ihr zwar dachtet, aber ihr hattet keine Gelegenheit.}
\verstab{11}{ESRA}{Nicht dass ich das aufgrund von Mangel sage, denn ich habe gelernt, worin ich bin, genügsam zu sein.}
\verstab{12}{ESRA}{Ich weiss erniedrigt zu sein, und ich weiss übrig zu haben. In jedes und in alles bin ich eingeweiht: satt sein und hungern, übrig haben und Mangel leiden.}
\verstab{13}{ESRA}{Alles vermag ich durch den, der mich \es{fortwährend} kräftigt.}
\verstab{14}{ESRA}{Und doch, ihr habt gut \es{daran} getan, an meiner Bedrängnis Anteil zu nehmen.}

\verstab{15}{ESRA}{Ihr wisst auch selbst \es{liebe} Philipper, dass im Anfang \es{der Verkündigung} des Evangeliums, als ich wegzog, von Mazedonien, keine Gemeinde Gemeinschaft hatte mit mir im  \es{gegenseitigen} Geben und Empfangen als nur ihr allein.}
\verstab{16}{ESRA}{Nämlich auch in Thessalonich schicktet ihr mir einmal, sogar zweimal \es{etwas} für meinen Bedarf.}
\verstab{17}{ESRA}{Nicht dass ich die Gabe suche, sondern ich suche die sich für eure Rechnung mehrende Frucht.}
\verstab{18}{ESRA}{Ich habe alles erhalten und habe übrig; ich bin erfüllt, nachdem ich von Epaphroditus die \es{Gabe} von euch empfangen habe, einen lieblichen Geruch. Ein willkommenes Opfer, Gott wohgefällig.}
\verstab{19}{ESRA}{Mein Gott aber wird all euren Bedarf erüllen nach seinem Reichtum in Herrlichkeit in Christus Jesus.}
\verstab{20}{ESRA}{Unserem Gott und Vater sei die Herrlichkeit in alle Ewigkeit! Amen.}

\verstab{21}{ESRA}{Grüsst jeden Heiligen in Jesus, dem Gesalbten.}
\verstab{22}{ESRA}{Alle Heiligen Grüssen euch, am meisten die dem Haus des Kaisers.}

\verstab{23}{ESRA}{Die Gnade des Herrn Jesus, des Gesalbten, \es{sei} mit eurem Geist!}
      
        \section{1. Timotheus}
        \setdefault{Schlachter}
        \setversions{Schlachter}
        Standart-Bibel: \getdefault \\
        Ausdruch: \getversions \\        
        \subsection*{Kapitel 1}
\addcontentsline{toc}{subsection}{Kapitel 1}
\verstab{1}{Schlachter}{Paulus, Apostel Jesu Christi nach dem Befehl Gottes, unseres Retters, und des Herrn Jesus Christus, der unsere Hoffnung ist,}
\verstab{2}{Schlachter}{an Timotheus, [mein] echtes Kind im Glauben: Gnade, Barmherzigkeit, Friede [sei mit dir] von Gott, unserem Vater, und Christus Jesus, unserem Herrn!}
\verstab{3}{Schlachter}{Ich habe dich ja bei meiner Abreise nach Mazedonien ermahnt, in Ephesus zu bleiben, dass du gewissen Leuten gebietest, keine fremden Lehren zu verbreiten}
\verstab{4}{Schlachter}{und sich auch nicht mit Legenden und endlosen Geschlechtsregistern zu beschäftigen, die mehr Streitfragen hervorbringen als göttliche Erbauung im Glauben;}
\verstab{5}{Schlachter}{das Endziel des Gebotes aber ist Liebe aus reinem Herzen und gutem Gewissen und ungeheucheltem Glauben.}
\verstab{6}{Schlachter}{Davon sind einige abgeirrt und haben sich unnützem Geschwätz zugewandt;}
\verstab{7}{Schlachter}{sie wollen Lehrer des Gesetzes sein und verstehen doch nicht, was sie verkünden und als gewiss hinstellen.}
\verstab{8}{Schlachter}{Wir wissen aber, dass das Gesetz gut ist, wenn man es gesetzmäßig anwendet}
\verstab{9}{Schlachter}{und berücksichtigt, dass einem Gerechten kein Gesetz auferlegt ist, sondern Gesetzlosen und Widerspenstigen, Gottlosen und Sündern, Unheiligen und Gemeinen, solchen, die Vater und Mutter misshandeln, Menschen töten,}
\verstab{10}{Schlachter}{Unzüchtigen, Knabenschändern, Menschenräubern, Lügnern, Meineidigen und was sonst der gesunden Lehre widerspricht,}
\verstab{11}{Schlachter}{nach dem Evangelium der Herrlichkeit des glückseligen Gottes, das mir anvertraut worden ist.}
\verstab{12}{Schlachter}{Und darum danke ich dem, der mir Kraft verliehen hat, Christus Jesus, unserem Herrn, dass er mich treu erachtet und in den Dienst eingesetzt hat,}
\verstab{13}{Schlachter}{der ich zuvor ein Lästerer und Verfolger und Frevler war. Aber mir ist Erbarmung widerfahren, weil ich es unwissend im Unglauben getan habe.}
\verstab{14}{Schlachter}{Und die Gnade unseres Herrn wurde über alle Maßen groß samt dem Glauben und der Liebe, die in Christus Jesus ist.}
\verstab{15}{Schlachter}{Glaubwürdig ist das Wort und aller Annahme wert, dass Christus Jesus in die Welt gekommen ist, um Sünder zu retten, von denen ich der größte bin.}
\verstab{16}{Schlachter}{Aber darum ist mir Erbarmung widerfahren, damit an mir zuerst Jesus Christus alle Langmut erzeige, zum Vorbild für die, die künftig an ihn glauben würden zum ewigen Leben.}
\verstab{17}{Schlachter}{Dem König der Ewigkeit aber, dem unvergänglichen, unsichtbaren, allein weisen Gott, sei Ehre und Ruhm von Ewigkeit zu Ewigkeit! Amen.}
\verstab{18}{Schlachter}{Dieses Gebot vertraue ich dir an, mein Sohn Timotheus, gemäß den früher über dich ergangenen Weissagungen, damit du durch sie [gestärkt] den guten Kampf kämpfst,}
\verstab{19}{Schlachter}{indem du den Glauben und ein gutes Gewissen bewahrst. Dieses haben einige von sich gestoßen und darum im Glauben Schiffbruch erlitten.}
\verstab{20}{Schlachter}{Zu ihnen gehören Hymenäus und Alexander, die ich dem Satan übergeben habe, damit sie gezüchtigt werden und nicht mehr lästern.} 
        \subsection*{Kapitel 2}
\addcontentsline{toc}{subsection}{Kapitel 2}
\verstab{1}{Schlachter}{So ermahne ich nun, dass man vor allen Dingen Bitten, Gebete, Fürbitten und Danksagungen darbringe für alle Menschen,}
\verstab{2}{Schlachter}{für Könige und alle, die in hoher Stellung sind, damit wir ein ruhiges und stilles Leben führen können in aller Gottesfurcht und Ehrbarkeit;}
\verstab{3}{Schlachter}{denn dies ist gut und angenehm vor Gott, unserem Retter,}
\verstab{4}{Schlachter}{welcher will, dass alle Menschen gerettet werden und zur Erkenntnis der Wahrheit kommen.}
\verstab{5}{Schlachter}{Denn es ist ein Gott und ein Mittler zwischen Gott und den Menschen, der Mensch Christus Jesus,}
\verstab{6}{Schlachter}{der sich selbst als Lösegeld für alle gegeben hat. [Das ist] das Zeugnis zur rechten Zeit,}
\verstab{7}{Schlachter}{für das ich eingesetzt wurde als Verkündiger und Apostel — ich sage die Wahrheit in Christus und lüge nicht —, als Lehrer der Heiden im Glauben und in der Wahrheit.}
\verstab{8}{Schlachter}{So will ich nun, dass die Männer an jedem Ort beten, indem sie heilige Hände aufheben ohne Zorn und Zweifel.}
\verstab{9}{Schlachter}{Ebenso [will ich] auch, dass sich die Frauen in ehrbarem Anstand mit Schamhaftigkeit und Zucht schmücken, nicht mit Haarflechten oder Gold oder Perlen oder aufwendiger Kleidung,}
\verstab{10}{Schlachter}{sondern durch gute Werke, wie es sich für Frauen geziemt, die sich zur Gottesfurcht bekennen.}
\verstab{11}{Schlachter}{Eine Frau soll in der Stille lernen, in aller Unterordnung.}
\verstab{12}{Schlachter}{Ich erlaube aber einer Frau nicht, zu lehren, auch nicht, dass sie über den Mann herrscht, sondern sie soll sich still verhalten.}
\verstab{13}{Schlachter}{Denn Adam wurde zuerst gebildet, danach Eva.}
\verstab{14}{Schlachter}{Und Adam wurde nicht verführt, die Frau aber wurde verführt und geriet in Übertretung;}
\verstab{15}{Schlachter}{sie soll aber [davor] bewahrt werden durch das Kindergebären, wenn sie bleiben im Glauben und in der Liebe und in der Heiligung samt der Zucht.} 
        \subsection*{Kapitel 3}
\addcontentsline{toc}{subsection}{Kapitel 3}
\verstab{1}{Schlachter}{Glaubwürdig ist das Wort: Wer nach einem Aufseherdienst trachtet, der begehrt eine vortreffliche Tätigkeit.}
\verstab{2}{Schlachter}{Nun muss aber ein Aufseher untadelig sein, Mann einer Frau, nüchtern, besonnen, anständig, gastfreundlich, fähig zu lehren;}
\verstab{3}{Schlachter}{nicht der Trunkenheit ergeben, nicht gewalttätig, nicht nach schändlichem Gewinn strebend, sondern gütig, nicht streitsüchtig, nicht geldgierig;}
\verstab{4}{Schlachter}{einer, der seinem eigenen Haus gut vorsteht und die Kinder in Unterordnung hält mit aller Ehrbarkeit}
\verstab{5}{Schlachter}{-- wenn aber jemand seinem eigenen Haus nicht vorzustehen weiß, wie wird er für die Gemeinde Gottes sorgen? --,}
\verstab{6}{Schlachter}{kein Neubekehrter, damit er nicht aufgeblasen wird und in das Gericht des Teufels fällt.}
\verstab{7}{Schlachter}{Er muss aber auch ein gutes Zeugnis haben von denen außerhalb [der Gemeinde], damit er nicht in üble Nachrede und in die Fallstricke des Teufels gerät.}
\verstab{8}{Schlachter}{Gleicherweise sollen auch die Diakone ehrbar sein, nicht doppelzüngig, nicht vielem Weingenuss ergeben, nicht nach schändlichem Gewinn strebend;}
\verstab{9}{Schlachter}{sie sollen das Geheimnis des Glaubens in einem reinen Gewissen bewahren.}
\verstab{10}{Schlachter}{Und diese sollen zuerst erprobt werden; dann sollen sie dienen, wenn sie untadelig sind.}
\verstab{11}{Schlachter}{[Die] Frauen sollen ebenfalls ehrbar sein, nicht verleumderisch, sondern nüchtern, treu in allem.}
\verstab{12}{Schlachter}{Die Diakone sollen jeder Mann einer Frau sein, ihren Kindern und ihrem Haus gut vorstehen;}
\verstab{13}{Schlachter}{denn wenn sie ihren Dienst gut versehen, erwerben sie sich selbst eine gute Stufe und viel Freimütigkeit im Glauben in Christus Jesus.}
\verstab{14}{Schlachter}{Dies schreibe ich dir in der Hoffnung, recht bald zu dir zu kommen,}
\verstab{15}{Schlachter}{damit du aber, falls sich mein Kommen verzögern sollte, weißt, wie man wandeln soll im Haus Gottes, welches die Gemeinde des lebendigen Gottes ist, der Pfeiler und die Grundfeste der Wahrheit.}
\verstab{16}{Schlachter}{Und anerkannt groß ist das Geheimnis der Gottesfurcht: Gott ist geoffenbart worden im Fleisch, gerechtfertigt im Geist, gesehen von den Engeln, verkündigt unter den Heiden, geglaubt in der Welt, aufgenommen in die Herrlichkeit.} 
        \subsection*{Kapitel 4}
\addcontentsline{toc}{subsection}{Kapitel 4}
\verstab{1}{Schlachter}{Der Geist aber sagt ausdrücklich, dass in späteren Zeiten etliche vom Glauben abfallen und sich irreführenden Geistern und Lehren der Dämonen zuwenden werden}
\verstab{2}{Schlachter}{durch die Heuchelei von Lügenrednern, die in ihrem eigenen Gewissen gebrandmarkt sind.}
\verstab{3}{Schlachter}{Sie verbieten zu heiraten und Speisen zu genießen, die doch Gott geschaffen hat, damit sie mit Danksagung gebraucht werden von denen, die gläubig sind und die Wahrheit erkennen.}
\verstab{4}{Schlachter}{Denn alles, was Gott geschaffen hat, ist gut, und nichts ist verwerflich, wenn es mit Danksagung empfangen wird;}
\verstab{5}{Schlachter}{denn es wird geheiligt durch Gottes Wort und Gebet.}
\verstab{6}{Schlachter}{Wenn du dies den Brüdern vor Augen stellst, wirst du ein guter Diener Jesu Christi sein, der sich nährt mit den Worten des Glaubens und der guten Lehre, der du nachgefolgt bist.}
\verstab{7}{Schlachter}{Die unheiligen Altweiberlegenden aber weise ab; dagegen übe dich in der Gottesfurcht!}
\verstab{8}{Schlachter}{Denn die leibliche Übung nützt wenig, die Gottesfurcht aber ist für alles nützlich, da sie die Verheißung für dieses und für das zukünftige Leben hat.}
\verstab{9}{Schlachter}{Glaubwürdig ist das Wort und aller Annahme wert;}
\verstab{10}{Schlachter}{denn dafür arbeiten wir auch und werden geschmäht, weil wir unsere Hoffnung auf den lebendigen Gott gesetzt haben, der ein Retter aller Menschen ist, besonders der Gläubigen.}
\verstab{11}{Schlachter}{Dies sollst du gebieten und lehren!}
\verstab{12}{Schlachter}{Niemand verachte dich wegen deiner Jugend, sondern sei den Gläubigen ein Vorbild im Wort, im Wandel, in der Liebe, im Geist, im Glauben, in der Keuschheit!}
\verstab{13}{Schlachter}{Bis ich komme, sei bedacht auf das Vorlesen, das Ermahnen und das Lehren.}
\verstab{14}{Schlachter}{Vernachlässige nicht die Gnadengabe in dir, die dir verliehen wurde durch Weissagung unter Handauflegung der Ältestenschaft!}
\verstab{15}{Schlachter}{Dies soll deine Sorge sein, darin sollst du leben, damit deine Fortschritte in allen Dingen offenbar seien!}
\verstab{16}{Schlachter}{Habe acht auf dich selbst und auf die Lehre; bleibe beständig dabei! Denn wenn du dies tust, wirst du sowohl dich selbst retten als auch die, welche auf dich hören.} 
        \subsection*{Kapitel 5}
\addcontentsline{toc}{subsection}{Kapitel 5}
\verstab{1}{Schlachter}{Einen älteren Mann fahre nicht hart an, sondern ermahne ihn wie einen Vater, jüngere wie Brüder,}
\verstab{2}{Schlachter}{ältere Frauen wie Mütter, jüngere wie Schwestern, in aller Keuschheit.}
\verstab{3}{Schlachter}{Ehre die Witwen, die wirklich Witwen sind.}
\verstab{4}{Schlachter}{Wenn aber eine Witwe Kinder oder Enkel hat, so sollen diese zuerst lernen, am eigenen Haus gottesfürchtig zu handeln und den Eltern Empfangenes zu vergelten; denn das ist gut und wohlgefällig vor Gott.}
\verstab{5}{Schlachter}{Eine wirkliche und vereinsamte Witwe aber hat ihre Hoffnung auf Gott gesetzt und bleibt beständig im Flehen und Gebet Tag und Nacht;}
\verstab{6}{Schlachter}{eine genusssüchtige jedoch ist lebendig tot.}
\verstab{7}{Schlachter}{Sprich das offen aus, damit sie untadelig sind!}
\verstab{8}{Schlachter}{Wenn aber jemand für die Seinen, besonders für seine Hausgenossen, nicht sorgt, so hat er den Glauben verleugnet und ist schlimmer als ein Ungläubiger.}
\verstab{9}{Schlachter}{Eine Witwe soll nur in die Liste eingetragen werden, wenn sie nicht weniger als 60 Jahre alt ist, die Frau eines Mannes war}
\verstab{10}{Schlachter}{und ein Zeugnis guter Werke hat; wenn sie Kinder aufgezogen, Gastfreundschaft geübt, die Füße der Heiligen gewaschen, Bedrängten geholfen hat, wenn sie sich jedem guten Werk gewidmet hat.}
\verstab{11}{Schlachter}{Jüngere Witwen aber weise ab; denn wenn sie gegen [den Willen des] Christus begehrlich geworden sind, wollen sie heiraten}
\verstab{12}{Schlachter}{und kommen [damit] unter das Urteil, dass sie die erste Treue gebrochen haben.}
\verstab{13}{Schlachter}{Zugleich lernen sie auch untätig zu sein, indem sie in den Häusern herumlaufen; und nicht nur untätig, sondern auch geschwätzig und neugierig zu sein; und sie reden, was sich nicht gehört.}
\verstab{14}{Schlachter}{So will ich nun, dass jüngere [Witwen] heiraten, Kinder gebären, den Haushalt führen und dem Widersacher keinen Anlass zur Lästerung geben;}
\verstab{15}{Schlachter}{denn etliche haben sich schon abgewandt, dem Satan nach.}
\verstab{16}{Schlachter}{Wenn ein Gläubiger oder eine Gläubige Witwen hat, so soll er sie versorgen, und die Gemeinde soll nicht belastet werden, damit diese für die wirklichen Witwen sorgen kann.}
\verstab{17}{Schlachter}{Die Ältesten, die gut vorstehen, sollen doppelter Ehre wertgeachtet werden, besonders die, welche im Wort und in der Lehre arbeiten.}
\verstab{18}{Schlachter}{Denn die Schrift sagt: »Du sollst dem Ochsen nicht das Maul verbinden, wenn er drischt!«, und »Der Arbeiter ist seines Lohnes wert«.}
\verstab{19}{Schlachter}{Gegen einen Ältesten nimm keine Klage an, außer aufgrund von zwei oder drei Zeugen.}
\verstab{20}{Schlachter}{Die, welche sündigen, weise zurecht vor allen, damit sich auch die anderen fürchten.}
\verstab{21}{Schlachter}{Ich ermahne dich ernstlich vor Gott und dem Herrn Jesus Christus und den auserwählten Engeln, dass du dies ohne Vorurteil befolgst und nichts aus Zuneigung tust!}
\verstab{22}{Schlachter}{Die Hände lege niemand schnell auf, mache dich auch nicht fremder Sünden teilhaftig; bewahre dich selbst rein!}
\verstab{23}{Schlachter}{Trinke nicht mehr nur Wasser, sondern gebrauche ein wenig Wein um deines Magens willen und wegen deines häufigen Unwohlseins.}
\verstab{24}{Schlachter}{Die Sünden mancher Menschen sind allen offenbar und kommen vorher ins Gericht; manchen aber folgen sie auch nach.}
\verstab{25}{Schlachter}{Gleicherweise sind auch die guten Werke allen offenbar; und die, mit welchen es sich anders verhält, können auch nicht verborgen bleiben.} 
        \subsection*{Kapitel 6}
\addcontentsline{toc}{subsection}{Kapitel 6}
\verstab{1}{Schlachter}{Diejenigen, die als Knechte unter dem Joch sind, sollen ihre eigenen Herren aller Ehre wert halten, damit nicht der Name Gottes und die Lehre verlästert werden.}
\verstab{2}{Schlachter}{Die aber, welche gläubige Herren haben, sollen diese darum nicht gering schätzen, weil sie Brüder sind, sondern ihnen umso lieber dienen, weil es Gläubige und Geliebte sind, die darauf bedacht sind, Gutes zu tun. Dies sollst du lehren und dazu ermahnen!}
\verstab{3}{Schlachter}{Wenn jemand fremde Lehren verbreitet und nicht die gesunden Worte unseres Herrn Jesus Christus annimmt und die Lehre, die der Gottesfurcht entspricht,}
\verstab{4}{Schlachter}{so ist er aufgeblasen und versteht doch nichts, sondern krankt an Streitfragen und Wortgefechten, woraus Neid, Zwietracht, Lästerung, böse Verdächtigungen entstehen,}
\verstab{5}{Schlachter}{unnütze Streitgespräche von Menschen, die eine verdorbene Gesinnung haben und der Wahrheit beraubt sind und meinen, die Gottesfurcht sei ein Mittel zur Bereicherung — von solchen halte dich fern!}
\verstab{6}{Schlachter}{Es ist allerdings die Gottesfurcht eine große Bereicherung, wenn sie mit Genügsamkeit verbunden wird.}
\verstab{7}{Schlachter}{Denn wir haben nichts in die Welt hineingebracht, und es ist klar, dass wir auch nichts hinausbringen können.}
\verstab{8}{Schlachter}{Wenn wir aber Nahrung und Kleidung haben, soll uns das genügen!}
\verstab{9}{Schlachter}{enn die, welche reich werden wollen, fallen in Versuchung und Fallstricke und viele törichte und schädliche Begierden, welche die Menschen in Untergang und Verderben stürzen.}
\verstab{10}{Schlachter}{Denn die Geldgier ist eine Wurzel alles Bösen; etliche, die sich ihr hingegeben haben, sind vom Glauben abgeirrt und haben sich selbst viel Schmerzen verursacht.}
\verstab{11}{Schlachter}{Du aber, o Mensch Gottes, fliehe diese Dinge, jage aber nach Gerechtigkeit, Gottesfurcht, Glauben, Liebe, Geduld, Sanftmut!}
\verstab{12}{Schlachter}{Kämpfe den guten Kampf des Glaubens; ergreife das ewige Leben, zu dem du auch berufen bist und worüber du das gute Bekenntnis vor vielen Zeugen abgelegt hast.}
\verstab{13}{Schlachter}{Ich gebiete dir vor Gott, der alles lebendig macht, und vor Christus Jesus, der vor Pontius Pilatus das gute Bekenntnis bezeugt hat,}
\verstab{14}{Schlachter}{dass du das Gebot unbefleckt und untadelig bewahrst bis zur Erscheinung unseres Herrn Jesus Christus,}
\verstab{15}{Schlachter}{welche zu seiner Zeit zeigen wird der Glückselige und allein Gewaltige, der König der Könige und der Herr der Herrschenden,}
\verstab{16}{Schlachter}{der allein Unsterblichkeit hat, der in einem unzugänglichen Licht wohnt, den kein Mensch gesehen hat noch sehen kann; ihm sei Ehre und ewige Macht! Amen.}
\verstab{17}{Schlachter}{Den Reichen in der jetzigen Weltzeit gebiete, nicht hochmütig zu sein, auch nicht ihre Hoffnung auf die Unbeständigkeit des Reichtums zu setzen, sondern auf den lebendigen Gott, der uns alles reichlich zum Genuss darreicht.}
\verstab{18}{Schlachter}{Sie sollen Gutes tun, reich werden an guten Werken, freigebig sein, bereit, mit anderen zu teilen,}
\verstab{19}{Schlachter}{damit sie das ewige Leben ergreifen und so für sich selbst eine gute Grundlage für die Zukunft sammeln.}
\verstab{20}{Schlachter}{O Timotheus, bewahre das anvertraute Gut, meide das unheilige, nichtige Geschwätz und die Widersprüche der fälschlich so genannten »Erkenntnis«!}
\verstab{21}{Schlachter}{Zu dieser haben sich etliche bekannt und haben darüber das Glaubensziel verfehlt. Die Gnade sei mit dir! Amen.}       
        \section{Philemon}
        \setdefault{Schlachter}
        \setversions{Schlachter}
        Standart-Bibel: \getdefault \\
        Ausdruch: \getversions \\        
        
\subsection*{Kapitel 1}
\addcontentsline{toc}{subsection}{Kapitel 1}
\verstab{1}{ESRA}{
    \person{Paulus} \bindV{und} \person{Timotheus}, \person{Knechte Jesu Christi}, alle \person{Heiligen} in \person{Jesus}, der, \person{\person{Christus}}, die in \ort{Philippi} sind, zusammen mit \person{Aufsehern} \bindV{und} \person{Dienern}:} 

\verstab{2}{ESRA}{
    Gnade euch \bindV{und} Friede von \person{Gott}, unserem \person{Vater}, \bindV{und} dem \person{Herrn Jesus, dem Gesalbten}!} 
\verstab{3}{ESRA}{Ich \verbN{danke} meinem \person{Gott} bei jedem Gedenken an euch,}
\verstab{4}{ESRA}{
    allezeit in all meinem Beten für euch alle, \bindA{dabei} das Gebet mit Freuden verrichtend
    }
\verstab{5}{ESRA}{
    \bindB{wegen} eurer Teilnahme am Evangelium vom ersten Tag an bis jetzt,
    }
\verstab{6}{ESRA}{ 
    \bindA{weil} ich davon \verbP{überzeugt bin}, \bindA{dass} der, der ein gutes Werk in euch \verbN{angefangen hat}, \es{es} zu Ende \verbN{führen wird} bis zum Tag Jesu Christi;}
\verstab{7}{ESRA}{
    \bindA{so} wie es für mich \verbN{recht ist}, dies über euch alle zu \verbN{denken}, weil ich euch im Herzen \verbN{habe}, da ihr alle sowohl in meinen Fesseln als auch in der Verteidigung \bindV{und} Bekräftigung des Evangeliums zusammen mit mir Teilhaber \verbP{seid} an der Gnade.}
\verstab{8}{ESRA}{
    Denn \person{Gott} \verbN{ist} mein Zeuge, wie ich mich nach euch allen \verbN{sehne} mit dem herzlichen Empfinden \person{Jesu, des Gesalbten}.}
\verstab{9}{ESRA}{
    \bindV{Und} dieses \verbN{erbete} ich, \bindA{dass} eure Liebe noch mehr \bindV{und} mehr \verbN{zunehme} in der Erkenntnis \bindV{und} allem Empfinden,}
\verstab{10}{ESRA}{
    \bindA{sodass} ihr \verbN{prüfen könnt}, was das Vorzuziehende \verbN{sei} \bindB{damit} ihr lauter \bindV{und} ohne Anstoss \verbN{seid} am Tag Christi,}
\verstab{11}{ESRA}{
\verbP{erfüllt} mit der Frucht der Gerechtigkeit, die durch \person{Jesus, den Gesalbten}, \es{\verbN{ist}}, zur Herrlichkeit \bindV{und} zum Lob \person{Gottes}.}
\verstab{12}{ESRA}{
    Ich \verbN{will} aber, \bindA{dass} ihr \verbN{wisst}, \person{Brüder}, \bindA{dass} meine Umstände mehr zum Fortschreiten des Evangeliums \verbN{geführt haben},}
\verstab{13}{ESRA}{
    \bindA{sodass} meine Fesseln \es{als Fesseln} in \person{Christus} offenbar \verbN{geworden sind} im ganzen \ort{Prätorium} \bindV{und} den übrigen allen,}
\verstab{14}{ESRA}{
    \bindV{und} \bindA{dass} die meisten der \person{Brüder}, \bindB{da} sie im Herrn \verbP{Vertrauen haben} durch meine Fesseln, \bindA{umso} mehr \verbN{wagen}, das Wort Gottes zu \verbN{sagen} ohne Furcht.}
\verstab{15}{ESRA}{Zwar \verbN{verkündigen} einige den \person{Christus} gar aus Neid \bindV{und} Streit, andere \bindB{dagegen} aus gutem Willen.}
\verstab{16}{ESRA}{
    Die einen aus Liebe, \bindB{da} sie \verbP{wissen}, \bindA{dass} ich zur Verteidigung des Evangeliums \verbP{bestimmt bin};}
\verstab{17}{ESRA}{
    die anderen \verbN{verkünden} \person{Christus} aus Eigennutz, nicht lauter, \bindB{da} sie \verbN{meinen}, \es{mir} in meinen Fesseln Begrängnis zu \verbP{erwecken}.}
\verstab{18}{ESRA}{
    Doch was \es{\verbN{tut}\textquotesingle s}? Jedenfalls \verbN{wird} auf alle Weise, \verbN{sei} es zum Vorwand \bindV{oder} in Wahrheit, \person{Christus} \verbP{verkündet}, \bindV{und} darüber \verbN{freue} ich mich, ich \verbN{werde} mich \bindB{auch} \es{weiterhin} \verbN{freuen}.}
\verstab{19}{ESRA}{
    Ich \verbN{weiss} nähmlich: \enquote{Dies \verbN{wird} mir zum Heil ausgehen} durch euer Bitten \bindV{und} durch die Unterstützung des Geistes \person{Jesu, des Gesalbten},}
\verstab{20}{ESRA}{
    \bindA{gemäss} meinem erwartungsvollen Harren \bindV{und} der Hoffnung, \bindA{dass} ich in nichts \verbP{werde beschämt werden}, \bindA{sondern} mit allem Freimut, \bindA{wie} allezeit, \bindA{so auch} jetzt \person{Christus} \verbN{gross gemacht wird} an meinem Leib, \bind{ob} durch Leben \bindV{oder} Tod.}
\verstab{21}{ESRA}{
    \bindA{Denn} zu \verbN{leben ist} für mich \person{Christus} \bindV{und} zu \verbN{sterben} Gewinn.}
\verstab{22}{ESRA}{\bindA{Wenn} \bindV{aber} im Fleisch zu \verbN{leben} -- das \es{\verbN{hiesse}} für mich Frucht aus \es{weiterem} Wirken. \bindV{Und} was ich \verbN{wählen soll}, \verbN{weiss} ich nicht.}
\verstab{23}{ESRA}{
    Ich \verbP{werde bedrängt} von beidem, \bindB{da} ich Lust \verbN{habe}, aufzubrechen \bindV{und} bei \person{Christus} zu \verbN{sein}, \bindA{denn} \es{das \verbN{wäre}} um vieles \verbN{besser};}
\verstab{24}{ESRA}{
    \bindB{doch} das Verbleiben im Fleisch \verbN{ist nötiger} euretwegen.}
\verstab{25}{ESRA}{
    \bindA{Weil} ich von diesem \verbN{überzeugt bin}, \verbN{weiss} ich: Ich \verbN{werde bleiben} \bindV{und} bei euch allen \verbN{verbleiben} zu eurem Fortschreiten \bindV{und} eurer Freude im Glauben,}
\verstab{26}{ESRA}{
    \bindA{damit} euer Rühmen an mir in Jesus, dem Gesalbten, \verbN{zunehme} durch meine erneute Ankunft bei euch.}
\verstab{27}{ESRA}{
    Nur: \verbI{Führt} euer Leben \es{im Gemeinwesen} würdig des Evangeliums des \person{Christus}, \bindA{damit}, \bindB{ob} ich \verbN{ankomme} \bindV{und} euch \verbN{erblicke} oder \verbN{abwesend bin}, ich von euren Umständen \verbN{höre}, \bindA{dass} ihr \es{fest}steht in einem Geist, \bindV{mit} einer Seele zusammen \verbN{kämpfend} für den Glauben des Evangeliums}
\verstab{28}{ESRA}{
    \bindV{und} durch nichts \verbN{eingeschüchtert} von den \person{Widerstreitenden}, was für sie ein Anzeichen des Verderbens \verbN{ist}, aber eures Heils -- \bindV{und} das von Gott her;} 
\verstab{29}{ESRA}{
    \bindA{denn} euch \verbN{ist} es hinsichtlich \person{Christi} \verbP{geschenkt worden}, nicht allein an ihn zu verbN{glauben}, \bindA{sondern auch} für ihn zu \verbN{leiden},}
\verstab{30}{ESRA}{
    die ihr ja den gleichen Kampf \verbN{habt}, so beschaffen, wie ihr \es{ihn} an mir \verbN{gesehen habt} \bindV{und} von mir \verbN{hört}.}

      
      
        %%%%%%%%%%%%%%%%%%%%%%%%%%%%%%%%%%%%%%%%%%%%
       
      
    \end{spacing}
\end{adjustwidth*}
\end{document}

