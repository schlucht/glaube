\subsection*{Kapitel 3}
\addcontentsline{toc}{subsection}{Kapitel 3}
\verstab{1}{ESRA}{
    \bindB{Des Weiteren}, meine \person{Brüder}, \verbI{freut} euch \es{stets} im Herrn! Euch das Gleiche \es{wiederholt} zu \verbN{schreiben}, \verbN{macht} mir keine Bedenken, auch \bindA{aber} \es{\verbN{gibt} es} Festigkeit.}

\verstab{2}{ESRA}{
    \verbI{Habt} ein Auge auf die \person{Hunde}, \verbI{habt} ein Auge auf die bösen \person{Arbeiter}, \verbI{habt} ein Auge auf die Zerschneidung.}
\verstab{3}{ESRA}{
    \bindA{Denn} wir \verbN{sind} die Beschneidung, die im Geist Gottes \es{Gott} \verbN{dienen} \bindV{und} uns in \person{Jesus, dem Gesalbten}, \verbN{rühmen} \bindV{und} nicht auf Fleisch \verbN{vertrauen},}
\verstab{4}{ESRA}{
    \bindB{obwohl} auch ich \es{und} \verbN{hätte}, auf Fleisch zu \verbN{vertrauen}. \bindV{Wenn} irgendein anderer \verbN{meint}, er \es{\verbN{habe} Grund}, auf Fleisch zu \verbN{vertrauen}, ich mehr:}
\verstab{5}{ESRA}{
    Beschneidung als Achtjähriger, aus dem Geschlecht \ort{Israel}, dem Stamm \ort{Benjamin}, Hebräer von Hebräern; dem Gesetz nach Pharisäer;}
\verstab{6}{ESRA}{dem Eifer nach Verfolger der Gemeinde; der Gerechtigkeit nach, die im Gesetz \es{\verbN{ist}}, \verbN{untadelig geworden}.}

\verstab{7}{ESRA}{\bindB{Jedoch}, was irgend mit Gewinn \verbN{war}, das \verbN{habe} ich des \person{Gesalbten} wegen für Verlust \verbN{geachtet};}
\verstab{8}{ESRA}{ja, vielmehr, ich verbN{achte} noch alles für Verlust aufgrund des überragenden \es{Wertes} der Erkenntnis Christi Jesu, meines Herrn, dessentwegen ich alles \verbN{verloren habe}, \bindV{und} ich halte es für Unrat, \bindA{damit} ich Christus \verbN{gewinne}}
\verstab{9}{ESRA}{\bindV{und} in ihm \verbP{erfunden werde}, wobei ich nicht meine Gerechtigkeit \verbN{habe} -- die aus dem Gesetz --, \bindB{sondern} die durch den Glauben an den \person{Gesalbten}, die Gerechtigkeit aus Gott aufgr\bindV{und} des Glaubens,}
\verstab{10}{ESRA}{\bindB{um} die Erkenntnis zu \verbN{erlangen} von ihm \bindV{und} von der Kraft seiner Auferstehung \bindV{und} die Gemeinschaft mit seinen Leiden, womit ich seinen Tod \verbN{gleichgestaltet werde},}
\verstab{11}{ESRA}{\bind{ob} ich \es{vielleicht} \verbN{hingelange} zur Auferstehung aus den Toten.}

\verstab{12}{ESRA}{Nicht \bindB{dass} ich \es{es} schon \verbN{ergriffen habe} \bindV{oder} schon \verbN{vollendet bin}; ich \verbN{jage} \es{ihm} \bindA{aber} nach, \bind{ob} ich es auch \verbN{ergreifen möge}, \bindA{weil} ich \es{ja} \bindB{auch} \verbN{ergriffen wurde} von \person{Jesus, dem Gesalbten}.}
\verstab{13}{ESRA}{\person{Brüder}, ich selbst \verbN{halte} mich nicht \verbN{dafür}, \es{es} \verbN{ergriffen} zu \verbN{haben}; eines \bindA{aber}: Indem ich \verbN{vergesse}, was dahinten \verbN{ist}, \bindV{und} indem ich mich \verbN{ausstrecke} nach dem, was vorn \verbN{ist},}
\verstab{14}{ESRA}{\verbN{jage} ich nach dem Ziel, hin zum Siegespreis, dem Ruf Gottes nach oben in \person{Jesus, dem Gesalbten}.}
\verstab{15}{ESRA}{So viele also \verbN{vollkommen} \es{\verbN{sind}}, \verbI{lasst} uns so \verbI{gesinnt sein}! \bindV{Und wenn} ihr anders gesinnt seid, \bindV{auch} das wird Gott euch \verbN{aufdecken}.}
\verstab{16}{ESRA}{Doch wozu wir \verbN{gelangt sind}; \verbI{Richten} wir uns nach derseleben \es{Ordnung} \verbI{aus}!}

\verstab{17}{ESRA}{\verbI{Seid} zusammen meine Nachahmer, \person{Brüder}, \bindV{und} \verbI{achtet} \es{stets} auf jene, die so \verbN{wandeln}, wie ihr uns zum Vorbild \verbN{habt}!}
\verstab{18}{ESRA}{\bindB{Denn} viele \verbN{wandeln}, von denen ich euch oft \verbN{gesagt habe}, jetzt \bindA{aber} sogar \verbN{weinend sage}: Sie \es{\verbN{sind}} die \person{Feinde} des Kreuzes Christi,}
\verstab{19}{ESRA}{deren Ende Verderben, deren Gott der Bauch \bindV{und} die Herrlichkeit in ihrer Schande \verbN{ist}, die auf das Irdische \verbN{sinnen}.}
\verstab{20}{ESRA}{\bindA{Aber} unser Gemeinwesen ist in den Himmeln, von woher wir auch als \person{Retter} den Herrn \person{Jesus, den Gesalbten}, \verbN{erwarten},} 
\verstab{21}{ESRA}{der unseren Leib der Niedrigkeit \verbN{umwandeln wird}, \bindB{sodass} er \verbN{gleichgestaltet wird} seinem Leib der Herrlichkeit, nach der Wirkkraft, mit der er sich auch Alles zu \verbN{unterwerfen vermag}.}