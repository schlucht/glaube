\subsection*{Kapitel 4}
\addcontentsline{toc}{subsection}{Kapitel 4}
\verstab{1}{ESRA}{Daher meine geliebten und ersehnten Brüder, meine Freude und mein Siegeskranz: Auf diese Weise steht \es{fest} im Herrn, Geliebte!}

\verstab{2}{ESRA}{Evodia rufe ich auf, und Syntyche rufe ich auf, das Gleiche zu sinnen im Herrn.}
\verstab{3}{ESRA}{Ja, ich bitte auch dich, echter Jochgenosse, stehe ihnen bei, die im Evangelium mit mir gekämpft haben, samt Clemens und meinen übrigen Mitarbeitern, deren Namen im Buch des Lebens \es{stehen}.}

\verstab{4}{ESRA}{Freut euch im Herrn allezeit! Nochmals will ich sagen: Freut euch!}
\verstab{5}{ESRA}{Eure Milde werde allen Menschen bekannt! Der Herr ist nahe.}
\verstab{6}{ESRA}{Macht euch um nichts Sorgen, sondern in allem sollen eure Bitten durch Gebet und Flehen mit Danksagung vor Gott kundwerden,}
\verstab{7}{ESRA}{und der alles Denken überragende Friede Gottes wird eure Herzen und eure Gedanken in Gewahrsam halten in Jesus, dem Gesalbten.}

\verstab{8}{ESRA}{Des Weiteren, Brüder, alles, was wahr, was ehrbar, was gerecht, was rein, was liebenswert ist, was wohltuend ist, ob eine Tugend, ob ein Lob -- diese Dinge bedenkt.}
\verstab{9}{ESRA}{ Was ihr auch gelernt und übernommen und gehört und an mir gesehen habt, das tut, und der Gott des Friedens wird mit euch sein.}

\verstab{10}{ESRA}{Ich habe mich im Herrn hoch gefreut, dass ihr endlich wieder aufgeblüht seid, an mich zu denken; woran ihr zwar dachtet, aber ihr hattet keine Gelegenheit.}
\verstab{11}{ESRA}{Nicht dass ich das aufgrund von Mangel sage, denn ich habe gelernt, worin ich bin, genügsam zu sein.}
\verstab{12}{ESRA}{Ich weiss erniedrigt zu sein, und ich weiss übrig zu haben. In jedes und in alles bin ich eingeweiht: satt sein und hungern, übrig haben und Mangel leiden.}
\verstab{13}{ESRA}{Alles vermag ich durch den, der mich \es{fortwährend} kräftigt.}
\verstab{14}{ESRA}{Und doch, ihr habt gut \es{daran} getan, an meiner Bedrängnis Anteil zu nehmen.}

\verstab{15}{ESRA}{Ihr wisst auch selbst \es{liebe} Philipper, dass im Anfang \es{der Verkündigung} des Evangeliums, als ich wegzog, von Mazedonien, keine Gemeinde Gemeinschaft hatte mit mir im  \es{gegenseitigen} Geben und Empfangen als nur ihr allein.}
\verstab{16}{ESRA}{Nämlich auch in Thessalonich schicktet ihr mir einmal, sogar zweimal \es{etwas} für meinen Bedarf.}
\verstab{17}{ESRA}{Nicht dass ich die Gabe suche, sondern ich suche die sich für eure Rechnung mehrende Frucht.}
\verstab{18}{ESRA}{Ich habe alles erhalten und habe übrig; ich bin erfüllt, nachdem ich von Epaphroditus die \es{Gabe} von euch empfangen habe, einen lieblichen Geruch. Ein willkommenes Opfer, Gott wohgefällig.}
\verstab{19}{ESRA}{Mein Gott aber wird all euren Bedarf erüllen nach seinem Reichtum in Herrlichkeit in Christus Jesus.}
\verstab{20}{ESRA}{Unserem Gott und Vater sei die Herrlichkeit in alle Ewigkeit! Amen.}

\verstab{21}{ESRA}{Grüsst jeden Heiligen in Jesus, dem Gesalbten.}
\verstab{22}{ESRA}{Alle Heiligen Grüssen euch, am meisten die dem Haus des Kaisers.}

\verstab{23}{ESRA}{Die Gnade des Herrn Jesus, des Gesalbten, \es{sei} mit eurem Geist!}