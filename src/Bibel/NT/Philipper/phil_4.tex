\subsection*{Kapitel 4}
\addcontentsline{toc}{subsection}{Kapitel 4}
\verstab{1}{ESRA}{\bindB{Daher} meine geliebten \bindV{und} ersehnten Brüder, meine Freude \bindV{und} mein Siegeskranz: Auf diese Weise \verbI{steht} \es{\verbI{fest}} im \person{Herrn}, \person{Geliebte}!}

\verstab{2}{ESRA}{\person{Evodia} \verbN{rufe} ich \verbN{auf}, \bindV{und} \person{Syntyche} \verbN{rufe} ich \verbN{auf}, das Gleiche zu \verbN{sinnen} im \person{Herrn}.}
\verstab{3}{ESRA}{Ja, ich \verbN{bitte} \bind{auch} dich, echter \person{Jochgenosse}, \verbI{stehe} ihnen \verbI{bei}, die im Evangelium mit mir \verbN{gekämpft haben}, \bind{samt} \person{Clemens} \bindV{und} meinen übrigen \person{Mitarbeitern}, deren Namen im Buch des Lebens \es{\verbN{stehen}}.}

\verstab{4}{ESRA}{\verbI{Freut} euch im \person{Herrn} allezeit! Nochmals \verbN{will} ich \verbN{sagen}: \verbI{Freut} euch!}
\verstab{5}{ESRA}{Eure Milde \verbI{werde} allen Menschen \verbI{bekannt}! Der Herr verbN{ist} nahe.}
\verstab{6}{ESRA}{\verbI{Macht} euch um nichts Sorgen, \bindB{sondern} in allem \verbI{sollen} eure Bitten durch Gebet \bindV{und} Flehen mit Danksagung vor Gott \verbI{kundwerden},}
\verstab{7}{ESRA}{\bindV{und} der alles Denken überragende Friede Gottes \verbN{wird} eure Herzen \bindV{und} eure Gedanken in Gewahrsam \verbN{halten} in \person{Jesus, dem Gesalbten}.}

\verstab{8}{ESRA}{\bindA{Des Weiteren}, Brüder, alles, was wahr, was ehrbar, was gerecht, was rein, was liebenswert \verbN{ist}, was wohltuend \verbN{ist}, \bindB{ob} eine Tugend, \bindB{ob} ein Lob -- diese Dinge \verbI{bedenkt}.}

\verstab{9}{ESRA}{ Was ihr \bindA{auch} \verbN{gelernt} \bindV{und} \verbN{übernommen} \bindV{und} \verbN{gehört} \bindV{und} an mir \verbN{gesehen habt}, das tut, \bindV{und} der Gott des Friedens \verbN{wird} mit euch \verbN{sein}.}

\verstab{10}{ESRA}{Ich \verbN{habe} mich im \person{Herrn} hoch \verbN{gefreut}, \bindB{dass} ihr endlich wieder \verbN{aufgeblüht seid}, an mich zu \verbN{denken}; woran ihr zwar \verbN{dachtet}, \bindB{aber} ihr \verbN{hattet} keine Gelegenheit.}
\verstab{11}{ESRA}{Nicht \bindB{dass} ich das aufgrund von Mangel \verbN{sage}, \bindB{denn} ich \verbN{habe gelernt}, worin ich \verbN{bin}, genügsam zu \verbN{sein}.}
\verstab{12}{ESRA}{Ich \verbN{weiss} erniedrigt zu \verbN{sein}, \bindV{und} ich \verbN{weiss} übrig zu \verbN{haben}. In jedes \bindV{und} in alles \verbN{bin} ich \verbN{eingeweiht}: \verbN{satt sein} \bindV{und} \verbN{hungern}, übrig \verbN{haben} \bindV{und} Mangel \verbN{leiden}.}
\verstab{13}{ESRA}{Alles \verbN{vermag} ich durch den, der mich \es{fortwährend} \verbN{kräftigt}.}
\verstab{14}{ESRA}{\bindV{Und} \bindB{doch}, ihr \verbN{habt} \verbN{gut} \es{daran} \verbN{getan}, an meiner Bedrängnis Anteil \verbN{zu nehmen}.}

\verstab{15}{ESRA}{Ihr \verbN{wisst} \bindB{auch} selbst \es{liebe} \person{Philipper}, \bindB{dass} im Anfang \es{der Verkündigung} des Evangeliums, \bindB{als} ich wegzog, von \ort{Mazedonien}, keine Gemeinde Gemeinschaft hatte mit mir im  \es{gegenseitigen} Geben \bindV{und} Empfangen als nur ihr allein.}
\verstab{16}{ESRA}{\bindB{Nämlich} \bindB{auch} in \ort{Thessalonich} \verbN{schicktet} ihr mir einmal, sogar zweimal \es{etwas} für meinen Bedarf.}
\verstab{17}{ESRA}{Nicht \bindB{dass} ich die Gabe \verbN{suche}, \bindB{sondern} ich \verbN{suche} die sich für eure Rechnung mehrende Frucht.}
\verstab{18}{ESRA}{Ich \verbN{habe} alles \verbN{erhalten} \bindV{und} \verbN{habe übrig}; ich \verbN{bin erfüllt}, \bindB{nachdem} ich von \person{Epaphroditus} die \es{Gabe} von euch \verbN{empfangen habe}, einen lieblichen Geruch. Ein willkommenes Opfer, \person{Gott} wohgefällig.}
\verstab{19}{ESRA}{Mein \person{Gott} \bindA{aber} \verbN{wird} all euren Bedarf \verbN{erüllen} nach seinem Reichtum in Herrlichkeit in \person{Christus Jesus}.}
\verstab{20}{ESRA}{Unserem \person{Gott} \bindV{und} \person{Vater} \verbI{sei} die Herrlichkeit in alle Ewigkeit! Amen.}

\verstab{21}{ESRA}{\verbI{Grüsst} jeden \person{Heiligen} in \person{Jesus, dem Gesalbten}. Die Brüder, die bei mir \verbN{sind} \verbN{grüssen} euch.}
\verstab{22}{ESRA}{Alle \person{Heiligen} \verbN{grüssen} euch, am meisten die dem \ort{Haus des Kaisers}.}

\verstab{23}{ESRA}{Die Gnade des \person{Herrn} \person{Jesus, des Gesalbten}, \es{\verbI{sei}} mit eurem Geist!}