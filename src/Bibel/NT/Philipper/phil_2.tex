\subsection*{Kapitel 2}
\addcontentsline{toc}{subsection}{Kapitel 2}
\verstab{1}{ESRA}{
    \bindA{Wenn} es \bindA{also} \es{so ist}, \bindA{dass} es Ermunterung \verbN{gibt} in \person{Christus}, \bindA{wenn} Zuspruch der Liebe, \bindA{wenn} Gemeinschaft des Geistes, \bindA{wenn} inniges Mitgefühl \bindV{und} Erbarmungen,}
\verstab{2}{ESRA}{
    \bindA{dann} \verbI{macht} meine Freude \es{damit} \verbI{voll}, \bindA{dass} ihr \es{auf} das Gleiche \verbN{sinnt}, \bindB{indem} ihr dieselbe Liebe \verbN{habt}, in einer Seele \verbN{verbunden seid} \bindV{und} \bindB{indem} ihr auf eines \verbN{sinnt},} 
\verstab{3}{ESRA}{
    \bindB{indem} ihr nichts aus Eigennutz \bindV{oder} leerer Ruhmsucht \es{\verbN{tut}}, \bindA{sondern} in der Demut einer den anderen für \verbN{höher hält} als sich selbst,} 
\verstab{4}{ESRA}{
    \bindB{indem} ein jeder nicht auf das Seine \verbN{schaut}, \bindA{sondern} ein jeder \bindV{auch} auf das der anderen. }
\verstab{5}{ESRA}{
    Unter euch \verbI{sei} diese Gesinnung, die \bindV{auch} in \person{Jesus, dem Gesalbten}, \verbN{war},} 
\verstab{6}{ESRA}{
    der, \bindB{obwohl} in Gestalt \person{Gottes} \verbN{seiend}, das Gott Gleichsein nicht wie eine Beute \verbN{ansah},} 
\verstab{7}{ESRA}{
    \bindA{sondern} sich selbst \verbN{entäusserte}, \bindB{indem} er die Gestalt eines Knechtes \verbN{annahm}. Den \person{Menschen} \verbN{gleich geworden} \bindV{und} in der äusseren Erscheinung wie ein \person{Mensch} \verbN{erfunden},} 
\verstab{8}{ESRA}{
    \verbN{erniedrigte} er sich selbst, \bindB{indem} er \verbN{gehorsahm wurde} bis zum Tod, zum Tod an einem Kreuz.} 
\verstab{9}{ESRA}{
    \bindB{Darum} \verbN{erhöhte} \person{Gott} ihn \bindV{auch} über \es{alles} \bindV{und} \verbN{gab} ihm den Namen, der über jeden Namen \verbN{ist}.} 
\verstab{10}{ESRA}{
    \bindA{damit} im Namen \person{Jesu} sich \verbN{beuge} jedes Knie, \es{der} Himmlischen der Irdischen \bindV{und} Unterirdischen,} 
\verstab{11}{ESRA}{
    \bindV{und} jede Zunge \verbN{bekenne}, \bindA{dass} \person{Jesus, der Gesalbte}, Herr \verbN{ist}, zur Verherrlichung \person{Gottes}, des Vaters.}

\verstab{12}{ESRA}{
    \bindB{So denn}, meine \person{Geliebte}, wie ihr allezeit \verbN{gehorcht habt}, nicht nur wie in meiner Anwesenheit, \bindA{sondern} jetzt vielmehr in meiner Abwesenheit, \verbI{bringt} euer eigenes Heil \verbI{hervor} mit Furcht \bindV{und} Zittern;} 
\verstab{13}{ESRA}{
    \bindB{denn} \person{Gott} \verbN{ist} der in euch Wirkende \bindA{sowohl} das Wollen als \bindV{auch} das Wirken  \es{seines} Wohlgefallens wegen.}

\verstab{14}{ESRA}{
    \verbI{Tut} alles ohne Murren \bindV{und} Bedenken,} 
\verstab{15}{ESRA}{
    \bindA{damit} ihr \verbN{untadelig} \bindV{und} \verbN{unverfälscht werdet}, \person{Kinder Gottes} ohne Makel inmitten eines krummen \bindV{und} verdrehten \person{Geschlechts}, unter dem ihr \verbN{aufscheint} wie Lichter in der Welt,} 
\verstab{16}{ESRA}{
    \bindB{indem} ihr \verbN{festhaltet} das Wort des Lebens, mir zum \es{Gegenstand des} Rühmens auf den Tag Christi, \bindA{weil} ich \es{dann} nicht vergeblich \verbN{gelaufen bin}, noch \bindV{auch} vergeblich \verbN{gearbeitet \verbN{habe}}.} 
\verstab{17}{ESRA}{
    \bindA{Wenn} ich aber \bindV{auch} \es{als Gussopfer} \verbN{ausgegossen werde} über das Opfer \bindV{und} den Priesterdienst für euren Glauben, \verbN{freue} ich mich mit euch allen.} 
\verstab{18}{ESRA}{
    \bindA{Ebenso} \verbI{freut} \bindV{auch} ihr euch \bindV{und} \verbI{freut} euch zusammen mit mir.}

\verstab{19}{ESRA}{
    Ich \verbN{hoffe} aber in dem \person{Herrn Jesus}, \person{Timotheus} bald zu euch zu \verbN{senden}, \bindV{damit} \bindV{auch} ich frohgemut sei, \bindA{wenn} ich eure Umstände erfahre.}
 
\verstab{20}{ESRA}{
    Ich \verbN{habe} nämlich niemand gleichgesinnt, der in echter Weise für das eure \verbN{besorgt sein wird};} 
 \verstab{21}{ESRA}{
    \bindB{denn} alle \verbN{suchen} das Eigene, nicht das, \es{was} \person{Jesu Christi} \es{\verbN{ist}}.} 
 \verstab{22}{ESRA}{
    \bindB{Aber} seine Bewährtheit \verbN{kennt} ihr, \bindA{dass} er wie ein \person{Kind} dem \person{Vater} zusammen mit mir \verbN{gedient hat} im Evangelium.} 
 \verstab{23}{ESRA}{
    Diesen \bindA{also} \verbN{hoffe} ich, sofort zu \verbN{schicken}, \bindA{sobald} ich \verbN{absehe} wie es um mich \verbN{steht}.} 
 \verstab{24}{ESRA}{
    \bindB{Doch} ich \verbN{bin zuversichtlich} im Herrn, \bindA{dass} \bindV{auch} ich selbst bald \verbN{kommen werden}.} 
 \verstab{25}{ESRA}{
    Ich \verbN{hielt} es \bindA{aber} für notwendig, \person{Epaphroditus} meinen Bruder \bindV{und} \person{Mitarbeiter} \bindV{und} \person{Mitkämpfer}, \bindA{aber} euren \person{Abgesandten} \bindV{und} \person{Diener} meines Bedarfs, zu euch zu \verbN{schicken},} 
 \verstab{26}{ESRA}{
    \bindV{da} er sich nach euch allen \verbN{sehnte} \bindV{und} in Unruhe \verbN{war}, \bindB{weil} ihr \verbN{gehört hattet}, \bindA{dass} er \verbN{erkrankt war}, dem Tod nahe. \bindA{Doch} \person{Gott} \verbN{erbarmte} sich über ihn, \bindV{und} nicht nur über ihn, \bindV{sondern auch} über mich, damit ich nicht Kummer über Kummer \verbN{bekäme}.} 
 \verstab{27}{ESRA}{
    \bindA{Also} \verbN{habe} ich ihn \es{umso} eiliger \verbN{geschickt}, damit ihr, \bindA{wenn} ihr ihn \verbN{seht}, wieder \verbN{froh werdet} \bindV{und} ich weniger \verbN{bekümmert sei}.} 
 \verstab{28}{ESRA}{
    \verbI{Nehmt} ihn \bindA{also} \verbI{auf} im \person{Herrn} mit aller Freude, \bindV{und} \verbI{haltet} solche in Ehren;} 
 \verstab{29}{ESRA}{
    \bindA{denn} wegen des Werkes Christi \verbN{kam} er dem Tod nahe, \bindB{indem} er sein Leben \verbN{gering achtete}, \bindA{um} euren Mangel im Dienst für mich \verbN{aufzufüllen}.}