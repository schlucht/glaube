\subsection*{Kapitel 18}
\addcontentsline{toc}{subsection}{Kapitel 18}
\verstab{1}{Schlachter}{Zu jener Stunde traten die Jünger zu Jesus und sprachen: Wer ist wohl der Größte im Reich der Himmel?}
\verstab{2}{Schlachter}{Und Jesus rief ein Kind herbei, stellte es in ihre Mitte}
\verstab{3}{Schlachter}{und sprach: Wahrlich, ich sage euch: Wenn ihr nicht umkehrt und werdet wie die Kinder, so werdet ihr nicht in das Reich der Himmel kommen!}
\verstab{4}{Schlachter}{Wer nun sich selbst erniedrigt wie dieses Kind, der ist der Größte im Reich der Himmel.}
\verstab{5}{Schlachter}{Und wer ein solches Kind in meinem Namen aufnimmt, der nimmt mich auf.}
\verstab{6}{Schlachter}{Wer aber einem von diesen Kleinen, die an mich glauben, Anstoß [zur Sünde] gibt, für den wäre es besser, dass ein großer Mühlstein an seinen Hals gehängt und er in die Tiefe des Meeres versenkt würde.}
\verstab{7}{Schlachter}{Wehe der Welt wegen der Anstöße [zur Sünde]! Denn es ist zwar notwendig, dass die Anstöße [zur Sünde] kommen, aber wehe jenem Menschen, durch den der Anstoß [zur Sünde] kommt!}
\verstab{8}{Schlachter}{Wenn aber deine Hand oder dein Fuß für dich ein Anstoß [zur Sünde] wird, so haue sie ab und wirf sie von dir! Es ist besser für dich, dass du lahm oder verstümmelt in das Leben eingehst, als dass du zwei Hände oder zwei Füße hast und in das ewige Feuer geworfen wirst.}
\verstab{9}{Schlachter}{Und wenn dein Auge für dich ein Anstoß [zur Sünde] wird, so reiß es aus und wirf es von dir! Es ist besser für dich, dass du einäugig in das Leben eingehst, als dass du zwei Augen hast und in das höllische Feuer geworfen wirst.}
\verstab{10}{Schlachter}{Seht zu, dass ihr keinen dieser Kleinen verachtet! Denn ich sage euch: Ihre Engel im Himmel schauen allezeit das Angesicht meines Vaters im Himmel.}
\verstab{11}{Schlachter}{Denn der Sohn des Menschen ist gekommen, um das Verlorene zu retten.}
\verstab{12}{Schlachter}{Was meint ihr? Wenn ein Mensch hundert Schafe hat, und es verirrt sich eines von ihnen, lässt er nicht die neunundneunzig auf den Bergen, geht hin und sucht das verirrte?}
\verstab{13}{Schlachter}{Und wenn es geschieht, dass er es findet, wahrlich, ich sage euch: Er freut sich darüber mehr als über die neunundneunzig, die nicht verirrt waren.}
\verstab{14}{Schlachter}{So ist es auch nicht der Wille eures Vaters im Himmel, dass eines dieser Kleinen verlorengeht.}
\verstab{15}{Schlachter}{Wenn aber dein Bruder an dir gesündigt hat, so geh hin und weise ihn zurecht unter vier Augen. Hört er auf dich, so hast du deinen Bruder gewonnen.}
\verstab{16}{Schlachter}{Hört er aber nicht, so nimm noch einen oder zwei mit dir, damit jede Sache auf der Aussage von zwei oder drei Zeugen beruht.}
\verstab{17}{Schlachter}{Hört er aber auf diese nicht, so sage es der Gemeinde. Hört er aber auch auf die Gemeinde nicht, so sei er für dich wie ein Heide und ein Zöllner.}
\verstab{18}{Schlachter}{Wahrlich, ich sage euch: Was ihr auf Erden binden werdet, das wird im Himmel gebunden sein, und was ihr auf Erden lösen werdet, das wird im Himmel gelöst sein.}
\verstab{19}{Schlachter}{Weiter sage ich euch: Wenn zwei von euch auf Erden übereinkommen über irgendeine Sache, für die sie bitten wollen, so soll sie ihnen zuteilwerden von meinem Vater im Himmel.}
\verstab{20}{Schlachter}{Denn wo zwei oder drei in meinem Namen versammelt sind, da bin ich in ihrer Mitte.}
\verstab{21}{Schlachter}{Da trat Petrus zu ihm und sprach: Herr, wie oft soll ich meinem Bruder vergeben, der gegen mich sündigt? Bis siebenmal?}
\verstab{22}{Schlachter}{Jesus antwortete ihm: Ich sage dir, nicht bis siebenmal, sondern bis siebzigmalsiebenmal!}
\verstab{23}{Schlachter}{Darum gleicht das Reich der Himmel einem König, der mit seinen Knechten abrechnen wollte.}
\verstab{24}{Schlachter}{Und als er anfing abzurechnen, wurde einer vor ihn gebracht, der war 10 000 Talente schuldig.}
\verstab{25}{Schlachter}{Weil er aber nicht bezahlen konnte, befahl sein Herr, ihn und seine Frau und seine Kinder und alles, was er hatte, zu verkaufen und so zu bezahlen.}
\verstab{26}{Schlachter}{Da warf sich der Knecht nieder, huldigte ihm und sprach: Herr, habe Geduld mit mir, so will ich dir alles bezahlen!}
\verstab{27}{Schlachter}{Da erbarmte sich der Herr über diesen Knecht, gab ihn frei und erließ ihm die Schuld.}
\verstab{28}{Schlachter}{Als aber dieser Knecht hinausging, fand er einen Mitknecht, der war ihm 100 Denare schuldig; den ergriff er, würgte ihn und sprach: Bezahle mir, was du schuldig bist!}
\verstab{29}{Schlachter}{Da warf sich ihm sein Mitknecht zu Füßen, bat ihn und sprach: Habe Geduld mit mir, so will ich dir alles bezahlen!}
\verstab{30}{Schlachter}{Er aber wollte nicht, sondern ging hin und warf ihn ins Gefängnis, bis er bezahlt hätte, was er schuldig war.}
\verstab{31}{Schlachter}{Als aber seine Mitknechte sahen, was geschehen war, wurden sie sehr betrübt, kamen und berichteten ihrem Herrn den ganzen Vorfall.}
\verstab{32}{Schlachter}{Da ließ sein Herr ihn kommen und sprach zu ihm: Du böser Knecht! Jene ganze Schuld habe ich dir erlassen, weil du mich batest;}
\verstab{33}{Schlachter}{solltest denn nicht auch du dich über deinen Mitknecht erbarmen, wie ich mich über dich erbarmt habe?}
\verstab{34}{Schlachter}{Und voll Zorn übergab ihn sein Herr den Folterknechten, bis er alles bezahlt hätte, was er ihm schuldig war.}
\verstab{35}{Schlachter}{So wird auch mein himmlischer Vater euch behandeln, wenn ihr nicht jeder seinem Bruder von Herzen seine Verfehlungen vergebt.}