\subsection*{Kapitel 22}
\addcontentsline{toc}{subsection}{Kapitel 22}
\verstab{1}{Schlachter}{Da begann Jesus und redete wieder in Gleichnissen zu ihnen und sprach:}
\verstab{2}{Schlachter}{Das Reich der Himmel gleicht einem König, der für seinen Sohn das Hochzeitsfest veranstaltete.}
\verstab{3}{Schlachter}{Und er sandte seine Knechte aus, um die Geladenen zur Hochzeit zu rufen; aber sie wollten nicht kommen.}
\verstab{4}{Schlachter}{Da sandte er nochmals andere Knechte und sprach: Sagt den Geladenen: Siehe, meine Mahlzeit habe ich bereitet; meine Ochsen und das Mastvieh sind geschlachtet, und alles ist bereit; kommt zur Hochzeit!}
\verstab{5}{Schlachter}{Sie aber achteten nicht darauf, sondern gingen hin, der eine auf seinen Acker, der andere zu seinem Gewerbe;}
\verstab{6}{Schlachter}{die Übrigen aber ergriffen seine Knechte, misshandelten und töteten sie.}
\verstab{7}{Schlachter}{Als der König das hörte, wurde er zornig, sandte seine Heere aus und brachte diese Mörder um und zündete ihre Stadt an.}
\verstab{8}{Schlachter}{Dann sprach er zu seinen Knechten: Die Hochzeit ist zwar bereit, aber die Geladenen waren nicht würdig.}
\verstab{9}{Schlachter}{Darum geht hin an die Kreuzungen der Straßen und ladet zur Hochzeit ein, so viele ihr findet!}
\verstab{10}{Schlachter}{Und jene Knechte gingen hinaus auf die Straßen und brachten alle zusammen, so viele sie fanden, Böse und Gute, und der Hochzeitssaal wurde voll von Gästen.}
\verstab{11}{Schlachter}{Als aber der König hineinging, um sich die Gäste anzusehen, sah er dort einen Menschen, der kein hochzeitliches Gewand anhatte;}
\verstab{12}{Schlachter}{und er sprach zu ihm: Freund, wie bist du hier hereingekommen und hast doch kein hochzeitliches Gewand an? Er aber verstummte.}
\verstab{13}{Schlachter}{Da sprach der König zu den Dienern: Bindet ihm Hände und Füße, führt ihn weg und werft ihn hinaus in die äußerste Finsternis! Da wird das Heulen und Zähneknirschen sein.}
\verstab{14}{Schlachter}{Denn viele sind berufen, aber wenige sind auserwählt!}
\verstab{15}{Schlachter}{Da gingen die Pharisäer und hielten Rat, wie sie ihn in der Rede fangen könnten.}
\verstab{16}{Schlachter}{Und sie sandten ihre Jünger samt den Herodianern zu ihm, die sprachen: Meister, wir wissen, dass du wahrhaftig bist und den Weg Gottes in Wahrheit lehrst und auf niemand Rücksicht nimmst; denn du siehst die Person der Menschen nicht an.}
\verstab{17}{Schlachter}{Darum sage uns, was meinst du: Ist es erlaubt, dem Kaiser die Steuer zu geben, oder nicht?}
\verstab{18}{Schlachter}{Da aber Jesus ihre Bosheit erkannte, sprach er: Ihr Heuchler, was versucht ihr mich?}
\verstab{19}{Schlachter}{Zeigt mir die Steuermünze! Da reichten sie ihm einen Denar.}
\verstab{20}{Schlachter}{Und er spricht zu ihnen: Wessen ist dieses Bild und die Aufschrift?}
\verstab{21}{Schlachter}{Sie antworteten ihm: Des Kaisers. Da spricht er zu ihnen: So gebt dem Kaiser, was des Kaisers ist, und Gott, was Gottes ist!}
\verstab{22}{Schlachter}{Als sie das hörten, verwunderten sie sich, und sie ließen ab von ihm und gingen davon.}
\verstab{23}{Schlachter}{An jenem Tag traten Sadduzäer zu ihm, die sagen, es gebe keine Auferstehung, und sie fragten ihn}
\verstab{24}{Schlachter}{und sprachen: Meister, Mose hat gesagt: Wenn jemand ohne Kinder stirbt, so soll sein Bruder dessen Frau zur Ehe nehmen und seinem Bruder Nachkommen erwecken.}
\verstab{25}{Schlachter}{Nun waren bei uns sieben Brüder. Der erste heiratete und starb; und weil er keine Nachkommen hatte, hinterließ er seine Frau seinem Bruder.}
\verstab{26}{Schlachter}{Gleicherweise auch der andere und der dritte, bis zum siebten.}
\verstab{27}{Schlachter}{Zuletzt, nach allen, starb auch die Frau.}
\verstab{28}{Schlachter}{Wem von den Sieben wird sie nun in der Auferstehung als Frau angehören? Denn alle haben sie zur Frau gehabt.}
\verstab{29}{Schlachter}{Aber Jesus antwortete und sprach zu ihnen: Ihr irrt, weil ihr weder die Schriften noch die Kraft Gottes kennt.}
\verstab{30}{Schlachter}{Denn in der Auferstehung heiraten sie nicht, noch werden sie verheiratet, sondern sie sind wie die Engel Gottes im Himmel.}
\verstab{31}{Schlachter}{Was aber die Auferstehung der Toten betrifft, habt ihr nicht gelesen, was euch von Gott gesagt ist, der spricht:}
\verstab{32}{Schlachter}{»Ich bin der Gott Abrahams und der Gott Isaaks und der Gott Jakobs«? Gott ist aber nicht ein Gott der Toten, sondern der Lebendigen.}
\verstab{33}{Schlachter}{Und als die Menge dies hörte, erstaunte sie über seine Lehre.}
\verstab{34}{Schlachter}{Als nun die Pharisäer hörten, dass er den Sadduzäern den Mund gestopft hatte, versammelten sie sich;}
\verstab{35}{Schlachter}{und einer von ihnen, ein Gesetzesgelehrter, stellte ihm eine Frage, um ihn zu versuchen, und sprach:}
\verstab{36}{Schlachter}{Meister, welches ist das größte Gebot im Gesetz?}
\verstab{37}{Schlachter}{Und Jesus sprach zu ihm: »Du sollst den Herrn, deinen Gott, lieben mit deinem ganzen Herzen und mit deiner ganzen Seele und mit deinem ganzen Denken«.}
\verstab{38}{Schlachter}{Das ist das erste und größte Gebot.}
\verstab{39}{Schlachter}{Und das zweite ist ihm vergleichbar: »Du sollst deinen Nächsten lieben wie dich selbst«.}
\verstab{40}{Schlachter}{An diesen zwei Geboten hängen das ganze Gesetz und die Propheten.}
\verstab{41}{Schlachter}{Als nun die Pharisäer versammelt waren, fragte sie Jesus}
\verstab{42}{Schlachter}{und sprach: Was denkt ihr von dem Christus? Wessen Sohn ist er? Sie sagten zu ihm: Davids.}
\verstab{43}{Schlachter}{Er spricht zu ihnen: Wieso nennt ihn denn David im Geist »Herr«, indem er spricht:}
\verstab{44}{Schlachter}{»Der Herr hat zu meinem Herrn gesagt: Setze dich zu meiner Rechten, bis ich deine Feinde hinlege als Schemel für deine Füße«?}
\verstab{45}{Schlachter}{Wenn also David ihn Herr nennt, wie kann er dann sein Sohn sein?}
\verstab{46}{Schlachter}{Und niemand konnte ihm ein Wort erwidern. Auch getraute sich von jenem Tag an niemand mehr, ihn zu fragen.}