\subsection*{Kapitel 2}
\addcontentsline{toc}{subsection}{Kapitel 2}
\verstab{1}{ESRA}{Und es geschah, im Monat Nisan des zwanzigsten Jahres des \person{Königs Artasasta} [war] Wein vor ihm. Und ich trug den Wein auf und gab ihn dem \person{König}; und ich war nicht missmutig vor ihm.
}
\verstab{2}{ESRA}{
Doch der \person{König} sagte zu mir: Warum ist dein Gesicht missmutig? Und du bist nicht krank; dies ist nichts anderes als Missmut des Herzens. Da fürchtete ich mich gar sehr.
}
\verstab{3}{ESRA}{
Und ich sagte zum \person{König}: Der \person{König} lebe ewig! Warum sollte mein Gesicht nicht missmutig sein, wo die Stadt, das Haus der Gräber meiner \person{Väter}, verwüstet ist und ihre Tore vom Feuer verzehrt sind.
}
\verstab{4}{ESRA}{
Und der \person{König} sagte zu mir: Worum bittest du denn? Da betete ich zum \person{Gott des Himmels},
}
\verstab{5}{ESRA}{
und ich sagte zum \person{König}: Wenn es für den \person{König} gut ist, und wenn dein \person{Knecht} gut steht vor dir: Du wollest mich nach \ort{Juda} senden, nach der \ort{Stadt der Gräber meiner Väter}, und ich will sie aufbauen.
}
\verstab{6}{ESRA}{
Da sagte der \person{König} zu mir – und die \person{Königin} saß neben ihm –: Bis wann wird deine Reise währen, und wann kehrst du zurück? Und es war gut vor dem \person{König}, und er sandte mich; und ich gab ihm eine Frist an.
}
\verstab{7}{ESRA}{
Und ich sagte zum \person{König}: Wenn es dem \person{König} recht ist, gebe man mir Briefe an die \person{Statthalter} \ort{jenseits des Stromes}, dass sie mich durchreisen lassen, bis ich nach \ort{Juda} komme;
}
\verstab{8}{ESRA}{
und einen Brief an \person{Asaph}, den \person{Hüter des Forstes} des \person{Königs}, dass er mir Holz gebe, um die Tore der Burg, die zum Haus gehört, mit Balken zu bauen, und für die Mauer der Stadt und für das Haus, in das ich ziehen soll. Und der \person{König} gab [es] mir, gemäß der guten Hand meines \person{Gottes} über mir.
}
\verstab{9}{ESRA}{
Und ich kam zu den \person{Statthaltern} \ort{jenseits des Stromes}, und ich gab ihnen die Briefe des \person{Königs}. Und der \person{König} hatte \person{Heeroberste} und \person{Reiter} mit mir gesandt.
}
\verstab{10}{ESRA}{
Und \person{Sanballat}, der Choroniter, hörte [das], und \person{Tobija}, der ammonitische Knecht. Und es erschien ihnen sehr übel, dass ein \person{Mensch} gekommen war, das Wohl der \person{Söhne Israels} zu suchen.
}
\verstab{11}{ESRA}{
Als ich nach \ort{Jerusalem} gekommen und drei Tage dort gewesen war,
}
\verstab{12}{ESRA}{
machte ich mich bei Nacht auf, ich und wenige \person{Männer} mit mir. Und ich hatte keinem \person{Menschen} mitgeteilt, was mein \person{Gott} mir ins Herz gab, für \ort{Jerusalem} zu tun. Und kein Tier war bei mir, außer dem Tier, auf dem ich ritt
}
\verstab{13}{ESRA}{
Und ich zog bei Nacht aus durch das Taltor Richtung \ort{Drachenquelle} und zum \ort{Misttor}. Und ich untersuchte gründlich die Mauern von \ort{Jerusalem}, die niedergerissen waren; und ihre Tore waren vom Feuer verzehrt.
}
\verstab{14}{ESRA}{
Und ich zog hinüber zum Quellentor und zum \person{Königsteich}, und es war kein Raum, um durchzukommen für das Tier unter mir.
}
\verstab{15}{ESRA}{
Und ich stieg bei Nacht das Tal hinauf und prüfte sorgfältig die Mauer. Und ich kehrte wieder durch das Taltor und kam zurück.
}
\verstab{16}{ESRA}{
Und die \person{Vorsteher} wussten nicht, wohin ich gegangen war und was ich tue; denn den Juden und den Priestern und den Edlen und den \person{Vorstehern} und den Übrigen, die das Werk tun sollten, hatte ich bis dahin nichts mitgeteilt.
}
\verstab{17}{ESRA}{
Da sagte ich zu ihnen: Ihr seht das Übel, in dem wir sind, dass \ort{Jerusalem} verwüstet ist und seine Tore mit Feuer verbrannt sind. Auf! Lasst uns die Mauer \ort{Jerusalems} bauen! Und wir werden nicht länger ein [Anlass zur] Schmähung sein.
}
\verstab{18}{ESRA}{
Und ich berichtete ihnen von der Hand meines \person{Gottes}, dass sie gut über mir gewesen war, und auch die Worte des \person{Königs}, die er zu mir geredet hatte. Da sagten sie: Wir wollen uns aufmachen und bauen. Und sie stärkten ihre Hände zum Guten.
}
\verstab{19}{ESRA}{
Und \person{Sanballat}, der  \ort{Choroniter}, hörte es, und \person{Tobija}, der \ort{ammonitische Knecht}, und \person{Geschem}, der  \ort{Araber}. Und sie spotteten über uns und verachteten uns, und sie sagten: Was ist das für eine Sache, die ihr tun wollt? Wollt ihr euch gegen den \person{König} auflehnen?
}
\verstab{20}{ESRA}{
Und ich gab ihnen Antwort und sagte zu ihnen: Der \person{Gott des Himmels}, er wird es uns gelingen lassen. Und wir, seine \person{Knechte}, wollen uns aufmachen und bauen. Doch ihr habt weder Anteil noch Anrecht noch Andenken in \ort{Jerusalem}.
}