\documentclass{../inc/mybibbook}

\title{Gibt es Gott wirklich?}
\author{Ken Ham, Jonathan Sarfati, Carl Wieland}
\date{2011}

\begin{document}

\maketitle
\begin{bibeltext}{ELB}{1Petr}{3:15-16}
    Seid jederzeit bereit zur Verantwortung jedem gegenüber, der Rechenschaft von euch über die Hoffnung in euch fordert, aber mit Sanftmut und Ehrerbietung! Und habt ein guter Gewissen...
\end{bibeltext}
\tableofcontents
\section{Vorwort zur deutschen Auflage}
Dieses Minibuch \cite{keith1} ist das erste Kapitel aus dem Buch \flqq Fragen an den Anfang - Die Logik der Schöpfung\frqq. Es handelt sich um die Übersetzung des Bestsellers \flqq The Answers Book\frqq{} aus dem englischsprachigen Bereich. Die vier Verfasser sind australische Wissenschaftler, die durch zahlreiche Publikationen und Vorträge weit über den eigenen Erdteil hinaus bekannt sind und Grundlegendes zum Thema Schöpfung erforscht haben.

Den Menschen unserer Tage ist durch das Überangebot an evolutionistischen Abhandlungen in Presse, Literatur und Wissenschaft die Botschaft der Bibel abhanden gekommen und damit -- das ist die tragische Folge -- auch das Heil. Der Mensch kennt keine Ewigkeit mehr. Wer das Buch liest, findet nicht nur viele wissenschaftliche Details in allgemeinverständlicher Form aufbereitet, sondern er bekommt ein tieferes Verständnis vom Evangelium und damit auch von der Rettung durch Jesus Christus. Die Autoren haben in einzigartiger Weise herausgearbeitet, in welch starkem Masse das Evangelium mit vielen Schöpfungsdetails verknüpft sind. Dabei kommen sie zu dem wichtigen Ergebnis: Von der Bibel her können wir nichts, aber auch wirklich gar nichts aufgeben, ohne Schaden zu erleiden. Diese wichtige Lehre vermittelt uns das Buch in brillanter Weise und an einer grossen Fülle einsichtiger Beweise.

Wer Vorträge zu dem Thema Schöpfung/Evolution hält, stösst mit Regelmässigkeit auf ein ganzes Paket von Fragen, die sich erfahrungsgemäss stark wiederholen. Einige aus dem Buch \flqq Fragen an den Anfang\frqq{} seien hier schon einmal stellvertretend genannt:\\
Hat Gott die Welt wirklich in sechs Tagen geschaffen? Was kann man über die Radiokarbonmethode sagen? Warum können wir weit entfernte Sterne in einem jungen Universum sehen? Warum gibt es so viele destruktive Merkmale bei Lebewesen? Was ist von den Argumenten für die Evolutionslehre zu halten? Woher kam die Frau Kains? War die Sintflut weltweit? Passten alle Tierarten in Noahs Arche hinein? Wie gelangten die Tiere nach Australien? Wie kam es zu den verschiedenen Rassen der Menschen? Was geschah mit den Dinosaurier?

Der Leser wird zuerst mit den häufigsten Gegenpositionen vertraut gemacht. Danach weisen die Autoren nach, welche Denk- und Glaubenskonsequenzen es hat, wenn man einer gegen die Bibel gerichteten Auffassung folgt. Der Leser ist dann eingeladen, sich dafür zu entscheiden, was Gott gesagt hat, und diesen Weg des Vertrauens zu beschreiten.

Der mit der Bibel bisher nicht Vertraute und dem Glauben noch Fernsehenden findet in kompetenter wissenschaftlicher Weise dargelegt, dass der ganzen Bibel auch im 21. Jahrhundert glauben kann. Das Buch dürfte für diese Personengruppe ein Augenöffner besonderer Art sein. Diejenigen, die sich bereits zum christlichen Glauben bekennen und in Jesus Christus den Retter gefunden haben, werden staunen, welch tiefe Verankerung das Evangelium in Schöpfung und Sündenfall hat.

Das vorliegende Minibuch geht jener ganz grundlegenden Frage nach, ob es Gott überhaupt gibt. Auch schwierige Sachverhalte werden von den Autoren dem Leser in leicht verständlicher Weise nachegebracht. Damit bekommt er einen Vorgeschmack auf das kommende Werk. Beiden Büchern wünsche ich eine weite Verbreitung.

\begin{flushright} Werner Gitt\end{flushright}
\newpage
\section{Gibt es Gott wirklich?}
\textit{Gibt es objektive Beweise für die Existenz Gottes? Welche Konsequenzen hat der Atheismus? Woher kam Gott? Kann man Gott persönlich kennenlernen?}\\
Die Bibel beginnt mit der Aussage: \flqq Am Anfang schuf Gott Himmel und Erde\frqq{}(1. Mose 1). Die Existenz Gottes wird also in der Bibel vorausgesetzt und als selbstverständlich angesehen; dennoch gibt es Menschen, die diese Tatsache ignorieren. Psalm 14,1 spricht von solchen: \flqq Die Toren sprechen in ihrem Herzen: \flq Es ist kein Gott.\frq{} Sie taugen nichts; ihr Treiben ist ein Gräuel; da ist keiner, der gutes tut\frqq.

Hier sehen wir, dass die Bibel abfällige Gedanken über Gott  -- insbesonder das Leugnen seiner Esistenz -- mit einer verdorbenen Moral verbindet. Und es stimmt: Wenn es keinen Gott gibt, keinen Schöpfer, der die Lebensregeln festsetzt, dann treiben wir ohne moralische Orientierung dahin. Als das Volk Israel in der Zeit der Richter seinen Schöpfer vergass und niemand Gott die Treue hielt, regierte das Chaos, denn \flqq jeder tat, was ihn Recht dünkte\frqq{}(Richter 21,25)

In unserer Zeit wiederholt sich genau das selbe.Länder, in denen Gott geehrt wurde und man glaubte, dass \flqq Gott in Christus war und die Welt mit sich selbst versöhnte\frqq{}(2. Korinther 5,19), haben Sicherheit und Wohlstand erfahren, wie es bis dahin nie der Fall war. Doch genau dieselben Länder befinden sich im Niedergang, seit die Menschen Gott den Rücken zukehren. So heisst es schon in Sprüchen 14,34 \flqq Gerechtigkeit erhöht ein Volk; aber die Sünde ist der Leute Verderben.\frqq

Je weiter sich ganze Völker von Gott entfernen und so leben, als gäbe es ihn gar nicht, desto mehr nimmt die Sünde überhand. Es häufen sich politische Korruption, Lügen, Verleumdung, ausschweifende Lebensweise, Gewaltverbrechen, Abtreibung, Diebstahl, Ehebruch, Drogensucht, Alkoholprobleme und Spielsucht. Mit der Wirtschaft geht es bergab, die Steuern steigen, und die Regierungen verschulden sich einnehmend. Ein immer grösseres Polizeiaufgebot, grössere Gefängnisse und grössere soziale Sicherheitssysteme sind erforderlich, um die Probleme auch nur notdürftig in den Griff zu bekommen.

Die Verse 18 bin 32 des ersten Kapitels des Römerbriefes lesen sie wie ein Kommentar zu Welt von heute:
\begin{bibeltext}{HFA}{Rom}{1:18-32}
    Gott lässt aber auch seinen Zorn sichtbar werden. Vom Himmel herab trifft er alle Menschen, die sich gegen Gott und seinen Willen auflehnen. Sie tun, was Gott missfällt, und treten so die Wahrheit mit Füßen. 19 Dabei gibt es vieles, was sie von Gott erkennen können, er selbst hat es ihnen ja vor Augen geführt. 20 Gott ist zwar unsichtbar, doch an seinen Werken, der Schöpfung, haben die Menschen seit jeher seine ewige Macht und göttliche Majestät sehen und erfahren können. Sie haben also keine Entschuldigung. 21 Denn obwohl sie schon immer von Gott wussten, verweigerten sie ihm die Ehre und den Dank, die ihm gebühren. Stattdessen kreisten ihre Gedanken um Belangloses, und da sie so unverständig blieben, wurde es schließlich in ihren Herzen finster. 22 Sie hielten sich für besonders klug und waren die größten Narren. 23 Statt den ewigen Gott in seiner Herrlichkeit anzubeten, verehrten sie Götzenstatuen von sterblichen Menschen, von Vögeln und von vierfüßigen und kriechenden Tieren. 24 Deshalb hat Gott sie all ihren Trieben und schmutzigen Leidenschaften überlassen, so dass sie sogar ihre eigenen Körper entwürdigten. 25 Sie haben die Wahrheit über Gott verdreht und ihrer eigenen Lüge geglaubt. Sie haben die Schöpfung angebetet und ihr gedient und nicht dem Schöpfer. Ihm allein aber gebühren Lob und Ehre bis in alle Ewigkeit. Amen. 26 Weil die Menschen Gottes Wahrheit mit Füßen traten, gab Gott sie ihren Leidenschaften preis, durch die sie sich selbst entehren: Die Frauen haben die natürliche Sexualität aufgegeben und gehen gleichgeschlechtliche Beziehungen ein. 27 Ebenso haben die Männer die natürliche Beziehung zur Frau mit einer unnatürlichen vertauscht: Männer treiben es mit Männern, ohne sich dafür zu schämen, und lassen ihrer Lust freien Lauf. So erfahren sie die gerechte Strafe für ihren Götzendienst am eigenen Leib. 28 Gott war ihnen gleichgültig; sie gaben sich keine Mühe, ihn zu erkennen. Deshalb überlässt Gott sie einer inneren Haltung, die ihr ganzes Leben verdirbt. Und folglich tun sie Dinge, mit denen sie nichts zu tun haben sollten: 29 Sie sind voller Unrecht und Gemeinheit, Habgier, Bosheit und Neid, ja sogar Mord; voller Streit, Hinterlist und Verlogenheit, Klatsch 30 und Verleumdung. Sie hassen Gott, sind gewalttätig, anmaßend und überheblich. Beim Bösen sind sie sehr erfinderisch. Sie weigern sich, auf ihre Eltern zu hören, 31 haben weder Herz noch Verstand, lassen Menschen im Stich und sind erbarmungslos. 32 Dabei wissen sie ganz genau, dass sie nach dem Urteil Gottes dafür den Tod verdient haben. Trotzdem machen sie so weiter wie bisher, ja, sie freuen sich sogar noch, wenn andere es genauso treiben.
\end{bibeltext}
Die Bibel würde viele Leute in Spitzenpositionen in der Regierung und im Bildungssystem der ehemals christlichen Länder als \flqq Narren\frqq{} bezeichnen. Sie behaupten zwar, weise zu sein, da sie aber die Existenz Gottes leugnen sowie sein noch heute erfahrbares Handeln, \flqq sind sie zu Narren geworden\frqq.

Die weitverbreitete Akzeptanz des evolutionistischen Denkens -- die Auffassung, alles sein durch natürliche Prozesse von selbst entstanden und Gott sei nicht nötig -- untermauert zusätzlich diese Preisgabe des Glaubens an Gott. Alles, was offensichtlich nach einem Plan entworfen ist, soll diesen Plan selbst entworfen haben! Dieses Denken, das die klar erkennbaren Indizien für die die Existens Gottes (Römer 1,19-20) wegerklärt, führt unausweichlich zum Atheismus (dem Glauben, dass es keinen Gott gibt) und zum säkularen Humanismus (der Mensch kann über sein Leben selbst bestimmen, und zwar ohne Gott). In den Schulen und Universitäten bis hin zu den Regierungen herrscht heutzutage eine solche Denkweise vor.

Einige der schlimmsten Gräueltaten der Weltgeschichte wurde von Lenin, Hitler, Stalin, Mao Tse-tung und Pol Pot verübt. Wenn sie auch unterschiedlichen Ideologien und Moralauffassungen folgten, so hatten sie doch ein der evolutionistischen Demkweise eine gemeinsame Basis für ihr Handeln. Der Atheist und Vertreter der Evolutionslehre Sir Arthur Keith schreibt über Hitler: 
\begin{quote}  Der deutsche Führer ... hat in konsequenter Weise versucht, das Leben in Deutschland konform zur Evolutionslehre zu gestalten. \cite{keith1} \end{quote}
Viele Millionen haben die Konsequenzen, einer atheistischen Denkweise erleiden müssen; sie haben Schreckliches erlebt oder sind sogar ums Leben gekommen. Atheismus kann tötend wirken, denn ohne Gott ist man auch nicht seinem Gesetz verpflichtet. So ist letztlich alles erlaubt! In den Bestrebungen, Abtreibung, Euthanasie, Rauschgift, Prostitution, Pornographie und Promiskuität zu legalisieren, stehen die Atheisten an vorderster Front. Dies alles verursacht Elend, Leid und Tod. Atheismus ist die Philosophie des Todes.

Nun verweisen Atheisten gern auf die Gräueltaten, die von angeblichen \flqq Christen\frqq{} verübt wurde -- die Kreuzzüge und der Nordirland-Konflikt gehören dabei zu den Favoriten. Falls die Menschen, die so etwas Schreckliches verübt, tatsächlich Christen waren, denn handelten bzw. handeln sie in krassem Widerspruch zu den Massstäben der Gebote Gottes (z.B. \flqq Du sollst nicht töten \frq{}, \flqq Liebet eure Feinde \frq{}).


\printbibliography


\end{document} 