\documentclass[12pt,a4paper]{scrarticle}

\usepackage[headsepline, footsepline]{scrlayer-scrpage}
\usepackage{blindtext}% nur für Fülltext
\usepackage[ngerman]{babel}
\usepackage[utf8]{inputenc}
\usepackage[T1]{fontenc}
\usepackage{lmodern}
\usepackage{blindtext}
\usepackage{graphicx}
\usepackage{xcolor}
\usepackage[most]{tcolorbox}
\usepackage{geometry}
\usepackage[version=4]{mhchem}
\usepackage{bibleref-german}
\usepackage{setspace}
%%%%%%%%%%%%%%%%%%%%%%Link Formatierung%%%%%%%%%%%%%%%%%%%%%%%%%%%
\usepackage[colorlinks = true,
linkcolor = black,
urlcolor  = blue,
citecolor = black,
anchorcolor = blue]{hyperref}
%\documentclass[xcolor=dvipsnames]{beamer}

\pagestyle{scrheadings}

\chead{\headmark}
\automark{section}



\cfoot{Seite \pagemark}



%%%%%%%%%%%%%%%%%%%%%Griechische Zeichen%%%%%%%%%%%%%%%%%%%%%%%%%%%%%%%%%%%
%\begin{otherlanguage}{polutonikogreek} 
%	Δοκεῖ δέ μοι καὶ Καρχηδόνα μὴ εἶναι. 
%\end{otherlanguage} 



%%%%%%%%%%%%%%%%%%%%%%%%%%%%%%%%%%%%%%%%%%%%%%%%%%%%%%%%%%%%%%%%%%
%
% Makro zur Namensvervollstädigung
% Parameter: Kürzel --- Bibel
%			LuN			Neue Luther Übersetzung
%			Sch2		Schlachter 2000
%			Elb			Elbfelder
%%%%%%%%%%%%%%%%%%%%%%%%%%%%%%%%%%%%%%%%%%%%%%%%%%%%%%%%%%%%%%%%%%%
	\newcommand{\bib}[1]{%
	\ifthenelse{\equal{#1}{EI}}{Einheitsübersetzung}{%
		\ifthenelse{\equal{#1}{Sch2}}{Schlachter 2000}{%
			\ifthenelse{\equal{#1}{HFA}}{Hoffnung für Alle}{%
				\ifthenelse{\equal{#1}{ELB}}{Elbfelder}{%
					\ifthenelse{\equal{#1}{Neü}}{neue ev. Übersetzung}{%
						\ifthenelse{\equal{#1}{Lut}}{Luther}{#1}%
					}%
				}%
			}%
		}%
	}%
}%

%%%%%%%%%%%%%%%%%%%%%%%%%%%%%%%%%%%%%%%%%%%%%%%%%%%%%%%%%%%%%%%%%%
%
% Makro für Bibelzitate
% 
% Beispiel: 
%		\begin{bibeltext}{ELB}{Matt}{1:1-4}
%			Ich bin der zitierte Bibeltext.
%		\end{bibeltext}
%1Petr, Matt, Offb, 1Mos, Joh, IThess, 2Thess, Kol, Ps, Jak, ITim
%%%%%%%%%%%%%%%%%%%%%%%%%%%%%%%%%%%%%%%%%%%%%%%%%%%%%%%%%%%%%%%%%%
\newcommand{\bibtit}[1]{\large\bib{#1}}
\newcommand{\tmpbib}{Hallo}
\newcommand{\herr}{HERR}

\newcommand{\lied}[1]{\colorbox{yellow}{\textcolor{blue}{Lied: #1}}}
\newcommand{\green}[1]{\colorbox{green}{\textcolor{black}{#1}}}
\newcommand{\red}[1]{\colorbox{red}{\textcolor{white}{#1}}}
\newcommand{\beten}{\red{Ich möchte noch beten!}}
\newcommand{\hh}[1]{\textsuperscript{#1}}
\newenvironment{bibeltext}[3]{%	
        \renewcommand{\tmpbib}{\textbf{\bibleverse{#2}( #3)}}
		\biblerefformat{lang}
        \begin{tcolorbox}[
        title=\tmpbib \hspace{1.5cm} (\bibtit{#1}),
        colback=black!2!white,
        colframe=black!55!black]        
	}{%	
		\newline		
        \end{tcolorbox}	
	\endquote		
}
% \begin{tcolorbox}[title=Bibeltext,
%     title filled=false,
%     colback=blue!5!white,
%     colframe=blue!75!black]
% \end{tcolorbox}


\title{\includegraphics[height=48pt]{assets/images/logo.png}\\Die Gemeinde}
\author{Lothar Schmid}
\begin{document}
\maketitle
\section{Was ist die Gemeinde?}
\begin{bibeltext}{Sch2}{Kol}{1:18}
    Und er [Jesus] ist das Haupt des Leibes, der Gemeinde, er, der der Anfang ist, der
Erstgeborene aus den Toten, damit er in allen der Erste ist.
\end{bibeltext}
Hier schreibt Paulus, das Jesus, das Haupt des \textbf{Leibes der Gemeinde} ist. Was ist das Haupt des
Leibes? Das Haupt steuert alles, aber nicht nur das, sondern ohne das Haupt stirbt der Körper.
Er ist nicht lebensfähig.

Da Jesus jetzt als das \textbf{Haupt der Gemeinde} bezeichnet wird, heisst das, dass eine Gemeinde
ohne Jesus nicht überlebensfähig ist. Die Gemeinde ist keine kleine Gruppe in Naters, die Gemeinde
ist nicht MNR oder FEG oder einer der anderen tausend Namen. \textbf{\textit{Die Gemeinde ist Jesus.}} Der Erstgeborene aus den Toten. Der neue Adam.

Eine Gemeinde kann noch so tolle Musik machen, extrem gute Prediger aus Zürich einladen,
wenn die Gemeinde das Haupt Jesus nicht hat, so ist sie tot. Mausetot. Das Kirchensterben das
aktuell hier in der westlichen Welt passiert, ist nicht weil die Kirchen zu langweilig sind, sondern
weil Jesus nicht vorhanden ist.
\begin{bibeltext}{Sch2}{Offb}{2:5}
Bedenke nun, wovon du gefallen bist, und tue Busse und tue die ersten Werke! Sonst
komme ich rasch über dich und werde deinen Leuchter von seiner Stelle wegstossen,
wenn du nicht Busse tust!
\end{bibeltext}
Das lässt Jesus an die Gemeinde Ephesus ausrichten. Das ist heute so. Sobald eine Gemeinde von
Jesus wegrückt zerfällt sie. Sie kann erst wieder Frucht bringen, wenn sie Busse tut. Das heisst,
wenn sie wieder Umkehrt und wieder Jesus in ihren Mittelpunkt stellt.
\subsection{Wer gehört zur Gemeinde?}
Ja wer ist denn jetzt in der Gemeinde? Wer macht die Gemeinde aus? Der Pastor? Die Musiker?
Oder vielleicht die, die am meisten Spenden oder die meiste Arbeit machen?

Nein ein Gemeindemitglied oder ein \textbf{Glied am Leib der Gemeinde} ist der, der Jesus Christus
als sein Retter annimmt. Der glaubt, dass er ein Sünder ist und Jesus für seine Sünde am Kreuz
gestorben ist. Wer so im Glauben, Jesus in sein Herz aufnimmt, der gehört zum Leib der Gemeinde
Christi. Egal ob er tätowiert ist, am Sonntag auf dem Sofa liegt, lange Haare hat, ob er viel oder
wenig in der Gemeinde arbeitet, ob er spendet oder nicht. Egal, wenn man Jesus Christus von
herzen aufnimmt, gehört man zur Gemeinde Jesus. Das Übergabegebet, die Taufe oder sonst ein
Ritual kann das nicht, sondern einzig allein der Glaube an den Jesus, der für dich und mich am
Kreuz gestorben ist.

Es muss uns bewusst sein, dass viele verschiedene Charakteren in einer Gemeinde sind. Wir sollen aber alle Mitglieder in der Gemeinde lieben. Lieben heisst nicht anhimmeln, sonder gegenseitig wertschätze, anständig sein, keine Vorurteile bilden, nicht hinter dem Rücken über die Person reden und demütig sein. Man sieht einem Bruder oder Schwester nicht an ob sie wirklich Jesus angenommen hat. Darum lasst uns kein Urteil bilden. Wenn Mitglieder gegen die Gebote verstösst, ohne Reue und Einsicht, kann man diese ansprechen und sie in Liebe auf die Fehler aufmerksam machen. Das ist unsere Pflicht. Liebe heisst nicht, zu allem und alles ja sagen. Es ist keine Liebe wenn man die Person nicht vor dem Abgrund warnt.
Wir sollten auch nicht vergessen, dass wir selber Sünder sind. Darum sagte Jesus

\begin{bibeltext}{Sch2}{Matt}{7:3}
    Was siehst du aber den Splitter im Auge deines Bruders, und den Balken in deinem
Auge bemerkst du nicht?
\end{bibeltext}

Darum immer bescheiden bleiben. Im Wallis gibt es ein Sprichwort: \glqq{}Eine mischt der ander dräck\grqq.

Und weil wir so sind und uns Jesus trotzdem liebt und auserwählt hat, lasst uns ihn loben und preisen mit Liedern, Gebeten aber auch mit unserem Handeln und Auftreten. Diese Welt braucht gute Jesus und gute Christen als Glieder an seinem Leib.
\end{document}
