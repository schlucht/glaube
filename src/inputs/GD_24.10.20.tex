\documentclass{../inc/mybib}

\title{\includegraphics[height=48pt]{assets/images/logo.png}\\Danken}
\author{Lothar Schmid}
\begin{document}
\maketitle
\section{Begrüssung}

Ich möchte euch alle und Norbert herzlich zu diesem Gottesdienst begrüßen.
Schön Norbert, dass du den langen Weg auf dich genommen hast, um uns das Wort auszulegen und zu verkündigen. 
\beten{} und dann singen wir zusammen das Lied \lied{Du bist der Weg und die Wahrheit und das Leben}.

\section{Ankündigungen}
\begin{itemize}
    \item \green{Bibel und Gebetsabend:} Do 24.10.2024 20:00Uhr mit Samuel Rindlisbacher ab Römer 10.16 - 21. Wenn alles klappt wird das der letzte Gebetsabend hier in diesem Lokal sein. Also kommt nochmals alle.
    \item \green{Nächster Gottesdienst:} So 27.10.2024 14:30Uhr mit Thomas Lieth
    \item \green{Info zum neuem Lokal:} Die Arbeiten schreiten voran. Aktuell ist alles noch im Zeitplan und dem 3. November als Eröffnung steht nichts im Wege. Liebe Gäste werden erwartet und das ganze wird mit Profimusikern musikalisch zu einem Highlight werden.
    \begin{quote}
        Neue Adresse:
        Furkastrasse 46
        Ortsbus-Haltestelle: Amerika
    \end{quote}
    Wir werden auch die Gottesdienst-Zeiten anpassen. Neu wird der GD um 15Uhr sein. Der Grund ist, dass es manchmal hektisch ist und wir zu wenig Zeit vor dem GD haben, um uns mit dem Gast aus Zürich auszutauschen und beten zu können. Auch können wir unser Mittagsschläfchen ein bisschen ausdehnen.
    \item Die Kollekte, die wir hier sammeln geht an den MNR und wird dort für die Weltweite Mission eingesetzt.    
\end{itemize}

\section{ Input }
\begin{spacing}{1.5}
\subsection{Danken}\\
Letztes mal habe ich erzählt wie man sein Leben glücklich gestalten kann, auch wenn es einem nicht so gut geht. Das Glück hängt nicht am Reichtum oder an der Gesundheit. Es ist zwar schön, wenn man gesund ist und sich keine Sorgen um Essen machen muss, aber zum glücklich sein hilft uns, wenn wir nach dem Gesetz Gottes leben. Dazu habe ich euch exemplarisch die ersten Verse von Psalm 119 vorgelesen.
\begin{bibeltext}{ELB}{Ps}{119:1-3}
1 Glücklich sind, die im Weg untadelig sind, die im Gesetz des Herrn wandeln\\
2 Glücklich sind, die seine Zeugnisse bewahren, die ihn vom ganzen Herzen suchen\\
3 Die auch kein Unrecht tun, die auf seinen Wegen wandeln.
\end{bibeltext}
Es gibt aber noch andere Aspekte, die helfen ein glückliches Leben zu führen. Einer davon ist die Dankbarkeit. Früher bekam ich zu Weihnachten von der Grossmutter, den Geschwistern oder von Verwandten oft Rasierwasser als Geschenk. Da habe ich brav danke gesagt und dabei gedacht, was mache ich jetzt bloss mit dem ganzen Zeugs? Ich habe jetzt noch zwei Flaschen davon; die sind jetzt sicher gegen 30 Jahre alt.

Ich denke, wir haben verlernt wirklich Dankbar zu sein und richtig Danke zu sagen. In der freievangelikalen Szene wird das Danken im Gebet sehr gross geschrieben. Es gibt Beter, die jedes Anliegen mit Danke anfangen. Ja wir sollen dem Herrn danken. Gottes Wort ermahnt uns auch immer wieder zu Danken und Dankbar zu sein.
\begin{bibeltext}{Lut}{IThess}{5:16-18}
Freut euch allezeit! Betet immerzu! Sagt Gott in allem Dank! Das ist es, was Gott will, und was er euch durch Christus Jesus möglich macht.
\end{bibeltext}
\begin{bibeltext}{Lut}{Kol}{3:17}
Überhaupt alles, was ihr tut und sagt, sollt ihr im Namen des Herrn Jesus tun und durch ihn Gott, dem Vater, danken!
\end{bibeltext}
\begin{bibeltext}{Lut}{Kol}{4:2}
Seid treu, ausdauernd und wach im Gebet und im Dank an Gott!
\end{bibeltext}

Auch David dankt Gott in den Psalmen (7.18, 9.2, 18.50, 28.7) und ruft uns auf Gott zu danken (in Psalm 30.5, 67.6, 92.2, 100.4, 105.1 ,um ein paar Stellen zu nennen).\\
Wenn ich meiner Grossmutter für die 5 Flaschen Rasierwasser gedankt habe, hoffe ich doch, dass sie den Dank angenommen hat, auch wenn ich mich in meinem Innern nicht unbedingt darüber gefreut habe. Es gibt doch den Spruch: "Der Wille macht`s aus!". Bei meiner Grossmutter könnte das funktioniert haben. Bei Gott funktioniert das nicht. Manchmal leiern wir ein Tischgebet runter, danken Gott für die tollen Speisen, aber denken im Innern \textcolor{green}{ICH} habe dafür gearbeitet und \textcolor{green}{ICH} habe es mit meinem Geld gekauft. Gott weiss ja schon im voraus was wir beten und kennt unser innerstes besser als wir selber.\\ Unser Wohlstand und Reichtum steht uns mehr im Weg um Gott zu preisen und zu ehren, als wir es uns eingestehen wollen.\\ Hätte ich nur alle 5 Jahre Rasierwasser bekommen, wäre ich wirklich dankbarer gewesen. Aber mit 3-5  Flaschen pro Weihnachten?\\ Echte, innige Dankbarkeit macht Glücklich. Wir Leben hier und alle die wir Jesus Christus in unser Herz aufgenommen haben, haben das Leben in Ewigkeit geschenkt bekommen. Und glaubt mir, dass nächste Leben wird unbeschreiblich und wir werden erkennen wie mickrig unsere Dankbarkeit in diesem Leben war. Darum lasst uns unsere Dankgebete nicht einfach nur runter leiern, sondern von Herzen und ehrlich oder wie Luther sagt mit Inbrunst beten. David drückt seine Dankbarkeit in Psalm 100 aus. Lasst uns diesen zusammen als Gebet lesen.

Danach singen wir das \lied{Danke, für diesen guten Morgen } und ich geben das Wort an Norbert weiter.

\end{spacing}

\section{Predigt}
Die wird von Norbert verkündigt.\\

\textbf{Nach der Predigt}\\

Danke für die Predigt.\\

Zum Schluss singen wir das Lied \lied{Herr du gibst uns Hoffnung}

\section{Abschluss}
Vielen Dank für eure Teilnahme und das Gebet. Im Anschluss seid ihr zu Kaffee und guten Gesprächen eingeladen.
Ich will \beten{} und dann mit dem Segen aus Epheser 6.24 abschliessen.
\begin{bibeltext}{Sch2}{Eph}{6:24}
Die Gnade sei mit allen, die unseren Herrn Jesus Christus lieb haben mit unvergänglicher Liebe. Amen
\end{bibeltext}
Maranatha Amen
\end{document}