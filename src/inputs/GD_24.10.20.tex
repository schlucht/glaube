\documentclass[12pt,a4paper]{scrarticle}

\usepackage[headsepline, footsepline]{scrlayer-scrpage}
\usepackage{blindtext}% nur für Fülltext
\usepackage[ngerman]{babel}
\usepackage[utf8]{inputenc}
\usepackage[T1]{fontenc}
\usepackage{lmodern}
\usepackage{blindtext}
\usepackage{graphicx}
\usepackage{xcolor}
\usepackage[most]{tcolorbox}
\usepackage{geometry}
\usepackage[version=4]{mhchem}
\usepackage{bibleref-german}
\usepackage{setspace}
%%%%%%%%%%%%%%%%%%%%%%Link Formatierung%%%%%%%%%%%%%%%%%%%%%%%%%%%
\usepackage[colorlinks = true,
linkcolor = black,
urlcolor  = blue,
citecolor = black,
anchorcolor = blue]{hyperref}
%\documentclass[xcolor=dvipsnames]{beamer}

\pagestyle{scrheadings}

\chead{\headmark}
\automark{section}



\cfoot{Seite \pagemark}



%%%%%%%%%%%%%%%%%%%%%Griechische Zeichen%%%%%%%%%%%%%%%%%%%%%%%%%%%%%%%%%%%
%\begin{otherlanguage}{polutonikogreek} 
%	Δοκεῖ δέ μοι καὶ Καρχηδόνα μὴ εἶναι. 
%\end{otherlanguage} 



%%%%%%%%%%%%%%%%%%%%%%%%%%%%%%%%%%%%%%%%%%%%%%%%%%%%%%%%%%%%%%%%%%
%
% Makro zur Namensvervollstädigung
% Parameter: Kürzel --- Bibel
%			LuN			Neue Luther Übersetzung
%			Sch2		Schlachter 2000
%			Elb			Elbfelder
%%%%%%%%%%%%%%%%%%%%%%%%%%%%%%%%%%%%%%%%%%%%%%%%%%%%%%%%%%%%%%%%%%%
	\newcommand{\bib}[1]{%
	\ifthenelse{\equal{#1}{EI}}{Einheitsübersetzung}{%
		\ifthenelse{\equal{#1}{Sch2}}{Schlachter 2000}{%
			\ifthenelse{\equal{#1}{HFA}}{Hoffnung für Alle}{%
				\ifthenelse{\equal{#1}{ELB}}{Elbfelder}{%
					\ifthenelse{\equal{#1}{Neü}}{neue ev. Übersetzung}{%
						\ifthenelse{\equal{#1}{Lut}}{Luther}{#1}%
					}%
				}%
			}%
		}%
	}%
}%

%%%%%%%%%%%%%%%%%%%%%%%%%%%%%%%%%%%%%%%%%%%%%%%%%%%%%%%%%%%%%%%%%%
%
% Makro für Bibelzitate
% 
% Beispiel: 
%		\begin{bibeltext}{ELB}{Matt}{1:1-4}
%			Ich bin der zitierte Bibeltext.
%		\end{bibeltext}
%1Petr, Matt, Offb, 1Mos, Joh, IThess, 2Thess, Kol, Ps, Jak, ITim
%%%%%%%%%%%%%%%%%%%%%%%%%%%%%%%%%%%%%%%%%%%%%%%%%%%%%%%%%%%%%%%%%%
\newcommand{\bibtit}[1]{\large\bib{#1}}
\newcommand{\tmpbib}{Hallo}
\newcommand{\herr}{HERR}

\newcommand{\lied}[1]{\colorbox{yellow}{\textcolor{blue}{Lied: #1}}}
\newcommand{\green}[1]{\colorbox{green}{\textcolor{black}{#1}}}
\newcommand{\red}[1]{\colorbox{red}{\textcolor{white}{#1}}}
\newcommand{\beten}{\red{Ich möchte noch beten!}}
\newcommand{\hh}[1]{\textsuperscript{#1}}
\newenvironment{bibeltext}[3]{%	
        \renewcommand{\tmpbib}{\textbf{\bibleverse{#2}( #3)}}
		\biblerefformat{lang}
        \begin{tcolorbox}[
        title=\tmpbib \hspace{1.5cm} (\bibtit{#1}),
        colback=black!2!white,
        colframe=black!55!black]        
	}{%	
		\newline		
        \end{tcolorbox}	
	\endquote		
}
% \begin{tcolorbox}[title=Bibeltext,
%     title filled=false,
%     colback=blue!5!white,
%     colframe=blue!75!black]
% \end{tcolorbox}


\title{\includegraphics[height=48pt]{assets/images/logo.png}\\Danken}
\author{Lothar Schmid}
\begin{document}
\maketitle
\section{Begrüssung}

Ich möchte euch alle und Norbert herzlich zu diesem Gottesdienst begrüßen.
Schön Norbert, dass du den langen Weg auf dich genommen hast. 
\beten{} und dann singen wir zusammen das Lied \lied{Danke, für diesen guten Morgen} zusammen.

\section{Ankündigungen}
\begin{itemize}
    \item Donnerstag 10.10.2024 ist Bibel und Gebetsabend mit Samuel Rindlisbacher ab Römer 10.16 - 21. Wird wohl der letzte Gebetsabend hier in diesem Lokal sein.
    \item Nächster Gottesdienst: 27.10.2024 mit Thomas Lieth
    \item Info zum neuem Lokal. Die Arbeiten schreiten voran. Aktuell ist alles noch im Zeitplan und dem 3. November als Eröffnung steht nichts im Wege. Es werden noch Vorhänge und Teppich montiert, was alles dann ein bisschen leiser und heimeliger wirken lässt.
    \begin{quote}
        Neue Adresse:
        Furkastrasse 46
        Ortsbus-Haltestelle: Amerika
    \end{quote}
    Wir werden auch die Gottesdienst Zeiten anpassen. Neu wird der GD ab 15Uhr sein. Der Grund ist, dass es manchmal hektisch ist und wir zuwenig Zeit haben, vor dem GDmit dem Gast aus Zürich uns auszutauschen und zu beten. Auch können wir unser Mittagschläfchen ein bisschen ausdehnen.
    \item Die Kollekte die wir hier sammeln geht an den MNR und wird dort für die Weltweite Mission eingesetzt.    
\end{itemize}

\section{ Input }
\begin{spacing}{1.5}
\subsection{Danken}\\
Letztes mal habe ich erzählt wie man sein Leben glücklich gestalten kann, auch wenn es einem nicht so gut geht. Das Glück hängt nicht am Reichtum oder an der Gesundheit. Es ist schön wenn man die Gesundheit und hat und sich keine Sorgen machen muss ob ich heute was zu Essen kriege oder nicht. Da habe ich euch die ersten Verse von Psalm 119 vorgelesen.
\begin{bibeltext}{ELB}{Ps}{119:1-3}
1 Glücklich sind, die im Weg untadelig sind, die im Gesetz des Herrn wandeln\\
2 Glücklich sind, die seine Zeugnisse bewahren, die ihn vom ganzen Herzen suchen\\
3 Die auch kein Unrecht tun, die auf seinen Wegen wandeln.
\end{bibeltext}
Es gibt aber noch ein anderen Aspekt um glücklich zu sein. Das ist die Dankbarkeit. Früher bekam ich zu Weihnachten von der Grossmutter, Geschwister, Verwandten oft Rasierwasser als Geschenk. Da habe ich brav danke gesagt und dabei gedacht, was mache ich jetzt bloss mit dem ganzen Zeugs? Ich habe jetzt noch zwei Flaschen von dem Zeugs, das jetzt sicher gegen 30 Jahre alt ist.

Ich glaube wir haben verlernt wirklich Dankbar zu sein und richtig Danke zu sagen. In der Freievangelikalen Szene wird das Danken im Gebet sehr gross geschrieben. Es gibt Beter die jedes Anliegen mit Danke anfangen. Ja wir sollen dem Herrn Danken. Gottes Wort ermahnt uns auch immer wieder zu Danken und Dankbar zu sein.
\begin{bibeltext}{Lut}{IThess}{5:16-18}
Freut euch allezeit! Betet immerzu! Sagt Gott in allem Dank! Das ist es, was Gott will, und was er euch durch Christus Jesus möglich macht.
\end{bibeltext}
\begin{bibeltext}{Lut}{Kol}{3:17}
Überhaupt alles, was ihr tut und sagt, sollt ihr im Namen des Herrn Jesus tun und durch ihn Gott, dem Vater, danken!
\end{bibeltext}
\begin{bibeltext}{Lut}{Kol}{4:2}
Seid treu, ausdauernd und wach im Gebet und im Dank an Gott!
\end{bibeltext}

Auch David dankt Gott in den Psalmen (7.18, 9.2, 18.50, 28.7) und ruft uns auf Gott zu danken in (30.5, 67.6, 92.2, 100.4, 105.1) um ein paar Stellen zu nennen.\\
Wenn ich meiner Grossmutter für die 5 Flasche Rasierwasser gedankt habe, hoffe ich doch dass sie den Dank angenommen hat, auch wenn ich mich im innern nicht unbedingt darüber gefreut habe. Es gibt doch der Spruch: "Der Wille machts aus!". Bei meiner Grossmutter könnte das funktioniert haben. Bei Gott funktioniert das nicht. Manchmal leiern wir ein Tischgebet runter, danken Gott für die tollen Speisen, aber denken im Innern \textcolor{yellow}{ICH} dafür gearbeitet und \textcolor{yellow}{ICH} habe es gekauft. Bei Gott funktioniert das nicht. Gott weiss ja schon im voraus was wir beten und kennst unser innerstes besser als wir selber.\\ Unser Wohlstand und Reichtum steht uns mehr im Weg Gott preisen und ehren zu können als wir merken.\\ Hätte ich nur alle 5Jahre Rasierwasser bekommen wäre ich wirklich dankbar gewesen. Aber mit 3 Flaschen pro Weihnachten?\\ Echte, innige Dankbarkeit macht Glücklich. Wir Leben hier und alles die wir Jesus Christus in unser Herz aufgenommen haben, haben das Leben in Ewigkeit. Und glaubt mir das nächste Leben wird unbeschreiblich und wir werden erkennen wie mickrig unsere Dankbarkeit in diesem Leben war.

Dazu passt der \textcolor{magenta}{Psalm} 100 und möchte diesen euch gerne vorlesen. Danach singen wir das \lied{Du bist der Weg und die Wahrheit und das Leben} und geben das Wort an Norbert weiter.

\end{spacing}
Darum singen wir jetzt unser \textbf{2. Lied} 

Nach dem Lied möchte ich Norbert bitten nach vorne zu kommen und uns das Wort des Herrn weiter zu geben.

\section{Predigt}
Die wird von Norbert verkündigt.\\

\textbf{Nach der Predigt}\\

Danke für die Predigt.\\

Zum Schluss singen wir das Lied \lied{Herr du gibst uns Hoffnung}

\section{Abschluss}
Vielen Dank für eure Teilnahme und das Gebet. Im Anschluss lade ich euch zu Kaffee und guten Gesprächen ein.
Ich will \beten{} und dann mit dem Segen aus 1. Moses 28.15 abschliessen.
\begin{bibeltext}{Sch2}{1Mos}{28:15}
Gott spricht: Siehe, ich bin mit dir,
ich behüte dich, wohin du auch gehst.
Denn ich verlasse dich nicht,
bis ich vollbringe, was ich dir versprochen habe.
\end{bibeltext}
Maranatha Amen
\end{document}