
\section{Begrüssung}

Ich möchte euch alle recht herzlich zu diesem besonderen Gottesdienst begrüssen. Schön das so viele gekommen sind, um unserem Herrn zu Danken ihn zu Loben und zu Preisen.

Besonders möchte ich unseren Prediger Thomas Lieth begrüssen. Thomas du bist sicher froh mal wieder Sonne geniessen zu können. Wir werden deine Worte geniessen und versuchen daraus zu lernen und Gottes Wort auch umzusetzen.

\noindent
\beten{} und anschliessend singen wir zusammen das Lied

\noindent
\lied{Wunderbar grosser Erlöser}.

\section{Ankündigungen}
\begin{itemize}
    \item \green{Bibel und Gebetsabend:} Do 21.11.2024 20:00Uhr mit Elia Morris ab Römer 11 17-21
    \item \green{Nächster Gottesdienst:} So 24.11.2024 15:00Uhr mit Florian Lehmann
    \item \green{Diverses:} In der Cafeteria liegt eine Kuchenliste, wer will darf sich gerne da Eintragen.
    \item \green{Die Kollekte:} Die Kollekte geht an den Mitternachtsruf und wird dort für die Missionsarbeit in der Welt eingesetzt.
\end{itemize}

\section{ Input }
\begin{spacing}{1.5}
\subsection{Lügen}

\begin{bibelbox}{Neü}{Jak}{3:5}
So ist auch die Zunge nur ein kleines Glied und kann sich doch großer Wirkungen rühmen. Und ein kleines Feuer steckt einen großen Wald in Brand.
\end{bibelbox}
\begin{bibelbox}{Neü}{Spr}{4:24}
Entferne Unwarheit aus deinem Mund, die Falschheit von deinen Lippen.
\end{bibelbox}

Was haben wir bis jetzt zum Thema glücklicheres und zufriedeneres Leben gehört. Nicht alleine Reichtum und Gesundheit macht glücklich. Wenn es so wäre, würden ja keine gesunden Menschen sich von einer Brücke stürzen oder kranke Menschen trotzdem eine Lebensfreude ausstrahlen. Da muss noch was anderes sein. Man ist auch nicht automatisch glücklich, nur weil man Jesus Christus von Herzen aufgenommen und mit ihm unterwegs ist.\\
Hilfsbereitschaft, Dankbarkeit können Glücksgefühle und eine Zufriedenheit hervorrufen. Das haben wir alle schon erlebt. Darüber habe ich schon gesprochen.

Heute will ich ein anderes Thema ansprechen und das ist Ehrlichkeit. Ich weiss nicht wie es euch geht, aber wenn ich jemanden anlüge, habe ich danach ein schlechtes Gewissen. Jedes mal wenn ich mit dieser Person zusammen bin, habe ich das Gefühl, dass man mir meine Lüge ansieht. Das ist sehr unangenehm und man fängt an die Person zu meiden. Wenn ich meinen Partner angelogen habe, kann es die Partnerschaft sehr belasten.\\
Wenn ich gegenüber Freunden unehrlich war, Sachen erzählt habe um besser dazustehen, fühlte ich mich nie Wohl. Wie schnell ist unbedacht eine Lüge ausgesprochen? Manchmal beruhigen wir uns damit, dass wir diese Lüge als Notlüge bezeichnen. Aber auch wenn ich dem Regen, Sonne sage, werde ich trotzdem nass.

Das Sprichwort Lügen haben kurze Beine, stimmt doch oder? Oft wird man relativ schnell ertappt. Aber nicht nur das. Es zermürbt einem das Herz. Es ist sehr anstrengend wenn man in der Angst leben muss, dass die Lügen aufgedeckt werden.

Auch dann als ich Jesus nicht kannte, hat sich mein Gewissen wenn ich gelogen habe, angemeldet. Ich hatte aber oft nicht den Mut die Lügen einzugestehen. Jetzt werde ich oft schon von meine inneren gewarnt, bevor ich die Lüge ausspreche. Man lebt einfach entspannter ohne Lügen. Auch wenn es im Augenblick so aussieht, dass jetzt eine Lüge helfen würde, passiert gewöhnlich das Gengenteil.

Jede nicht verarbeitete Unwahrheit macht einem das Leben ein bisschen schwerer. Die Wahrheit kann einem das Leben ein bisschen leichter machen.
Alle Menschen lügen. Es gibt aber einer der nie lügt und noch nie gelogen hat, unser Herr Jesus Christus. Auf ihn können wir unser ganzes vertrauen setzen.
\begin{bibelbox}{Neü}{Lev}{23:19}
Gott ist ja kein Mensch, der lügt, kein Menschensohn, der etwas bereut. Wenn er etwas sagt, dann tut er es auch, und was er verspricht, das hält er gewiss.
\end{bibelbox}
Darum dürfen wir unser Wissen und Hilfe unbedenklich aus der Bibel, dem Wort Gottes, nehmen. Da ist keine einzige Lüge in diesem Buch.
Am Anfang habe ich gesagt, dass die Aufnahme Jesus Christus in dein Herz, dein Leben nicht automatisch glücklich macht, aber wenn du Jesus folgst, hast eine Person neben dir, der dir hilft und dich trägt.

\end{spacing}

\lied{Wir haben eine Hoffnung}

\section{Predigt}
\green{Schriftlesung}

Danach gebe ich das Wort an Thomas weiter.

\textbf{Nach der Predigt}

Danken für die Predigt.


\section{Abschluss}
Hinten habt ihr die Möglichkeit euch mit Bücher einzudecken. Auf dem Tisch sind die Neuerscheinungen frisch aus der Presse.

Inzwischen ist das NAVI-Gottes eingetroffen. Dieses Buch hilft einem die Bibel in einzelne Heilszeiten einzuteilen und so die Geschichte Gottes besser zu verstehen. Für die Bibelschüler nächstes Jahr ist dieses Buch eine Pflichtlektüre.

Jetzt wollen wir Gott mit dem Lied \lied{Halleluja lobet Gott in seinem Heiligtum} danken.

Vielen Dank für eure Teilnahme und das Gebet. Im Anschluss seid ihr zu Kaffee und guten Gesprächen eingeladen.
\beten{} 


Der Friede Gottes, der höher ist als alle Vernunft,

bewahre eure Herzen und Sinne in Christus Jesus.

Maranatha Amen
