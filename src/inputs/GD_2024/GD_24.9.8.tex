
\section{Begrüssung}

Ich möchte euch alle herzlich zu diesem Gottesdienst begrüßen.\\
Lasst uns beten und dann das erste Lied singen.
\textbf{GEBET}\\

\textbf{1. Lied}
\textit{O Gott dir sei Ehre der Grosses getan}

\section{Ankündigungen}
\begin{itemize}
    \item Gebetsabend und Bibelstunde hier um 20:00 Uhr
    \item nächsten Sonntag kein Gottesdienst. Wir fahren zum Freundestreffen von MNR Bern nach Bern. Der Zug fährt um 10:48 Uhr ab Brig. Das Treffen findet um 13:00 Uhr in Bern statt.
\end{itemize}

\section{ Input }
\begin{spacing}{1.5}
Die letzte Woche bin ich am Morgen mit dem Fahrrad zu Arbeit gefahren. Mein Rennrad ist ein leichtes Karbonrad knapp 10kg, dünne 28mm breite Reifen. Jetzt ist es am Morgen dunkel, also montiere ich ein Licht und dann geht's los. So nach 400 Meter kommt das steilste Stück, wo man so um die 60km/h auf dem Tacho hat. \\
Wieso fahre ich da so schnell? Ich vertraue meinem Fahrrad, dass es dies aushält. Dieses Vertrauen kommt nicht ich einfach so. Dieses Vertrauen hat sich über die Jahre mit vielen tausend Kilometer und Abfahrten aufgebaut. \\
Das gleiche gilt mit unseren Mitmenschen. Bevor wir einer Person vollständig vertrauen, müssen wir sie zuerst kennen lernen. Vertraut man zu früh kann es böse Überraschungen geben. Habe das selber erlebt und es hat mich viel Geld und Nerven gekostet.  \\
Das gleiche ist mit Gott. Um Gott vollständig vertrauen zu können, musste ich ihn kennen lernen. Wie lerne ich jetzt, jemanden kennen, der nicht sichtbar ist, und nicht direkt mit mir redet? Mein Fahrrad redet auch nicht mit mir, es brauchte auch eine gewisse Zeit bis ich ohne Sorgen, den Nuvenen Pass mit fast 100km/h runter fahren durfte. Es braucht also Zeit und Übung. Ausserdem habe ich am Fernseher gesehen wie das die Profis machten. Das hat mir Sicherheit gegeben. \\
Gott lernen wir durch die Bibel kennen. In der Bibel steht auch, dass wir im Bedingungslos vertrauen können.
\begin{bibelbox}{SCHL}{Joh}{14:1}
Euer Herz erschrecke nicht! Glaubt an Gott und glaubt an mich!
\end{bibelbox}
Für mich war das Lesen von Biographien von Gläubigen Menschen sehr hilfreich. Was die alles erlebt haben nur weil sie 100\% Gott vertraut haben, ist unglaublich. Das hat mich gestärkt.
Schon im alten Testament ermutigt Gott die Israeliten ihm zu vertrauen. Aber das Vertrauen hielt nicht lange. Schon kurz nach dem Auszug von Ägypten haben sie lieber ihren eigenen Gott gebastelt. \\
Auch wir stehen in Gefahr unsere eigenen Götter zu basteln. Gott hat auf dieser Welt alles im Griff. Es fällt kein Haar von unserem Kopf, ohne dass er es weiss. Alles was geschieht ist Gottes Wille. Wir dürfen ihm vertrauen. Wir brauchen uns keine Sorgen um unsere kleine Gemeinde zu machen. Gott hat seinen Plan. Lasst uns einfach sein Werkzeug sein und freudig gespannt sein, wohin die Reise führt.
Ich möchte euch den Psalm 104 vorlesen, der, wie ich finde, dieses Vertrauen ganz gut beschreibt.

\end{spacing}
Wir singen jetzt unser \textbf{2. Lied} \textit{Lob und Dank}

\section{Predigt}
Wer am 18. August hier im Gottesdienst war konnte eine Auslegung von Wolfgang Nestvogel über den 1. Johannesbrief hören. Er hat aufgezeigt, dass schon damals, knapp 60 Jahre nach Jesus kommen als Mensch auf dieser Welt, an der Richtigkeit des Evangeliums gezweifelt wurde. Johannes schrieb dann in seinem Brief: wir haben gesehen und bezeugen und verkündigen euch das ewige Leben, das bei dem Vater und uns offenbart worden ist. 1.Joh 2b.\\
Noch heute dürfen wir hier Johannes 100\% vertrauen, dass es so war. Er war Augenzeuge. Heute hören wir uns den zweiten Teil an und Wolfgang Predigt über die Versen 1, 5 -10. Der Titel der Predigt lautet: Vorsicht Selbsttäuschung.

\textbf{Nach der Predigt}

Danke für die Predigt

Nach dem Beten singen wir unser \textbf{3. Lied}: \textit{Bleiben ist deine treu}\\

\section{Abschluss}

\begin{bibelbox}{SCHL}{1Mos}{28:15}
Gott spricht: Siehe, ich bin mit dir,
ich behüte dich, wohin du auch gehst.
Denn ich verlasse dich nicht,
bis ich vollbringe, was ich dir versprochen habe.
\end{bibelbox}
Und was hat er versprochen? Das ewige Leben!
Amen
Ich wünsche euch einen schönen restlichen Tag und eine gute Woche, bis Donnerstag..
