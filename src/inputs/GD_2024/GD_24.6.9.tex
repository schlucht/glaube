

\section{Begrüssung}
Ich möchte euch alle, Fredy und Jean-Lucca herzlich zu diesem Gottesdienst Begrüßen. Schön das ihr diese Reise auf euch genommen habt, um uns das Wort zu bringen. \\Lasst uns beten und dann das erste Lied, Jesus höchster Name singen.

\textbf{1. Lied}
\textit{Das höchste meines Lebens}

\section{Ankündigungen}
\begin{itemize}
    \item Nächster Bibel und Gebetsabend am Donnerstag: 
13.06.2024 Römer 8 23-25 Fredy Peter
    \item Nächster Gottesdienst: 16.06.2024 
\end{itemize}

\section{ Input }

Wir sind doch alles gewissenhafte Christen, die täglich Bibel lesen und dauerhaft beten. Oder? Weil, das wird ja von uns Christen verlangt. Die Bibel ist voll von Ratschlägen zum Lesen der Bibel. 
(Psalm 1 1-2):
\begin{bibelbox}{SCHL}{Ps}{1:2}
Wohl dem der nicht wandelt nach dem Rat der Gottlosen, noch tritt auf den Weg der Sünder, noch sitzt, wo die Spötter sitzen, sondern seine Lust hat am Gesetz des Herrn und über sein Gesetz nachsinnt Tag und Nacht.
\end{bibelbox}
(Hebräer 4 12):
\begin{bibelbox}{SCHL}{Hebr}{4:12}
Denn das Wort Gottes ist lebendig und wirksam und schärfer als jedes zweischneidige Schwert, und es dringt durch, bis es scheidet sowohl Seele als auch Geist, sowohl Mark als auch Bein, \textbf{und es ist ein Richter der Gedanken und der Gesinnung des Herzens.}
\end{bibelbox}
(Psalm 119 105):
\begin{bibelbox}{SCHL}{Ps}{119:105}
Dein Wort ist meines Fusses Leuchte und ein Licht auf meinem Weg.
\end{bibelbox}
(Jakobus 1 22):
\begin{bibelbox}{SCHL}{Jak}{1:22}
Seid aber Täter des Wortes und nicht bloss Hörer, die sich selbst betrügen.
\end{bibelbox}
und noch viele mehr.

Ich war schon als Kind ein Viel-Leser. Ich bin stundenlang im Bett gelegen und habe Bücher gelesen. Laut meiner Mutter konnte ich schon sehr früh lesen, aber nur leise oder in Gedanken, nicht aber laut vorlesen. Auch als Jugendlicher war ich dauernd am Lesen. Sehr zum Leidwesen meines Vaters, der als Bergbauer es nicht gerade gerne sah, dass sein Sohn faul auf dem Bett lag und las.

Ich bekam, als ich 12 oder so war, diese Bibel vom Pfarrer Pfaffen Josef von Naters geschenkt. Ich war ein fleissiger Messdiener und wollte immer Priester werden. Diese Bibel habe ich dann auch einmal als Kind und dann später nochmals mit über 30 gelesen. Ich glaube, ich kam nicht bis zum Ende des Alten Testamentes hinaus. Ich weiss noch als ich, als  junger Mann und nicht gläubig an den Herrn Jesus, diese Bibel gelesen habe, mir hat dieses Buch nur zeigte wie blutgierig und brutal dieser Gott, der darin gezeigt wird, ist. Der liess ganze Völker vernichten, liess Kinder und Frauen töten. Und das soll ein liebender Gott sein?

Später, nach dem ich Jesus kennen lernen durfte und Gott mir die Gnade geschenkt hat ein Glied seiner Gemeinde sein zu dürfen, habe ich die Bibel nochmals gelesen. Ich verstand vieles nicht, aber ich sah vieles in einem anderen Licht. Die Bibel war immer noch die gleiche, es gab immer noch Kriege, Mord und Totschlag. Aber es gab auch noch etwas anderes, es gab da einen gerechten und gnädigen Gott. Gott hat die Vernichtung eines Volkes erst zugelassen, als die Vollzahl der Sünden erreicht wurden und dazu wartete er mehrere Jahrhunderte. Solche Gedanken hatte ich plötzlich.

Es war so als hätte sich ein Vorhang aufgetan. Bin ich über Nacht schlauer geworden? Wohl kaum! Ich schreibe das dem hl. Geist zu, den ich, als ich Jesus als meinen Retter angenommen habe, bekommen habe.\\
Das war für mich wie dieses Sprachen reden, welche die Jünger an Pfingsten erhalten haben. Ich verstand plötzlich viel mehr über Gott.

Bei einem Gespräch mit einem Christen hat mir dieser erzählt, er lese nur das Neue Testament, der Gott im Alten Testament sei nicht der gleiche. Es könne nicht sein, dass ein Gott wie im Neuen Testament beschrieben ist, solche Sachen wie das Töten der Menschen, die um das goldene Kalb getanzt sind, anordnet. Hmm... da frage ich mich, hat er den hl. Geist? 

Jetzt, ein paar Jahre später, habe ich immer mehr dazugelernt. Inzwischen habe ich die Bibel schon mehrmals von vorne nach hinten durchgelesen. Es ist unglaublich, wie stimmig die Bibel über diese 2500 Jahren geschrieben wurde. Unmöglich für Menschen. Ich finde es aber wichtig, dass man nicht nur Verse liest, sondern die Bibel wie ein Roman von vorne nach hinten durchliest. Nur wenn man auch das Alte Testament kennt, kann man vieles im Neuen Testament verstehen und nur wenn man das Neue Testament kennt, sieht man im Alten Testament, wo Jesus drin steckt. Wie will man die Aussage von Jesus verstehen: \glqq{}Es ist vollbracht\grqq ohne das Alte Testamant zu lesen? Was ist denn vollbracht? Der alte Bund ist vollbracht, er stirbt für unsere Erlösung am Kreuz. Er ist der neue Adam, der alte Adam ist tot. Jetzt liegt es allein an uns, diesen neuen Adam anzunehmen, nur wenn wir ihn annehmen, das Alte sterben lassen und durch ihn, mit einem neuen Gewand, wieder auferstehen, kriegen wir den wahren hl. Geist und können Jesus unseren \herr{} verstehen.
Darum singen wir jetzt das Lied \textbf{Das Höchste meines Lebens}

\textbf{2. Lied} \textit{Das Höchste meines Lebens}

- Oswald nimmt die Schriftlesung vor und betet vor der Predigt

\section{Predigt}
Die wird von Fredy Peter verkündigt.
\\
\\
\textbf{Nach der Predigt}\\
Es wäre schön wenn ein Bruder oder Schwester kurz für Susanne, die Frau von Fredy Peter beten könnte. 
Sie befindet sich jetzt in der Chemo-Behandlung.

Und ein zweiter Teilnehmer für unsere kleine Gemeinde.

Nach dem Beten singen wir unser \textbf{3. Lied}: Du sollst nicht m\"ude werden (Wir werden sein wie die Tr\̈"aumenden)

\section{Abschluss}
(Psalm 25 20-21):
\begin{bibelbox}{SCHL}{Ps}{25:20-21}
Bewahre meine Seele und rette mich! Lass mich nicht zuschanden werden, denn ich vertraue auf dich! Lauterkeit und Redlichkeit, mögen mich behüten, denn auf dich harre ich.
Maranatha, Herr komme bald.
\end{bibelbox}
