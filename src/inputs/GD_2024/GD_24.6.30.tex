
\section{Begrüssung}
Ich möchte euch alle und Eberhard herzlich zu diesem Gottesdienst begrüßen. Schön, dass du diese Reise auf dich genommen hast, um uns das Wort zu bringen. \\Lasst uns beten und dann das erste Lied singen: Halleluja, Halleluja, singt und jubelt.

\textbf{1. Lied}
\textit{Halleluja, Halleluja, singt und jubelt}

\section{Ankündigungen}
\begin{itemize}
    \item Nächster Bibel und Gebetsabend am Donnerstag: 4.7.2024 Römer 8,31-34
    \item Nächster Gottesdienst: 7.7.2024 mit Abendmahl mit Samuel Rindlisbacher
    \item Allgemein: Letzten Sonntag haben wir einen Flyer abgeben, für das Gemeindefest am 18. August der MNR Bern. Da habe ich fälschlicherweise gesagt, dass wir da gemeinsam hin gehen. Ich habe den Flyer nicht genau gelesen und dachte der Flyer würde vom Freundestreffen handeln. Wer will kann aber gerne am 18.8. an dem Gemeindefest der MNR Bern teilnehmen. Wir werden aber nicht geschlossen hingehen. An diesem Sonntag Nachmittag findet wie normal ein Gottesdienst hier in diesem Raum statt.
\end{itemize}

\section{ Input }
\begin{spacing}{1.5}
Aktuell läuft ja die Fussball-EM. Überall wurden öffentliche Bildschirme aufgestellt, wo die Fans zusammen kommen, um die Spiele zu gucken. 
In der EM sind 24 Fussballmannschaften aus ebenso vielen Ländern. Länder mit unterschiedlichen Kulturen, Ansichten und Weltanschauungen. Der Südländer ist anders gestrickt, als der Mann oder Frau aus dem hohen Norden. Trotzdem reden beim Fußball alle über das gleiche. Es werden von allen die gleichen Regeln akzeptiert. Die Regeln wurden von einem Vorstand erstellt und für alle verbindlich in einem Reglement aufgeschrieben. Diese Regeln werden angewendet und wenn nötig anhand dieser Regeln auch Strafen verteilt. Es wird zwar über die Regeln diskutiert und man kann dafür oder dagegen sein, aber die Mannschaften ordnen sich diesen unter.

Wie würde jetzt so ein Turnier mit Christen ablaufen? Jede Mannschaft hat seine eigenen Regeln. Die katholische Mannschaft hat ihre Regeln, die evangelische Mannschaft hat ihre Regeln, dann kommen die konservativen Puritaner, die FEG, der Mitternachtsruf und viele mehr. Die verschiedenen Mannschaften kommen zusammen, es wird lang und breit diskutiert welche Regeln die besten sind. Alle sagen: \glqq Meine sind die besten. Wenn ihr nicht nach meinen Regeln spielt, dann machen wir ein eigenes Turnier. Ihr seid keine echten Spieler.\grqq{} Jeder baut sein eigenes Stadion und spielt für sich alleine und schaut immer skeptisch auf die anderen, ob diese es auch wirklich richtig machen.
Ich habe jetzt etwas überspitzt formuliert, aber mir kommt es manchmal so vor.
Paulus formuliert es in 1.Kor 1,12 so: 
\begin{bibelbox}{SCHL}{ICor}{1:12}
Ich rede aber davon, dass jeder von euch sagt: Ich gehöre zu Paulus! - Ich aber zu Apollos! - Ich aber zu Kephas! - Ich aber zu Christus!
\end{bibelbox}
Es gibt aber auch ein klares Reglement für Christen. 2. Timotheus 3.16-17
\begin{bibelbox}{SCHL}{IITim}{3:16-17}
Alle Schrift ist von Gott eingegeben und nützlich zur Belehrung, zur Überführung, zur Zurechtweisung, zur Erziehung in der Gerechtigkeit, damit der Mensch Gottes ganz zubereitet sei, zu jedem guten Werke völlig ausgerüstet.
\end{bibelbox}
Es gibt einen Erfinder und Gründer der Gemeinde Math 16.18. 
\begin{bibelbox}{SCHL}{Mat}{16:18}
Und ich sage dir auch: Du bist Petrus und auf diesen Felsen will ich \textbf{meine} Gemeinde bauen,...
\end{bibelbox}
Da sagt Jesus \textbf{ICH will meine} Gemeinde bauen.
Und wer darf dann da alles mitspielen? Das steht in  Galater 3.28
\begin{bibelbox}{SCHL}{Gal}{3:28}
Da ist weder Jude noch Grieche, da ist weder Knecht noch Freier, da ist weder Mann noch Frau; denn ihr seid alle einer in Christus Jesus.
\end{bibelbox}
also alle und wer macht die Regeln? Gal 1.12.
\begin{bibelbox}{SCHL}{Gal}{1:12}
ich habe es auch nicht von einem Menschen empfangen noch erlernt, sondern durch die Offenbarung Jesu Christi.
\end{bibelbox}
Es gibt Weltweit auch nur einen Chef für unseren Verein. Kolosser 1.18
\begin{bibelbox}{SCHL}{Kol}{1:18}
Und er (Christus Jesus) ist das Haupt der Gemeinde...
\end{bibelbox}
Lasst uns also eine Gemeinde nach den Regeln der Bibel und nach Gottes Plan leben. 
Nicht die Prediger auf YouTube, nicht die Regierung oder irgendein selbsternannter Prophet ist der Maßstab, sondern nur das Wort Gottes. Und wie wissen wir nun wie die Regeln heissen? Indem wir Bibellesen und beten. 
\end{spacing}{1.5}
So wir singen jetzt unser \textbf{2. Lied} \textit{Herr füll mich neu}

- Oswald nimmt die Schriftlesung vor und betet vor der Predigt

\section{Predigt}
- Die wird von Eberhard Hanisch verkündigt.

\textbf{Nach der Predigt}

Danke für die Predigt

Nach dem Beten singen wir unser \textbf{3. Lied}: \textit{Herr weil mich fest hält.}\\

\section{Abschluss}

\begin{bibelbox}{SCHL}{Eph}{6:24}
Die Gnade sei mit allen, die unseren Herrn Jesus Christus lieb haben mit unvergänglicher Liebe. Amen
\end{bibelbox}

Ich wünsche euch einen schönen restlichen Tag und eine gute Woche, bis Donnerstag..
