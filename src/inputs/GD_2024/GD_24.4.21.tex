\section{Begrüssung}
Ich möchte euch alle und Thomas herzlich zu diesem Gottesdienst Begrüßen. Lasst uns beten und dann das erste Lied singen.
\\
\textbf{1. Lied}
\\ 
\\
\section{Ankündigungen}
Nächster Bibel und Gebetsabend am Donnerstag: 25.4.2024
Gottesdienst: 28.04.2024

Wir haben neu eine E-Mail Adresse: mnr-naters@jagolo.ch. Wer intresse hat, kann mich gerne nach dem Gottesdienst kontaktieren.

\section{Input Kaleb}
Kennt ihr Kaleb?
Von Kaleb wird nur wenig in der Bibel erwähnt. Trotzdem ist er ein Beispiel von Mut und Treue in der Nachfolge Gottes. Kaleb wurde aus erwählt, um für den Stamm Juda das von Gott versprochene Land zu erkunden. Alle waren sich einig, dass das Land gut war, in dem Milch und Honig fliesst. Sie fürchteten sich aber vor den Einwohner des Landes. Die 10 Fürsten, der anderen Stämme, die an der Erkundung mit dabei waren, rebellierten und das Volk hörte auf sie. In diese Rebellion hinein sprach Kaleb (4.Mose 13,30):
\begin{bibelbox}{SCHL}{4Mos}{13:30}
Kaleb aber beschwichtigte das Volk gegenüber Mose und sprach: \glqq{}Lasst uns doch hinaufziehen und das Land einnehmen, denn wir werden es gewiss bezwingen!\grqq
\end{bibelbox}
Haben wir das nicht auch schon zu unseren Liebsten gesagt? \glqq{}Lasst uns doch zu Jesus gehen!\grqq Niemand hörte auf ihn, das Resultat war (4. Mose 14, 1-4)
\begin{bibelbox}{SCHL}{4Mos}{14:1-4}
Da erhob die ganze Gemeinde ihre Stimme und schrie, und das Volk weinte in dieser Nacht. Und alle Kinder Israels murrten gegen Mose und Aaron; und dass wir doch im Land Ägypten gestorben wären, oder noch in dieser Wüste sterben würden! Und warum führt uns der \herr{} in dieses Land, dass wir durch das Schwert fallen und dass unsere Frauen und unsere kleinen Kinder zum Raub werden? Ist es nicht besser für uns, wenn wir wieder nach Ägypten zurückkehren? Und sie sprachen zu einander: \glqq{}Wir wollen uns selbst einen Anführer geben und wieder nach Ägypten zurückkehren!\grqq
\end{bibelbox}
Kaleb musste eine Entscheidung treffen. Josua die Rechte Hand Mose und Kaleb waren die einzigen der 12 Kundschafter, die überzeugt waren, dass sie es mit der Hilfe von Gott schaffen können, diese Land einzunehmen und zu besiedeln. Das ganze Volk aber entschied sich dagegen. Diese Rebellion gegen Gott hatte weitreichende folgen für Israel.
Auch wir müssen uns entscheiden. Jesus sagt in (Johannes 3,36)
\begin{bibelbox}{SCHL}{Joh}{3:36}
Wer an den Sohn glaubt, der hat ewiges Leben; wer aber dem Sohn nicht glaubt, der wird das Leben nicht sehen, sondern der Zorn Gottes bleibt auf ihm.
\end{bibelbox}
Treffen wir die Entscheidung für Jesus, gehören wir ab diesem Zeitpunkt einer Minderheit an. Kaleb entging wegen seiner Entscheidung nur knapp einer Steinigung. Bei uns ist es hier nicht so extrem, trotzdem stösst man auf Unverständnis oder teilweise auch auf Spott.
In (4. Mose 14,24)
\begin{bibelbox}{SCHL}{4Mos}{14:24}
Aber meinen Knecht Knecht Kaleb, in dem ein anderer Geist ist und der mir völlig nachgefolgt ist, ihn will ich in das Land bringen, in das er gegangen ist, und sein Same soll es als Erbe besitzen.
\end{bibelbox}
Ist das nicht wunderbar wenn Gott das von dir und mir sagen würde? Kaleb blieb Gott treu. Trotzdem wurde er nicht aus der Gemeinschaft gerissen und direkt ins gelobte Land gesetzt, sondern musste, wie alle anderen 40 Jahre auf die Besitznahme des Landes warten. Aber Gott hat ihm versprochen, dass er hinein kommt. Er und Josua waren die Einzigen. 
Auch uns hat Gott versprochen das wir eines Tages in sein Reich kommen, wo er viele Wohnungen für uns vorbereitet hat. Ausserdem verspricht er uns in (Johannes 6,50-51)
\begin{bibelbox}{SCHL}{Joh}{6:50-51}
dies ist das Brot, das aus dem Himmel herab kommt, damit, wer davon isst, nicht stirbt. Ich bin das lebendige Brot, das aus dem Himmel herab gekommen ist. Wenn Jemand von diesem Brot isst, so wird er leben in Ewigkeit. Das Brot aber, das ich geben werde, ist mein Fleisch, das ich geben werde für das Leben der Welt.
\end{bibelbox}
Wenn wir unser Mut und Vertrauen auf Gottes Wort stützen, werden wir seine Hilfe und seinen Segen persönlich erleben. Durch unser Verhalten können wir andere Menschen ermutigen, sich für Gott zu entscheiden, aber auch sie entmutigen ihm zu folgen, wie es die anderen 10 Fürsten der Stämme Israel getan haben.

Darum lasst uns dem \herr N danken, dass er uns auserwählt hat. Wollen wir unseren "40 Jahren" treu folgen und geduldig auf die Belohnung warten.

\section{2. Lied}

\textbf{2. Lied}

\section{Schriftlesung und Gebet}
Oswald nimmt die Schriftlesung vor und betet vor der Predigt

\section{Predigt}
Das wird von Thomas Lieth verkündigt.

Es wäre schön wenn zwei Geschwister in kurzen Worten Gott für die Worte von Thomas und den Dienst von Thomas danken könnten.

\section{Abschluss}
Stehen wir doch, denen es möglich ist, auf für das letzten Lied.
\textbf{3. Lied}

Der Friede Gottes, der höher ist als alle Vernunft,
bewahre eure Herzen und Sinne in Christus Jesus.

Wünsche einen schönen restlichen Tag und eine gute Woche.
