\documentclass[12pt,a4paper]{scrarticle}

\usepackage[headsepline, footsepline]{scrlayer-scrpage}
\usepackage{blindtext}% nur für Fülltext
\usepackage[ngerman]{babel}
\usepackage[utf8]{inputenc}
\usepackage[T1]{fontenc}
\usepackage{lmodern}
\usepackage{blindtext}
\usepackage{graphicx}
\usepackage{xcolor}
\usepackage[most]{tcolorbox}
\usepackage{geometry}
\usepackage[version=4]{mhchem}
\usepackage{bibleref-german}
\usepackage{setspace}
%%%%%%%%%%%%%%%%%%%%%%Link Formatierung%%%%%%%%%%%%%%%%%%%%%%%%%%%
\usepackage[colorlinks = true,
linkcolor = black,
urlcolor  = blue,
citecolor = black,
anchorcolor = blue]{hyperref}
%\documentclass[xcolor=dvipsnames]{beamer}

\pagestyle{scrheadings}

\chead{\headmark}
\automark{section}



\cfoot{Seite \pagemark}



%%%%%%%%%%%%%%%%%%%%%Griechische Zeichen%%%%%%%%%%%%%%%%%%%%%%%%%%%%%%%%%%%
%\begin{otherlanguage}{polutonikogreek} 
%	Δοκεῖ δέ μοι καὶ Καρχηδόνα μὴ εἶναι. 
%\end{otherlanguage} 



%%%%%%%%%%%%%%%%%%%%%%%%%%%%%%%%%%%%%%%%%%%%%%%%%%%%%%%%%%%%%%%%%%
%
% Makro zur Namensvervollstädigung
% Parameter: Kürzel --- Bibel
%			LuN			Neue Luther Übersetzung
%			Sch2		Schlachter 2000
%			Elb			Elbfelder
%%%%%%%%%%%%%%%%%%%%%%%%%%%%%%%%%%%%%%%%%%%%%%%%%%%%%%%%%%%%%%%%%%%
	\newcommand{\bib}[1]{%
	\ifthenelse{\equal{#1}{EI}}{Einheitsübersetzung}{%
		\ifthenelse{\equal{#1}{Sch2}}{Schlachter 2000}{%
			\ifthenelse{\equal{#1}{HFA}}{Hoffnung für Alle}{%
				\ifthenelse{\equal{#1}{ELB}}{Elbfelder}{%
					\ifthenelse{\equal{#1}{Neü}}{neue ev. Übersetzung}{%
						\ifthenelse{\equal{#1}{Lut}}{Luther}{#1}%
					}%
				}%
			}%
		}%
	}%
}%

%%%%%%%%%%%%%%%%%%%%%%%%%%%%%%%%%%%%%%%%%%%%%%%%%%%%%%%%%%%%%%%%%%
%
% Makro für Bibelzitate
% 
% Beispiel: 
%		\begin{bibeltext}{ELB}{Matt}{1:1-4}
%			Ich bin der zitierte Bibeltext.
%		\end{bibeltext}
%1Petr, Matt, Offb, 1Mos, Joh, IThess, 2Thess, Kol, Ps, Jak, ITim
%%%%%%%%%%%%%%%%%%%%%%%%%%%%%%%%%%%%%%%%%%%%%%%%%%%%%%%%%%%%%%%%%%
\newcommand{\bibtit}[1]{\large\bib{#1}}
\newcommand{\tmpbib}{Hallo}
\newcommand{\herr}{HERR}

\newcommand{\lied}[1]{\colorbox{yellow}{\textcolor{blue}{Lied: #1}}}
\newcommand{\green}[1]{\colorbox{green}{\textcolor{black}{#1}}}
\newcommand{\red}[1]{\colorbox{red}{\textcolor{white}{#1}}}
\newcommand{\beten}{\red{Ich möchte noch beten!}}
\newcommand{\hh}[1]{\textsuperscript{#1}}
\newenvironment{bibeltext}[3]{%	
        \renewcommand{\tmpbib}{\textbf{\bibleverse{#2}( #3)}}
		\biblerefformat{lang}
        \begin{tcolorbox}[
        title=\tmpbib \hspace{1.5cm} (\bibtit{#1}),
        colback=black!2!white,
        colframe=black!55!black]        
	}{%	
		\newline		
        \end{tcolorbox}	
	\endquote		
}
% \begin{tcolorbox}[title=Bibeltext,
%     title filled=false,
%     colback=blue!5!white,
%     colframe=blue!75!black]
% \end{tcolorbox}


\title{\includegraphics[height=48pt]{assets/images/logo.png}\\Wieso Bibel lesen?}
\author{Lothar Schmid}
\begin{document}
\maketitle

\section{Begrüssung}
Ich möchte Nathanel und euch alle herzlich zu diesem Gottesdienst begrüssen.
\\
Lasst uns beten und dann ein Lied zu ehre Gottes singen, Jesus höchster Name singen

\textbf{1. Lied}
\textit{Das höchste meines Lebens}

\section{Ankündigungen}
\begin{itemize}
    \item Nächster Bibel und Gebetsabend am Donnerstag: 
27.06.2024 Römer 8 
    \item Nächster Gottesdienst: 30.06.2024 
\end{itemize}

\section{ Input }

(Psalm 1 1-2):
\begin{bibeltext}{Schl2}{Ps}{1:2}
Wohl dem der nicht wandelt nach dem Rat der Gottlosen, noch tritt auf den Weg der Sünder, noch sitzt, wo die Spötter sitzen, sondern seine Lust hat am Gesetz des Herrn und über sein Gesetz nachsinnt Tag und Nacht.
\end{bibeltext}

Darum singen wir jetzt das Lied \textbf{Das Höchste meines Lebens}

\textbf{2. Lied} \textit{Das Höchste meines Lebens}

- Oswald nimmt die Schriftlesung vor und betet vor der Predigt

\section{Predigt}
Die wird von Nathanael verkündigt.
\\
\\
\textbf{Nach der Predigt}\\
Nathanel vielen Dank für deine Worte 

Nach dem Beten singen wir unser \textbf{3. Lied}: Du sollst nicht müde werden (Wir werden sein wie die Träumenden)

\section{Abschluss}
(Psalm 25 20-21):
\begin{bibeltext}{Schl}{Ps}{25:20-21}
Bewahre meine Seele und rette mich! Lass mich nicht zuschanden werden, denn ich vertraue auf dich! Lauterkeit und Redlichkeit, mögen mich behüten, denn auf dich harre ich.
Maranatha, Herr komme bald.
\end{bibeltext}
\end{document}