\documentclass{../inc/mybib}

\title{\includegraphics[height=48pt]{../assets/images/logo.png}\\Wichtige Hinweise}
\author{Lothar Schmid}
\begin{document}
\maketitle
\section{Begrüssung}

Ich möchte euch alle recht herzlich zu diesem besonderen Gottesdienst begrüssen. Es ist schön, dass ihr trotz dem schönen Wetter hier zu uns gefunden habt, um unseren Herrn zu loben und zu ehren.

Auch möchte ich Samuel recht herzlich begrüßen, der den Weg von Zürich über Bern hier gemacht hat, um uns das Wort des Herrn zu verkünden. 

\noindent
\beten{} und anschließend singen wir zusammen das Lied

\noindent
\lied{Christus ist mein ganzer halt}.

\section{Ankündigungen}
\begin{itemize}
    \item \green{Bibel und Gebetsabend:} Do 23.1.2025 20:00Uhr mit Philipp Ottenburg ab Römer 12:3-5
    \item \green{Nächster Gottesdienst:} So 26.1.2025 10:00Uhr Livestream aus Dübendorf. Am Nachmittag findet kein Gottesdienst statt.
    \item \green{Diverses:} -
    \item \green{Die Kollekte:} Die Kollekte geht an den Mitternachtsruf und wird dort für die Missionsarbeit in der Welt eingesetzt.
\end{itemize}

\section{ Input }
\begin{spacing}{1.5}
\subsection{ Wichtige Hinweise }

Letzten Sonntag, waren Janina und ich krank zu Hause. Was macht ein guter Christ an einem Sonntag wenn er krank ist? Genau er hört sich Livegottesdienste an. So auch wir beide. Auf der Kautsch mit heissem Tee und einer warmen Decke, haben wir Gottesdienste gehört. Wir haben 3 Predigten gehört und alle drei waren sehr gut mit sehr unterschiedlichen Themen. Am Montagabend in meiner Bibellese las ich das Kapitel 1.Timotheus 6. In den vorgehenden Kapitel hat Paulus viele gute Ratschläge gegeben, auf was er bei einer Gemeinde achten soll. Hier im Kapitel 6 ab Vers 11, gibt Paulus Ratschläge für ihn persönlich. Es fängt so an:

\begin{bibelbox}{NUE}{ITim}{6:11}
Aber du, als Mann Gottes fliehe vor alledem...
\end{bibelbox}

 \glqq Vor alledem \grqq{} nimmt Bezug auf die Verse vorher, in denen es über Falsche Lehre, Habgier und Reichtum geht. Da Paulus hier seinen Freund als Mann Gottes anspricht, können wir das nicht auch auf uns wiedergeborene Christen anwenden? Als Männer und Frauen Gottes? Ab Vers 11 bis 17 ist alles drin, was wir als Christen wissen und befolgen müssen. Darum lese ich euch diese 7 Verse jetzt aus der bibel.heute Übersetzung vor.

Habt ihr gehört? Ist das nicht fantastisch? In so einem kurzen Abschnitt ist der Inhalt von mehreren Predigten zusammengefasst. 

Wir singen jetzt gemeinsam ein Lied und dann möchte ich Samuel gerne bitten uns sein Wort zu verkündigen.

\end{spacing}

\lied{Nur durch Christus in mir}

\section{Predigt}
\green{Schriftlesung}

Danach gebe ich das Wort an Samuel weiter.

\textbf{Nach der Predigt}

Danken für die Predigt

\section{Abschluss}

Jetzt wollen wir Gott mit dem Lied \lied{Schau ich zurück} danken.


Vielen Dank für eure Teilnahme und das Gebet. Im Anschluss seid ihr zu Kaffee und guten Gesprächen eingeladen.
\beten{}

\begin{bibelbox}{SCHL}{Phil}{4:20.23}
Unser Gott und Vater aber sei die Herrlichkeit von Ewigkeit zu Ewigkeit!
Die Gnade des Herrn Jesus Christus sei mit eurem Geist!
\end{bibelbox}

Maranatha Amen
\end{document}