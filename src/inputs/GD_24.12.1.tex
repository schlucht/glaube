\documentclass{../inc/mybib}

\title{\includegraphics[height=48pt]{../assets/images/logo.png}\\Gottesdienst}
\author{Lothar Schmid}
\begin{document}
\maketitle
\section{Begrüssung}

Ich möchte euch alle recht herzlich zu diesem besonderen Gottesdienst begrüssen. Schön das so viele gekommen sind, um unserem Herrn zu Danken ihn zu Loben und zu Preisen.

Besonders möchte ich Nathanael begrüssen, der den Weg von Zürich über Bern hier gemacht hat um uns das Wort des Herrn zu verkünden. 

\noindent
\beten{} und anschliessend singen wir zusammen das Lied

\noindent
\lied{Schöpfer aller Himmel}.

\section{Ankündigungen}
\begin{itemize}
    \item \green{Bibel und Gebetsabend:} Do 9.1.2025 20:00Uhr Es gibt keine Bibelauslegung aus Dübendorf. In Dübendorf ist eine Gebetswoche. Wir wollen an dem Abend im Gebet dem Herrn für 2024 Danken und ihn für Bewahrung 2025 bitten. Hat jemand Gebetsanliegen, können wir diese gerne mit aufnehmen.
    \item \green{Nächster Gottesdienst:} So 12.1.2025 15:00Uhr mit Philip Ottenburg
    \item \green{Diverses:} Am letzten Sonntag jeden Monats geht eine Delegation von hier nach Dübendorf an die Bibelschule. An diesem Sonntag sind sehr wenig Gottesdienst Besucher hier. An diesen Tagen wollen wir die Gottesdienste am Morgen um 10Uhr durchführen. Wir wollen gemeinsam den Livestream von Dübendorf als Gottesdienst durchführen. Eine gute Gelegenheit für Leute die keine Lust haben 15Uhr zu Gottesdienst, den Mitternachtsruf kennen zu lernen.
    \item \green{Die Kollekte:} Die Kollekte geht an den Mitternachtsruf und wird dort für die Missionsarbeit in der Welt eingesetzt.
\end{itemize}

\section{ Input }
\begin{spacing}{1.5}
\subsection{ Geschenke }

Weihnachten 2024 ist vorbei, Weihnachten 2025 ist wieder vor uns. 

Teilweise werden Weihnachtsgeschenke wieder eingesammelt und diese dann als zweite Weihnacht an ärmere Familien weitergegeben. Diese Aktion wird dann als zweite Weihnacht bezeichnet.

Janina und ich waren über Silvester Neujahr in Moldawien und haben dort beim Verteilen von Päckli der HMK Weihnachtspäckli Aktion mitgeholfen. Die Schulen haben Theaterstücke eingeübt und uns dann vorgetragen. Danach haben wir dann ihnen die Geschenke gegeben. Die Freude der Kinder war gross. Plüschtiere hatten bei kleineren Kindern einen riesigen Wert.

Wir waren aber auch in armen Familien und haben die Päckli direkt der Familie gebracht. Alles in allem für mich eine tolle Erfahrung die ich nicht mehr missen möchte.

Kleine Geschenke erhalten die Freundschaft. Auch in einer Ehe hilft ab und zu ein kleines Geschenk diese erstrahlen zu lassen. Wie schon am letzten Gottesdienst uns Oswald erkärt hat, kamen ja die 3 Weisen aus dem Morgenland mit Geschenke zu Jesus. Gold, Weihrauch und Myrrhe. Gold für die Reinheit und Göttlichkeit Jesus, Weihrauch für seine Sündlosikeit und Myrrhe für seinen Leidensweg hier auf Erden. Ob Maria und Joseph die genau Symbolik dieser Geschenke kannten weiss ich nicht, aber sie konnten sie sicher sehr gut gebrauchen. Das half ihnen auf dem Weg nach Agypten, Weihrauch und Myrrhe könnte sicher manches stinkige Herbergszimmer angenehmer machen.

Jesus möchte aber immer noch das wir ihm Geschenke machen. Zum Beispiel Gottesdienst. Das ist ein Geschenk für Jesus, Lobpreis, Gebet, Abendmahl alles zu ehren Gottes. Gottestdienst darf keine Pflichtübung sein. Wir sitzen auch nicht nur hier einem so guten Prediger wie Nathanel zuzuhören. Wir wollen doch mit diesem Gottesdienst den Herrn beschenken, ihm Danken für sein grösstes Geschenk, dass er uns gegeben hat. Darum lasst uns mit dem Herzen den Gottesdienst feieren und in der Predigt darauf achten, was Gott uns damit sagen will.

\begin{bibelbox}{Neü}{Luk}{2:26}
...und hatte ihm Gewissheit gegeben, dass er nicht sterben werde, bevor er den vom Herrn gesandten Messias gesehen habe.
\end{bibelbox}

\begin{bibelbox}{Neü}{Luk}{2:37}
...und war jetzt eine Witwe von 84 Jahren. Sie verliess den Tempel gar nicht mehr und diente Gott Tag und Nacht mit Fasten und Beten.
\end{bibelbox}

Sie warteten ihr Leben lang auf den Erlöser und haben ihn gesehen. Lasst uns also Täglich ihn erwarten. In dieser Wartezeit dürfen wir seinen Geburtstag zu Weihnachten und seine Erlösung zu Ostern, sowie das Abendmahl feiern.

\end{spacing}

\lied{Jesus kam für dich}

\section{Predigt}
\green{Schriftlesung}

Danach gebe ich das Wort an Nathanael weiter.

\textbf{Nach der Predigt}

Danken für die Predigt.

\section{Abendmal}

Beten für das Brot

\lied{Das Blut der Lammes 1. Strophe}

Beten für den Wein

\lied{Das Blut der Lammes 2. Strophe}


\section{Abschluss}

Jetzt wollen wir Gott mit dem Lied \lied{Er ist der Erlöser} danken.


Vielen Dank für eure Teilnahme und das Gebet. Im Anschluss seid ihr zu Kaffee und guten Gesprächen eingeladen.
\beten{}

\begin{bibelbox}{Sch2}{1Mos}{28:15}
Gott spricht: Siehe, ich bin mit dir,
ich behüte dich, wohin du auch gehst.
Denn ich verlasse dich nicht,
bis ich vollbringe, was ich dir versprochen habe.
\end{bibelbox}

Maranatha Amen
\end{document}