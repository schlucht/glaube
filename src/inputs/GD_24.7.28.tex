\documentclass[12pt,a4paper]{scrarticle}

\usepackage[headsepline, footsepline]{scrlayer-scrpage}
\usepackage{blindtext}% nur für Fülltext
\usepackage[ngerman]{babel}
\usepackage[utf8]{inputenc}
\usepackage[T1]{fontenc}
\usepackage{lmodern}
\usepackage{blindtext}
\usepackage{graphicx}
\usepackage{xcolor}
\usepackage[most]{tcolorbox}
\usepackage{geometry}
\usepackage[version=4]{mhchem}
\usepackage{bibleref-german}
\usepackage{setspace}
%%%%%%%%%%%%%%%%%%%%%%Link Formatierung%%%%%%%%%%%%%%%%%%%%%%%%%%%
\usepackage[colorlinks = true,
linkcolor = black,
urlcolor  = blue,
citecolor = black,
anchorcolor = blue]{hyperref}
%\documentclass[xcolor=dvipsnames]{beamer}

\pagestyle{scrheadings}

\chead{\headmark}
\automark{section}



\cfoot{Seite \pagemark}



%%%%%%%%%%%%%%%%%%%%%Griechische Zeichen%%%%%%%%%%%%%%%%%%%%%%%%%%%%%%%%%%%
%\begin{otherlanguage}{polutonikogreek} 
%	Δοκεῖ δέ μοι καὶ Καρχηδόνα μὴ εἶναι. 
%\end{otherlanguage} 



%%%%%%%%%%%%%%%%%%%%%%%%%%%%%%%%%%%%%%%%%%%%%%%%%%%%%%%%%%%%%%%%%%
%
% Makro zur Namensvervollstädigung
% Parameter: Kürzel --- Bibel
%			LuN			Neue Luther Übersetzung
%			Sch2		Schlachter 2000
%			Elb			Elbfelder
%%%%%%%%%%%%%%%%%%%%%%%%%%%%%%%%%%%%%%%%%%%%%%%%%%%%%%%%%%%%%%%%%%%
	\newcommand{\bib}[1]{%
	\ifthenelse{\equal{#1}{EI}}{Einheitsübersetzung}{%
		\ifthenelse{\equal{#1}{Sch2}}{Schlachter 2000}{%
			\ifthenelse{\equal{#1}{HFA}}{Hoffnung für Alle}{%
				\ifthenelse{\equal{#1}{ELB}}{Elbfelder}{%
					\ifthenelse{\equal{#1}{Neü}}{neue ev. Übersetzung}{%
						\ifthenelse{\equal{#1}{Lut}}{Luther}{#1}%
					}%
				}%
			}%
		}%
	}%
}%

%%%%%%%%%%%%%%%%%%%%%%%%%%%%%%%%%%%%%%%%%%%%%%%%%%%%%%%%%%%%%%%%%%
%
% Makro für Bibelzitate
% 
% Beispiel: 
%		\begin{bibeltext}{ELB}{Matt}{1:1-4}
%			Ich bin der zitierte Bibeltext.
%		\end{bibeltext}
%1Petr, Matt, Offb, 1Mos, Joh, IThess, 2Thess, Kol, Ps, Jak, ITim
%%%%%%%%%%%%%%%%%%%%%%%%%%%%%%%%%%%%%%%%%%%%%%%%%%%%%%%%%%%%%%%%%%
\newcommand{\bibtit}[1]{\large\bib{#1}}
\newcommand{\tmpbib}{Hallo}
\newcommand{\herr}{HERR}

\newcommand{\lied}[1]{\colorbox{yellow}{\textcolor{blue}{Lied: #1}}}
\newcommand{\green}[1]{\colorbox{green}{\textcolor{black}{#1}}}
\newcommand{\red}[1]{\colorbox{red}{\textcolor{white}{#1}}}
\newcommand{\beten}{\red{Ich möchte noch beten!}}
\newcommand{\hh}[1]{\textsuperscript{#1}}
\newenvironment{bibeltext}[3]{%	
        \renewcommand{\tmpbib}{\textbf{\bibleverse{#2}( #3)}}
		\biblerefformat{lang}
        \begin{tcolorbox}[
        title=\tmpbib \hspace{1.5cm} (\bibtit{#1}),
        colback=black!2!white,
        colframe=black!55!black]        
	}{%	
		\newline		
        \end{tcolorbox}	
	\endquote		
}
% \begin{tcolorbox}[title=Bibeltext,
%     title filled=false,
%     colback=blue!5!white,
%     colframe=blue!75!black]
% \end{tcolorbox}


\title{\includegraphics[height=48pt]{assets/images/logo.png}\\Was ist Gnade?}
\author{Lothar Schmid}
\begin{document}
\maketitle
\section{Begrüssung}

Ich möchte euch alle und Thomas herzlich zu diesem Gottesdienst begrüßen. Schön Thomas, dass du diese Reise auf dich genommen hast, um uns das Wort zu bringen. \\
\textbf{GEBET}\\
Lasst uns beten und dann das erste Lied singen: O Gott dir sei Ehre der Grosses getan.

\textbf{1. Lied}
\textit{O Gott dir sei Ehre der Grosses getan}

\section{Ankündigungen}
\begin{itemize}
    \item In Dübendorf findet kein Gebetsabend statt. Janina und ich sind nicht da. 
    \item Nächster Gottesdienst: 4.8.2024 mit Erich Maag und Abendmahl    
\end{itemize}

\section{ Input }
\begin{spacing}{1.5}
Die Gnade, verstehen wir überhaupt was die Gnade wirklich ist? Seit wir auf die Welt kamen, versuchen wir Leistung zu erbringen und anderen Menschen zu gefallen. Das Lob der Eltern, wenn man etwas tolles gemacht hat, gab einem das Gefühl, ein bisschen mehr geliebt zu sein. Dann in der Schule, die guten Noten, bei manchen war es das Lob der Lehrer, bei anderen  die Anerkennung der Mitschüler bei Streichen .\\
Später dann in der Lehre und im Beruf, der versuch die Besten zu sein. Immer versuchen wir den anderen zu gefallen. Ich ertappe mich, wo ich denke ob ich wohl ein guter Christ bin, wenn ich dies und jenes mache? Ich bin sicher ein besserer Christ und Gott liebt mich ein bisschen mehr, wenn ich am Donnerstag immer zum Gebet gehe.\\
Ich glaube das ist menschlich und wir ticken so und das es fördert und auch in unserem Leben. Aber es ist nicht göttlich, Gott tickt anders. Wie die Gnade ist, habe ich in einem Buch gelesen und möchte das hier zitieren:
\begin{quote}
    Aus Gnade statt aus Werken zu leben bedeutet, dass Sie frei sind von der Lebensmaschinerie. Es bedeutet, dass Gott Ihnen bereits eine \glqq Sechs\grqq{} gegeben hat, obwohl Sie eine \glqq Eins\grqq{} verdient hätten. Er hat Ihnen bereits den vollen Tagelohn gegeben, obwohl Sie vielleicht nur eine Stunde gearbeitet haben. Es bedeutet, dass Sie nicht bestimmte geistliche Pflichten erfüllen müssen, um Gottes Gunst zu erlangen. Jesus Christus hat das bereits für Sie getan. Sie werden aufgrund dessen, was Jesus getan hat, von Gott geliebt und angenommen, und Sie werden aufgrund dessen was Jesus getan hat, von Gott gesegnet. Nichts, was Sie jemals zustande bringen, wird ihn dazu veranlassen, Sie mehr oder weniger zu lieben. Er liebt Sie allein aus seiner Gnade, die er Ihnen durch Jesus geschenkt hat.\\
    Jerry Bridges
\end{quote}
Und nicht vergessen, Gott liebt auch die Queeren und nicht Christen genau so wie er mich und dich liebt.

Also was können wir machen? Ihm gehorchen, ihn Loben und Danke für seine Gnade, die wir unverdient bekommen haben. Und das machen wir jetzt am besten mit einem Lied.

% \begin{bibeltext}{Sch2}{Kol}{1:18}
% Und er (Christus Jesus) ist das Haupt der Gemeinde...
% \end{bibeltext}

\end{spacing}{1.5}
Wir singen jetzt unser \textbf{2. Lied} \textit{Du bist würdig}

- Oswald nimmt die Schriftlesung vor und betet vor der Predigt

\section{Predigt}
- Die wird von Thomas verkündigt.

\textbf{Nach der Predigt}

Danke für die Predigt

Nach dem Beten singen wir unser \textbf{3. Lied}: \textit{Herr, füll mich neu}\\

\section{Abschluss}

\begin{bibeltext}{Sch2}{1Mos}{28:15}
Gott spricht: Siehe, ich bin mit dir,
ich behüte dich, wohin du auch gehst.
Denn ich verlasse dich nicht,
bis ich vollbringe, was ich dir versprochen habe.
\end{bibeltext}
Amen
Ich wünsche euch einen schönen restlichen Tag und eine gute Woche, bis Donnerstag..
\end{document}