\author{OTS}
\documentclass{../../inc/mybib}

\setincpath{../../inc/}

\usepackage{bible_style}
\usepackage{header}

\begin{document}
    \tableofcontents
    \section{Gottesreich in der Bibel}
    \textbf{Video 01 - Das Königreich - das Thema der Schrift}
    \subsection{Allgemein}
    Oft fragt man sich, was will Gott mit der Bibel? Und was ist der Grund seines Heilsplanes?

    Gott hatte schon in \bibleverse{IMos}(1:26-28) einen Plan mit dem Menschen. Dieser Plan wird dann in \bibleverse{Rev}(22:1-5) mit dem kommen des neuen Reiches beendet. Auch wenn das Reich Gottes in der ganzen Bibel ein Thema ist, ist die Vollendung noch in der Zukunft. Aktuell sind wir auf dem Weg dahin aber noch nicht am Ziel.
    \subsection{Die 5 Phasen bei der Entfaltung Gottes Reich}
    \subsubsection*{1. Die Schöpfung}
    \bibleverse{IMos}(1:); Gott erschafft den Menschen. Der Mensch wird als Verwalter der restlichen Schöpfung eingesetzt. Er bekommt von Gott einen Aufrag.
    \begin{bibelbox}{ELB}{IMos}{1:26-28}
        Und Gott sprach: \flqq Lasst uns Menschen machen in unserem Bild, uns ähnlich! Sie sollen herrschen über die Fische des Meeres und über die Vögel des Himmels und über das Vieh und über die ganze Erde und über alle kriechenden Tiere, die auf der Erde kriechen.\frqq{}

        Und Gott schuf den Menschen nach seinem Bilde, nach dem Bild Gottes schuf er ihn; als Mann und Frau schuf er sie. 
        
        Und Gott segnete sie, und Gott sprach zu ihnen: \flqq Seid fruchtbar und vermehrt euch, und füllt die Erde, und mach sie euch unteran; und herrscht über die Fische des Meeres und über die Vögel des Himmels und über alle Tiere, die sich auf Erden regen.\frqq{}
    \end{bibelbox}
    Gott hat uns schon im Paradies nicht erschaffen um auf der faulen Haut zu liegen sondern um sein Schöpfung pflegen und zu verwalten. Aber!
    \subsubsection*{2. Der Sündenfall}
    \bibleverse{IMos}(3:); Der Sündenfall wird im Kapitel 3 beschrieben. Wieviel Zeit dazwischen vergangen ist, wissen wir nicht. Gott hat zusätzlich zum Auftrag die Schöpfung zu verwalten, ihnen ein Gebot gegeben. Er erschaffte mitten im Garten zwei Bäume, \textbf{den Baum des Lebens} und \textbf{den Baum zur Erkenntnis des Guten und Bösen}. Dann kommt das Verbot von Gott an den Mensch:
    \begin{bibelbox}{IMos}{ELB}{2:16-17}
        Und Gott, der \herr{}, gebot dem Menschen und sprach: \flqq Von jedem Baum des Gartens darfst du essen; aber vom Baum der Erkenntnis des Guten und Bösen, davon darfst du nicht essen; denn an dem Tag, da du davon isst, musst du sterben.\frqq{}
    \end{bibelbox}
    Das war das Verbot. Von einem Baum einfach die Frucht nicht essen. Aber die ersten zwei Menschen hielten sich nicht daran. Sie assen von der Frucht. Auch wenn der Satan in Form einer Schlange sie verführte gibt es kein Entschuldigung. Und nun was macht Gott? Schwamm drüber? Einmal ist keinmal? Nein er bestraft die zwei Menschen, wie er es angekündigt hat. \red{Aber!}
    \subsubsection*{3. Die Verheissung}
    \bibleverse{IMos}(3:15) bis \bibleverse{Mal}; Hier kommt der berühmte Vers \bibleverse{IMos}(3:15) ins Spiel. Dieser Vers gibt dem Paar wieder Mut. Es soll also jemand kommen, durch den sie wieder zurück ins Paradies können.
    \begin{bibelbox}{IMos}{ELB}{3:15}
        Und ich werde Feindschaft setzen zwischen dir und der Frau, zwischen deinem Samen und ihrem Samen; er wird dir den Kopf zermalmen, und du, du wirst ihm die Ferse zermalmen.
    \end{bibelbox}
    Auf diesen Samen haben die Menschen gewartet. Eva nannte ihren ersten Sohn Kain, was soviel heisst wie, Erworbenes oder auch Gewinn. Vielleicht dachte sie, dass sei jetzt der versprochene Same. Jede Generation dachte, jetzt kommt er wir können zurück ins Paradies. Auch Lamech der Vater von Noah dachte, dass sein Sohn der sein wird, der der Schlange der Kopf zermalmen wird. Noah war ein Retter aber nicht der verheissene Retter. Die Sindflut war ein einschneidendes Ereignis. Nach der Sindflut hat Gott mit 8 Menschen neu angefangen. Die Menschen waren immer noch  nicht besser, aber Gottes Langmut ist grenzenlos. Er erwählte sich ein Volk, mit dem ER seinen Plan, das Reich Gottes zu errichten, zu Ende bringen wollte. Das Volk war sehr widerspenstig und gehorchte Gott nicht. \red{Aber!}
    \subsubsection{4. Ankunft der Königs}
    \bibleverse{Math} bis \bibleverse{Jud}; Aus seinem auserwählten Volk wurde der Retter geboren. Der Retter Jesus Christus ist Gottes Sohn und verkündigte das Reich Gottes unter den Juden.
    \begin{bibelbox}{ELB}{Math}{6:33}
        \textbf{Trachtet aber zuerst nach dem Reich Gottes und nach seiner Gerechtigkeit!} (Und dies alles wird euch hinzugefügt.)
    \end{bibelbox}
    Die Juden wollten aber nicht nach dem Reich Gottes trachten. Darum verheisst Jesus ihnen am Schluss:
    \begin{bibelbox}{ELB}{Math}{21:43}
        Deswegen sage ich euch: Das Reich Gottes wird von euch weggenommen und einer anderen Nation gegeben werden, die seine Früchte bringen wird.
    \end{bibelbox}
    Wenn man die Suche anwirft, kommt das Reich Gottes in Matthäus 5x, in Markus 14x, in Lukas 31x und in Johannes 2 mal vor. Man sieht es ist in den Evangelien ein wichtiges Thema. Wir wissen, dass Jesus in erster Linie zu den Juden gepredigt hat und die Heiden erst später in der Apostelgeschichte zum Zuge kamen.
    \begin{bibelbox}{ELB}{Mk}{7:27}
        Und er sprach: \flqq Lass zuerst die Kinder satt werden, denn es ist nicht schön, das Brot der Kinder zu nehmen und den Hunden hinzuwerfen.\frqq{}
    \end{bibelbox}
\end{document}