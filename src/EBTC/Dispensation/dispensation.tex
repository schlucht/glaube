\author{OTS}
\documentclass{../../inc/mybib}

\setincpath{../../inc/}

\usepackage{bible_style}
\graphicspath{{../../assets/images/}}
\usepackage{longtable}
\usepackage{tabularx} % Stelle sicher, dass dieses Paket im Präambel geladen ist
\newcommand{\st}{Substitutionstheolog}
\newcommand{\sz}{Superzessionismus}

\begin{document}
\section{Dispensationalismus}
    Der Dispensationalismus ist eine realtiv junge Theorie, die sich im frühen bis mittleren 19.\,Jahrhundert entwickelte. Es handelt sich um ein nachreformatisches System, obgleich viele seiner Vorstellungen bis auf 
die frühe Gemeindezeit zurückgehen.

Der Ursprung des systematischen Dispensationalis ist mit John Nelson Darby (1800–1882) verbunden.Durch das Studium von Jesaja 32 kam Darby zu dem Schluss, das Israel in einer zukünftigen heidenchristlichen Epoche irdische Segnungen erfahren wird. Er erkannte ein klarer Unterschied zwischen der Gemeinde und Israel.

Der Mensch muss in jeder heilsgeschichtlichen Epoche bestimmte Bedingungen erfüllen; der Mensch hat vor Gott eine gewisse Verantwortung. Darby stellte auch fest, dass jede dieser Epochen im Scheitern endete.

Darby unterschied sieben heilsgeschichtliche Epochen: 
\begin{enumerate}
    \item Vom Paradies bis zur Sintflut;
    \item Noah;
    \item Abraham;
    \item Israel;
    \item Zeitalter der Nationen;
    \item Zeitalter des Heiligen Geistes;
    \item das Tausendjährige Reich.
\end{enumerate}
Die American Bible and Prophetic-Konferenzen in den Jahren 1878–1914 trugen zur Verbreitung dispensationalistischer Theologie bei.

Die Scofield Bibel trug in beträchtlichen Mass, der Verbreitung dieser Theorie bei.

\subsection{Variationen des Dispensationalismus}
Es gibt drei Hauptpositionen in der Geschichte dieser Theologie:
\begin{enumerate}
    \item \textbf{Der klassische Dispensationalismus}\\
    Der klassische Dispensationalismus nahm seinen Anfang in den 1830er Jahren und dauerte bis in die 1940er Jahre. Vertreter John Nelson Darby. Ein wichtiges Mekmal des klassischen Dispensationalismus ist der dualistische Heilschluss Gottes für die Völker Gottes. Gott verfolgt demnach zwei unterschiedliche Heilsratschlüsse. Der eine ist auf den Himmel bezogen und der andere auf die Erde. Im klassischen Dispensationalismus existiert eine sehr scharfe Trennung zwischen Israel und der Gemeinde. Die Gemeinde ist für den Himmel bestimmt, während Israel die Erde erben wird. Der klassische Dispensationalismus könnte auch als \enquote{traditioneller Dispensationalismus} bezeichnet werden.
    \item \textbf{Revidierter oder modifiziert Dispensationalismus}\\
    \item Die Ära des revidierten oder modifiziert Dispensationalismus kann ungefähr auf die Zeitspanne von 1950--1985 datiert werden. In dieser Form wird die scharfe Trennung zwischen Israel und der Gemeinde abgemildert. Sie legten dennoch Wert auf die Unterscheidung zweier anthropologischer Gruppierungen: Israel und die Gemeinde, die stets voneinander unterschieden werden. Für die meisten Vertreter gibt es nicht einen zweifachen Neuen Bund, sondern lediglich einen Neuen Bund, während Israel die Erfüllung des Neuen Bundes erst in einem zukünftigen Tausendjährigen Reich vollum
    fänglich erfahren wird. Er lehrt ferner, dass Jesus im Zeitalter der Gemeinde nicht auf dem Thron Davids sitzt oder von diesem Thron her herrscht. Folglich liegt die davidische Herrschaft Jesu in der Zukunft.     
    \begin{itemize}    
        \item Pentecost        
        \item Charles Ryrie
        \item Charles Feinberg
        \item Alva j.McCain
    \end{itemize}    
    Heute gibt es viele revidierte Dispensationalisten
    \item \textbf{Progressiver Dispensationalismus} \blockquote[Progressiv]{fortschrittlich, fortschreitend im Sinne von Modern}\\
    Zitat Charles Ryrie: \blockquote{Das Adjektiv \enquote{progressiv} bezieht sich auf die zentrale Lehre, dass sich der abrahamitische Bund, der davidische Bund und der neue Bund heute progressiv erfüllen (wobei sich einige Aspekte erst im Tausendjährigen Reich erfüllen werden).} 
    
    Bei der Gemeinde handelt es scih nicht um eine gesonderte Menschengruppe, sondern um die erlöste Menschheit in dieser gegenwärtigen Epoche der Heilsgeschichte. 
    
    Das \enquote{Volk Gottes} setzt sich, was das Heil angeht, sowohl aus Israel als auch aus der Gemeinde zusammen, und beide stehen unter den Segnungen des Neuen Bundes. Alle progressiven Dispensationalisten vertreten, dass sich die davidische Herrschaft für Israel im Tausendjährigen Reich vollumfänglich erfüllen wird. Die Mehrzahl der Dispensationalisten glaubt weiterhin, dass das Sitzen Jesu zur Rechten des Vaters nach \bibleverse{Ps}(110:1) vom Thron der Gottheit spricht, nicht vom davidischen Thron. Die Besteigung des davidischen Throns und die davidische Herrschaft werden sich aus ihrer Sicht erst mit dem zweiten Kommen Christi erfüllen. \bibleverse{Mat}(19:28; 25:31); \bibleverse{Offb}(3:21).
\end{enumerate}
\section{Grundzüge des Dispensationalismus}
1965 Charles Ryrie stellt drei wesentliche Kennzeichen des Dispensationalismus heraus:
\begin{enumerate}
    \item Die Unterscheidung zwischen Israel und der Gemeinde;
    \item die wörtliche Auslegung der Schrift;
    \item Offenbarung von Gottes Herrlichkeit ist Gottes grundlegende Heilsabsicht in der Welt.
\end{enumerate}
1988 legte John Feinberg sechs Grundzüge des Dispensationalismus dar:
\begin{enumerate}
    \item die Glaubenslehre, dass die Bibel Begriffe wie \enquote{Jude} und \enquote{Same Abrahams} mehrere Bedutungen beimisst
    \item eine hermeneutische Methode, die unterstreicht, dass die Aussagen des Alten Testaments in sich selbst Gültigkeit haben und nicht Lichte des Neuen Testaments interpretiert werden sollten.
    \item die Glaubenslehre, dass alttestamentlichen Verheissungen an der Nation Israel erfüllen werden;
    \item die Glaubenslehre, dass das ethnische Israel eine wesensgebundene Zukunkft haben wird;
    \item die Glaubenslehre, dass die Gemeinde ein spezieller Organismus ist;
    \item eine Geschichtsphilosophie, die nicht nur soterologische und geistliche Themen berücksichtigt, sondern auch soziale, ökonomische und politische Themen.
\end{enumerate}
Craig Blaising und Darrell Bock fassen die Grundzüge des Dispensationalismus wie folgt zusammen:
\begin{enumerate}
    \item die Autorität der Schrift;
    \item Epochen der Heilsgeschichte;
    \item Einzigartigkeit der Gemeinde;
    \item praktische Bedeutung der universellen Gemeinde;
    \item Wichtigkeit biblischer Prophetie;
    \item futuristische Prämillennialismus;
    \item unmittelbare bevorstehende Wiederkunft Christi;
    \item eine nationale Zukunft Israels;
\end{enumerate}
Alle Dispensationalisten sind auch Prämillennialisten, glauben also an eine Wiederkunft Christi vor dem Tausendjährigen Reich. Dispensationalisten betonen die vollständige Erfüllung sowohl der geistlichen als auch der physischen Verheissungen der biblischen Bündnisse.

Nichtdispensationalisten beginnen mit dem Neuen Testament um die phrophetischen Schriftstellen des Alten Testaments zu verstehen. Das Neue Testament ist die Brille, durch welche sie das Alte Testament betrachten. Dies führt des Öfteren zu einem \enquote{nicht wörtlichen} Verständnis alttestamentlicher Texte.
    
\end{document}