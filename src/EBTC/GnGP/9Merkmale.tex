\author{OTS}
\documentclass{../../inc/mybib}

\setincpath{../../inc/}

\usepackage{bible_style}
\graphicspath{{../../assets/images/}}


\begin{document}

\section*{Merkmal 1: Auslegungspredigt}
\subsection*{Auslegungspredigten}
Es gibt zwei Arten von Predigten:
\begin{itemize}
    \item Themenpredigten
    \item Auslegungspredigten
\end{itemize}
Themenpredigten beschäftigen sich mit einem bestimmten Thema, die der Prediger auswählt. Die Predigt rund um dieses Thema aufgebaut, mit Bibelversen und eigenen Erfahrungen.

Die Auslegungspredigt nimmt einen Bibeltext und ist an diesen Text gebunden. Eine Auslegungspredigt ist nicht nur ein mündlicher Kommentar zu einem Bibeltext, vielmehr geht darum herauszufinden, welche Hauptaussage diesem bestimmten Bibeltext entspricht. Der Wille zur Auslegungspredigt ist der Wille, Gottes Wort zu hören. NIcht bloss zu bejahen, sondern sich auch tatsächlich diesem zu unterstellen.

Die Hauptaufgabe wie auch die Hauptaufgabe eines jeden Pastors ist das Halten von Auslegungspredigten. Prediger haben ausdrücklich den Auftrag, hinzugehen und das Wort Gottes zu predigen.

Jemanden mit der geistlichen Leitung einer Gemeinde zi beauftragen, der nicht im praktischen Leben den festen Willen an den Tag legt, Gottes Wort zu hören und zu lehren, heisst, das Wachstum der Gemeinde zu hemmen.

Wonach und Christen verlangen sollte ist das Wort Gottes zu hören.

\subsection*{Die Zentrale Rolle des Wortes Gottes}
Die Predigt sollte immer oder so gut immer eine Auslegungspredigt sein. Sein Wort ist das Werkzeug, das er auserwählt hat, um uns Leben zu bringen.
\begin{itemize}
    \item Gott schuf die Himmel und die Erde mit seinem Wort.
    \item Gottes Volk wurde durch die Worte Gottes geschaffen.
    \item Gottes Wort erging nicht nur an Mose und seine Nachkommen, sondern auch an das ganze Volk Israel. Er berief es zu seinem Volk.
    \item In Hesekiel 37 sehen wir am deutlichsten, dass das Leben aus dem Wort Gottes kommt.
    \item In Christus ist das Wort Gottes vollständig und endgültig zu uns gekommen.
\end{itemize}
Wegen unserer eigenen Sündhaftigkeit können wir Gott nicht kennenlernen. Gott muss sich selbst offenbaren.
\begin{bibelbox}{SCHL}{Rom}{10:7}
    Demnach kommt der Glaube aus der Verkündigung, die Verkündigung aber durch Gottes Wort.
\end{bibelbox}
\begin{bibelbox}{SCHL}{Rom}{10:9}
    Denn wenn du mit deinem Mund Jesus als den Herrn bekennst und ih deinem Herzen glaubst, dass Gott ihn aus den Toten auferweckt hat, so wirst du gerettet.
\end{bibelbox}
Das Wort Gottes muss von zentraler Bedeutung sein, weil Gottes Geist sein Volk durch seine Worte erschafft. Letztendlich kann das Volk Gottes und die Gemeinde Gottes nur um das Wort Gottes herum erschaffen werden. Das Wort Gottes bringt Leben.
\textbf{Die Rolle des Wortes Gottes in der Heiligung}\\
Das Wort Gottes wird dazu gebraucht uns wachsen zu lassen.
\begin{bibelbox}{SCHL}{Mt}{4:4}
    Der Mensch lebt nicht von Brot allein, sondern von einem jedem Wort, das aus dem Mund Gottes hervorgeht.
\end{bibelbox}
\begin{bibelbox}{SCHL}{Ps}{119:105}
    Dein Wort ist meines Fusses Leuchte und ein Licht auf meinem Weg.
\end{bibelbox}
Wir brauchen Gottes Wort um gerettet zu werden. Eine Gesunde Gemeinde ist eine Gemeinde, die das Wort Gottes hört, und die es immer wieder hört.
\textbf{Die Rolle des Predigers des Wortes Gottes}\\
Ein Prediger der von Gott besonders begabt und zu diesem Dienst berufen worden ist, ist das Wichtigste wonach Sie in einer Gemeinde Ausschau halten können. Prediger sind nicht dazu berufen, das zu predigen, was laut Meinungsumfrage beliebt ist. Das Wort Gottes muss in aufrichtiger Weise vermittelt werden, damit wir nicht bloss das hören, was wir hören wollen, sondern damit vielmehr das hören, was Gott wirklich gesagt hat. Warum diese Priorität? Weil dieses Wort, \enquote{das Wort des Lebens} ist.

Gott gibt uns sein Wort und Gott gibt uns Glauben. Wir leben in einer Zeit des Glaubens. Gottes Wort ist das Wort, das wir heute hören müssen. Was heisst es, wenn wir sagen, dass die Bibel Gottes Wort ist? Es heisst, dass wir sie hören und ihr folgen müssen. Was nützt es als zu denken, man habe das Wort Gottes, wenn man es nicht beachtet. Die Gemeinde soll für die Prediger beten und nach solche Predigten streben.
\section*{Merkmal 2: Biblische Theologie}
Die Gleichgültigkeit gegenüber religiösen Überzeugungen passt zu dem Merkmal unserer Kultur, sich nicht an Details aufzuhängen. Wir brauchen das, was die Literaten unter uns eine \enquote{Metaerzählung} nennen -- einen Sinn. Heutzutage wird solchen übergeordneten Sinnkonstrukten eher Gleichgültigkeit, wenn nicht gar Feindseligkeit entgegengebracht.
Gottes Metaerzählung unterdrückt nicht, sondern befreit. Die Bibel lehrt und das Gott der Schöpfer ist, dass er heilig ist, dass er treu ist, dass er die Liebe ist und das er allmächtig ist.
\subsection*{Der Gott der Bibel ist ein Schöpfergott}
Gott ist ein Schöpfergott der die Welt erschaffen und der sich auf dieser Welt ein besonderes Volk erschaffen hat. Das Alte Testament gibt uns sehr konkrete, irdische Offenbarung davon wer Gott ist und wie er ist. Wir müssen erkennen, dass Gott der grosse Initiator ist, der grosse Geber, der Schöpfer der Welt, der Schöpfer seines Volkes, der Urheber unseres Glaubens. So ist Gott.
\subsection*{Der Gott der Bibel ist ein heiliger Gott}
Der Gott der Bibel ist ein heiliger Gott. Gott steht seiner Schöpfung nicht gleichgültig gegenüber. Sie lesen in der Bibel, dass es da ein Gott gibt, der eine Leidenschaft für Heiligkeit besitzt.
Die Sühne war nötig, weil wir mit diesem heiligen Gott irgendwie versöhnt werden mussten. Im Alten Testament ist dieser Sühnegedanke an ein Opfer gebunden -- als der Weg, auf dem Gott uns diese Wiedergutmachung und die Wiederherstellung unserer Beziehung zu ihm anbieten. Ein heiliger Gott musste von einem sündigen Volk getrennt sein. Diese Opfer wiesen auf die Wiederherstellung der Beziehung zwischen Gott und seinem Volk hin. Alle Opfer mussten freiwillig geschehen, teuer sein, vom Opfernden selbst kommen, von einem Sündenbekenntnis begleitet sein und Gottes Vorschriften entsprechen. Diese Opfer lehrten, dass Sünde den Menschen beschmutzte. Dieses Sühneopfer musste einmal Jährlich wiederholt werden, weil sie, wie gut oder Schlecht die Lage war, in einem permanenten sündigen Zustand waren, dass die Sünde den Menschen von Gott trennt, dass sie nie ein vollkommenes Opfer bringen konnte und dass es Gott selbst ist, der den Zugang zu ihm selbst eröffnet, indem er unsere Sünden vergibt.
\subsection*{Der Gott der Bibel ist ein treuer Gott}
Gott ist ein Schöpfergott, ein heiliger Gott und auch ein treuer Gott. Er ist ein Gott der Übertretungen und Sünde vergibt, aber nicht ungestraft lässt. Dies ist die Verheissung der Hoffnung auf die Erlösung von Gottes Volk. Gott verband die Prophetie über den Messias als König mit anderen Prophetien über den Messias als dem, der anstelle seines Volkes leiden würde. Er verband diese beiden prophetischen Linien miteinnander. Sowohl das Alte Testament als auch das neue Testament lehren, dass dieser königliche, leidende Messias unsere einzige Hoffnung ist. Jesus löst das Rätsel. Er war das fleischgewordene Wort Gottes. Er war das Lamm Gottes, das für die Sünden seines Volkes geschlachtet wurde. Jesus Christus war die treue Erfüllung von Gottes Verheissung. Unser Schöpfergott und unser heiliger Gott ist auch ein erstaunlich treuer Gott.
\subsection*{Der Gott der Bibel ist ein liebender Gott}
Im NT lesen wir, dass Gott aus Liebe zu seinem Bundevolk, alle seine Versprechen einhält. Der Gott der Bibel ist ein Gott der erstaunlichen Liebe!
\subsection*{Der Gott der Bibel ist ein allmächtiger Gott}
Gott bring seine Pläne zum Ziel. Er erfüllt seine Verheissungen. Gott wird weiterhin für uns sorgen und dass seine anhaltende Fürsorge nicht auf unserer Treue geründet, sondern auf seiner. Die Vorstellungen, die Sie von Gott haben, haben einen Einfluss auf die Art und Weise, wie Sie ihr Leben führen, und darauf, wie Sie sich Ihre Gemeinde wünschen. Sie müssen ein biblisches Verständnis von Gott haben. Wenn wir Gottes Wort hören und ihm glauben, dann beginnen wir von Neuem, diese Beziehung zu ihm zu haben, zu der er uns geschaffen hat. Das ist der Gott, dem wir vertrauen können und auch vertrauen sollen, weil sein Wort uns nicht enttäuschen wird. Darum und um nichts anderes geht es in der Bibel.
\section*{Merkmal 3: Das Evangelium}
Evangelium heisst gute Nachricht. Auch Jesus spricht von der Guten Nachricht. Was ist die Gute Nachricht?  Ist die Gute Nachricht, dass ich besser bin? Oder dass Gott Liebe ist? Oder Jesus mein Freund ist? Oder dass ich ein geordneteres Leben führen soll?

\subsection{Die Gute Nachricht lautet nicht nur, dass ich OK bin}
Die Bibel lehnt den Gedanken Grundlegend ab, dass wir Ok seien, dass der Zustand des Menschen ganz in Ordnung sie, dass jedemann in Wirklichkeit nur seine gegenwärtige Situation, seine Endlichkeit, seine Begrenztheit und seine Unvollkommenheit zu akzeptieren brauche oder dass wir uns einfach auf die Sonnenseite des Lebens zu konzentrieren brauchen.

Und deshalb sind wir weder Gerecht, noch stehen wir zu Gott in einer guten Beziehung. In der Tat sind unsere Sünden so schwerwiegend, dass wir ein ganz neues Leben brauchen.
So, dass sich Gottes Gesetz in unserem Charakter wiederspiegelt.

Es geht aber darum, was es über unsere Beziehung zu Gott selbst aussagt, wenn wir wissentlich sein Gesetz missachten. Wir können nicht sagen, dass wir Christen seien, und trotzdem wissentlich, widerholt und fröhlich das Gesetz brechen.

Wir fühlen uns nicht nur schuldig, wir sind tatsächlich vor ihm schuldig. Wir fühlen uns nicht nur in einem inneren Konflikt, sondern wir befinden uns auch tatsächlich in einem Konflikt mit Gott. Die Botschaft Jesus Christi ist die, uns lehren, mit einer Sehnsucht nach Veränderung, mit einem wachsenden Glauben und mit einer festen und sicheren Hoffnung auf das, was kommt, zu leben.
\subsection{Die Gute Nachricht lautet nicht bloss, dass Gott Liebe ist}
Die Bibel spricht von der Heiligung, ohne die niemand den Herrn sehen wird. Erst wenn wir beginnen zu erfassen, dass seine Liebe eine Tiefe, eine Struktur, eine Fülle und eine Schönheit hat, über die wir in unserem gegenwärtigen Zustand nur staunen können.
\subsection{Die Gute Nachricht lautet nicht bloss, dass Jesus unser Feund sein will}
Es geht nicht bloss darum, eine Beziehung zu plfegen oder einem Vorbild zu folgen. Der Sohn des Menschen ist gekommen, um sein Leben zu geben als Lösegeld für viele.

Das Werk Christi wird also als Erlösung beschrieben -- als ein Freikauf, durch den die Freiheit bestimmter unterdrückter Menschen sichergestellt wird. Das Werk Christi wird als Versöhnung bezeichnet -- wo die Feindselilgkeiten zwischen zwei Menschen beigelegt werden. Das Werk Christi wird auch als Versöhnung beschrieben -- als eine Besänftigung von Gottes gerechtem Zorn gegen die Menschen wegen ihrer Sünden. Gottes Zorn wird besänftigt, sodass er die Sünder gerecht behandeln kann -- auf der Grundlage seiner Liebe statt seines Zorns.
\subsection{Die Gute Nachricht lautet nicht bloss, dass ich ein geordneteres Leben führen soll}
Das evangelium ist eine Botschaft der wunderbaren Guten Nachricht für diejenigen, die ihre Verzweiflung vor Gott kennen und sie sich eingestehen. Nach der Bibel sollte ihre Antwort heissen, dass sie Busse tun und zum Glauben kommen.

Mit der Busse kommt der Glaube. Als Erstes müssen wir ehrlich meinen, dass das, was das Evangelium sagt, wahr ist. In dieser Hinsicht müssen wir daran glauben. Der Glaube das völlige Vertrauen auf die Gute Nachricht von der Erlösung. Wir sollen zu der Erkenntnis gelangen, dass wir uns ganz auf Gott verlassen und, was unsere Erlösung betrifft, auf Christus allein vertrauen.Er fordert, dass sich unser Leben tatsächlich andert. Jetzt hat ein Neues Leben begonnen. Eine persönliche Beziehung zu ihrem Gott, ihrem SChöpfer, jetzt und für immer.
\section{Merkmal 4: Ein biblisches Verständnis von Bekehrung}
\subsection{Ist Veränderung nötig?}
Ja, dass er ihnen eine bestimmte Sünde ins Bewusstsein bringt und dass diese bestimmte Sünde dann ernster erschein als bisher.
\subsection{Ist Veränderung wirklich möglich?}
Die sagt nicht nur, dass wir Veränderung brauchen, sondern das Veränderung auch möglich ist.
\subsection{Welche Veränderung ist nötig?}
Die whare Veränderung, die wir brauchen, ist diese Veränderung weg von der Anbetung des eigenen Ichs hin zur Anbetung Gottes, weg von der eigenen Schuld vor Gott hin zur Vergebung in Christus. Das ist die Veränderung, die wir brauchen.
\subsection{Was gehört zu dieser Veränderung?}
Das Abwenden von unseren Sünden und die Hinwendung zu Gott. Dazu gehört, dass wir über unsere Sünden Busse tun und Gott nachfolgen.

Die Bibel sagt, dass die wahre Veränderung in der christlichen Bekehrung im Vertrauen allein auf Christus liegt. Bei dieser grossen Veränderung geht es vor allem um das Erkennen, dass wir nie oft genug darum bemühen können, Gottes Wohlwollen zu verdienen. Wir müssen erkennen, dass wir aufgrund unserer Sünden vor Gott wirklich in einer verzweifelten Lage sind.
\subsection{Wie geschieht diese grosse Veränderung?}
Jesus sagt allen Menschen, dass sie von ihren Sünden umkehren und sich Gott zuwenden müssen. Es muss diese Veränderung in uns bewirken, damit wir die geistlichen Wahrheiten der Bibel annehmen können.

Deshalb beten wir, dass Gott in seiner grossen Liebe seinen Geist ausgiessen möge, damit das Evangelium treu gepredigt wird und er Menschen retten möge.

Laut der Bibel sind unsere Bussen und unser Glaube Gottes Geschenk an uns; unsere Bekehrung -- unsere grosse Veränderung -- geschieht nur durch Gottes Gnade. Gott braucht viele Wege um uns die Gaben der Busse und des Glaubens zu schenken.

Die Veränderung, die sie nötig haben, könnte ihr eigenes Vermögen zu übersteigen scheinen. Doch die Gute Nachricht ist: sie übersteigt nicht Gottes Vermögen. Sie müssen nur die Worte Jesu beherzigen: \enquote{Kehrt um und glaubt an das Evangelium.}
\end{document}