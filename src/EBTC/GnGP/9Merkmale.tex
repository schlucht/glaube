\author{OTS}
\documentclass{../../inc/mybib}

\setincpath{../../inc/}

\usepackage{bible_style}
\graphicspath{{../../assets/images/}}


\begin{document}
\section*{Neun Merkmale einer gesunden Gemeinde}
\subsection*{Merkmal 1: Auslegungspredigt}
\subsubsection*{Auslegungspredigten}
Es gibt zwei Arten von Predigten:
\begin{itemize}
    \item Themenpredigten
    \item Auslegungspredigten
\end{itemize}
Themenpredigten beschäftigen sich mit einem bestimmten Thema, die der Prediger auswählt. Die Predigt rund um dieses Thema aufgebaut, mit Bibelversen und eigenen Erfahrungen.

Die Auslegungspredigt nimmt einen Bibeltext und ist an diesen Text gebunden. Eine Auslegungspredigt ist nicht nur ein mündlicher Kommentar zu einem Bibeltext, vielmehr geht darum herauszufinden, welche Hauptaussage diesem bestimmten Bibeltext entspricht. Der Wille zur Auslegungspredigt ist der Wille, Gottes Wort zu hören. NIcht bloss zu bejahen, sondern sich auch tatsächlich diesem zu unterstellen.

Die Hauptaufgabe wie auch die Hauptaufgabe eines jeden Pastors, ist das Halten von Auslegungspredigten. Prediger haben ausdrücklich den Auftrag, hinzugehen und das Wort Gottes zu predigen.

Jemanden mit der geistlichen Leitung einer Gemeinde zi beauftragen, der nicht im praktischen Leben den festen Willen an den Tag legt, Gottes Wort zu hören und zu lehren, heisst, das Wachstum der Gemeinde zu hemmen.

Wonach und Christen verlangen sollte ist das Wort Gottes zu hören.

\subsubsection*{Die Zentrale Rolle des Wortes Gottes}
\subsection*{Merkmal 2: Biblische Theologie}
\subsubsection*{Der Gott der Bibel ist ein Schöpfergott}
\subsubsection*{Der Gott der Bibel ist ein heiliger Gott}
\subsubsection*{Der Gott der Bibel ist ein treuer Gott}
\subsubsection*{Der Gott der Bibel ist ein liebender Gott}
\subsubsection*{Der Gott der Bibel ist ein allmächtiger Gott}


\end{document}