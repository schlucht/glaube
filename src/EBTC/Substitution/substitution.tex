\author{OTS}
\documentclass{../../inc/mybib}

\setincpath{../../inc/}

\usepackage{bible_style}
\graphicspath{{../../assets/images/}}
\usepackage{header}
\newcommand{\st}{Substitutionstheolog}
\newcommand{\sz}{Superzessionismus}
\begin{document}
    \tableofcontents
    \newpage
    \section{Allgemein}
    Zusammenfassung zu dem Buch \enquote{Hat die Gemeinde Israel ersetzt?}. Ein Buch von Michael J. Vlach.
    \section{Kapitel 1}
    \subsection{Eine Definition der \st{}}
    Die Substitutionstheologie vertritt die Auffassung, dass die Gemeinde das Israel oder das Volk Gottes abgelöst hätte.

    Eine gemeinsame Bezeichnung, die in der wissenschaftlichen Literatur verwendet wird, ist der Begriff \textit{\sz}. Der Begriff Ersatztheologie und \sz{} kann vielfach synonym betrachtet werden.

    Der \sz{} beruht auf zwei grundlegenden Überzeugungen:
    \begin{itemize}
        \item das Volk Israel hat die Rolle als Volk Gottes abgeschlossen und erlangt diese nie mehr.
        \item die Gemeinde ist das wahre Volk Israel. Sie ist neu das eigentliche Volk Gottes und hat Israel als Volk Gottes für immer ersetzt.
    \end{itemize}
    \subsection{Variationen innerhalb des \sz{}}
    Drei Hauptformen lassen sich voneinander unterscheiden:
    \begin{enumerate}
        \item puntitive \sz
        \item ökonomische \sz
        \item struckturelle \sz
    \end{enumerate}
    \subsubsection{puntitive \sz}
    \begin{description}
        \item[punitiv] bestrafende \sz 
    \end{description}
    Israel wurde von der Gemeinde abgelöst, weil das Volk böse gehandelt und seine Stellung als Volk Gottes deswegen eingebüsst hatte.\\
    Der punitive \sz{} beruht auf der Auffassung, dass Gott die Juden aufgrund ihres Ungehorsams und ihrer Ablehnung Christi verstiess.\\
    
    \textbf{Vertreter dieser Auffassung sind:}
    \begin{itemize}
        \item Hippolyt (ca. 205)
        \item Origenes (ca. 185--245)
        \item Lactantius (ca. 303--313)
        \item Martin Luther
    \end{itemize}
    \subsubsection{ökonomische \sz}
    Der ökonomische \sz behauptet, Gott habe von Anfang an geplant, dass Volk Israels Rolle als Volk Gottes mit dem Kommen Christi und der Gründung der Gemeinde zu Ende gehen sollte.\\
    Letzten Endes wurde Israel nicht so sehr wegen seines Ungehormsams, sondern wegen der Beendigung seiner geschichtlichen Rolle im Heilsplan durch die Gemeinde abgelöst.
    \textbf{Vertreter dieser Auffassung sind:}
    \begin{itemize}        
        \item Karl Barth
    \end{itemize}
    \subsubsection{strukturelle \sz}
    Diese Form wird ist von Craig Soulen. Bei der strukturellen \sz{} geht es um einen hermeneutischen Ansatz oder eine bestimmte Sichtweise hinsichtlich der jüdischen Schriften.\\
    Soulen behauptet, dass die meisten \sz{} einen hermeneutischen Ansatz vertreten, der die hebräischen Schriften des Alten Testaments entweder ignorieren oder ihr Bedeutung beraubt.
    \begin{quote}
        \enquote{die strukturelle Variante des \sz{} hat eine tief sitzende Tradition begründet, durch die das ethische und nationale Israel bei der theologischen Auseinandersetzung mit der Schrift ausgeschlossen wird.}
    \end{quote}
    \section{Kapitel 2}
    \subsection{Die \st ie und die Zukunft Israel}
    Die \st ie führt stehts zu der Erkenntnis, das Israel im Plan Gottes überhaupt keine Zukunft mehr hat. Gott ist mit Israel fertig, und daran lässt sich nichts ändern.
    Es gibt aber andere die an eine Zukunft für Israel glauben.
    \subsection{Rettung und Wiederherstellung}
    Einige \st en glauben zwar an eine zukünftige Errettung Israels; es gibt aber nicht an die vollständige Wiederherstellung von Israel.\\
    Die Errettung besagt, dass das Volk der Juden am Ende der Zeit zum Glauben an Christus kommen.\\
    Die Wiederherstellung Israels ist nicht nur die Errettung, sondern auch die Rückführung in ihr Land und eine einzigartige Rolle erlangen.

    \subsection{Der gemässigte \sz}
    In einer milderen oder moderateren Form wird zwar gegelaubt, dass die Gemeinde Israel das Volk Gottes abgelöst hat, dass sich aber auch eine grosse Anzahl von Juden bekehren und der christlichen Gemeinde anschliessen wird.\\
    Nur so können die Verfechter die Aussagen von Paulus in Römer 9-11 und Galater 3 erklären.
    \section{Kapite 3}
    \subsection{Die Ursache der \st ie}
    Die \st ie entwickelte sich im Laufe der Kirchengeschichte zu einer populären Ansicht, die von vielen Christen vertreten wurde und noch vertreten wird. Wie kam es dazu, dass der \sz{} eine so weite Verbreiteitung fand und weithin akzeptiert wurde?
    \subsubsection{Verschiedene Erklärungen für den Ursprung der \st ie}
    Die Enwicklung begann bereits in der frühen Kirchengeschichte.Justinius ist darin eine wichtige Persönlichkeit, weil er der erste Kirchenvater war die Gemeine explizit als das \blockquote{wahre geistliche Israel} bezeichnete.\\
    Historisch gesehen, war Justinius keineswegs der Begründer dieser Sichtweise. Es gab schon vor Justinius eine Verbreitung des \sz. Es ist schwierig ein konkretes Datum festzulegen.

    Bei der Verbreitung und Azeptanz des \sz{} spielen drei Faktoren eine wesentliche Rolle:
    \begin{enumerate}
        \item die wachsende Anzahl von Nichtjuden in der frühen Gemeinde.
        \item die Position der Gemeinde zu den Zerstörungen Jerusalem 70 n. Chr. und 135 n. Chr.
        \item die Entwicklung einer Hermeneutik, die es der Gemeinde erlaubt, die Verheissungen Israels für sich selbst in Anspruch zu nehmen.
    \end{enumerate}
    \subsection{Die wachsende Anzahl von Nichtjuden in der frühen Gemeinde}
    Die Anzahl der Nichtjuden innerhalb der Gemeinde nahm immer zu, während der Einfluss der jüdischen Mitglieder zusehends abnahm. Der fehlgeschlagene zweite Aufstand unter Bar-Kochba 132--135 n. Chr. trug ebenfalls dazu bei, dass der jüdischen Einfluss auf die Gemeinde abnahm. Es ging soweit, dass im vieren Jahrhundert im Konzil von Nicäa kein einziger Jude mehr unter den Bischöfen. Diese wachsende heidenchristliche Gegenwart in der Gemeinde führte zu theologischen Debatten über den Status der Juden vor Gott.
    \subsection{Die Position der Gemeinde zu den zwei Zerstörungen Jerusalems}
    In Augen vieler Juden waren die jüdischen Christen Verräter, weil die sich geweigert hatten, ihren Landsleuten in den Kämpfen gegen Rom beizustehen. Als folge nahm der jüdischen Widerstand gegen das Christentum zu.

    Viele Christen sahen die Zerstörungen Jerusalems als Gottes Strafgericht über Israel, weil sie Christus abgelehnt hatten. Zitat Justinius.
    \begin{quote}
        Deswegen sind diese Dinge mit Recht und Gerechtigkeit an euch geschehen, denn ihr habt den Gerechten ermordet und die verstossenen, die ihre Hoffnung auf ihn setzen.
    \end{quote}
    Zitat Richardson:
    \begin{quote}
        Der Krieg bewirkte, was der Synagogenbann nicht bewirken konnte: Er führte zu einer Trennung der beiden Gruppen, befreit die Christen von der Pflicht, einen engen Kontakt zum Judentum aufrechthalten zu müssen, und liefert ihnen den Beweis für das göttliche Strafgericht über Israel.
    \end{quote}
    Zitat McDonald:
    \begin{quote}
        Die Kirchenväter zogen aus der offensichtlichen Verstossung der Juden durch Gott, wie sie an der Zerstörung des Tempels und der Vertreibung der Bewohner Jerusalems sichtbar wurde, den Schluss, dass es hinfort ein neues Israel gab, das aus Christen bestand.
    \end{quote}
    \subsection{Hermeneutischen Ansätze}
    Der erste Faktor, die Gemeinde übernahm auch die Bibel der Juden die Septuaginta(LXX). Der zweite Einfluss der zunehmende Einfluss der griechischen Philosophie auf die Auslegung der Bibel. Die zunehmende Dominanz der allegorischen Methode der Interpretation. Der dritte Faktor war die Auffassung der Gemeinde, sie sei die echte Fortsetzung des alttestamentlichen Glaubens. Die Gemeinde glaubte auch, dass sie die Erbin der Bündnisse Israels sei.
    
\end{document}