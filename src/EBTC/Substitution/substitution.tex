\author{OTS}
\documentclass{../../inc/mybib}

\setincpath{../../inc/}

\usepackage{bible_style}
\graphicspath{{../../assets/images/}}
\usepackage{longtable}
\usepackage{tabularx} % Stelle sicher, dass dieses Paket im Präambel geladen ist
\newcommand{\st}{Substitutionstheolog}
\newcommand{\sz}{Superzessionismus}
\begin{document}
    \tableofcontents
    \newpage
    \section{Allgemein}
    Zusammenfassung zu dem Buch \enquote{Hat die Gemeinde Israel ersetzt?} Ein Buch von Michael J. Vlach.
    \section{Kapitel 1}
    \subsection{Eine Definition der \st{}}
    Die Substitutionstheologie vertritt die Auffassung, dass die Gemeinde das Israel oder das Volk Gottes abgelöst hätte.

    Eine gemeinsame Bezeichnung, die in der wissenschaftlichen Literatur verwendet wird, ist der Begriff \textit{\sz}. Der Begriff Ersatztheologie und \sz{} kann vielfach synonym betrachtet werden.

    Der \sz{} beruht auf zwei grundlegenden Überzeugungen:
    \begin{itemize}
        \item Das Volk Israel hat die Rolle als Volk Gottes abgeschlossen und erlangt diese nie mehr.
        \item Die Gemeinde ist das wahre Volk Israel. Sie ist neu das eigentliche Volk Gottes und hat Israel als Volk Gottes für immer ersetzt.
    \end{itemize}
    \subsection{Variationen innerhalb des \sz{}}
    Drei Hauptformen lassen sich voneinander unterscheiden:
    \begin{enumerate}
        \item puntitive \sz
        \item ökonomische \sz
        \item strukturelle \sz
    \end{enumerate}
    \subsubsection{puntitive \sz}
    \begin{description}
        \item[punitiv] bestrafende \sz 
    \end{description}
    Israel wurde von der Gemeinde abgelöst, weil das Volk böse gehandelt und seine Stellung als Volk Gottes deswegen eingebüsst hatte.
    Der punitive \sz{} beruht auf der Auffassung, dass Gott die Juden aufgrund ihres Ungehorsams und ihrer Ablehnung Christi verstiess.\\
    
    \textbf{Vertreter dieser Auffassung sind:}
    \begin{itemize}
        \item Hippolyt (ca. 205)
        \item Origenes (ca. 185--245)
        \item Lactantius (ca. 303--313)
        \item Martin Luther
    \end{itemize}
    \subsubsection{ökonomische \sz}
    Der ökonomische \sz behauptet, Gott habe von Anfang an geplant, dass Volk Israels Rolle als Volk Gottes mit dem Kommen Christi und der Gründung der Gemeinde zu Ende gehen sollte.\\
    Letzten Endes wurde Israel nicht so sehr wegen seines Ungehorsams, sondern wegen der Beendigung seiner geschichtlichen Rolle im Heilsplan durch die Gemeinde abgelöst.
    \textbf{Vertreter dieser Auffassung sind:}
    \begin{itemize}        
        \item Karl Barth
    \end{itemize}
    \subsubsection{strukturelle \sz}
    Diese Form wird ist von Craig Soulen. Bei der strukturellen \sz{} geht es um einen hermeneutischen Ansatz oder eine bestimmte Sichtweise hinsichtlich der jüdischen Schriften.\\
    Soulen behauptet, dass die meisten \sz{} einen hermeneutischen Ansatz vertreten, der die hebräischen Schriften des Alten Testaments entweder ignorieren oder ihr Bedeutung beraubt.
    \begin{quote}
        \enquote{Die strukturelle Variante des \sz{} hat eine tief sitzende Tradition begründet, durch die das ethische und nationale Israel bei der theologischen Auseinandersetzung mit der Schrift ausgeschlossen wird.}
    \end{quote}
    \section{Kapitel 2}
    \subsection{Die \st ie und die Zukunft Israel}
    Die \st ie führt stets zu der Erkenntnis, das Israel im Plan Gottes überhaupt keine Zukunft mehr hat. Gott ist mit Israel fertig, und daran lässt sich nichts ändern.
    Es gibt aber andere die an eine Zukunft für Israel glauben.
    \subsection{Rettung und Wiederherstellung}
    Einige \st en glauben zwar an eine zukünftige Errettung Israels; es gibt aber nicht an die vollständige Wiederherstellung von Israel.\\
    Die Errettung besagt, dass das Volk der Juden am Ende der Zeit zum Glauben an Christus kommen.\\
    Die Wiederherstellung Israels ist nicht nur die Errettung, sondern auch die Rückführung in ihr Land und eine einzigartige Rolle erlangen.

    \subsection{Der gemässigte \sz}
    In einer milderen oder moderateren Form wird zwar geglaubt, dass die Gemeinde Israel das Volk Gottes abgelöst hat, dass sich aber auch eine grosse Anzahl von Juden bekehren und der christlichen Gemeinde anschliessen wird.\\
    Nur so können die Verfechter die Aussagen von Paulus in Römer 9-11 und Galater 3 erklären.
    \section{Kapitel 3}
    \subsection{Die Ursache der \st ie}
    Die \st ie entwickelte sich im Laufe der Kirchengeschichte zu einer populären Ansicht, die von vielen Christen vertreten wurde und noch vertreten wird. Wie kam es dazu, dass der \sz{} eine so weite Verbreitung fand und weithin akzeptiert wurde?
    \subsubsection{Verschiedene Erklärungen für den Ursprung der \st ie}
    Die Entwicklung begann bereits in der frühen Kirchengeschichte. Justinus ist darin eine wichtige Persönlichkeit, weil er der erste Kirchenvater war die Gemeine explizit als das \blockquote{wahre geistliche Israel} bezeichnete.\\
    Historisch gesehen, war Justinus keineswegs der Begründer dieser Sichtweise. Es gab schon vor Justinus eine Verbreitung des \sz. Es ist schwierig ein konkretes Datum festzulegen.

    Bei der Verbreitung und Akzeptanz des \sz{} spielen drei Faktoren eine wesentliche Rolle:
    \begin{enumerate}
        \item die wachsende Anzahl von Nichtjuden in der frühen Gemeinde.
        \item Die Position der Gemeinde zu den Zerstörungen Jerusalem 70 n. Chr. und 135 n. Chr.
        \item die Entwicklung einer Hermeneutik, die es der Gemeinde erlaubt, die Verheissungen Israels für sich selbst in Anspruch zu nehmen.
    \end{enumerate}
    \subsection{Die wachsende Anzahl von Nichtjuden in der frühen Gemeinde}
    Die Anzahl der Nichtjuden innerhalb der Gemeinde nahm immer zu, während der Einfluss der jüdischen Mitglieder zusehends abnahm. Der fehlgeschlagene zweite Aufstand unter Bar-Kochba 132--135 n. Chr. trug ebenfalls dazu bei, dass der jüdische Einfluss auf die Gemeinde abnahm. Es ging so weit, dass im vierten Jahrhundert im Konzil von Nicäa kein einziger Jude mehr unter den Bischöfen. Diese wachsende heidenchristliche Gegenwart in der Gemeinde führte zu theologischen Debatten über den Status der Juden vor Gott.
    \subsection{Die Position der Gemeinde zu den zwei Zerstörungen Jerusalems}
    In Augen vieler Juden waren die jüdischen Christen Verräter, weil die sich geweigert hatten, ihren Landsleuten in den Kämpfen gegen Rom beizustehen. Als Folge nahm der jüdische Widerstand gegen das Christentum zu.

    Viele Christen sahen die Zerstörungen Jerusalems als Gottes Strafgericht über Israel, weil sie Christus abgelehnt hatten. Zitat Justinus.
    \begin{quote}
        Deswegen sind diese Dinge mit Recht und Gerechtigkeit an euch geschehen, denn ihr habt den Gerechten ermordet und die verstossenen, die ihre Hoffnung auf ihn setzen.
    \end{quote}
    Zitat Richardson:
    \begin{quote}
        Der Krieg bewirkte, was der Synagogenbann nicht bewirken konnte: Er führte zu einer Trennung der beiden Gruppen, befreit die Christen von der Pflicht, einen engen Kontakt zum Judentum aufrecht halten zu müssen, und liefert ihnen den Beweis für das göttliche Strafgericht über Israel.
    \end{quote}
    Zitat McDonald:
    \begin{quote}
        Die Kirchenväter zogen aus der offensichtlichen Verstossung der Juden durch Gott, wie sie an der Zerstörung des Tempels und der Vertreibung der Bewohner Jerusalems sichtbar wurde, den Schluss, dass es hinfort ein neues Israel gab, das aus Christen bestand.
    \end{quote}
    \subsection{Hermeneutischen Ansätze}
    Der erste Faktor, die Gemeinde übernahm auch die Bibel der Juden die Septuaginta(LXX). Der zweite Einfluss der zunehmende Einfluss der griechischen Philosophie auf die Auslegung der Bibel. Die zunehmende Dominanz der allegorischen Methode der Interpretation. Der dritte Faktor war die Auffassung der Gemeinde, sie sei die echte Fortsetzung des alttestamentlichen Glaubens. Die Gemeinde glaubte auch, dass sie die Erbin der Bündnisse Israels sei.
    \section{Kapitel 4}
    \subsection{Die \st ie in der Zeit der Kirchenväter}
    Während dieser Zeit die Lehre sowohl Verstossung als auch der Hoffnung.
    Folgende Kirchenväter vertraten die \st ie:
    \begin{itemize}
        \item Justinus (ca. 100-165)
        \item Irenäus (130-200)
        \item Clemens von Alexandria (ca. 195)
        \item Tertullian (ca. 197)
        \item Cyprian (ca. 250)
    \end{itemize}
    
    Justinus war der erste Autor, der die Gemeinde ausdrücklich als \enquote{Israel} bezeichnete. Justinus war nicht der Begründer des \sz, doch er trat im Hinblick auf Israel und die Gemeinde offen für eine Ersatztheologie ein.

    Origenes ein einflussreicher Theologe lehrte, dass Israel von Gott verstossen worden war und dass die Gemeinde das neue Israel sei. Sein Wirken war ausschlaggebend dafür, dass die allegorische Methode bei der Auslegung von Schrifttexten über Israel im Christentum allgemein akzeptiert wurde. Für Origenes war die geistliche Bedeutung hinter dem wörtlichen Sinn sehr wichtig.

    Augustinus stellte ausdrücklich fest, dass der Name \enquote{Israel} der christlichen Gemeinde gehört. Die superzessionischten Ansichten, die Augustinus vertrat, stammte zum Grossteil nicht von ihm selbst, sondern entsprachen in den meisten Punkten der patristischen Tradition, die ihm voranging. Die Existenz nicht christlicher Juden war für ihn somit kein Problem, sondern vielmehr ein notwendiges Zeugnis für die Echtheit des Christentums.

    \subsection{Hoffnung für Israel}
    Viele Kirchenväter vertraten die moderatere Variante dieser superzessionischten Ansicht. Viele glaubten auch an eine zukünftige Errettung Israels.

    Für Justinus war die Hoffnung für Israel, wie sie im Alten Testament beschrieb, eine lebendige Hoffnung.

    Auch Tertullian vertrat eine gemässigte Variante:
    \blockquote[Terullian]{Er [Gott] wird auch die Beschneidung, die Nachkommenschaft Abrahams, die ihn nach  und nach anerkennen, wird, mit seiner Annahme und Segnung beschenken}

    Der Glaube an eine zukünftige Errettung der Juden wurde auch von Kyrill von Jerusalem (ca. 350) und von Chysostomus (349-407) vertreten. 

    Augustinus bezeichnet seine Ansicht über die Errettung der Juden als einen Gedanken, mit dem die Gläubigen seiner Zeit vertraut waren. Der Glaube an die Errettung der Juden war somit nicht nur seine eigene Ansicht, sondern war in seiner Generation weit verbreitet.

    Folgende Kirchenväter vertraten die Hoffnung, dass die Juden errettet werden:
    \begin{itemize}
        \item Hieronymus (ca. 347-420)
        \item Tiro von Aquitanien (ca. 390-455)
        \item Ambrosius (ca. 340-397)
        \item Theodor von Cyrus (ca. 393-457)
    \end{itemize}
    Auch wenn die Kirchenväter glaubten, dass das Volk Israel aufgrund seines Ungehorsams und seiner Ablehnung Christi von Gott verstossen worden war, so glaubten sie doch an eine zukünftige Errettung Israels. Sie glaubten, dass Gott die Juden am Ende der Zeit retten würde. Diese Hoffnung wurde von den Kirchenvätern in ihren Schriften immer wieder erwähnt.
    \section{Kapitel 5}
    \subsection{Die \st ie im Mittelalter}
    Wie in der Ära der Kirchenväter davor war der \sz{} auch im Mittelalter eine weit verbreitete Ansicht. Die Christen im Mittelalter meinten, dass die Juden schlussendlich Christus annehmen und errettet werden würden, doch gleichzeitig betrachteten sie die Juden als gefährliche Ungläubige, die von Gott verstossen und bestraft worden waren.

    In der mittelalterlichen Kunst, wurde die Gemeinde oft als die Siegerin gesehen, während Israel als besiegt und verstossen galt.

    Thomas von Aquin (1225-1274) sah die Juden vor allem wegen ihrer historischen Rolle wichtig. Er stimmte mit Augustinus darin überein, dass die Juden ein Beweis für die Echtheit des christlichen Glaubens und die Erfüllung alttestamentlicher Prophetien sind. Aquin glaubte ausserdem auch an eine zukünftige Bekehrung der Juden als Volk, und er begründete dies mit seiner Interpretation von \bibleverse{Rom}(11:25-26).
    \section{Kapitel 6}
    \subsection{Die \st ie in der Reformation}
    Von Seiten der Reformatoren gab es keine einheitliche Stellungnahme zur \st ie. Einige Reformatoren vertraten die \st ie, während andere sie ablehnten.

    Der frühe Luther hat wohlwollend über die Juden geschrieben. Luther glaubte auch an eine besondere Stellung der Juden im Plan Gottes.\\
    In seinen späteren Jahren änderte sich Luthers Einstellung zu den Juden jedoch dramatisch. 1543 veröffentlichte er seine Schrift \enquote{Von den Juden und ihren Lügen}, in der er die Juden als Teufel bezeichnete. Luther glaubte, dass die Juden von Gott verworfen worden waren, weil sie Christus abgelehnt hatten. Er vertrat die Ansicht, dass die Juden nicht mehr das Volk Gottes seien und dass sie von Gott bestraft würden. Luther machte auch Aussagen die mit einer punitiven Ersatztheologie übereinstimmen.

    Calvin vertrat die antijüdischen Haltung seiner Zeit, war in seiner Haltung gegenüber den Juden und Israel jedoch gemässigter als Luther. Calvin betrachtet die Gemeinde als das neue Israel. An anderer Stellen findet sich auch Hinweise, die seinen Glauben an eine konkrete Zukunft für das jüdische Volk bezeugen.
    \section{Kapitel 7}
    \subsection{Die \st ie in der Neuzeit}
    Die superzessionistische Position ist nach wie vor populären und wird von vielen vertreten.

    Immanuel Kant (1724-1804) lehnte die Meinung ab, das Christentum hätte eine direkte Beziehung zum Judentum und dem jüdischen Volk. Er sagt, die christliche Lehre sei somit nichts anderes als eine Geschichte des Triumphs des erschaffenen, universellen Geistes über das geschichtliche individuelle Fleisch.

    Friedrich Schleiermacher (1768-1834) leugnete wie Kant jegliche Verbindung zwischen dem alttestamentlichen Judentum und dem Christentum. Er vertrat die Form des strukturellen \sz, in dem die hebräischen Schriften nicht mehr als Faktor für die Erkenntnis des Handelns Gottes mit seiner Schöpfung betrachtet werden.

    Karl Barth (1886-1968) sah einen notwendigen Zusammenhang zwischen Israel und der Gemeinde, die auf dem Auserwählten, auf Jesus Christus, beruhte. Er lehnte den punitive \sz{} ab. Er vertrat eine Form des ökonomischen \sz, in der Israel einzigartige Rolle mit dem Tod und der Auferstehung Christi ein Ende fand.

    \subsection{Der Holocaust und der moderne Staat Israel}
    Der Holocaust die Überlegungen zu den Juden und dem Judentum neu aufgeworfen. Folgende Fragen wurden gestellt:
    \begin{itemize}
        \item Wie sollen die Christen über die Existenz des jüdischen Volkes denken?
        \item Ist die Gemeinde tatsächlich das neue Israel?
        \item Welche Bedeutung haben Israels Sünden?
        \item Welche Bedeutung haben das Land und der Staat Israel?
    \end{itemize}

    Die Antworten auf diese Fragen in jüngeren Jahren deuten auf zunehmenden Widerstand gegen die \st ie hin.

    \subsection{Gemeinde und Konfessionen seit dem Holocaust}
    Im Dokument \textit{Nostra Aetate} des Zweiten Vatikanischen Konzils (1965) wird die \st ie verworfen. Die katholische Kirche erkennt an, dass die Juden nicht für den Tod Christi verantwortlich gemacht werden können. Sie betont, dass die Juden Gottes auserwähltes Volk sind und dass sie eine besondere Rolle im Heilsplan Gottes spielen.

    Das protestantische Konzil sagt: \enquote{Die Behauptung, die Gemeinde sei das einzige neue Israel Gottes, lässt sich keineswegs anhand der Bibel begründen.} 
    Holwerda weist darauf hin, dass die traditionelle Ansicht, die christliche Gemeinde habe das jüdische Volk verdrängt und dieses spiele im Erlösungsplan Gottes keine Rolle mehr, nicht mehr dominant ist.

    \subsection{Historische Jesusforschung}
    In jüngerer Zeit argumentierten einige Theologen, dass die Mission des historischen Jesus im Zusammenhang mit seiner Vision für ein wiederhergestelltes Israel verstanden werden muss. Die Überzeugung, wonach sich die Mission Jesu direkt auf die Wiederherstellung des Volkes Israel bezog, hat wesentliche Auswirkungen auf die \st.

    \subsection{Zusammenfassung}
    Obwohl es schwer vorstellbar ist, dass die \st{} aus der theologischen Landschaft völlig verschwinden wird, ist es unwahrscheinlich, dass der \sz{} in der nahen Zukunft seinen dominanten Status zurückgewinnen wird. Die jüngere Geschichte steht somit auf der Seite des Nonsuperzessionismus.

    \section{Kapitel 11 - 13}
    \subsection{Theologische Argumente für die \sz}
    Diese häufigsten 5 Argumente werde oft aufgeführt:
    \begin{enumerate}
        \item Israel wurde als Volk Gottes für immer verworfen. \bibleverse{Mat}(21:43)
        \item Die Anwendung alttestamentlicher Begriffe auf die Gemeinde deuten darauf hin, dass die Gemeinde als das neue Israel angesehen werden muss \bibleverse{Gal}(6:16), \bibleverse{1Petr}(2:9-10), \bibleverse{Gal}(3:7.29).
        \item Die Einheit von Juden und NIchfuden schliesst eine zukünftige Rolle oder Funtkion des Volkes Israel aus. \bibleverse{Eph}(2:11-22), \bibleverse{Rom}(11:17-24).
        \item Die Beziehung der Gemeinde zum neuen Bund macht deutlich, dass die Gemeinde die alleinige Erbin der alttestamentlichen Bundesverheissungen ist, die ursprünglich dem Volk Israel galten. \bibleverse{Hebr}(8:6-13).
        \item Das Schweigen des Neuen Testaments im Hinblick auf eine nationale Wiederherstellung Israels beweist, dass mit einer solchen Wiederherstellung nicht zu rechnen ist.
    \end{enumerate}
    \newpage
   
\renewcommand{\arraystretch}{1.3} % Optional für besseren Zeilenabstand

\begin{longtable}{|p{7cm}|p{7cm}|}
    \caption{Biblische Argumente – Pro und Kontra Israel-Ersatz} \\
    \hline
    \multicolumn{1}{|c|}{\textbf{Pro}} & 
    \multicolumn{1}{c|}{\textbf{Kontra}} \\
    \hline
    \endfirsthead

    \multicolumn{2}{c}%
    {{\bfseries Fortsetzung von vorheriger Seite}} \\
    \hline
    \multicolumn{1}{|c|}{\textbf{Pro}} & 
    \multicolumn{1}{c|}{\textbf{Kontra}} \\
    \hline
    \endhead

    \hline \multicolumn{2}{r}{\textit{Fortsetzung auf nächster Seite}} \\
    \endfoot

    \hline
    \endlastfoot

    \multicolumn{2}{|c|}{\textbf{\bibleverse{Mat}(21:43)}} \\
    \hline
    Jesus lehnt die Juden ab und sagt, dass das Reich Gottes den Juden weggenommen wird und einem anderen Volk gegeben wird. &
    Das Reich Gottes wird einem damaligen ungläubigen Volk Israel weggenommen und einem zukünftigen gläubigen Volk Israel gegeben werden. \\
    \hline
    \multicolumn{2}{|c|}{\textbf{\bibleverse{Gal}(6:16)}} \\
    \hline
    Die Gemeinde wird an dieser Stelle explizit als das \enquote{Israel Gottes} bezeichnet. Das Wörtchen \textit{kay} (und) sollte in einem erklärenden Sinn verstanden werden und nicht als und übersetzt werden. &
    Der Kontext bestätigt jene Ansicht, wonach Paulus mit dem \enquote{Israel Gottes} die Judenchristen meinte. Es empfiehlt sich, \textit{kai} (und) so zu verstehen wie auch in den meisten anderen Stellen, wo dieses Wort vorkommt, und es in einem kopulativen oder kollektiven Sinn mit \enquote{und} zu übersetzen. \\
    \hline
    \multicolumn{2}{|c|}{\textbf{\bibleverse{Rom}(9:6)}} \\
    \hline
    Paulus unterscheidet angeblich zwischen dem ethnischen Israel und einem geistlichen Israel, das aus allen Gläubigen besteht, auch den nicht jüdischen. &
    Hier spricht der Vers von einem Israel innerhalb des ethnischen Volkes Israel. Paulus bezeichnet hier nicht gläubige Nichtjuden, sondern gläubige Juden als das wahre Israel. \\
    \hline
    \multicolumn{2}{|c|}{\textbf{\bibleverse{Rom}(2:28-29)}} \\
    \hline
    Diese Stelle wird oft als Begründung für die Behauptung, der Begriff \enquote{Jude} beziehe sich auch auf die nicht jüdische Gemeinde. &
    Mit \enquote{echten Juden} sind jene ethnischen Juden gemeint, die aufgrund ihres Glaubens eine Beziehung zu Gott haben. \\
    \hline
    \multicolumn{2}{|c|}{\textbf{\bibleverse{1Petr}(2:9-10)}} \\
    \hline
    Die Anwendung alttestamentlicher Bezeichnungen Israels auf die Gemeinde ist in ihren Augen der Beweis dafür, dass die neutestamentlichen Schreiber die Gemeinde als das neue Israel definierten. &
    Kann allenfalls als Hinweis dafür verstanden werden, dass der Begriff Volk Gottes auf die Nichtjuden ausgeweitet wurde. Dass die Gemeinde jedoch Israel ersetzt, ist keine zwingende Schlussfolgerung. Auch wenn Petrus seine Leser mit israelischen Bildern beschreibt, bezeichnet er sie nirgendwo als Israel. \\
    \hline
    \multicolumn{2}{|c|}{\textbf{\bibleverse{Gal}(3:7.29)}} \\
    \hline
    Hier wird darauf hingewiesen, dass die \enquote{Nachkommenschaft Abrahams}, Jesus Christus, verwandt ist. Auch dies sei ein Beweis dafür, dass die Gemeinde mit Israel identisch ist. &
    Im geistlichen Sinn können gläubige Nichtjuden als Nachkommen Abrahams bezeichnet werden. Abraham ist der Vater aller Gläubigen, sowohl der Nichtjuden als auch der Juden. Die Verwandtschaft mit Abraham durch den Glauben macht einen aber nicht automatisch zum Juden. \\
    \hline
    \pagebreak
    \multicolumn{2}{|c|}{\textbf{\bibleverse{Eph}(2:11-22)}} \\
    \hline
    Die Ansicht ist, dass es unwahrscheinlich ist, dass Gott Juden und Nichtjuden zusammengeführt hat, nur um sie in der Zukunft wieder voneinander zu unterscheiden. Dies käme einem Schritt nach rückwärts gleich. &
    Die Tatsache, dass die Nichtjuden einst ferne waren und jetzt nahe sind, dass sie einst ausgeschlossen waren und es jetzt nicht mehr sind, bedeutet nicht, dass sie Israels Identität angenommen hätten. Im Hinblick auf ihre Errettung, Segnungen und Stellung vor Gott stehen Heiden auf der gleichen Stufe wie gläubige Juden. \\
    \hline
    \multicolumn{2}{|c|}{\textbf{\bibleverse{Rom}(11:17-24)}} \\
    \hline
    In einer Analogie werden die ungläubigen Juden durch die abgebrochenen Zweige eines natürlichen Ölbaumes dargestellt, während die gläubigen Nichtjuden die Zweige eines wilden Ölbaumes sind, die in den natürlichen Ölbaum eingepfropft wurden. &
    Auch in diesem Beispiel werden gläubige Juden und Nichtjuden voneinander unterschieden. Durch diese heilsbezogene Einheit wird jedoch die einzigartige Rolle des Volkes Israel nicht aufgehoben. \\
    \hline
    \multicolumn{2}{|c|}{\textbf{\bibleverse{Hebr}(8:8-13)}} \\
    \hline
    Für viele Stellen ist dies der Beweis dafür, dass sich der Neue Bund gänzlich in der Gemeinde erfüllt. &
    Neben der Gemeinde wird also auch Israel bei seiner Errettung Anteil am Neuen Bund erhalten. Die Tatsache, dass die Gemeinde nur an den geistlichen Segnungen des Neuen Bundes Anteil erhielt, beweist, dass sie Israel nicht ersetzt hat. \\
    \hline
    \multicolumn{2}{|c|}{\textbf{Das Schweigen des NT}} \\
    \hline
    Das Fehlen einer ausdrücklichen Stelle über die Wiederherstellung des Volkes Israel im NT als zusätzlichen Beweis dafür, dass Israel durch die Gemeinde ersetzt wurde. &
    Da die neutestamentlichen Autoren als Juden mit dem Alten Testament vertraut waren, bestand in ihren Augen keine Notwendigkeit, alttestamentliche Verheißungen für Israels Zukunft zu wiederholen. Überdies stimmt es nicht, dass das Neue Testament nichts über eine Wiederherstellung Israels sagt. \bibleverse{Apg}(1:6), \bibleverse{Mat}(19:28) \\
    \hline
    \pagebreak
    \multicolumn{2}{|c|}{\textbf{\bibleverse{Rom}(11:26)}} \\
    \hline
    \begin{enumerate}
        \item Wenn Paulus davon spricht, dass \enquote{ganz Israel errettet werden wird}, dann meint er damit den gegenwärtigen Prozess der Errettung, wie er unter gläubigen Juden und Nichtjuden stattfindet, die gemeinsam das wahre Israel, die Gemeinde, bilden.
        \item Eine Ansicht besagt, dass mit dem Begriff \enquote{ganz Israel} die Gesamtzahl aller auserwählten Juden in der Geschichte gemeint ist.
        \item Paulus spricht von einer zukünftigen, umfassenden Bekehrung der Juden zum Christentum.
    \end{enumerate} &
    Wenn \bibleverse{Rom}(11:) eine zukünftige Errettung Israels auf der Grundlage des AT verheißt, dann dürfen wir auf derselben Grundlage auch eine Wiederherstellung der Nation erwarten. Da Israels Errettung in den alttestamentlichen Verheißungen verwurzelt ist, stellt sich die berechtigte Frage, warum die verheißene Wiederherstellung der Nation nicht stattfinden sollte. \\
    \hline
\end{longtable}
\section{Kapitel 14}
\subsection{Gottes Plan für die Zukubnft der Nationen}
Im Wesentlichen gibt es zwei eschatologische Denkmodelle:
\begin{itemize}
    \item das Modell der geistlichen Vision
    \item das Modell der neuen Schöpfung
\end{itemize}
\blockquote[geistliche Vision]{Der Himmel ist der Bereich des Geistes im Gegensatz zur Materie. Es ist ein nicht irischer geistlicher Ort für geistliche Wesen, die ausschliesslich an geistlichen aktivitäten teilnehmen. Das Modell der geistlichen Vision weist einen engen Bezug zu allegorischen oder geistlichen Interpretationsmethoden auf. Das Modell geistlicher Vision kommt im Mittelalter zu einer allegemeinen Akzeptanz.}
\blockquote[neue Schöpfung]{Das Modell der neuen Schöpfung erwartet eine neue Erde, die Erneuerung des Lebens auf dieser neuen Erde, die körperliche AAuferstehung der Erlösten und eine Existenz, zu der auch soziale und politische Aktivitäten gehören. Zu dieser erneuerten Erde gehören unter anderem auch Nationen, Könige, Wirtschaft, Kultur und andere Aspekte eines physischen Planeten. Dieses Modell steht der biblischen Lehre näher als das Model der geistlichen Vision. Die Endphase ist in \bibleverse{Offb}(21-22:) beschrieben. Die Wiederkunft Jesu wird eine Umgestaltung aller Bereiche der Gesellschaft einleiten.}
\subsection{Die Auswirkungen des Modells der neuen Schöpfung auf die Nationen und auf Israel}
In der Ewigkeit werden wir für immer in buchstäbliche physischen Körpern auf einer buchstäblich neuen Erde leben. Die Nationen haben definitiv einen Platz in Gottes Zukunftplänen. Paulus führt die Diversität der Nationen in \bibleverse{Apg}(17:26)auf den Schöpfer zurück. Es war Gott der jede einzelne Nation in Leben rief und ihre Zeiten und Grenzen festlegte. Das Volk Israel wurde ins Leben gerufen, weil Gott alle Nationen segnen und seine Schöpfung wiederherstellen wollte.

An dieser Stelle sind einige Dinge zu beachten. 
Bei \bibleverse{Jes}(19:23-25) Ist Folgendes zu beachten. Erstens: Gott hat einen Plan für die Nationen. Zweitens: Israel ist nicht die einzige Nation, die von Gott gesegnet werden wird. Drittens: Israel wird im AT als Gottes Volk bezeichnet, der gleiche Begriff wird aber auch auf Ägypten angewendet. Viertens: Nachdem Ägypten, mein Volk sowie Assyrien mein Händewerk genannt werden, spricht die Stelle von Israel, meinem Erbteil.

Es scheint, dass Gottes Pläne für die Nationen mehr umfassen als die Errettung einzelner Menschen. \bibleverse{Rom}(11:12) deutet eine stufenweise Zunahme des Segens der Nationen an.

Die Nationen befinden sich derzeit im Kriegszustand, was mit folgendene Bibelverseren belegt wird: \bibleverse{Ps}(2:1-3), \bibleverse{Ps}(2:8-9), \bibleverse{Mat}(25:31-32). Zum Zeitpunkt der Wiederkunft Jesu mit seinen Engeln wird er auf seinem Thron Platz nehmen, um die Nationen zu richten.

In \bibleverse{Offb}(2:26) Müssen wir folgendes beachten. In der jetzigen Zeit haben die Christen keine Autorität über die Nationen, doch sie werden diese Autorität verliehen bekommen, wenn Jesus wiederkommt. Nach der Wiederkunft Jesu wird es nach wie vor Nationen geben.

\bibleverse{Offb}(5:9), \bibleverse{Offb}(5:10), \bibleverse{Offb}(20:4-6), \bibleverse{Offb}(19:11-12) zeigen dass die Herrschaft Jesu über die Nationen gemeinsam mit seinen Heiligen con unserem Gesichtspunkt aus ein zukünftiges Ereignis.
Die Bibelstellen \bibleverse{Offb}(21:23-24,26;22:1-2) zeigen dass die Nationen in Ewigkeit bestehen werden.

Wenn wir anerkennen,dass es in der Ewigkeit Nationen geben wird, die konkrete Aufgaben und Identitäten haben werden, dann stellt sich die Fragem warum dies nicht auch für das Volk Israel der Fall sein sollte. Israeks Rolle besteht darin, den Nationen ein Segen zu sein, nicht allen Nationen zu Israeliten zu machen.

Die Dreieinigkeit ist ein Beispiel für Einheit (ein Gott) und Vielfalt (drei Personen). Das Gleiche gilt für die Gemeinde als Leib Christi, auch sie ist ein Beispiel für Einheit (ein Leib) und Vielfalt (viele Glieder des einen Leibes). Gott hat einen konkreten Plan für die Zukunft der Nationen. Eine davon wird das Volk Israel sein.

\section{Kapitel 15--16}
\subsection{Argumente für die nationale Wiederherstellung Israels}
Zahlreiche Stellen in der Bibel begründen die Wiederherstellung des Volkes Israel. Die Begründung kann in sieben Punkten zusammengefasst werden:
\begin{enumerate}
    \item Die Bibel spricht ausdrücklich von einer Wiederherstellung des Volkes Israel.\\ 
    \begin{itemize}
        \item \bibleverse{5Mos}(30:1-6)
        \item \bibleverse{5Mos}(28:0;29:0)
        \item \bibleverse{Jer}(16:14-15)
        \item \bibleverse{Jer}(30:1-3)
        \item \bibleverse{Hes}(36:22-30)
        \item \bibleverse{Hes}(37:21-29)
        \item \bibleverse{Rom}(11:26-28)
    \end{itemize}    
    \item Die Bibel spricht ausdrücklich von dem Fortbestand des Volkes Israel\\
    \begin{itemize}
        \item \bibleverse{Jer}(31:35-37)
    \end{itemize}
    \item Das Neue Testament bestätigt die zukünftige Wiederherstellung des Volkes Israel.\\
    \begin{itemize}
        \item \bibleverse{Mat}(19:28) und \bibleverse{Lk}(22:30) Jesu Worte in diesen Bibelstellen beweisen ausdrücklich, dass er mit einer nationalen Wiederherstellung Israels rechnet. Die Apostel verstanden Jesus Worte zweifellos als Hinweis auf das wiederhergestellte Volk Israel. Die zwölf Stämme Isreals sollten als Hinweis auf das Israel des AT verstanden werden.
        \item \bibleverse{Mat}(23:37-39) und \bibleverse{Lk}(13:34-35) In diesen beiden Paralellstellen warnt Jesus vor der bevorstehenden Zerstörung Jerusalems und des Tempels, wozu es deshalb kam, weil ihn die jüdischen Bewohner ablehnten.\\ Der Satz \enquote{Gepriesen sei, der da kommt im Namen des \herr n} ist ein freudiger Ausruf eines bussfertigen Volkes. Zitat aus \bibleverse{Ps}(118:)\\ Das Reich wird den Juden nicht in dem Sinn weggenommen werden, dass sie für immer verstossen sind; sondern es wird \flq ganz Israel\frq{} errettet und in den erlösenden Plan Gottes aufgenommen werden.
    \end{itemize}
    \item Das Neue Testament bestätigt, dass die alttestamentlichen Verheissungen und die Bündnisse nach wie vor dem Volk Israel gehören.
    \begin{itemize}
        \item \bibleverse{Rom}(9:3-4) Laut Paulus sind die \enquote{Bündnisse}, die \enquote{Verheissungen} und \enquote{der Gottesdienst} immer noch Israels Eigentum.
    \end{itemize}
    \item Die neutestamentliche Prophetie bestätigt die Zukunft Israels.
    \begin{itemize}
        \item \bibleverse{Mat}(24:15-20), \bibleverse{Offb}(11:1-2) In Gottes Plan haben das verheissene Land und der Tempel eine zukünftige Bedeutung.
        \item \bibleverse{Offb}(7:4-8) Die konkrete Erwänung alles zwölf Stämme ist ein deutlicher Hinweis auf die bleibende Bedeutung Israels im Plan Gottes.
        \item \bibleverse{Offb}(21:10-14) Die Unterscheidung zwischen Israel und der Gemeinde wird auch in der Ewigkeit aufrechterhalten.
    \end{itemize}
    \item Das NT unterscheidet zwischen Israel und der Gemeinde
    \begin{itemize}
        \item \bibleverse{Apg}(3:12;4:8.10;5:21.31.35;21:28) schliessen eine gelichsetzung aus.
    \end{itemize}
    \item Die Lehre der Auserwählung ist ein Beweis dafür, dass Gott für Israel eine Zukunft bestimmt hat.
    \begin{itemize}
        \item \bibleverse{5Mos}(7:6-8)
        \item \bibleverse{Rom}(11:) Die Wiederherstellung liegt im Wesen Gottes selbst begründet, von dem wir wissen, dass er treu ist und sein Wort hält, so dass an Israels Wiederherstellung kein Zweifel sein kann.
    \end{itemize}    
\end{enumerate}
\section{Schlussfolgerung}
Der Superzessionismus ist keine biblische Lehre. Es lässt sich eine zukünftige Errettung und Wiederherstellung Israels mit dem Modell der neuen Shöpfung begründen. Es war nie Gottes Vorhaben, jeden Menschen zu einem Israeliten zu machen. Eindeutige Bibelstellen sprechen von der zukünftigen Errettung Israels.
\begin{itemize}
    \item \bibleverse{5Mos}(30:)
    \item \bibleverse{Jes}(2:)
    \item \bibleverse{Hes}(36:-37:)
    \item \bibleverse{Zef}(3:)
    \item \bibleverse{Sach}(14:)
    \item \bibleverse{Mat}(19:28;23:39)
    \item \bibleverse{Apg}(1:6)
    \item \bibleverse{Rom}(11:0)
\end{itemize}
Die angeführten Bibelstellen der Superzessionisten
\begin{itemize}
    \item \bibleverse{Gal}(6:16)
    \item \bibleverse{Rom}(9:6)
    \item \bibleverse{Jes}(19:24-25)
    \item \bibleverse{Eph}(2:11-22)
\end{itemize}
Das NT bestätigt die alttestamentliche Erwartung einer Wiederherstellung Israels. Zu behaupten, Gott hätte Israel durch die Gemeinde ersetzt, bedeute einen enormen Bestand an biblischen Beweisen zu ignorieren.

Letzlich ich Gottes Treue gegenüber Israel auch ein Zeugnis dafür, dass seine Verheissungen für uns nie versagen werden:
\begin{bibelbox}{Schl}{Rom}{11:36}
    Denn aus im und durch ihn und zu ihm hin sind alle Dinge! Ihm sei die Herrlichkeit in Ewigkeit! Amen
\end{bibelbox}
\section{Ursprung der Gemeinde}
Der Begriff \textit{ekklesia} stammt von dem Verb \textit{ek-kaleo} ab. Ürsprünlich für die Einberufung einer Armee verwendet. Im sekulären Sinn bezeichnet \textit{ekklesia} meist eine versammelte Menschenmenge. (\bibleverse{Apg}(19:32.39.41))

Im AT wird noch nicht von der Gemeinde gesprochen in dem Sinne wie im NT:

\textit{ekklesia} kommt 114mal im NT vor. 5mal ist nicht von neutestamentlichen Gemeinden die Rede. 109 mal ist die Gemeinde gemeint. Im Neuen Testament bezieht sich der Begriff konkret auf die Nachfolge Christi.

Die Gemeinde wurde durch den Dienst Christi ins Leben gerufen.
\end{document}