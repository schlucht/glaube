\documentclass[
    12pt,
    a4paper,
    letter,
    parskip=yes,
    ]{scrlttr2}
\usepackage{lipsum} % Remove if you don't need sample text

\setkomavar{subject}{Wie biblisch ist katholische Kirche}

\date{\today}

\begin{document}
\setlength{\baselineskip}{1.2\baselineskip}  % Erhöht den Zeilenabstand auf das 1,5-fache
\begin{letter}{
Brunner Jean-Pierre\\
Judengasse 33\\
3904 Naters}

\opening{Sehr geehrter Herr Pfarrer Brunner,}


Ich heisse Schmid Lothar, bin 57 und wohne in Birgisch. Ich bin katholisch aufgewachsen. Meine Eltern sind konservative Katholiken und gehen regelmäßig zur Messe und pflegen auch regelmässig die anderen Sakramente. Auch ich ging bis zu meinem 18 Lebensjahr regelmässig zur Messe. Wollte eigentlich auch Priester werden, was ich aber wegen meinen schulischen Leistungen nicht machen konnte. Noch immer faszinieren mich die Rituale die von der Kirche durchgeführt werden. Ich finde, wir haben nichts hier auf der Erde, mit dem wir Gott auch nur ansatzweise ehren können. Darum bleibe ich lieber bescheiden. Er will uns persönlich und nicht irgendwelche Rituale.

2015 bin ich durch meine damaligen Freundin in die FEG in Visp gegangen. Ich bin einfach nur wegen ihr dahin, halt um der Liebe willen. Zu der Zeit glaubte ich nicht an einen Gott. Habe auch viele Bücher gelesen, die sich mit diesem Thema beschäftigen. Darum fühlte ich mich überlegen und hatte keine Probleme in diese Freikirche zu gehen. Nur habe ich nicht mit dem hl. Geist gerechnet. Der Pastor in der Gemeinde hat gerade mit einer dreiteilige Predigtserie über den Galater angefangen. Diese Predigtreihe hat mich motiviert, die Bibel, die ich von Pfarrer Josef Pfaffen gekriegt habe, aus dem Regal zu nehmen diese abzustauben und zu lesen. Nach ein paar Kapitel legte ich das Buch wieder hin und meine Frau meinte, ich solle es doch mit der Hoffnung für alle ausprobieren. Wir haben dann gemeinsam die Bibel gelesen. Dieses mal war es einfacher und ich habe die Bibel in 3 Monaten gelesen. Vieles habe ich nicht verstanden, aber ich konnte nichts darin finden, was mit der Tradition der katholischen Kirche übereinstimmte. Keine Päpste, keine Pfarrer, kein Sakramente. Nichts von alle dem. Ich ging mit meinen Fragen zu Dany Rhoner, dem Pastor von der FEG. Wir haben nur biblische Fragen besprochen, nicht über die kath. oder ev. Kirche und ihre Politik. Ein halbes Jahr nach der ersten Galater Predigt, habe ich mich auf dem Weg zur Arbeit zu Jesus Christus bekennt. Danach hat sich vieles verändert. Jetzt als Christ wollte ich wieder in eine hl. Messe. Ich ging dann auch an einem Sonntag, es war das Ernte Dank Fest, in Brig in die Messe. Der erste Teil war gut. Fürbitten, Bibellesen und beten. Die Predigt war schrecklich. Vielleicht 15min lang und der Pfarrer hat irgendwas über Foot Wast gepredigt. War ein bisschen irritiert da in der Freikirche, jede Predigt irgendwas mit der Bibel und Jesus zu tun hatte. Am schlimmsten war aber dann die Eucharistie Feier. Das Brot wurde aus dem Tabernakel geholt, nach dem Austeilen wieder zurückgestellt und abgeschlossen? Ich fand das schrecklich. Wenn jetzt das Brot zum Fleisch Jesus wurde, denkt der Pfarrer jetzt wirklich, dass sich Jesus in das kleine Schränkchen sperren lässt? Von da an ging ich nur noch bei Beerdigungen oder Hochzeiten in eine Kirche. Irgendwie fühle ich mich darin nicht wohl. Diese ganzen Heiligen und Marien Bilder, Jesus am Kreuz wo er doch auferstanden ist. Jesus hängt ja nicht mehr am Kreuz, sondern sitzt zur Rechten Gottes.

Jetzt 10 Jahre später habe ich mich inzwischen intensiver mit dem christlichen Glauben und der Bibel befasst. Je mehr ich mich mit der Bibel beschäftigt habe, umso unverständlich ist mir, wie studierte Theologen zu so einem System wie die kath. Kirche stehen können. Es gibt die Schäfchen, die sich gerne führen lassen und das machen was ihnen der Pfarrer sagt. Dann gibt es die Gruppe von Gelehrten, die nicht glauben, aber von der Organisation profitieren. Dann gibt es aber auch die ganzen anderen, die das Theologiestudium machen, um wirklich Gott zu dienen und den Menschen helfen wollen. Wenn das so ist, wieso nimmt man denn die Bibel und die Worte Jesus nicht ernst? Schon nur das Johannes Evangelium widerlegt doch in jedem Kapitel die kath. Organisation. Jesus hat nie die Schäfchen oder die Besatzungsmacht kritisiert, sondern nur die Priester der damaligen Zeit. Die Pharisäer waren streng gläubige Menschen und alles was sie machten, machten sie aus gutem Gewissen um Gott zu dienen und ihn zu ehren. Eigentlich das, was auch heute viele Priester machen. Was sagte aber Jesus über diese Pharisäer? Schlangenbrut nannte er sie, Sauerteig der Gesellschaft, Verführer, Heuchler u.\,s.\,w. Wieso? Eigentlich nur weil sie Gottes Wort umgedeutet haben und die Menschen so von Gott weg- und nicht zu ihm hingeführt haben. Sie haben viele neue Gesetze erstellt, um die original Gesetze nicht zu übertreten. Viele der Gesetze haben sie so verbogen, dass sie davon profitieren können. Sie waren so stur, dass sie Jesus nicht erkannten und ihn am Kreuz töten liessen.

Wie würde heute die kath. Kirche mit dem Jesus der vor 2000 Jahren auf die Erde kam umgehen? Seine Lehre würde mit vielen Punkten der kath. Kirche nicht übereinstimmen. Der Kern ist da, aber in 1700 Jahren wurde soviel drum herrum gebaut, dass diese Schicht nicht mehr bis zum Kern durchdrungen werden kann. 

Das schlimmste ist, diese Heiligen und Marien Verehrung. Sie wissen genau, dass das falsch ist. Das ganze Alte Testament warnt davor sich neben Gott irgend welche anderen Götter anzuschaffen. Kurz nach dem Auszug von Ägypten, haben sich die Israeliten ein Kalb gebaut und dieses verehrt. Sie meinten es ja gut. Gott war nicht zu sehen, Moses war auch schon länger weg. Also war die Idee etwas zu bauen, was der Retter aus Ägypten repräsentiert. Was hat der Gott, der die Liebe und die Gnade in Person ist, aber sich selber der Eifersüchtige nennt, gemacht? Er wollte das ganze Volk vernichten. Verstehen Sie? "Nur"{} wegen einem goldenen Kalb wollte er eine Million Menschen töten. Dank der bitte Mose, die direkt an Gott gerichtet war, hat Gott es nicht gemacht. Die Strafe war trotzdem schrecklich. Dieser Gott ist immer noch der gleiche. Was wird wohl mit denen passieren die statt Gott zu Maria oder den Heiligen beten? Denken Sie, Jesus wird Verständnis voll nicken, wenn sie vor dem Richterstuhl erklären, dass Sie ihren Schäfchen zwar gesagt haben, anbeten darf man nicht, aber verehren geht und dann Marien und Heiligenbildchen verteilt haben? Oder die Reliquien Geschichte. Es gibt sogar Reliquien von der Vorhaut Jesus. Habe gelesen, die ganzen Holzreliquien vom Kreuz zusammensetzt, dieses mehrere Tonnen schwer wäre. Was passiert mit ihren Schäfchen, wenn sie vor Jesus stehen und sich verteidigen, dass sie doch jeden Tag 3 Ave Maria gebetet hätten? Verstehen Sie mich und mein Problem? Auf der einen Seite Gott, der Eifersüchtige, der Allmächtige, der will, dass wir nur ihn anbeten, auf der anderen Seite sind Sie ein kleiner Mensch der sagt, zur Maria zu beten ist schon OK und so Gott widerspricht. Macht das ihnen nicht Angst? Also mir schon.

Es gibt für mich noch ein anderes gefährliches Problem in der kath. Kirche. Das ist: Wie komme ich in den Himmel? Jesus höchst persönlich hat diesen Weg beschrieben: Johannes 14,\,6 "Ich bin der Weg und die Wahrheit und das Leben. Niemand kommt zum Vater als nur durch mich." Die katholische Kirche sagt etwas anderes. Das heisst also, die Kirche stempelte Jesus als Lügner ab. Stellen Sie sich das vor. Jesus Christus ist Gott, der wahrhaftige, der ohne Sünde und wird als Lügner bezeichnet. Es sind nicht die Werke, die Taufe, die Beichte oder letzte Ölung, die dem Menschen den Weg zu Jesus öffnet, sondern der Glaube an Jesus Christus selber. Die Liebe zu IHM, die Hingabe zu IHM, das Vertrauen auf IHN.

Mein Vater wird ratz fatz in die Hölle gehen, weil er denkt, dass die Taufe und jeden Sonntag zur Kommunion gehen, reicht. Er kennt Jesus nicht mal richtig. Und Sie als Priester, als studierter Theologe, haben es ihm so gesagt, Sie haben sein Leben auf dem Gewissen. Sie wissen, das laut Jakobus 3,\,1 (übrigends ein Sohn von Maria und Josef) die Lehrer noch vor einem strengeren Gericht stehen, als die Schäfchen? Macht das Ihnen keine Angst? Mein Vater legt Geld auf die Seite, dass wir ihm nach seinem Tod die Messen lesen, wegen dem Fegefeuer. Der Verbrecher am Kreuz neben Jesus, ist noch am gleichen Tag ins Paradies gekommen und ich glaube jetzt nicht, dass dieser wegen seinen guten Taten am Kreuz hing. Auch die Geschichte von Lazarus zeigt, dass das Fegefeuer eine erfundene Sache ist, um den Menschen Angst zu machen und ihnen ihr Geld aus der Tasche zu ziehen. 

Wieso schreibe ich das Ihnen überhaupt? Ich glaube jetzt nicht, dass Sie sich jetzt wegen einem Brief von einem Laien ändern und den Menschen ab jetzt, das wahre Evangelium predigen, aber ich bin einfach sauer auf die katholische Kirche. Nicht auf Sie persönlich. Viele Menschen gehen wegen dieser falschen Lehre verloren. Auch meine Eltern und meine Geschwister, die sich sicher fühlen, weil sie als Kinder getauft wurden. Mein Vater hat mir gesagt, ich habe das so gelernt und das halt so. Ich habe nicht Theologie studiert. Ich habe die Bibel gelesen und auch Bücher gelesen die, die Bibel erklären. Ich bin überzeugt, dass die Bibel Gottes Wort ist. Wird ja auch in der Bibel mehrmals erwähnt. Ich glaube an den Gott den Schöpfer, an seinen Sohn, der für meine Sünden am Kreuz gestorben ist und auferstanden ist, an den hl. Geist der ausgegossen wurde um uns zu führen und zu leiten. Und an die drei Einigkeit. Ich glaube, dass meine Rettung nur durch die Gnade Gottes erfolgt, die ich selber bewusst annehmen muss. Was hat mich überzeugt, dass es so ist? Nein nicht die Bibel, nicht die Pastoren, sondern der hl. Geist auf dem Weg am Morgen zu Arbeit. Ich war plötzlich überzeugt, ich hatte plötzlich absolute Gewissheit. Es viel mir wie Schuppen von den Augen und ich bekam einen inneren Frieden. Von da an hat sich mein Leben verändert. Mir wurden meine Sünden klarer, ich habe aufgehört zu trinken, mir kam vieles viel logischer vor. Nein, ich bin nicht heiliger geworden. Ich bin immer noch Sünder, aber ich bin ruhiger geworden, habe ein Sinn im Leben habe Freude am Leben und freue mich auf den Tod um Jesus zu treffen. Johannes 16,\,13 „Wenn aber jener, der Geist der Wahrheit, gekommen ist, wird er euch in die ganze Wahrheit leiten.“ Er ist gekommen, zu Pfingsten und zu jedem der ihn will, lassen wir uns leiten? Wenn wir die Wahrheit nicht erkennen, dann: 1Kor 2,\,14 "Der natürliche Mensch aber nimmt nicht an, was vom Geist Gottes ist; denn es ist ihm eine Torheit, und er kann es nicht erkennen, weil es geistlich beurteilt werden muss.“? Ich habe es erlebt. 

Wieso versucht ihr verzweifelt Leute in die Kirche zu bekommen? Wegen dem Geld? Wieso sind Kirchen bei denen Jesus Christus im Mittelpunkt steht, voll trotz 60 minütiger Predigt? Kenne Gemeinden ohne Mitgliedschaft die jeden Sonntag 500-600 Teilnehmer haben. Ohne spezielles Programm. Wieso? Hat nicht Jesus zu der Gemeinde Ephesus gesagt, wenn ihr nicht zu mir umkehrt, werde ich den Leuchter wegnehmen? Hat Jesus den Leuchter von der kath. Kirche genommen? Ist die kath. Kirche so von sich Überzeugt, dass sie sich sich nie darüber Gedanken macht? Oder macht sich die Kirche keine Gedanken darüber, weil sie im Grunde nicht daran glaubt, was die Bibel sagt. Wenn das so ist, wird sie Milliarden von Menschen in den Abgrund führen. (der Blinde führt den Blinden) Laut der Theologie der kath. Kirche würden alle Apostel nicht in den Himmel kommen. 
\begin{enumerate}
    \item Sie sind Juden.
    \item Sie waren nie in einer Messe
    \item Sie haben nicht gebeichtet
    \item Hatten Sie keine letzte Ölung
\end{enumerate}
Jesus war weder evangelisch noch katholisch, sondern er war Jesus der Christus, der Sohn Gottes. Der Rausschmiss aus dem Paradies hat die Menschen nicht zum Nachdenken gebracht, die Sintflut hat nichts gebracht, das Volk Israel, dass eigentlich eine Vorbildnation sein sollte, hat versagt und zu guter Letzt wurde der Sohn Gottes getötet. Und jetzt? Jesus ist für uns gestorben, wir müssen doch nur seinen Sühnetod annehmen! Jesus hat nie eine Religion gegründet, sondern er hat uns Menschen vor Gott, der seit Ewigkeiten besteht, rein gemacht. Er hat nichts neues erfunden und ganz sicher nicht eine katholische Kirche. Da Sie gebildeter sind als ich, wissen Sie ja, dass diese Orgnisation, wie sie jetzt da steht, erst ab dem 4. Jahrhundert entstanden ist. Dann die Trennung der Kirche in die Ost- und Westkirche. Von da an kam dann das Papstum so richtig in fahrt. Haben Sie sich eigentlich auch mal gefragt, wieso die kath. Kirche so brutal gegen Menschen vorging, die nach den Vorgaben der Bibel leben wollten? Auch Jan Hus, Luther und andere die die Kirche kritisierten, weil sie nicht nach Gottes Wort handelt? Es ist doch offensichtlich, dass wenn das gemeine Volk mitbekommen hätte, dass ihre Regeln nichts mit der Bibel zu tun haben und sie ihre Macht verloren hätte. Wie auch die Zeugen Jehovas, versucht die kath. Kirche, ihre zusätzlichen Regeln mit Bibelvers zu untermauern. Diese sind aber oft aus dem Zusammenhang gerissen und teilweise mystifiziert. Wenn Sie ehrlich sind und die Sakramente näher betrachten, sind diese eigentlich nur dafür da, um die Schäfchen an die Organisation zu binden. Keine von denen kann mit der Bibel belegt werden. Mit der Traditio, auf die sich die Kirche beruft, kann man alles erklären und diese von Menschen ausgedachten Regeln, als von Gott eingegeben verkaufen. Genau das, was Jesus an den Pharisäern kritisierte. Viele Glaubenssätze wurden in der Geschichte verändert und umgedeutet. Darum hat Gott die Bibel abgeschlossen, so dass alle das gleiche Grundgerüst haben. Nach der Offenbarung kam nichts mehr dazu. Davon bin ich überzeugt. Eigentlich sollten ja die Dogmen Zitat \textit{"dass die Menschen in der Erkenntnis des einzigen wahren Gottes und seines Sohnes Jesus Christus das ewige Leben haben (Joh 17,\,3)"}. Wieso braucht es dazu Marien Dogmen? Es steht doch alles in der Bibel. Jesus hat es doch selber gesagt. Paulus und die anderen Briefschreiber haben es nochmals ausgelegt und erklärt. Das reicht doch. Und es ist ja jetzt nicht wirklich kompliziert. Wieso wurden diese Sachen mit dem Papsttum, Priestertum, Beichte, Fegefeuer, Maria und Heilige hinzugefügt? Wieso sieht man das nicht, wenn man Theologie und die Bibel studiert, dass da etwas nicht stimmen kann? Es fehlt der hl. Geist und der Widersacher versucht alles, dass es so bleibt. Stellen sie sich vor die kath. Kirche würde plötzlich das wahre Evangelium predigen. Der Teufel würde sich im Himmel grün und blau ärgern.


Ich bin verärgert. Ich hatte aber das Glück, dass mich Gott auserwählt hat, IHN zu erkennen und dass ich dies auch angenommen habe. Inzwischen habe ich mich bewusst taufen lassen, um die Verbundenheit mit meinem Jesus und Retter zu bezeugen. Ich weiss jetzt auch was Jesus meinte mit, dass sein Joch leicht ist. 


Ich hätte da noch einen Predigt Anfang für die nächste Messe:
\begin{quote}
    Liebe Leute, ich muss ihnen leider Mitteilen, dass die Taufe die sie als Kinder erhalten haben, sie nicht von dem ewigen Feuer der Hölle retten wird. Es gibt nur ein Weg Johannes 14,\,6
    ...
\end{quote}
Daraus können Sie jetzt ein mehrstündige Predigt erstellen.


Ich musste das einfach mal los werden. Sie können mir gerne auf jl.schmid@jagolo.ch antworten. Bin aber keine Theologe und verstehe vieles nicht. Vielleicht glaube ich einfach darum naiv an die Bibel und an die Worte von Jesus und den Apostel. 
Wenn Sie den Brief vollständig gelesen haben, ja ist lang geworden, danke ich Ihnen für Ihre Geduld und hoffe, dass ich nicht zu persönlich wurde. Wollte Sie in keinster weise persönlich angreifen, sondern die Organisation katholische Kirche.

\closing{Mit freundlichen Grüßen}
\end{letter}
\end{document}