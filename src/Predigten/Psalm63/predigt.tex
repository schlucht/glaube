\author{OTS}
\documentclass[14pt]{../../inc/mybib}

\setincpath{../../inc/}

\usepackage{bible_style}
\graphicspath{{../../assets/images/}}
\usepackage{header}
\usepackage{numprint}
% \usepackage{parskip}

\newcommand{\q}[1]{\blockquote{#1}}

\author{Lothar Schmid}

\begin{document}
\setlength{\baselineskip}{1.5\baselineskip}

\section*{Psalm 63}
    \subsection*{Einleitung}
    Guten Morgen miteineander. Ich bin Lothar Schmid und bin in der MNR Naters beheimatet. Seit zwei Jahren haben wir jetzt in Naters diese Mitternachtsrufgemeinde. Noch sind wir eine kleine Gruppe, aber so herrlich von Gott gesegnet. Wenn ihr mal einen schönen Sonntagsausflug machen wollt, kommt uns doch einfach besuchen. Unser Gottesdienst findet Sonntagnachmittags um 14:45Uhr in Naters, Furkastrasse 46, statt. So, fertig mit der Werbung.
    
    Ich freue mich sehr, heute hier zu sein zu dürfen, um Euch den Psalm 63 etwas näherzubringen. Psalm 63 ist für mich schon länger ein Begleiter in meinem Leben und gibt mir immer wieder Mut, Kraft und Zuversicht. 

    \subsection*{Allgemeines}
    \begin{block}[Allgemeines]
    Zuerst eine kleine Einleitung zu diesem Psalm der Vers 1.
        \begin{bibelbox}{SCHL}{Ps}{63:1}
            Ein Psalm Davids, als er in der Wüste Juda war.
        \end{bibelbox}
    Wie der erste Vers uns schon sagt, ist David der Autor von Psalm 63. Der Ort, an dem er diesen Psalm geschrieben hat, war wohl in der Wüste von Judäa. Die Wüste von Judäa liegt von Jerusalem aus gesehen, in Richtung Totes Meer. Sie wird auch die Wüste Siph genannt. In der Bibel wird zweimal erwähnt, dass David in der Wüste von Judäa war. Das erste Mal, in     
    \begin{bibelbox}{SCHL}{1Sam}{23:14-15}
        David aber blieb in der Wüste auf den Bergfesten und hielt sich im Bergland auf, in der Wüste Siph. Und Saul suchte ihn alle Tage, aber Gott lieferte ihn nicht in seine Hand.
    \end{bibelbox}    
    David flieht vor Saul mit 600 Mann in die Wüste Siph. Dort hielt er sich vor Saul im Bergland versteckt.
    
    Das zweite Mal, wie in \bibleverse{2Sam}(15:13-28) beschrieben, flüchtet David vor seinem Sohn Absolom aus Jerusalem. Zu dieser Zeit war David der König von Jerusalem und ganz Israel. So passt dieser Psalm besser in diese Zeit, weil sich in unserem Psalm David in Vers 12 als König bezeichnet. Aber davon später. Als David den Psalm schrieb, hatte er doppeltes Leid zu tragen. David wird aus Jerusalem vertrieben und von seinem Sohn verraten.
    \end{block}
    \begin{block}
        Ich habe diesen Psalm in drei Teilen eingeteilt. 
        \begin{enumerate}
            \item Das Suchen nach Gott Verse 2-7
            \item Das Finden von Gott Verse 8-9
            \item Das Vertrauen in Gott Verse 10-12
        \end{enumerate}
        Zusammen werden wir diese Punkte durchgehen und schauen, was \betonung{wir} von David lernen können und wie \betonung{wir} diesen Psalm für uns verwenden können.
    \end{block}

    \subsection*{Punkt 1: Gott suchen}
    \begin{block}
    In Vers 2 ist Davids ganzes Glaubensbekenntnis. 
    \begin{bibelbox}{SCHL}{Ps}{63:2}
        O Gott, Du bist mein Gott, früh suche ich Dich; meine Seele dürstet nach Dir, mein Fleisch schmachtet nach Dir in einem dürren und lechzenden Land, wo kein Wasser ist.
    \end{bibelbox}
    Sein Ausruf: \enquote{O Gott, Du bist mein Gott}, kommt aus der Schlachterübersetzung und ist ein richtig schöner Ausruf. In diesem Ausruf zeigt sich der Glaube Davids an den Gott der Schrift, an den Gott seiner Väter und er nimmt diesen Gott für sich persönlich in anspruch. \leise{Du bist mein Gott}. Ist das nicht herrlich? So  Gott zu vertrauen, dass man auch in grösster Not noch sagen kann, Du Herr bist \betonung{MEIN GOTT}. 

    \enquote{Früh suche ich Dich}, übersetzen kann man das auch mit in der \enquote{Frühe}, oder "Früh am Morgen". David hat also seinen persönlichen Gott schon in der Frühe gesucht und ihn angerufen.
    \end{block}    

    \begin{block}
        Wie ist das mit uns? Ist Gott auch mein persönlicher Gott? Suche ich ihn auch schon in \betonung{der Frühe}, oder warte ich erst mal ab, was so im Laufe des Tages passiert?

        Wie war das, als man frisch verheiratet zusammengezogen ist? Galt da unser erster Blick nicht direkt unserem Schatz? Heute geht mein erster Blick auch nicht mehr immer als Erstes zu meinem Schatz, sondern eher zum Wecker.

        Und wie ist das mit \betonung{meinem} Gott? Wir, die wir den Glauben bezeugt haben, wird es langsam zur Routine? Oder suchen wir Gott in der Früh? Und zwar nicht nur \enquote{einen Gott}, sondern \betonung{meinen} Gott. \betonung{Meinen persönlichen Gott}.

        David beschreibt seine Gefühle wie folgt, \enquote{Meine Seele dürstet nach Dir, mein Fleisch schmachtet nach Dir.} Sein ganzer Körper und seine ganze Seele schmachtet und dürstet nach seinem Gott. Klar ihm ging es in der aktuellen Situation schlecht. Aber geht es uns immer gut? Natürlich werden wir jetzt nicht verfolgt, aber jeder von uns hat so seine eigenen Probleme und Sorgen, bei denen wir Gott brauchen und suchen sollten. Gott \betonung{will}, dass wir mit unseren Problemen zu Ihm kommen.
    \end{block}

    \begin{block}
        In Vers 3 sagt er uns, dass er Gott in Macht und Herrlichkeit im Heiligtum Gottes gesehen hat.
        \begin{bibelbox}{SCHL}{Ps}{63:3}
            \dots das ich Deine Macht und Herrlichkeit sehen darf, gleichwie ich Dich schaute im Heiligtum.
        \end{bibelbox}
        Der Tempel war zu der Zeit noch nicht gebaut, aber David hatte die Bundeslade zurück nach Jerusalem gebracht. Bei dieser Aktion -- sie wird in \bibleverse{2Sam} (6:1-23) beschrieben -- hat David die Bundeslade aus Kirjat-Jearim geholt. Bei diesem Transport, strauchelten die Tiere und als Usa die Lade zum Festhalten berührte, wurde er von Gott mit dem Tod bestraft. Gott hatte im Gesetz mit Mose verordnet, dass nur ausgewählte Priester die Lade transportieren. Auch sollte diese getragen und nicht auf einem Wagen transportiert werden. Dies können wir lesen in:
        \begin{bibelbox}{SCHL}{4Mos}{4:15b}
            \dots die Söhne Kahats sollen hineingehen, um das Heiligtum zu tragen, sie sollen aber das Heiligtum nicht anrühren, sonst würden sie sterben.
        \end{bibelbox}
        \betonung{Diese Aktion} hatte David die Macht Gottes und seine Souveränität gezeigt. David brachte die Lade zuerst nach Kirjat-Jearim, danach nach Obed-Edom und später unter Tanz und Reigen nach Jerusalem.
    \end{block}

    \begin{block}
        Auch wir wiedergeborenen Christen haben die Herrlichkeit Gottes gesehen. Gott hat alles für uns gegeben, was er geliebt hat. Seinen Sohn Jesus Christus. Jeder, der Jesus in sein Herz aufgenommen hat, hat die Herrlichkeit Gottes gesehen.
        \begin{bibelbox}{SCHL}{2Kor}{4:6}
            Denn Gott, der dem Licht gebot, aus der Finsternis hervorzuleuchten, er hat es auch in unseren Herzen licht werden lassen, damit wir erleuchtet werden mit der Erkenntnis der Herrlichkeit Gottes im Angesicht Jesu Christi.
        \end{bibelbox}
    \end{block}

    \begin{block}
        Als David diesen Psalm schrieb, hatte er sicher Angst, Jerusalem und das Heiligtum nie mehr wiederzusehen. Haben wir nicht auch manchmal diese Angst, wenn etwas im Leben schiefgelaufen ist und wir gesündigt haben? Liebt Gott mich \betonung{so} noch? Darf ich \betonung{so} zu ihm kommen? Wir brauchen uns aber keine Sorgen zu machen. Wie David wusste, dass er, wenn er Gnade vor dem Herrn gefunden hat, die Herrlichkeit in Jerusalem wieder sehen wird, können auch wir getrost sein, in der Herrlichkeit Gottes zu bleiben. 
          \begin{bibelbox}{SCHL}{Joh}{10:11}
            Ich bin der gute Hirtte; der gute Hirte lässt sein Leben für die Schafe.
        \end{bibelbox}
        David war sich dessen so sicher, dass er die Lade, als sie ihm nach gebracht wurde, wieder zurück nach Jerusalem schickte.
        \begin{bibelbox}{SCHL}{2Sam}{15:25}
            Aber der König sprach zu Zadiok: Bringe die Lade Gottes wieder in die Stadt zurück! Wenn ich Gnade vor dem Herrn finde, so wird er mich zurückbringen, dass ich ihn und seine Wohnung wiedersehen darf
        \end{bibelbox}
        \betonung{Wir}, die wir mit Jesus unterwegs sind, haben diese Gnade auch bekommen. Wir können ihn enttäuschen und wir tun es auch täglich, aber ER verlässt uns nicht, ER lässt uns nicht alleine. Wir können immer zu IHM, \betonung{unserem Retter und Erlöser} zurückkommen. Wie auch David sich auf die Gnade Gottes berufen hat, so können wir das noch \betonung{viel mehr} durch die Erlösung unseres Herrn Jesus Christus.
    \end{block}
    \begin{block}
        David hat in seinem Leben die Gnade und die helfende Hand Gottes oft erlebt. Für David ist ein Leben ohne diese Gnade nicht lebenswert. Wie er dies in Vers 4 sagt.
        \begin{bibelbox}{SCHL}{Ps}{63:4}
            Denn Deine Gnade ist besser als Leben; meine Lippen sollen Dich rühmen.
        \end{bibelbox}
        Wie gross ist die Unzufriedenheit in dieser Welt? Auch hier, in unseren Breiten? Obwohl wir doch alles zum Leben haben, Essen, Trinken, ein Dach über dem Kopf, Medizin usw., ist diese Unzufriedenheit zu bemerken. Je reicher umso mehr. Medien und Influenzer gauckeln uns vor, wie toll das Leben wäre, wenn wir noch dieses und jenes hätten.
        
        Viele Reiche und erfolgreiche Prominente sind unzufrieden. Trotz ihres Reichtums und ihres Ruhms merken sie, dass ihnen etwas fehlt. Nichts können sie mitnehmen, das Leben verschwindet und sie stehen mit leeren Händen da.
        
        Nur die Gnade des Herrn kann unserem Leben einen Sinn geben. Wie David auf die Gnade Gottes vertraut, können auch wir auf die Gnade Gottes vertrauen. Diese Gnade gibt dem Leben einen Sinn und einen Halt.

        Übrigends 99\% der Menschen auf der Erde sind ärmer als wir.
    \end{block}
    \begin{block}
        Diese Sicherheit und Halt drückt David im Vers 4b - 7 aus.
        \begin{bibelbox}{SCHL}{Ps}{63:4-7}
            Denn Deine Gnade ist besser als Leben; meine Lippen sollen Dich rühmen. So will ich Dich Loben ein Leben lang, in Deinem Namen meine Hände aufheben. Meine Seele wird satt wie von Fett und Mark, und mit jauchzenden Lippen lobt Dich mein Mund, wenn ich an Dich gedenke auf meinem Lager, in den Nachtwachen nachsinne über Dich.
        \end{bibelbox}
        Dieses vertrauen zu seinem persönlichen Gott lässt David lobsingen und jubeln.
        Wenn ich hier fertig bin, wollen auch wir unseren Herrn gemeinsam mit unseren Lippen, mit Gebet und Lieder, rühmen und preisen.
    \end{block}    
    \subsection*{Teil 2: Gott finden}
    \begin{block}
        \betonung{David hat seinen Gott gefunden.}
        Die Verse 8-9 drücken aus, was David jetzt fühlte als er Gott gefunden hat.
        \begin{bibelbox}{SCHL}{Ps}{63:8-9}
            Denn Du bist meine Hilfe geworden, und ich juble unter dem Schatten Deiner Flügel. An Dir hängt meine Seele; Deine Rechte hält mich aufrecht.
        \end{bibelbox}
        Es ist dieses Vertrauen, das David auf Gott hat. Gut wir können sagen, dass David Gott in seinem Leben näher erlebt hat als wir. Wir wissen, dass sich David schon früh in seinem Leben unter den Schutz Gottes gestellt hat. Als er noch als Junge mit den Schafen unterwegs war, hat er schon mit Gottes Hilfe die Schafe vor Wölfen und Bären verteidigt und hat den Schutz des Herrn erlebte.

        Vom Schutz der Adlersflügel lesen wir öfters in der Bibel zum Beispiel in
        \begin{bibelbox}{SCHL}{2Mos}{19:4}
            Ihr habt gesehen, was Ich an den Ägyptern getan habe und wie ich Euch auf Adlersflügeln getragen und Euch zu Mir gebracht habe.
        \end{bibelbox}
         So wie Gott Israel auf Adlersflügel getragen hat, als sie vor den Ägyptern geflohen sind. So setzt David sein ganzes Vertrauen auf diesen Schutz. Auch im Kampf gegen Goliath hat er sich auf diesen Schutz von Gott verlassen. 
    \end{block}
    \begin{block}
        Wenn wir uns schon am Morgen in der Frühe auf die Suche nach \enquote{\betonung{meinem}} Gott machen, können wir uns auch schon Früh unter den Schutz \betonung{seiner} Flügel stellen. Wenn man seine Stille Zeit am Morgen vor dem Alltag hält, gibt es uns Schutz und Sicherheit für die Probleme die unweigerlich darauf folgen. Jeder Tag hat wieder seine eigenen Sorgen. 
        Ich habe schon erlebt, dass mir nach dem Gebet und dem Bibellesen Probleme, über die ich die halbe Nacht studiert habe, plötzlich unwichtig vorkamen. Oft denken wir, dass unser Problem klein und unwichtig ist und wollen damit Gott nicht belästigen.\\        
        \betonung{ABER!} \\        
        \textbf{Erstens:} \betonung{Wir} können Gott nicht belästigen, schon dieser Gedanke ist Gotteslästerung. Er sugeriert, dass wir über Gott stehen. \\        
        \textbf{Zweitens:} \betonung{Alle} noch so grossen Probleme, die wir kleinen Menschen hier auf dieser Erde haben, sind für Gott überhaupt \betonung{keine Probleme}. Er ist ein Gott, der mit einem Wort das ganze Universum und die ganze Erde erschaffen hat!

        Darum dürfen und sollen wir mit allen unseren Problemen zu Ihm kommen, egal wie klein sie uns scheinen.
    \end{block}
    \begin{block}
        Der Vers 9 ist einfach gewaltig.
        \begin{bibelbox}{SCHL}{Ps}{63:9}
            An Dir hängt meine Seele; Deine Rechte hält mich aufrecht.
        \end{bibelbox}
        Laut Benedikt Peters kann das Wort \leise{hängen} auch mit \leise{kleben} übersetzt werden. Das gleiche Wort wird auch in \bibleverse{1Mos} (2:24) benutzt, wo es heisst, 
        \begin{bibelbox}{SCHL}{IMos}{2:24}
            Darum wird ein Mann seinen Vater und seine Mutter verlassen und seiner Frau \betonung{anhängen}, und sie werden ein Fleisch sein.
        \end{bibelbox}
        Also nicht die Frau wird an den Mann geklebt, sondern der Mann an die Frau. Also liebe Männer verwöhnt Eure Frauen, Ihr klebt an ihnen.

        Kleben heisst, es ist fest miteinander verbunden. Wenn Du zwei Papierblätter zusammenklebst, dann sind diese so fest miteinander verbunden, dass Du diese nicht mehr ohne Schaden trennen kannst. Dieses \betonung{kleben} können wir auch in der Aussage von Jesus verwenden, wo er sagt, dass er uns nicht mehr loslässt. 
        \begin{bibelbox}{SCHL}{Joh}{10:27-29}
            Meine Schafe hören meine Stimme, und ich kenne sie, und sie folgen mir nach; und ich gebe ihnen ewiges Leben, und sie werden in Ewigkeit nicht verloren gehen, und niemand wird sie aus meiner Hand reissen.
        \end{bibelbox}
        Jeder, der Jesus Christus von Herzen aufnimmt, seine Sünden bekennt und vor den Herrn gelegt hat, klebt an ihm fest. Wenn wir fest an unserem Herrn kleben, können wir auch gewiss sein, dass uns seine Rechte aufrecht hält. Nicht falsch verstehen. Es geht nicht darum, dass es uns materiell und Wohlstandsmässig besser geht. Sondern das wir seine Unterstützung und Hilfe haben. Zwei zusammengeklebte Blätter sind stabiler als ein einzelnes. Wie diese \enquote{haltende Rechte} uns hält, sehen wir bei den Gottesmännern in den Paulus Briefen und auch bei den verfolgten Christen auf dieser Welt. 
        
        Ein Mitarbeiter von Open Doors hat mir einmal gesagt, dass wenn man die verfolgten Christen fragt, wofür und für was man für sie beten soll, wird selten gesagt, dass Gott ihnen das Leid wegnehmen soll, sondern für die Kraft, im Leid zu bestehen und standhaft zu bleiben. Das ist diese Rechte, die uns hält. Die ist immer dann da, wenn wir sie wirklich brauchen. Wenn man mit kleinen Kindern in die Berge geht, nimmt man diese auch erst dann an die Hand, wenn es gefährlich wird und nicht den ganzen Tag. Gott ist es wichtig, dass wir selbständig sind, aber auch ehrlich zu ihm kommen, wenn wir Ihn brauchen. Wie Du, Dein Kind nicht wegstossen würdest, wenn es Deine Hand ergreift, so wird auch Gott Dich nicht wegstossen, wenn Du seine Hilfe brauchst.
    \end{block}
    \begin{block}
        Diese Sicherheit und diesen Halt, hat David dadurch bekommen, dass er Tag und Nacht an Gott gedacht und über ihn nachgesonnen hat, und ihn im Wort studiert hat. Wie wir in Vers 7 gelesen haben, hat er auf dem Nachtlager und auch in den Nachtwachen über seinen Gott nachgedacht. Das ist jetzt nicht nur so eine Aussage, sondern es ist sogar ein Gebot von Gott. So lesen wir in Psalm 1,
        \begin{bibelbox}{SCHL}{Ps}{1:2}
            sondern hat Lust am Gesetz des HERRN und sinnt über sein Gesetz Tag und Nacht.
        \end{bibelbox}
        oder wie Gott Josua 1.8 geboten hat:
        \begin{bibelbox}{SCHL}{Jos}{1:8}
            Lass dieses Buch des Gesetzes nicht von Deinem Mund weichen, sondern forsche darin Tag und Nacht, damit Du darauf achtest, alles zu befolgen, was darin geschrieben steht; denn dann wirst Du Gelingen haben auf Deinen Wegen, und dann wirst Du weise handeln.
        \end{bibelbox}
        Das gleiche Buch benutzen auch wir, mit der Ergänzung des Neuen Testaments. Wie es Josua geholfen hat, ein ganzes Land zu erobern, hilft es uns heute unseren Alltag zu bewältigen.
        Jeder von uns hat schon mal schlaflose Nächte erlebt und ich vermute mal, dass ihr es ähnlich macht wie ich. Ich fange dann an zu beten und hoffe, bald wieder einzuschlafen. Das ist so ähnlich, wie wenn ihr mit Eurem Partner redet, um einzuschlafen. Aber immerhin eine gute Gelegenheit in der Stille mit Gott zu verweilen.
    \end{block}   
    \subsection*{Teil 3: Gott vertrauen}
    \begin{block}
    David hat also nach der Suche, Gott gefunden und setzt nun sein ganzes Vertrauen auf Ihn.
    Wenn wir Gott von Herzen gesucht und gefunden haben, können wir ihm voll vertrauen. David setzt in Vers 10 sein ganzes Vertrauen auf Gott. Er plant keinen Rachefeldzug gegen seine Widersacher. Er vertraut darauf, dass sein Gott seine Widersacher bestrafen wird.
    \begin{bibelbox}{SCHL}{Ps}{63:10}
        Jene aber, die meine Seele verderben wollen, werden hinabfahren in die untersten Örter der Erde.
    \end{bibelbox}
    Es ist ein Trost, wenn man weiss, dass Gott für Gerechtigkeit sorgt. Wenn wir selber auf einen Rachefeldzug gehen, ist dies mit \betonung{noch} mehr Frust und Leid verbunden. Ausserdem wissen wir nicht, ob er auch gerecht ist. Das sehen wir doch nur zur Genüge auf diese Welt. In dieser Welt herrscht die Ungerechtigkeit. Auch in der Kirchengeschichte sieht man, dass im Namen der Gerechtigkeit viele Bosheiten verrichtet wurden. Also überlassen wir die Rache lieber unserem Herrn. Der weiss ganz sicher was gut ist. Und uns erspart es viel Frust und Ärger. 
    \end{block}
    \begin{block}
    Wenn der Nachbar etwas gegen uns hat, dann lenken wir doch lieber ein, als dass wir nachsinnen, wie wir ihm das zurückzahlen können. Paulus schreibt in Römer 12:20-21 , 
    \begin{bibelbox}{SCHL}{Rom}{12:20-21}
        Wenn nun Dein Feind Hunger hat, so gib ihm zu essen; wenn er Durst hat, dann gib ihm zu trinken! Wenn Du das tust, wirst Du feurige Kohlen auf sein Haupt sammlen.
    \end{bibelbox}
    Dies zitiert Paulus aus dem Buch der \bibleverse{Spr} {25:21-22}. Seht Ihr, welche tollen Tipps uns dieses Buch gibt? \leise{bibel zeigen}

    Wenn wir dem Bösen mit Gutem begegnen, hat das Böse keine Chance. Wie ist es aber mit den Verfolgten Christen die in Gefägnissen sitzen und gefoltert werden? Kann man da den Feinden, die einen foltern oder vielleicht sogar die Familie getötet haben, mit Güte begegnen? Ich denke ja, das kann man. Nicht weil wir es können, sondern, wie schon gesagt, uns uns Gott in solcher Not hilft. Und wir können getrost sein, dass Gott die Peiniger bestrafen wird. Rache ändert ja nichts an der schlechten Situation in der man aktuell ist. 
    
    Ich habe ein Andachtsbuch von OpenDoors \enquote{Mit Jesus im Feuerofen}. In diesem Buch schildern verfolgte Christen, wie sie Jesus in ihrer schlimmsten Not erlebt haben. Ich bin überzeugt, wenn wir seine Hilfe brauchen, wird sie da sein. Viele von den verfolgten Christen haben ihren Peinigern vergeben. Vergebung heisst nicht, dass sie der gerechten Strafe entgehen, aber wir können uns von unseren negativen Gefühlen befreien. In dem Wissen wie schrecklich die Strafe später sein wird, hat Stephanus bei seiner Steinigung für seine Mörder gebetet: \enquote{Herr, rechne ihnen diese Sünde nicht an.} \bibleverse{Apg}(7:60).
\end{block}
\begin{block}
    In Vers 11 drückt David nochmals aus, wie sicher er ist, dass Gott seine Widersacher bestrafen wird.
    \begin{bibelbox}{SCHL}{Ps}{63:11}
        Man wird Sie der Gewalt des Schwertes preisgeben, eine Beute der Schakale werden Sie sein!
    \end{bibelbox}
    Mit dieser Gewissheit kann David auf das harren, was da kommt. So kann er auch seine Mitstreiter trösten, in dem er ihnen verspricht, dass Gott alles zum Guten wenden wird. 
    Nicht weil er so ein toller Kerl ist, sondern weil er die stärkste Hilfe hat, die es geben kann, nämlich Gott. 
\end{block}
\begin{block}
    \begin{bibelbox}{SCHL}{Ps}{63:12}
        Der König aber wird sich freuen in Gott; wer bei ihm schwört, wird sich glücklich preisen, doch der Mund der Lügenredner wird gestopft!
    \end{bibelbox}
    In der Neuen evangelistischen Übersetzung heisst es:
    \begin{bibelbox}{NGU}{Ps}{63:12}
        Doch der König wird sich freuen, weil Gott zu Ihm hält. Glücklich schätzen kann sich jeder, der Sich bei einem Eid auf Gott beruft. Der Mund der Lügner aber wird gestopft.
    \end{bibelbox}
    \begin{block}
        Ich bin überzeugt, dass wenn wir mit Gott unterwegs sind, auch unser Umfeld davon profitiert. Auch wenn Du der einzige Gläubge bist. Wenn Du nach dem Wort Gottes lebst, fällt das auf Deine Umgebung zurück. Wie Jakob bei Laban oder Josef in Agypten. Bei Jakob war es der Vater seiner Braut der gesegnet wurde und bei Josepf sogar ein ganzes Land, dass von seinem Glauben profitierte.

        Hier in diesem Vers redet David von sich selber in der 3. Person als König. Er weiss, das Gott bei Ihm ist. Er hat Gott gesucht, gefunden und jetzt kann er auf Ihn vertrauen. Weil er Gott gefunden hat und Ihm vertraut, wird auch seine Gefolgsschaft davon profitieren. Alle diejenigen, die jetzt zu ihm halten, werden auch die Gnade Gottes erleben.

        Wir sollen nicht schwören. Wenn wir uns aber für den Heiland Jesus Christus entscheiden, ist das wie ein Schwur auf Gott.
    \end{block}
    \begin{block}
        Wir, die wir Jesus Christus in unserem Herzen tragen, sollen ein Licht für unser Umfeld sein. Vor allem, wenn unsere Ehepartner oder Familie Jesus noch nicht kennengelernt haben, sollen wir ein Licht für Ihn sein.     

        Du kannst Dich in Gott freuen und diese Freude weitergeben, vor allem dort, wo Du bist; auf der Arbeit, in der Schule oder in der Familie.
\end{block}
    Suche täglich in der Früh Deinen persönlichen Gott und wenn Du ihn gefunden hast, klebe Dich an ihn und vertraue Dich ihm zu 100\% an.\\
    \leise{Amen}\\
    \beten
\end{block}  
\end{document}