\author{OTS}
\documentclass[14pt]{../../inc/mybib}

\setincpath{../../inc/}

\usepackage{bible_style}
\graphicspath{{../../assets/images/}}
\usepackage{header}
\usepackage{numprint}
% \usepackage{parskip}

\newcommand{\q}[1]{\blockquote{#1}}

\author{Lothar Schmid}

\begin{document}
\setlength{\baselineskip}{1.5\baselineskip}

\section*{Psalm 63}
    \subsection{Einleitung}
    \subsection{Allgemeines}
    \begin{block}[Allgemeines]
    Der Author von Psalm 63 ist der König David. Dies wird uns im schon im ersten Vers mitgeteilt. Der Ort wo er diesen Psam geschrieben hat war in der Wüste von Judäa. Die Wüste Judäa liegt von Jerusalem aus in Richtung Totes Meer und Jericho. In der Bibel wir zweimal erwähnt, dass David in der Wüste von Judäa war. Das erste Mal, in     
    \begin{bibelbox}{SCHL}{1Sam}{23:14-15}
        David aber blieb in der Wüste auf den Bergfesten und hielt sich im Bergland auf, in der Wüste Siph. Und Saul suchte ihn alle Tage, aber Gott lieferte ihn nicht in seine Hand.
    \end{bibelbox}    
    David flieht hier vor Saul mit 600 Mann in die Wüste Siph. Dort hielt er sich vor Saul im Bergland versteckt. Das zweite Mal in \bibleverse{2Sam}(15:23-28). Hier flüchtet David vor seinem Sohn Absolom aus Jerusalem. Zu dieser Zeit war David der König von Jerusalem. So passt dieser Psalm besser in diese Zeit, weil David in Vers 12 auf einen König bezug nimmt. Als David diesen Psalm schrieb, hatte er doppeltes Leid zu tragen. Einmal wird es aus Jerusalem vertrieben und zweitens von seinem Sohn verraten.
    \end{block}
    \begin{block}
        Ich habe diesen Psalm in drei Teilen eingeteilt. 
        \begin{enumerate}
            \item Das Suchen nach Gott \bibleverse{Ps}{63:2-7}
            \item Das Finden von Gott \bibleverse{Ps}{63:8-9}
            \item Das Vertrauen in Gott \bibleverse{Ps}{63:10-12}
        \end{enumerate}
        Ich möchte gerne mit Euch diese Punkte durchgehen und schauen, was \betonung{wir} von David lernen können und wie \betonung{wir} diesen Psalm für uns verwenden können.
    \end{block}
    \subsection*{Punkt 1: Gott suchen}
    \begin{block}
    In Vers 2 ist Davids ganzes Glaubensbekenntnis. Sein Ausruf: \enquote{O Gott, du bist mein Gott}, kommt aus der Schlachterübersetzung ist ein richtig schöner Ausruf. In diesem Ausruf zeigt sich der Glaube von David an den Gott der Schrift, an den Gott seiner Väter und nimmt diesen Gott für sich persönlich in anspruch. \leise{Du bist mein Gott}. Ist das nicht herrlich? So  Gott zu vertrauen, dass man auch in grösster Not noch sagen kann, du bist \betonung{MEIN GOTT}. 

    \enquote{Früh suche ich dich}, übersetzen kann man das auch mit in der \enquote{Frühe},  wie zum Beispiel früh am Morgen. David hat also seinen persönlichen Gott schon in der Frühe gesucht und ihn angerufen.
    \end{block}

    \begin{block}
        Wie ist das mit uns? Ist Gott auch mein persönlicher Gott? Suche ich ihn auch schon in der Frühe, oder warte ich erst mal ab, was so im Laufe des Tages passiert?

        Wie war das als man frisch verheiratet zusammengezogen ist? Am Morgen nach der ersten gemeinsamen Nacht? War da unser erster Blick am Morgen nicht direkt auf unseren Schatz im Bett neben mir? Heute, je nach der Anzahl der Ehejahre, geht mein erster Blick auch nicht mehr auf meinen Schatz, sondern wohl auf den Wecker.

        Und wie ist das mit \betonung{meinem} Gott? Die, die wir den Glauben bezeugt haben? Wird es langsam zur Routine? Suchen wir Gott in der Früh? Und zwar nicht nur \enquote{den Gott}, sondern meinen Gott. \betonung{Meinen persönlichen Gott}.

        David hat für diese frühe Suche eine herrliche Bildersprache benutzt. \zitat{Meine Seele dürstet nach Dir, mein Fleisch schmachtet nach Dir.} Sein ganzer Körper und seine ganze Seele schmachtete und dürstete nach seinem Gott. Klar ihm ging es in der aktuellen Situation schlecht. Aber geht es uns viel besser? Hat nicht jeder Tag seine Probleme und Sorgen wo wir Gott brauchen können?

    \end{block}

    \begin{block}
        In Vers 3 sagt er uns, dass er Gott in Macht und Herrlichkeit im Heiligtum Gottes gesehen hat. Der Tempel war noch nicht gebaut, aber David hat die Bundeslade zurück nach Jeruslam gebracht. Bei dieser Aktion -- die Bibelstelle \bibleverse{2Sam}(6:1-23) beschreibt sie -- hat David die Bundeslade aus Kirjat-Jearim geholt. Bei diesem Transport, strauchelten die Tiere und Usa hat die Lade festgehalten, wurde aber von Gott mit dem Tod bestraft. Gott hat dies so im Gesetz verordnet. 
        \begin{bibelbox}{SCHL}{4Mos}{4:15}
            \dots sie sollen aber das Heiligtum nicht anrühren, sonst würden sie sterben.
        \end{bibelbox}
        Diese Aktion hat David die Macht Gottes gezeigt. David brachte die Lade dann nach Obed-Edom und später unter Tanz und Reigen nach Jerusalem.
    \end{block}
    \begin{block}
        Auch wir wiedergeborenen Christen haben die Herrlichkeit Gottes gesehen. Gott hat alles für uns gegeben, was er geliebt hat. Seinen Sohn Jesus Christus. Jeder der Jesus gesehen hat und in sein Herz aufgenommen hat, hat die Herrlichkeit Gottes gesehen.
        \begin{bibelbox}{SCHL}{2Kor}{4:6}
            Denn Gott, der dem Licht gebot, aus der Finsternis hervorzuleuchten, er hat es auch in unseren Herzen licht werden lassen, damit wir erleuchtet werden mit der Erkenntnis der Herrlichkeit Gottes im Angesicht Jesu Christi.
        \end{bibelbox}
    \end{block}

    \begin{block}
        Als David diesen Psalm schrieb, hatte er sicher Angst, Jerusalem und das Heiligtum nie mehr zu sehen. Haben nicht auch wir manchmal diese Angst? Wenn etwas im Leben schiefgelaufen ist? Wir das Gefühl haben, dass Gott meilenweit entfernt ist? Liebt mich Gott noch? Darf ich zu ihm zurück? Wir brauchen uns aber keine Sorgen zu machen. David wusste, wenn er Gnade vor dem Herrn gefunden hat, würde er die Herrlichkeit in Jerusalem wieder sehen. Er schickte die Lade, als sie ihm gebracht wurde wieder zurück nach Jerusalem.
        \begin{bibelbox}{SCHL}{2Sam}{15:25}
            Aber der König sprach zu Zadiok: Bringe die Lade Gottes wieder in die Stadt zurück! Wenn ich Gnade cor dem Herrn finde, so wird er mich zurückbringen, dass ich ihn und seine Wohnung wiedersehen darf
        \end{bibelbox}
        \betonung{Wir}, die wir mit Jesus unterwegs sind, haben diese Gnade auch bekommen. Wir können ihn enttäuschen ja das können wir und tun es auch täglich, aber er geht nicht mehr weg, ER lässt uns nicht allein. Wir können immer zu IHM, \betonung{unserem Retter und Erlöser} zurückkommen. Wie auch David, sich auf die Gnade Gottes berufen hat, so können wir das noch viel mehr.
    \end{block}

    \begin{block}
        David hat in seinem Leben die Gnade und die helfende Hand Gottes oft erlebt. Für David ist ein Leben ohne diese Gnade nicht lebenswert. Wir er in Vers 4 sagt.

        Wie gross ist die Unzufriedenheit in dieser Welt? Auch hier in unseren Breiten, wo wir doch alles zum Leben haben. Essen, Trinken, ein Dach über dem Kopf, Medizin usw., ist diese Unzufriedenheit zu bemerken. Medien und Influenzer gauckeln uns vor wie toll das Leben wäre, wenn wir noch dieses und jenes hätten.
        
        Viele reiche und erfolgreiche Prominente sind unzufrieden. Trotz ihres Reichtums und Ruhm merken sie, dass ihnen etwas fehlt. Nichts können sie mitnehmen. Das Leben verschwindet und sie stehen mit leeren Händen da.
        
        Nur die Gnade des Herrn kann dem Leben einen Sinn geben. Wie David auf die Gnade Gottes vertraut, können auch wir auf die Gnade Gottes vertrauen. Diese Gnade gibt dem Leben einen Sinn und einen Halt. 
    \end{block}

    \begin{block}
        Diese Sicherheit und Halt drückt David im Vers 4 aus. \enquote{Meine Lippen sollen Dich rühmen.}

        Wenn ich hier endlich fertig bin, können wir unseren Herrn gemeinsam mit unseren Lippen mit Gebeten und Lieder rühmen.
    \end{block}
    \bibleverse{Ps}(63:2-7)
    \subsection*{Teil 2: Gott finden}
    \begin{block}
        Die Verse 8-9 drücken aus was David jetzt als er Gott gefunden hat mit dieser Gnade macht. Es dieses Vertrauen die David auf Gott hat. Gut wir können sagen, dass David Gott in seinem Leben viel näher erlebt hat als wir. Wir wissen, das David schon früh in seinem Leben sich unter Gottes schutz gestellt hat. Wo er noch als Junge mit den Schafen unterwegs war, hat er schon mit Gottes Hilfe Wölfe und Bären getötet. In \bibleverse{2Mos}(19:4) lesen wir. Wie Gott Israel auf Adlersflügel getragen als sie vor den Ägypter geflohen sind. Auch \bibleverse{5Mos}(32:11-12) wird das Symbol von Adlerflügel für den Schutz von Israel verwendet. Diesen Schutz hat David, nach seiner Suche gefunden. 
    \end{block}
    \begin{block}
        Wenn wir uns schon am Morgen in der Frühe auf die Suche nach Gott machen, können wir uns auch schon Früh unter den Schutz seiner Flügel stellen. Nirgends steht um welche Zeit man seine persönliche Stille Zeit halten soll, aber machen wir diese am Morgen vor dem Alltag gibt es uns Schutz und Sicherheit für die Probleme die unweigerlich darauf folgen. Ich habe schon erlebt, dass ich nach dem Gebet und Bibellesen Probleme, an denen ich die halbe Nacht studiert habe, plötzlich unwichtig vorkamen. Bei anderen Problemen mich hingesetzt habe und diese angegangen bin. Oft denken wir, dieses Problem ist doch so klein und unwichtig und wollen damit Gott nicht belästigen. Aber! Ersten: Wir können Gott nicht belästigen, dieser Gedanke sugeriert, dass wir über Gott stehen. Zweitens: Alle Probleme die wir kleinen Menschen hier auf dieser Welt haben sind für Gott klein. Er ist ein Gott, der mit einem Wort das ganze Universum und die ganze Erde geschaffen hat. Manchmal hört man von Christen, die sechs Tage könne man nicht so wörtlich nehmen. Ich sage mit eher, wieso hat er überhaupt so lange gebraucht. Ist aber jetzt nicht unser Thema und so kehren wir zurück zu David was er noch macht nachdem er jetzt seinen Gott gefunden hat.
    \end{block}
    \begin{block}
        Der Vers 9 ist einfach Gewaltig. Laut Benedikt Peters kann das Wort \green{hängen} auf mit \green{kleben} übersetzt werden. Das gleiche Wort wird auch in \bibleverse{1Mos}(2:24) benutzt, wo es heisst, dass der Mann in der Ehe seiner Frau \enquote{angehangen} werde.

        Kleben heisst, es ist fest verbunden. Wenn zwei Papierblätter zusammenklebt, dann sind diese so fest miteinnander verbunden das man die nicht mehr ohne schäden trennen kann. Dieses kleben können wir auch in der Aussage von Jesus verwenden wo er sagt dass er uns nicht mehr loslässt. \bibleverse{Joh}(10:27-29). Jeder der Jesus Christus von Herzen aufnimmt, seine Sünden vor den Herrn legt, klebt an ihm. Wenn wir dann fest an unserem Herrn kleben, können wir auch gewiss sein, dass uns seine Rechte aufrecht hält. Nicht falsch verstehen, es geht mir nicht drum, dass es uns Materiel und Wohlstandsmässig gut oder besser geht. Wie sich diese Rechte uns hält, finden wir in den Gottesmänner in den Paulus Briefen und auch in den Verfolgten Christen auf dieser Welt. Ein Missionar von Open Doors hat mir mal gesagt, dass er überrascht ist, dass wenn man die Verfolgten Cchristen fragt was man für sie Beten soll, gesagt selten gesagt wird, dass Gott ihnen das Leid nehmen soll, sondern die Kraft im Leid zu bestehen geben soll. Das ist iiese Rechte die uns hält. Die ist dann immer da, wenn wir sie wirklich brauchen. Wenn man mit kleinen Kinder in die Berge geht, nimmt man diese auch erst bei gefährlichen Passagen an die Hand und nicht den ganzen Tag. Gott ist es wichtig, dass wir selbständig sind, aber auch ehrlich zu ihm kommen, wenn wir das Gefühl haben jetzt wird es brenzlig für mich.
        
        Und da kommen wir zum letzten Teil dieser Predigt. Gott vertrauen. Wir haben meinen persönlichen Gott gesucht, ihn gefunden und jetzt dürfen wir Ihm vertrauen.
    \end{block}
    \bibleverse{Ps}(63:8-9)
    \subsection{Gott vertrauen}
    \bibleverse{Ps}(63:10-12)

   
\end{document}