\author{OTS}
\documentclass[14pt]{../../inc/mybib}

\setincpath{../../inc/}

\usepackage{bible_style}
\graphicspath{{../../assets/images/}}
\usepackage{header}
\usepackage{numprint}
% \usepackage{parskip}

\newcommand{\q}[1]{\blockquote{#1}}

\author{Lothar Schmid}

\begin{document}
\setlength{\baselineskip}{1.5\baselineskip}

\section*{Psalm 63}
    \subsection{Einleitung}
    \subsection{Allgemeines}
    \begin{block}[Allgemeines]
    Der Author ist der König David. Dies wird uns im schon im ersten Vers mitgeteilt. Der Ort wo er diesen Vers schreibt war in der Wüste von Judäa. Die Wüste Judäa liegt vor den Toren Jerusalem in Richtung Totes Meer. In der Bibel wir zweimal erwähnt, dass David in der Wüste von Judäa war. Das erste mal in \bibleverse{1Sam}(23:14-15). David flieht hier vor Saul mit 600 Mann in die Wüste Siph. Dort hielt er sich vor Saul im Bergland versteckt. Das zweite mal in \bibleverse{2Sam}(15:23-28). Hier flüchtet David vor seinem Sohn Absolom aus Jerusalem. Zu dieser Zeit war David also der König von Jerusalem. So passt dieser Vers besser in diese Zeit, weil David in Vers 12 auf einen König bezug nimmt. Als David diesen Psalm schrieb hat er also doppeltes Leid zu tragen. Einmal wird es aus Jerusalem vertrieben und zweitens von seinem Sohn verraten.
    \end{block}
    \begin{block}
        Ich habe diesen Psalm in drei Teilen eingeteilt. 
        \begin{enumerate}
            \item Das Suchen \bibleverse{Ps}{63:2-7}
            \item Das Finden \bibleverse{Ps}{63:8-9}
            \item Das Vertrauen \bibleverse{Ps}{63:10-12}
        \end{enumerate}
        Ich möchte gerne mit Euch diese Punkte durchgehen und schauen was wir von David lernen können und wie wir diesen Psalm für uns verwenden können.
    \end{block}
    \subsection{Gott suchen}
    \begin{block}
    In Vers 2 ist sein ganzes Glaubensbekenntnis. Sein Ausruf: \enquote{O Gott, du bist mein Gott}. In der Schlachterübersetzung ist dies ein richtig schöner Ausruf. In diesem Ausruf zeigt sich der Glaube an den Gott der Schrift, an den Gott seiner Väter und nimmt diesen Gott für sich persönlich. Du bist mein Gott. Ist das nicht herrlich? So ein Gott vertrauen, dass man auch in grösster Not noch sagen kann, du bist \red{MEIN GOTT}. 

    \enquote{Früh suche ich dich}, übersetzen kann man das auch mit in der \enquote{frühe}. Wie früh am Morgen. David hat also seinen persönlichen schon in der Frühe gesucht und ihn angerufen.
    \end{block}
    \begin{block}
        Wie ist das mit uns. Ist Gott auch mein persönlicher Gott? Suche ich ihn auch schon in der Frühe, oder warte ich erst mal ab was so im laufe des Tages passiert?\\
        Wie war das als man frisch verheiratet zusammengezogen ist? Nach der ersten gemeinsamen Nacht? War unser erster Blick am Morgen nicht auch auf unseren Schatz im Bett neben mir? Heute je nach der Anzahl der Ehejahre ist der erste Blick auch nicht auf meinem Schatz, sondern wohl eher auf den Wecker. Vielleicht, noch wenn man wegen dem Schnarchen aufgewacht ist.

        Und wir die wir den Glauben bezeugt haben? Wird es langsam zur routine? Suchen wir Gott in der Früh? Und zwar nicht nur \enquote{der Gott}, sondern meinen Gott. Meinen persönlicher Gott.\\

        David hat für diese seine Suche eine herrliche Bildersprache benutzt. \enquote{Meine Seele dürstet nach dir, mein Fleisch schmachtet nach dir.} Sein ganzer Körper und seine ganze Seele sucht Gott in der Frühe.

    \end{block}
    \bibleverse{Ps}(63:2-7)
    \subsection{Gott finden}
    \bibleverse{Ps}(63:8-9)
    \subsection{Gott vertrauen}
    \bibleverse{Ps}(63:10-12)

   
\end{document}