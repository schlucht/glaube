\author{OTS}
\documentclass[12pt]{../../inc/mybib}

\setincpath{../../inc/}

\usepackage{bible_style}
\usepackage{header}
\usepackage{numprint}
\author{Lothar Schmid}
\begin{document}
\setlength{\baselineskip}{2.5\baselineskip}

\section{Babilon 1 Mose 11}

\hh{1} Und die ganze Erde hatte eine einzige Sprache und dieselben Worte. \hh{2} Und es geschah, als sie nach Osten zogen, da fanden sie eine Ebene im Land Sinear, und sie liessen sich dort nieder.

\hh{3}Und sie sprachen zu einnander: Wohl an, lass uns Ziegel streichen und sie feuerfest brennen! Und sie verwendeten Ziegel statt Steine und Asphalt statt Mörtel. \hh{4}Und sie sprechen: Wohlan lasst uns eine Stadt bauen und eine Turm, dessen Spitze bis an den Himmel reicht, dass wir uns einen Namen machen, damit wir ja nicht über die ganze Erde zerstreut werden!

\hh{5}Da stieg der \herr herab, um die Stadt und den Turm anzusehen, den die Menschenkinder bauten. \hh{6}Und der \herr sprach: Siehe, sie sind ein Volk und sprechen, und sie sprechen eine Sprache, und dies ist [erst] der Anfang ihres Tuns! Und jetzt wird sie nichts davor zurückhalten, das zu tun, was sie sich vorgenommen haben. \hh{7}Wohlan, lasst uns hinabsteigen und dort ihre Sprache verwirren, damit keiner mehr die Sprache des anderen versteht!

\hh{8}So zerstreute der \herr sie von dort über die ganze Erde, und sie hörten auf, die Stadt zu bauen. \hh{9}Daher gab man ihr den Namen Babel, weil der \herr dort die Sprache der ganzen Erde verwirrte und sie von dort über die ganze Erde zerstreute.

\end{document}