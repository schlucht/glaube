\author{OTS}
\documentclass[14pt]{../../inc/mybib}

\setincpath{../../inc/}

\usepackage{bible_style}
\usepackage{header}
\usepackage{numprint}

\author{Lothar Schmid}
\begin{document}
\setlength{\baselineskip}{1.5\baselineskip}
\section{Babylon 1 Mose 11}

\subsection{Der Turmbau in der Bibel (15min)}        
    Babylon ist neben Jerusalem wohl die Stadt, die allen bekannt ist. Ob Christen, Theologen oder Menschen die nichts mit dem Christentum am Hut haben. Erwähnt man diese Stadt, denken auch nicht Bibelkenner direkt an den Turmbau von Babel. Schon als Kinder haben wir diese Geschichte aus der Bibel gehört. Babel ist auch für Zügellosigkeit und Sünde bekannt. Ausser dem gibt es ein Babel zum Sprachen lernen, oder ein Babel in der Programmierwelt, wo man eine Programmiersprache in verschiedene Sprachversionen umwanden kann.
    
    Die Geschichte vom Bau des Turmes zu Babel finden wir in Bibel in \bibleverse{IMos} {11:1-9}. Es sind nicht viele Verse, ist jetzt eigentlich auch nicht unbedingt eine spannende Geschichte, trotzdem hat diese Geschichte viele Maler, Gestalter, Dichter und Schriftsteller inspiriert, darüber zu schreiben und ihre Kunst zu verwirklichen. Wollen wir doch zusammen die Verse 1. Mose 11:1-9 einmal lesen. \green{Bibeltext lesen}.

    Die Geschichte findet zu einer Zeit statt, wo die Menschen die gleiche Sprache und Wortschatz hatten. (Vers1) In Vers 2 lesen wir, dass sie sich gemeinsam Richtung Westen auf den Weg gemacht haben. Im Land Schinar liessen liessen sie sich in einer Ebene nieder.  (\bibleverse{IMos}(9:1-3))

    Wer war dieses Volk? Wann hat diese Völkerwanderung stattgefunden? Wenn wir ein Kapitel vorher in den Versen \bibleverse{IMos} (10:7-8) nachlesen, finden wir da einen Stammbaum von Noah. Wir wissen, das Noah und seine Söhne noch zu den Menschen gehörten, die sehr alt wurden. Das finden wir im Kapitel 11 in den Versen 10-26, dort ist der Stammbaum von Sem bis Terach dem Vater von Abraham aufgelistet. Ich habe mir mal die Mühe gemacht, diese Alter in einem Balkendiagramm darzustellen, da sieht man, dass Abraham nur wenige Jahre nach dem Tod von Noah auf die Welt kam. Wenn wir jetzt beachten, dass Nimrod die 3. Generation nach Noah ist, wurde Nimrod sicher auch um die 500 Jahre alt. Die Menschen starben aber immer jünger. In der Bevölkerung gab es Menschen die Uralt waren und solche die jung starben. Dies kann schon speziell sein. Wie ist es wenn hier in Naters der Sepp wohnt, der schon alt war, als mein Urgrossvater, Grossvater und Vater noch lebten, und wenn ich dann auch alt bin lebt der gute Sepp immer noch fröhlich weiter. Von Nimrod wird ja in der Bibel gesagt, dass er der erste Gewaltsherrscher war. Nimrod war also der Machthaber und versammelt das Volk und will um seine Macht zu festigen eine grosse Stadt mit einem riesigen Turm bauen. Also gingen sie an die Arbeit. 

    Wollen wir erst mal anschauen, wieso Nimrod überhaupt dieser Turm bauen wollte. Nimrod kannte den Gott der Bibel. Er hat sicher erfahren, dass dieser Gott eine Sindflut ausgelöst hat und ausser Noah mit seiner Familie getötet hat. Noah und seine Söhne lebten ja zu der Zeit noch. Nimrod war ein Machtherrscher, er wollte sich verewigen, er wollte ein Denkmal. Er ist stärker als dieser unsichtbare Gott. Er hat sich über Gott hinweggesetzt und hat es in die eigene Hand genommen.
    
    Im Vers 5 steigt Gott vom Himmel runter und guckt sich das ganze Treiben an. Wieso war Gott nicht zufrieden, was er sah? Ist Gott dagegen, dass die Menschen eine Stadt und einen Turm bauten? Nein, das glaube ich nicht. Gott hatte Noah einen genauen Auftrag gegeben. In 9,1 steht Gottes Auftrag. Da steht nicht \flqq baut euch eine schöne Stadt\frqq{} sondern, \frqq verteilt euch auf der ganzen Welt und bearbeitet diese.\frqq{} Gottes Wort wurde also ignoriert und das Schicksal wurde selber in die Hände genommen. Menschen in Team können viel leisten. Gemeinsame Sprache, gemeinsamer Ausstausch, das ist die Voraussetzung für Fortschritt. Das sehen wir daran, dass sie jetzt nicht mehr Steine, sondern Ziegel selber brannten zum Bau. Ziegel sind leichter und einfacher damit eine Mauer zu machen.  Aber die Zeit war noch nicht reif. Das war nicht Gottes Plan. Roger Liebi hat da ein interessanten Gedanken. Die Menschen zu der Zeit waren ja nicht blöder als wir heute. Gemeinsam im Team hätten sie viel leisten können. Der Fortschritt wäre zu schnell gegangen. Liebi sagt, hätte Gott dem nicht ein Riegel vorgeschoben, hätten die Römer schon mit Raketen Krieg geführt. Gott hat eingriffen und den verschiedenen Sippen eine andere Sprache geben und die Kommunikation unter einander war vorbei. Ohne Kommunikation war die Teamarbeit sofort beendet. Also sind die einzelnen Sprachgruppen weggezogen und haben sich auf der ganzen Welt verteilt.

    Was zurück blieb, war ein unfertig gebauter Turm. Ist die Geschichte nun zu Ende? Nein, ich finde nicht. Wollen wir doch jetzt mal im Volk Israel schauen wo diese ihre Türme gebaut haben.
    
    \subsection{Turmbau von Israel}
    Wir haben jetzt also eine Stadt mit dem Namen Babel, einen halbfertigen Turm von dem noch heute das Fundament zu sehen ist. 

    Der Urvater des Israelischen Volkes ist Abraham. Abraham wohnte in Ur und wurde mit 75 Jahren von Gott aus erwählt, der Gründungsvater Gottes Volk zu werden. In Kapitel 12,1 sagt Gott zu Abraham, \flqq Geh hinaus aus deinem Land und aus deiner Verwandtschaft.\frqq{} Klare Anweisung, nicht leicht unzusetzen mit 75 aber die Anweisung war. Abraham hat aber sein eigenes Türmchen gebaut. Er nahm noch Lot mit. In der späteren Geschichte lesen wir dann von den Problemen mit Lot. Die Entführung, Sodom und Gomorra zb. Auch das Kind mit der Sklavin Hagars war ein Turm ohne Gottes Hilfe. 
    
    Seht ihr worauf ich raus will? Gott gibt eine Anweisung. Der Mensch denkt sich, er sei schlauer könne sein eigenen Turm bauen. Diese Türme werden nie beendet. Baut man mit Gottes willen funktionierts.

    Gehen wird doch weiter mit der Suche auf unfertige Türme. Die Probleme in der Wüstenwanderung kennen wir. Ein gutes Beispiel finden wir in 3.Mos 14, 39-45. Gott hat gesagt, ihr müsst jetzt 40 Jahre in der Wüste wandern, bis alles die >20Jahre sind gestorben sind. Puh, da haben sie wohl gesehen, dass die Rebellion nicht so gut war. Um ihr vergehen wieder gut zu machen, wollen sie ihre Zukunft in die eigene Hand nehmen. Moses warnt sie davor, ihren eigen Turm zu bauen. Sie hörten nicht auf Moses und zogen gegen die Amalekiter in den Krieg und wurden von diesen gläglich in die Fluch geschlagen.

    Auch im Buch Richter finden wir ein schönes Beispiel. Nämlich Gideon. Er hat doch erlebt, wie Gott hilft Ri 7:1-. Wieso will er jetzt am Ende seine Geschicke in die eigenen Hände nehmen?Ri 8:24-27.

    Gehen wir weiter zu Saul. Auch da ein Beispiel von einem persönlichen Turmbau der nie fertig wurde. Saul sollte (1. Samuel 15). Von da an war sein Königsturm nur noch ein Trümmerhaufen. Auch dieser Turm wurder nie beendet.

    Es sind jetzt nicht nur die einzelne Person in Israel die angefangen haben Türme zu bauen. Das Volk wollte einen König, Gott war nicht mehr gut genug. Gott hat schon in 5.Mose davor gewarnt einen König aus zu erwählen. Sobald Menschen an der Macht sind, verändern sie sich. Schauen wir uns ein Beispiel an, und zwar der König Ussijas. Ussija war 52 Jahre lang König von Judäa. Über ihn lesen wir in 2. Chronik 26:4-5, dass er das tat was der \herr{} wollte. In Vers 16 im gleichen Kapitel lesen wir dann, dass ihm die Macht in den Kopf gestiegen ist. Ab diesem Zeitpunkt fing er an seinen eigenen Turm zu bauen. Ussija opferte gegen die Anweisung des \herr n selber im Tempel. Die Priester warnten ihn, es nicht zu tun. Ussija wurde gegenüber die Priester zornig. Gott strafte ihn mit Aussatz. So wurde sein persönlicher Turmbau beendet.

    Nachdem die aus dem ersten Exil zurück nach Israel und Jerusalem kamen bauten sie den Tempel wieder auf. Sie fingen wieder mit den Opfer an. Es wurden Schriftgelehrte ausgebildet, welche die Menschen in der Schrift unterwiesen. In den Bücher der Makkabäer ist die Geschichte Israels beschrieben. Aus dieser Zeit entstanden auch die Pharisäer. Ich nehme an, dass diese zu beginn es sicher gut meinten und das Volk vor schaden und Sünde vor Gott bewahren wollten. Aber auch diese Gruppe fing an ihren eigenen Turm zu bauen. Sie verschärften die Gesetze, bereicherten sich an der Bevölkerung und spielte sich als die wahren Gläubigen auf. In den Evangelien können wir nachlesen wie Jesus zu diesen Pharisäer stand. Jesus hat diese Gruppe scharf kritisiert. Er nannte sie Blinde die Blinde führen. Wie fatal das ist, könnt ihre euch denken. Der Turmbau dieser Gruppe endete schrecklich. Einmal in den Tod des Herrn Jesu und dann später auch die Zerstörung von Jerusalem durch die Römer mit vielen Toten. Josephus redet von Millionen von Toten.

    \subsection{Der Turmbau der Gemeinde}
    Als Jesus unter uns weilte, gab er uns genaue Anweisungen. Diese Anweisungen hat uns der \herr{} mit den Evangelien mitgegeben. Die Briefe zeigen uns auf wie wir unser Christenleben leben sollen. Jesus hat uns gesagt wie wir von unseren Sünden befreit werden und wie wir die Gande Gottes erlangen können. Joh 11:25-26. Ausserdem gibt er uns Wiedergeborenen Christen einen genauen Auftrag bevor er in den Himmel aufgenommen wurde. Math 28:19, Mk 16:15, Apg 1:8. Eigentlich klar oder? Nur das Evangelium verkünden.

    Am Anfang klappte es auch. Nachdem Petrus das Evangelium verkündet hat, haben sich 3000 Menschen zu Jesus bekehrt. Petrus hat nur eine Predigt gehalten mehr nicht. Irgendwann reichte das den Menschen nicht mehr und fingen an einen Turm zu bauen. 
    

    

\end{document}