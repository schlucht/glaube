\author{OTS}
\documentclass[14pt]{../../inc/mybib}

\setincpath{../../inc/}

\usepackage{bible_style}
\graphicspath{{../../assets/images/}}
\usepackage{header}
\usepackage{numprint}

\newcommand{\q}[1]{\blockquote{#1}}

\author{Lothar Schmid}
\begin{document}
\setlength{\baselineskip}{1.5\baselineskip}
\section{Babylon 1 Mose 11}

\subsection{Der Turmbau in der Bibel (15min)}        
    Babylon ist neben Jerusalem wohl die Stadt aus der Bibel, die allen bekannt ist. Ob Christen, Theologen oder Menschen, die nichts mit dem Christentum am Hut haben, kennen diese Geschichte mit dem Turmbau. Schon als Kinder haben wir diese Geschichte aus der Bibel gehört und es spannend gefunden, wie hoch der Turm wohl war. Babel ist aber auch für Zügellosigkeit und Sünde bekannt geworden. Babel wird aber auch im Zusammenhang mit Sprachen benutzt. Es gibt eine Babel-App, um Sprachen zu lernen, oder ein Babel-Programm in der Programmierwelt, mit dem man eine Programmiersprache in verschiedene Sprachversionen übersetzen kann.
    
    Die Geschichte vom Bau des Turmes zu Babel finden wir in der Bibel in \bibleverse{IMos} {11:1-9}. Es ist eine kurze Geschichte, es ist jetzt auch nicht unbedingt eine spannende Geschichte, trotzdem hat diese Geschichte viele Maler, Gestalter, Dichter und Schriftsteller angeregt, darüber zu Schreiben, zu Malen und ihren Fantasien freien lauf zu lassen. 
    \\
    \underline{\textbf{Wollen wir doch zusammen die Verse 1. Mose 11:1-9 einmal lesen.}}
    \\
    Diese Geschichte findet zu einer Zeit statt, wo die Menschen noch eine gleiche Sprache und Wortschatz hatten.
    \begin{bibelbox}{SCHL}{1Mos}{11:1}
        Und die ganze Erde hatte eine einzige Sprache und dieselben Worte.
    \end{bibelbox}
    
    In Vers 2 lasen wir, dass sie sich gemeinsam Richtung Westen auf den Weg gemacht haben. Im Land Schinar liessen sie sich in einer Ebene nieder.  (\bibleverse{IMos}(9:1-3))

    Wer war dieses Volk? Wann hat diese Völkerwanderung stattgefunden? Wenn wir ein Kapitel vorher in den Versen \bibleverse{IMos} (10:7-8) nachlesen, finden wir da einen Stammbaum von Noah. Wir wissen, das Noah und seine Söhne noch zu den Menschen gehörten, die sehr alt wurden. Das finden wir im Kapitel 11 in den Versen 10-26, dort ist der Stammbaum von Sem bis Terach dem Vater von Abraham aufgelistet. Ich habe mir mal die Mühe gemacht, diese Alter in einem Balkendiagramm darzustellen, da sieht man, dass Abraham nur wenige Jahre nach dem Tod von Noah auf die Welt kam. Wenn wir jetzt beachten, dass Nimrod die 3. Generation nach Noah ist, wurde Nimrod sicher auch um die 500 Jahre alt. Jede Gemeration starb aber immer jünger. In der Bevölkerung gab es Menschen die Uralt waren und solche die jung starben. Dies kann schon speziell sein. Wie ist es, wenn hier in Naters der Sepp wohnt, der schon alt war, als mein Urgrossvater, Grossvater und Vater noch lebten, und wenn ich dann auch alt bin, lebt der gute Sepp immer noch fröhlich weiter. Von Nimrod wird ja in der Bibel gesagt, dass er der erste Gewaltsherrscher war. Es gibt auch Ausserbiblische Erzählung aus der Zeit die von Menschen zeugen die sehr alt wurden. zb Gilga-Mesh. Nimrod war also der Machthaber und versammelt das Volk und will um seine Macht zu festigen eine grosse Stadt mit einem riesigen Turm bauen. Also gingen sie an die Arbeit. 

    Wollen wir erst mal anschauen, wieso Nimrod überhaupt dieser Turm bauen wollte. Nimrod kannte den Gott der Bibel. Er hat sicher erfahren, dass dieser Gott eine Sindflut ausgelöst hat und ausser Noah und seine Familie alles getötet hat was an Land lebte. Noah und seine Söhne lebten ja zu der Zeit noch. Nimrod war ein Machtherrscher, er wollte sich verewigen, er wollte ein Denkmal. Er ist stärker als dieser unsichtbare Gott. Er hat sich über Gott hinweggesetzt und hat es in die eigene Hand genommen.
    
    Im Vers 5 steigt Gott vom Himmel runter und guckt sich das ganze Treiben an. Wieso war Gott nicht zufrieden, was er sah? Ist Gott dagegen, dass die Menschen eine Stadt und einen Turm bauten? Nein, das glaube ich nicht. Gott hatte Noah einen genauen Auftrag gegeben. In 9,1 steht Gottes Auftrag. Da steht nicht \q{baut euch eine schöne Stadt} sondern, \q{verteilt euch auf der ganzen Welt und bearbeitet diese.} Gottes Wort wurde also ignoriert und das Schicksal wurde selber in die Hände genommen. Menschen in Team können viel leisten. Gemeinsame Sprache, gemeinsamer Austausch, das ist die Voraussetzung für Fortschritt. Das sehen wir daran, dass sie jetzt nicht mehr Steine, sondern Ziegel selber brannten zum Bau. Ziegel sind leichter und einfacher damit eine Mauer zu bauen. Aber die Zeit war noch nicht reif. Das war nicht Gottes Plan. Roger Liebi hat da ein interessanten Gedanken. Die Menschen zu der Zeit waren ja nicht dummer als wir heute. Gemeinsam im Team hätten sie viel leisten können. Der Fortschritt wäre zu schnell gekommen. Liebi sagt, hätte Gott dem nicht ein Riegel vorgeschoben, hätten die Römer schon mit Raketen Krieg geführt.\\
    Gott hat eingriffen und den verschiedenen Sippen eine andere Sprache gegeben und die Kommunikation unter einander war vorbei. Ohne Kommunikation war die Teamarbeit schnell beendet. Also sind die einzelnen Sprachgruppen weggezogen und haben sich auf der ganzen Welt verteilt. \bibleverse{IMos}(12:)

    Was zurück blieb, war ein unfertig gebauter Turm. Ist damit das Thema Turmabau beendet? Wollen wir doch jetzt mal im Volk Israel schauen wo diese ihre Türme gebaut haben.
    
    \subsection{Turmbau von Israel}
    Wir haben jetzt also eine Stadt mit dem Namen Babel und einen halbfertigen Turm. Ein ausgegrabenes Fundament wird diesem Turm zugeteilt. 

    Der Urvater des Israelischen Volkes ist Abraham. Abraham wohnte in Ur und wurde mit 75 Jahren von Gott auserwählt, der Gründungsvater vom Gottes Volk zu werden. In Kapitel 12,1 sagt Gott zu Abraham, \q{Geh hinaus aus deinem Land und aus deiner Verwandtschaft.} Klare Anweisung, nicht leicht unzusetzen mit 75 aber die Anweisung war klar. Abraham hat aber sein eigenes Türmchen gebaut. Er nahm noch Lot mit. In der späteren Geschichte lesen wir, von den Problemen mit Lot. Die Entführung Lots, Sodom und Gomorra und der Inzucht mit seinen Töchtern.  
    
    Seht ihr worauf ich raus will? Gott gibt eine Anweisung und wir denken uns wir seien schlauer könne unser eigenen Turm bauen. Wie viele Türme ich, in meinem kurzen Glaubensleben schon gebaut?

    Gehen wir weiter mit der Suche nach unfertigen Türmen. Die Probleme in der Wüstenwanderung kennen wir. Ein gutes Beispiel finden wir in 3.Mos 14, 39-45. Gott hat gesagt, ihr müsst jetzt 40 Jahre in der Wüste wandern, bis alles die >20Jahre sind gestorben sind. Puh, da haben sie wohl gesehen, dass die Rebellion gegen den Einmarsch ins Gelobte Land \bibleverse{IIIMos}(14:) nicht so gut war. Um ihr vergehen wieder gut zu machen, wollen sie ihre Zukunft in die eigene Hand nehmen. Moses warnt sie davor, ihren eigen Turm zu bauen. Sie hörten nicht auf Moses und zogen stur gegen die Amalekiter in den Krieg und wurden von diesen kläglich in die Flucht geschlagen.

    Auch im Buch Richter finden wir ein schönes Beispiel. Nämlich Gideon. Er hat doch erlebt, wie Gott hilft Ri 7:1-. Wieso will er jetzt am Ende seine Geschicke in die eigenen Hände nehmen? \bibleverse{Ri}(8:24-27). 

    Gehen wir weiter zu Saul. Auch da ein Beispiel von einem persönlichen Turmbau der nie fertig wurde. Saul sollte \bibleverse{ISamuel}(15:). Von da an war sein Königsturm nur noch ein Trümmerhaufen. Auch dieser Turm wurde nie beendet.

    Es sind jetzt nicht nur die einzelne Person in Israel die angefangen haben Türme zu bauen. Das Volk wollte einen König, Gott war nicht mehr gut genug. Gott hat schon in \bibleverse{VMos}(1:) davor gewarnt einen König aus dem Volk zu erwählen. Sobald Menschen an der Macht sind, verändern sie sich. Schauen wir uns ein Beispiel an, und zwar der König Ussijas. Ussija war 52 Jahre lang König von Judäa. Über ihn lesen wir in \bibleverse{2Chronik}(26:4-5), dass er tat was der \herr{} wollte. In Vers 16 im gleichen Kapitel lesen wir dann, dass ihm die Macht in den Kopf gestiegen ist. Ab diesem Zeitpunkt fing er an seinen eigenen Turm zu bauen. Ussija opferte gegen die Anweisung des \herr n selber im Tempel. Die Priester warnten ihn, es nicht zu tun. Ussija wurde gegenüber den Priester zornig. Gott bestrafte ihn mit Aussatz. So wurde sein persönlicher Turmbau beendet.

    Nachdem die Juden aus dem ersten Exil zurück nach Israel und Jerusalem kamen, bauten sie den Tempel wieder auf. Sie fingen wieder mit den Opfer an. Es wurden Schriftgelehrte ausgebildet, welche die Menschen in der Schrift unterwiesen. In den Bücher der Makkabäer ist die Geschichte Israels beschrieben. Aus dieser Zeit entstanden auch die Pharisäer. Ich nehme an, dass diese zu beginn es sicher gut meinten und das Volk vor schaden und Sünde vor Gott bewahren wollten. Aber auch diese Gruppe fing an ihren eigenen Turm zu bauen. Sie verschärften die Gesetze, bereicherten sich an der Bevölkerung und spielte sich als die wahren Gläubigen auf. In den Evangelien können wir nachlesen wie Jesus zu diesen Pharisäer stand. Jesus hat diese Gruppe scharf kritisiert. Er nannte sie Blinde die Blinde führen. \red{bibelvers} Wie fatal das ist, könnt ihre euch denken. Der Turmbau dieser Gruppe endete schrecklich. Einmal in den Tod des Herrn Jesu und dann später auch die Zerstörung von Jerusalem durch die Römer mit vielen Toten. Und jetzt? Jetzt hat aber der Turmbau ein Ende. Ne, noch nicht.

    \subsection{Der Turmbau der Gemeinde}
    Als Jesus unter uns weilte, gab er uns genaue Anweisungen. Diese Anweisungen hat uns der \herr{} in den Evangelien mitgegeben. Die Briefe zeigen uns auf, wie wir unser Christenleben leben sollen. Jesus hat uns gesagt wie wir von unseren Sünden befreit werden und wie wir die Gnade Gottes und das ewige Leben erlangen können. Joh 11:25-26. Ausserdem gibt er uns Wiedergeborenen Christen einen genauen Auftrag bevor er in den Himmel aufgefahren ist. Math 28:19, Mk 16:15, Apg 1:8. Eigentlich klar oder? Das Evangelium die frohe Botschaft und unseren Mitmenschen verkünden.

    Am Anfang klappte es auch. Nachdem Petrus das Evangelium verkündet hat, haben sich 3000 Menschen zu Jesus bekehrt. Petrus hat nur eine Predigt gehalten mehr nicht. Die Menschen haben sich bekehrt und sind der Lehre Jesus nachgefolgt.\\
    Aber irgendwie ist das für uns Menschen zu einfache. Einfache nur Glauben und Vertrauen mehr nicht? Ich muss doch irgendwas tun. Was nichts kostet ist auch nichts Wert. So auch schon in der frühen Christenheit. Sehr früh wurden diverse Rituale und Bedingungen eingeführt, die erfüllt werden müssen um das Heil zu erlangen. So lesen wir in der Apostelgeschichte \bibleverse{Apg}(3:1) das sich auch Heiden beschneiden sollten und Moses Gesetze einhalten. Aus diesem Grund wurde das erste Konzil abgehalten um diverse Regeln zu besprechen und einzuführen. Dieses Konzil war aber auch nicht das letzte sondern es folgten 7 Konzile vor der Kirchenspaltung und dann 21 Konzile in der römisch-katholischen Kirche. Nebenbei noch unzählige regionale Synoden. Bei jedem Konzil wurde wieder ein Stein mehr auf den Turm gesetzt um zu versuchen unabhängiger von Gott zu werden. Waren es am Anfang noch Theme wie das Glaubensbekenntnis(325) die Dreifaltigkeit(381) oder ob Jesus ganz Gott oder Mensch ist(451), wurdes es später Themen wie Sakramente (1563), Unfehlbarkeit des Papstes(1869), Ökumene (1965). Irgendwo habe ich schon gelesen, dass geprüft wird, ob es nicht auch reichen würde an Maria zu glauben um das Heil zu erhalten. Es spitzt sich also immer mehr zu. Wann stört Gott diesen Turmbau?

    Aber nicht nur die katholische Kirche, sondern auch die evang. Kirche hat sich nach der Reformation auch wieder relativ schnell mit dem Bau eines eigenen Turms begonnen. Diverse Reformatoren haben miteinander gestritten, was man machen muss um das Heil zu erlagen zB. Taufen, Abendmahl usw.\\
    Daraus entstanden die Freikirchen die wieder zurück zum Wort Gottes wollten und darum eigene Gemeinde ins Leben gerufen. Die Gefahr vom Turmbau ist aber immer noch nicht gebannt und läuft tapfer weiter auch in den Freikirchen. Dürfen Frauen predigen? Lange Hosen, kurze Hosen? Haare bedecken oder nicht? Sabbath einhalten oder Sonntag? Das sind Themen die disskutiert werden, über die man Stunden lang disskutieren kann und dabei vergisst was Jesus gesagt hat. Wissen wir es noch? Haben wir das vor Augen wenn wir gläubigen und ungläubigen Mitmenschen begegnen? Beurteilen wir zuerst die Personen die in unser Gemeinde kommen wollen, ob diese in unserem gebauten Turm passen? Wie reagieren wir wenn jemand an unserem Turm unserer geliebten Gemeinde wackelt? Vielleicht sind ja die Kritiken angebracht. Wir haben aber lieber es werden Ziegel gestrichen um so hoch hinaus wie möglich zu kommen.
    \subsection{Mein privater Turmbau}
    Wenn wir zurückblicken auf die Türme die wir gesehen haben um was ging es eigentlich? Richtig, es geht um den Stolz im Menschen. Wir lassen uns nicht gerne irgenwas vorschreiben. Wir wissen alles besser. Jesus sagt in \red{??} \q{Mein Joch ist leicht}. Wir wollen aber kein leichtes Joch. Wir wollen ein schweres, wir wollen was leisten. Ich will das heiligste Lächeln meiner Geschwister haben, ein Jahr lang kein GD ausgelassen, 3mal im Jahr die Bibel durchgelesen und und und...\\
    Das wollte Jesus nicht. Er wollte keine verbissenen Christen, die auf biegen und brechen versuchen ihm zu gefallen. Er will das wir ehrlich im Herzen ihn lieben und an ihn glauben. Das ist alles. Mehr will er nicht. Wenn wir Jesus lieben und ihm dankbar sind, für das was er für uns gemacht hat, dann versuchen wir doch automatischen, Sachen zu vermeiden die verletzen können. Ist doch zu Hause auch so. Wenn ich mein Ehepartner liebe und dieser nicht mag wenn ich die Suppe schlürfe, setzte ich mich doch nicht an Tisch und fange an zu schlürfen. Sondern ich setze mich und versuche mein Suppe anständig ohne Geräusche zu essen. Sollte es doch mal passieren das es laut wurde, kann ich mich entschuldigen und ruhig weiter essen.\\
    Wir hören aber nicht auf Jesus. Wir zwar was er gesagt hat und lesen die Bibel, aber wir fangen an Ziegel zu streichen und Stein um Stein zu einem Turm zubauen. Je höher der Turm wird umso schmerzhafter ist es wenn dieser einstürzt oder je anstrengender ist es die Spitze zu erreichen. Sind wir dann oben, sind wir immer noch Meilen weit von Gott entfernt. Darum steht in \bibleverse{IMos}(11:) \q{...last uns hinuntersteigen}. Wir können soviel bauen wie wir wollen, erreichen tun wir so Gott nicht.\\
    Überprüfen wir uns doch selber mal, wie hoch ist mein Turm schon? Das heisst welche Hürden setzte ich mir im Leben um Jesus zu gefallen? zB Wenn eine Person anruft und um Hilfe bittet und du die Bibellese auslassen musst, hast du dann ein schlechtes Gewissen? Schleppst du dich mit einer Grippe in den Gottesdienst, obwohl du mit deiner Krankheit andere gefärden kannst? Denkst du wenn ein Tätowierter auftaucht, dass dieser nicht in den Himmel kommt?\\
    Welche Ziegel bin gerade am streichen? Lege ich mir selber irgend einen Zwang auf um Gott zu gefallen? Wenn schon jetzt die Zeit zum beten knapp wird vor der Arbeit, will ich mir das Gebetsheft von Open Doors noch extra zumuten?

    Lasst uns zum Schluss nochmal anhören was genau Jesus uns Aufgetragen hat. \bibleverse{Joh}(1:) \red{evangelisation und heil}

    Das ist es was er von uns verlangt. Aber das von ganzem Herzen. Die Dankbarkeit und Liebe sollte uns dazu bringen gute Werke zu tun und nicht die Ausssicht auf ein ewiges Leben.

    Die Gemeinde ist da um einander zu helfen zu unterstützen aber auch gegenseitig ermahnen wenn wieder mal am Ziegelstreichen ist. Die Geschwister sind aber auch dazu da, die Leitung aufmerksam zu machen wenn sie ihre Schäfchen benutzen um Ziegel zu streichen. Darum hat Jesus die Gemeinde ins Leben gerufen um eine Gemeinschaft zu bilden die sich gegenseitig hilft und unterstützt. Jeder nach seinen Gaben und Talenten.\\
    \beten{}

    
    

\end{document}