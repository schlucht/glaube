\author{OTS}
\documentclass[12pt]{../../inc/mybib}

\setincpath{../../inc/}

\usepackage{bible_style}
\usepackage{header}
\usepackage{numprint}

\author{Lothar Schmid}
\begin{document}
\section{1. Timoteus 6 11 - 16}

\subsection{Vorwort}
Sie kennen doch sicher die Predigerin Joyce Meyer? Als Christ muss man die ja kenne. Sie hat Auftritte in verschiedenen Fernsehsender wie Bibel-TV, in Predigten auf YouTube und ausser dem hat sie ca. 130 Bücher geschrieben die weltweit gelesen werden.

Die gute Frau verdient Millionen mit ihren Auftritten. Weltweit hat sie Hallen mit Menschen gefüllt und damit viel Geld verdient. Ihre Firma zahlt ihr einen Gehalt von 230000€. Weltweit unterhalten sie mit den Geldern Hilfswerke und Helfen Kindern in Afrika. Das Leben von Frau Meyer ist aber um einiges luxuriöser als ihr Gehalt. Bei Nachfrage sagt, das das Geld für die Villa und Privatflieger von den Buchverkäufen kommt. Bei jedem Auftritt muss sie von Bodyguards begleitet werden, weil sie viele Drohbriefe bekommt und beschützt werden muss. Viele dieser Briefe werden nicht geschrieben weil ihre Botschaft so falsch ist, sondern weil der Neid herrscht. Ich sehe jetzt die Frau auch nicht als christliche Predigerin, vor allem weil sie das sogenannte Wohlstandevangelium vertritt. Ich will hier aber nicht über den Sinn oder Unsinn ihrer Predigten erzählen, möchte aber trotzdem sagen dass ihre Botschaften nicht geeignet sind für neue Christen oder als Evangelisation, weil durch sie ein falsches Evangelium verkündet wird.

Wieso sage ich das euch? Frau Meyer ist ein gutes Beispiel was Paulus mit den Versen in 1. Timoteus 6 8 -10 meint.
\begin{bibelbox}{SCHL}{ITim}{6:11}
Denn die, welche reich werden wollen, fallen in Versuchung und Fallstricken und viele tötichte und schädliche Begierden, welche die Menschen in Untergang und Verderben stürzen. Denn die Geldgier ist eine Wurzel allen Bösen; etliche, die sich ihr hingegeben haben, sind vom Glauben abgeirrt und haben sich selbst viel Schmerzen verursacht.
\end{bibelbox}
Ich weiss es nicht, aber vielleicht war sie mal auf dem richtigen Weg, hat aber dann gemerkt, dass man mit diesem Wohlstandsevangelium viel mehr Menschen ansprechen kann. Am Anfang freut man sich über mehr Zuhörer und wenn dann der Rubel rollt, steigt einem das Geld in Kopf und man gedriftet immer weiter von Gotteswort weg. Jeder von uns hier würde sich ändern wenn er plötzlich 10 Millionen auf das Konto überwiesen bekommt. Glücklicher werden wir wohl nicht, sondern die Sorgen werden sich häufen. Ich habe viele Biografien von Prediger und Missionaren gelesen und allen ist gemeinsam, dass sie durch Gebet immer genau das bekamen was sie brauchten und kein Rappen mehr. Ich weiss nicht wie die Kasse von MNR ist, aber wenn alles bezahlt reicht das.

Kommen wir nun zu unserem Predigttext. Ich habe aus diesem Text drei Punkte rausgelesen.
\begin{itemize}
    \item Was ist ein Mensch Gottes?
    \item Wie soll ein Mensch Gottes durchs Leben geht?
    \item Wer hilft uns dabei?
\end{itemize}
\end{document}
