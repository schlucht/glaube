\section*{Kapitel 2}
\hh{1}\bindW{Wenn} es also \es{so \verbN{ist}, \bindW{dass} es} Ermunterung \verbN{gibt} in \person{Christus}, \bindW{wenn} Zuspruch der Liebe, \bindW{wenn} Gemeinschaft des \person{Geistes}, \bindW{wenn} inniges Mitgefühl \bindW{und} Erbarmungen, \hh{2}\bindW{dann} \verbN{macht} meine Freude \es{damit} voll, \bindW{dass} ihr \es{auf} das Gleiche \verbN{sinnt}, \bindW{indem} ihr dieselbe Liebe \verbN{habt}, in einer Seele \verbN{verbunden seid} \bindW{und} indem ihr auf eines \verbN{sinnt}, \hh{3}\bindW{indem} ihr nichts aus Eigennutz oder leerer Ruhmsucht \es{\verbN{tut}}, \bindW{sondern} in der Demut einer den anderen für höher \verbN{hält} als sich selbst, \hh{4}\bindW{indem} ein jeder nicht auf das Seine \verbN{schaut}, \bindW{sondern} ein jeder auch auf das der anderen. \hh{5}Unter euch \verbN{sei} diese Gesinnung, die auch in \person{Jesus, dem Gesalbten}, \verbN{war}, \hh{6}der, \bindW{obwohl} in Gestalt Gottes \verbN{seiend}, das Gott Gleichsein nicht wie eine Beute \verbN{ansah}, \hh{7}\bindW{sondern} sich selbst \verbN{entäusserte}, indem er die Gestalt eines \person{Knechtes} \verbN{annahm}. Den \person{Menschen} \verbN{gleich geworden} \bindW{und} in der äusseren Erscheinung wie ein Mensch \verbP{erfunden}, \hh{8}\verbN{erniedrigte} er sich selbst, indem er \verbN{gehorsahm wurde} bis zum Tod, \bindW{zum} Tod an einem Kreuz. \hh{9}\bindW{Darum} \verbI{erhöht} Gott ihn auch über \es{alles} \bindW{und} \verbN{gab} ihm den Namen, der über jeden Namen \verbN{ist}, \hh{10}\bindW{damit} im Namen \person{Jesu} sich \verbN{beuge} jedes Knie, \es{der} Himmlischen der Irdischen \bindW{und} Unterirdischen, \hh{11}\bindW{und} jede Zunge \verbN{bekenne}, \bindW{dass} \person{Jesus, der Gesalbte}, Herr \verbN{ist}, \bindW{zur} Verherrlichung \person{Gottes, des Vaters}.

\hh{12}\bindW{So denn}, meine Geliebte, wie ihr allezeit \verbN{gehorcht} \verbN{habt}, nicht nur wie in meiner Anwesenheit, \bindW{sondern} jetzt vielmehr in meiner Abwesenheit, \verbN{bringt} euer eigenes Heil hervor mit Furcht \bindW{und} Zittern; \hh{13}\bindW{denn} Gott \verbN{ist} der in euch Wirkende -- \bindW{sowohl} das Wollen als auch das Wirken -- wegen \es{seines} Wohlgefallens.

\hh{14}\verbI{Tut} alles ohne Murren \bindW{und} Bedenken, \hh{15}\bindW{damit} ihr \verbN{untadelig} \bindW{und} \verbP{unverfälscht werdet}, \person{Kinder Gottes} ohne Makel inmitten eines krummen \bindW{und} verdrehten Geschlechts, unter dem ihr \verbN{aufscheint} wie Lichter in der Welt, \hh{16}\bindW{indem} ihr \verbN{festhaltet} das Wort des Lebens, mir zum \es{Gegenstand des} Rühmens auf den Tag \person{Christi}, \bindW{weil} ich \es{dann} nicht vergeblich \verbN{gelaufen bin}, \bindW{noch} auch vergeblich \verbN{gearbeitet habe}. \hh{17}\bindW{Wenn} ich aber auch \es{als Gussopfer} \verbN{ausgegossen werde} über das \person{Opfer} \bindW{und} den Priesterdienst für euren Glauben, \verbN{freue} ich mich mit euch allen. \hh{18}\bindW{Ebenso} \verbN{freut} auch ihr euch \bindW{und} \verbN{freut} euch zusammen mit mir.

\hh{19}Ich \verbN{hoffe} aber in dem \person{Herrn Jesus}, \person{Timotheus} bald zu euch zu \verbN{senden}, \bindW{damit} auch ich frohgemut \verbN{sei}, \bindW{wenn} ich eure Umstände \verbN{erfahre}.
 \hh{20}Ich \verbN{habe} nämlich niemand gleichgesinnt, der in echter Weise für das eure \verbN{besorgt sein wird}; \hh{21}\bindW{denn} alle \verbN{suchen} das Eigene, nicht das, \es{was} \person{Jesu Christi} \es{\verbN{ist}}. \hh{22} \bindW{Aber} seine Bewährtheit \verbN{kennt} ihr, \bindW{dass} er wie ein \person{Kind} dem \person{Vater} zusammen mit mir \verbN{gedient} hat im Evangelium. \hh{23}Diesen also \verbN{hoffe} ich, sofort zu \verbN{schicken}, sobald ich \verbN{absehe} wie es um mich \verbN{steht}. \hh{24}\bindW{Doch} ich \verbN{bin} zuversichtlich im \person{Herrn}, \bindW{dass} auch ich selbst bald \verbN{kommen} werden. \hh{25}Ich \verbN{hielt} es aber für notwendig, \person{Epaphroditus} meinen \person{Bruder} \bindW{und} \person{Mitarbeiter} \bindW{und} \person{Mitkämpfer}, \bindW{aber} euren \person{Abgesandten} \bindW{und} \person{Diener} meines Bedarfs, zu euch zu \verbN{schicken}, \hh{26}da er sich nach euch allen \verbN{sehnte} \bindW{und} in Unruhe war, \bindW{weil} ihr \verbN{gehört hattet}, \bindW{dass} er \verbN{erkrankt}, dem Tod nahe. \bindW{Doch} \person{Gott} \verbN{erbarmte} sich über ihn, \bindW{und} nicht nur über ihn, \bindW{sondern} auch über mich, \bindW{damit} ich nicht Kummer über Kummer \verbN{bekäme}. \hh{28}\bindW{Also} \verbN{habe} ich ihn \es{umso} eiliger \verbN{geschickt}, \bindW{damit} ihr, wenn ihr ihn \verbN{seht}, wieder \verbN{froh werdet} \bindW{und} ich weniger \verbN{bekümmert sei}. \hh{29}\verbI{Nehmt} ihn also auf im \person{Herrn} mit aller Freude, \bindW{und} \verbI{haltet} solche in Ehren; \hh{30}\bindW{denn} wegen des Werkes \person{Christi} \verbN{kam} er dem Tod nahe, \bindW{indem} er sein Leben \verbN{gering achtete}, \bindW{um} euren Mangel im Dienst für mich \verbN{aufzufüllen}.

 \begin{block}[Gedanken zum Kapitel 2]
    \lineheight{0.5}
    \begin{itshape}
        Im Gegensatz zum Korintherbrief muss Paulus die Gemeinde in Philippi nicht tadeln. Er lobt ihre Zusammenarbeit \bindW{und} der Umgang miteinander. Er ermutigt sie, so weiter zu machen \bindW{und} nicht auf falsche Lehrer zu hören \bindW{und} nicht überheblich zu werden. Er ermutigt sie Jesus Christus den Gesalbten zu bekennen \bindW{und} ihm treu zu bleiben.

        Paulus will ihnen Timotheus schicken. Auf diesen kann er sich verlassen. Bei ihm weiss er, \bindW{dass} er sich gut um die Philipper kümmert.

        Den Mitarbeiter Epaphroditus schickt er, sobald dieser Ges\bindW{und} war, wieder direkt zurück. Der Mut \bindW{und} den Einsatz von Epaphrodites bringt Trost \bindW{und} Zuversicht in die Gemeinde von Philippi.
    \end{itshape}
\end{block}