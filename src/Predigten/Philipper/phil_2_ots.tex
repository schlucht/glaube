\section*{Kapitel 2}
\hh{1}Wenn es also \es{so \verbN{ist}, dass es} Ermunterung \verbN{gibt} in \person{Christus}, wenn Zuspruch der Liebe, wenn Gemeinschaft des \person{Geistes}, wenn inniges Mitgefühl und Erbarmungen, \hh{2}dann \verbN{macht} meine Freude \es{damit} voll, dass ihr \es{auf} das Gleiche \verbN{sinnt}, indem ihr dieselbe Liebe \verbN{habt}, in einer Seele \verbN{verbunden seid} und indem ihr auf eines \verbN{sinnt}, \hh{3}indem ihr nichts aus Eigennutz oder leerer Ruhmsucht \es{\verbN{tut}}, sondern in der Demut einer den anderen für höher \verbN{hält} als sich selbst, \hh{4}indem ein jeder nicht auf das Seine \verbN{schaut}, sondern ein jeder auch auf das der anderen. \hh{5}Unter euch \verbN{sei} diese Gesinnung, die auch in \person{Jesus, dem Gesalbten}, \verbN{war}, \hh{6}der, obwohl in Gestalt Gottes \verbN{seiend}, das Gott Gleichsein nicht wie eine Beute \verbN{ansah}, \hh{7}sondern sich selbst \verbN{entäusserte}, indem er die Gestalt eines \person{Knechtes} \verbN{annahm}. Den \person{Menschen} gleich \verbN{geworden} und in der äusseren Erscheinung wie ein Mensch \verbP{erfunden}, \hh{8}\verbN{erniedrigte} er sich selbst, indem er gehorsahm \verbN{wurde} bis zum Tod, zum Tod an einem Kreuz. \hh{9}Darum \verbN{erhöht} Gott ihn auch über \es{alles} und \verbN{gab} ihm den Namen, der über jeden Namen \verbN{ist}, \hh{10}damit im Namen Jesu sich \verbN{beuge} jedes Knie, \es{der} Himmlischen der Irdischen und Unterirdischen, \hh{11}und jede Zunge \verbN{bekenne}, dass \person{Jesus, der Gesalbte}, Herr \verbN{ist}, zur Verherrlichung Gottes, des Vaters.

\hh{12}So denn, meine Geliebte, wie ihr allezeit \verbN{gehorcht} \verbN{habt}, nicht nur wie in meiner Anwesenheit, sondern jetzt vielmehr in meiner Abwesenheit, \verbN{bringt} euer eigenes Heil hervor mit Furcht und Zittern; \hh{13}denn Gott \verbN{ist} der in euch Wirkende -- sowohl das Wollen als auch das Wirken -- wegen \es{seines} Wohlgefallens.

\hh{14}\verbN{Tut} alles ohne Murren und Bedenken, \hh{15}damit ihr untadelig und unverfälscht \verbP{werdet}, \person{Kinder Gottes} ohne Makel inmitten eines krummen und verdrehten Geschlechts, unter dem ihr \verbN{aufscheint} wie Lichter in der Welt, \hh{16}indem ihr \verbN{festhaltet} das Wort des Lebens, mir zum \es{Gegenstand des} Rühmens auf den Tag \person{Christi}, weil ich \es{dann} nicht vergeblich \verbN{gelaufen bin}, noch auch vergeblich \verbN{gearbeitet habe}. \hh{17}Wenn ich aber auch \es{als Gussopfer} \verbN{ausgegossen werde} über das \person{Opfer} und den Priesterdienst für euren Glauben, \verbN{freue} ich mich mit euch allen. \hh{18}Ebenso \verbN{freut} auch ihr euch und \verbN{freut} euch zusammen mit mir.

\hh{19}Ich \verbN{hoffe} aber in dem \person{Herrn Jesus}, \person{Timotheus} bald zu euch zu \verbN{senden}, damit auch ich frohgemut \verbN{sei}, wenn ich eure Umstände \verbN{erfahre}.
 \hh{20}Ich \verbN{habe} nämlich niemand gleichgesinnt, der in echter Weise für das eure \verbN{besorgt} sein \verbN{wird}; \hh{21}denn alle \verbN{suchen} das Eigene, nicht das, \es{was} \person{Jesu Christi} \es{\verbN{ist}}. \hh{22} Aber seine Bewährtheit \verbN{kennt} ihr, dass er wie ein \person{Kind} dem \person{Vater} zusammen mit mir \verbN{gedient} hat im Evangelium. \hh{23}Diesen also \verbN{hoffe} ich, sofort zu \verbN{schicken}, sobald ich \verbN{absehe} wie es um mich \verbN{steht}. \hh{24}Doch ich \verbN{bin} zuversichtlich im \person{Herrn}, dass auch ich selbst bald \verbN{kommen} werden. \hh{25}Ich \verbN{hielt} es aber für notwendig, \person{Epaphroditus} meinen \person{Bruder} und \person{Mitarbeiter} und \person{Mitkämpfer}, aber euren \person{Abgesandten} und \person{Diener} meines Bedarfs, zu euch zu \verbN{schicken}, \hh{26}da er sich nach euch allen \verbN{sehnte} und in Unruhe war, weil ihr \verbN{gehört hattet}, dass er \verbN{erkrankt}, dem Tod nahe. Doch \person{Gott} \verbN{erbarmte} sich über ihn, und nicht nur über ihn, sondern auch über mich, damit ich nicht Kummer über Kummer \verbN{bekäme}. \hh{28}Also \verbN{habe} ich ihn \es{umso} eiliger \verbN{geschickt}, damit ihr, wenn ihr ihn \verbN{seht}, wieder froh \verbN{werdet} und ich weniger bekümmert \verbN{sei}. \hh{29}\verbN{Nehmt} ihn also auf im \person{Herrn} mit aller Freude, und \verbN{haltet} solche in Ehren; \hh{30}denn wegen des Werkes \person{Christi} kam er dem Tod nahe, indem er sein Leben gering achtete, um euren Mangel im Dienst für mich \verbN{aufzufüllen}.

 \begin{block}[Gedanken zum Kapitel 2]
    \lineheight{0.5}
    \begin{itshape}
        Im Gegensatz zum Korintherbrief muss Paulus die Gemeinde in Philippi nicht tadeln. Er lobt ihre Zusammenarbeit und der Umgang miteinander. Er ermutigt sie, so weiter zu machen und nicht auf falsche Lehrer zu hören und nicht überheblich zu werden. Er ermutigt sie Jesus Christus den Gesalbten zu bekennen und ihm treu zu bleiben.

        Paulus will ihnen Timotheus schicken. Auf diesen kann er sich verlassen. Bei ihm weiss er, dass er sich gut um die Philipper kümmert.

        Den Mitarbeiter Epaphroditus schickt er, sobald dieser Gesund war, wieder direkt zurück. Der Mut und den Einsatz von Epaphrodites bringt Trost und Zuversicht in die Gemeinde von Philippi.
    \end{itshape}
\end{block}