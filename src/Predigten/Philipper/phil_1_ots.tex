
\section*{Kapitel 1}
\hh{1}\person{Paulus} \bindW{und} \person{Timotheus}, \person{Knechte Jesu Christi}, alle \person{Heiligen in Jesus}, der, Christus, die in \ort{Philippi} \verbN{sind}, zusammen mit \person{Aufsehern} \bindW{und} \person{Dienern}: \hh{2}Gnade euch \bindW{und} Friede von \person{Gott}, \person{unserem Vater}, \bindW{und} dem \person{Herrn Jesus}, dem Gesalbten! \hh{3}Ich \verbN{danke} meinem \person{Gott} bei jedem Gedenken an euch, \hh{4}allezeit in all meinem Beten für euch alle, dabei das Gebet mit Freuden \verbN{verrichtend} \hh{5}wegen eurer Teilnahme am Evangelium vom ersten Tag an bis jetzt, \hh{6}\bindW{weil} ich davon \verbN{überzeugt bin}, \bindW{dass} der, der ein gutes Werk in euch \verbN{angefangen hat}, \es{es} zu Ende \verbN{führen wird} bis zum Tag \person{Jesu Christi}; \hh{7}\bindW{so wie} es für mich \verbN{recht ist}, dies über euch alle zu \verbN{denken}, \bindW{weil} ich euch im Herzen \verbN{habe}, \bindW{da} ihr alle \bindW{sowohl} in meinen \ort{Fesseln} \bindW{als auch} in der Verteidigung \bindW{und} Bekräftigung des Evangeliums zusammen mit mir \person{Teilhaber} \verbN{seid} an der Gnade. \hh{8}Denn \person{Gott} \verbN{ist} mein \person{Zeuge}, \bindW{wie} ich mich nach euch allen \verbN{sehne} mit dem herzlichen Empfinden \person{Jesu, des Gesalbten}. \hh{9}\bindW{Und} dieses \verbN{erbete} ich, \bindW{dass} eure Liebe noch mehr \bindW{und} mehr \verbN{zunehme} in der Erkenntnis \bindW{und} allem Empfinden, \hh{10}\bindW{so dass} ihr \verbN{prüfen könnt}, was das Vorzuziehende \verbN{sei} \bindW{damit} ihr lauter \bindW{und} ohne Anstoss \verbN{seid} am \ort{Tag Christi}, \hh{11}\verbP{erfüllt} mit der Frucht der Gerechtigkeit, die durch \person{Jesus, den Gesalbten}, \es{ist}, zur Herrlichkeit \bindW{und} zum Lob \person{Gottes}.

\hh{12}Ich \verbN{will} aber, \bindW{dass} ihr \verbN{wisst}, \person{Brüder}, \bindW{dass} meine Umstände mehr zum Fortschreiten des Evangeliums \verbN{geführt haben}, \hh{13}\bindW{so dass} meine Fesseln \es{als Fesseln} in \person{Christus} \verbN{offenbar geworden sind} im ganzen \ort{Prätorium} \bindW{und} den übrigen allen, \hh{14}\bindW{und} \bindW{dass} die meisten der \person{Brüder}, da sie im \person{Herrn} \verbN{Vertrauen haben} durch meine Fesseln, umso mehr \verbN{wagen}, das Wort Gottes zu \verbN{sagen} ohne Furcht. \hh{15}Zwar \verbN{verkündigen} einige den \person{Christus} gar aus Neid \bindW{und} Streit, andere dagegen aus gutem Willen. \hh{16}Die einen aus Liebe, das sie \verbN{wissen}, \bindW{dass} ich zur Verteidigung des Evangeliums \verbP{bestimmt} \verbN{bin}; \hh{17}die anderen \verbN{verkünden} \person{Christus} aus Eigennutz, nicht lauter, da sie \verbN{meinen}, \es{mir} in meinen Fesseln Begrängnis zu \verbP{erwecken}. \hh{18}Doch was \es{\verbN{tut}'s}? Jedenfalls \verbN{wird} auf alle Weise, \verbN{sei} es zum Vorwand \bindW{oder} in Wahrheit, \person{Christus} \verbP{verkündet}, \bindW{und} darüber freue ich mich, ich \verbN{werde} mich auch \es{weiterhin} \verbN{freuen}. \hh{19}Ich \verbN{weiss} nähmlich: \enquote{Dies \verbN{wird} mir zum Heil \verbN{ausgehen}} durch euer Bitten \bindW{und} durch die Unterstützung des \person{Geistes Jesu, des Gesalbten}, \hh{20}gemäss meinem erwartungsvollen Harren \bindW{und} der Hoffnung, \bindW{dass} ich in nichts \verbP{werde beschämt werden}, \bindW{sondern} mit allem Freimut, wie allezeit, so auch jetzt \person{Christus} \verbP{gross gemacht wird} an meinem Leib, ob durch Leben oder Tod. \hh{21}Denn zu \verbN{leben ist} für mich \person{Christus} \bindW{und} zu \verbN{sterben} Gewinn. \hh{22}\bindW{Wenn} aber im Fleisch zu \verbN{leben} -- das \es{\verbN{hiesse}} für mich Frucht aus \es{weiterem} Wirken. \bindW{Und} was ich \verbN{wählen soll}, \verbN{weiss} ich nicht. \hh{23}Ich \verbP{werde bedrängt} von beidem, da ich Lust \verbN{habe}, \verbN{aufzubrechen} \bindW{und} bei \person{Christus} zu \verbN{sein}, \bindW{denn} \es{das \verbN{wäre}} um vieles besser; \hh{24}\bindW{doch} das Verbleiben im Fleisch \verbN{ist nötiger} euretwegen. \hh{25}Weil ich von diesem \verbN{überzeugt bin}, \verbN{weiss} ich: Ich \verbN{werde bleiben} \bindW{und} bei euch allen \verbN{verbleiben} zu eurem Fortschreiten \bindW{und} eurer Freude im Glauben, \hh{26}\bindW{damit} euer Rühmen an mir in \person{Jesus, dem Gesalbten}, \verbN{zunehme} durch meine erneute Ankunft bei euch.

\hh{27}Nur: \verbI{Führt} euer Leben \es{im Gemeinwesen} würdig des Evangeliums des \person{Christus}, \bindW{damit}, \bindW{ob} ich \verbN{ankomme} \bindW{und} euch \verbN{erblicke} \bindW{oder} \verbN{abwesend bin}, ich von euren Umständen \verbN{höre}, \bindW{dass} ihr \verbN{fest steht} in einem \person{Geist}, mit einer Seele zusammen \verbN{kämpfend} für den Glauben des Evangeliums \hh{28}\bindW{und} durch nichts \verbN{eingeschüchtert} von den Widerstreitenden, was für sie ein Anzeichen des Verderbens \verbN{ist}, \bindW{aber} eures Heils -- \bindW{und} das von Gott her; \hh{29}\bindW{denn} euch \verbN{ist} es hinsichtlich \person{Christi} \verbP{geschenkt worden}, nicht allein an ihn zu \verbN{glauben}, \bindW{sondern} auch für ihn zu \verbN{leiden}, \hh{30}die ihr ja den gleichen Kampf \verbN{habt}, \bindW{so} beschaffen, \bindW{wie} ihr \es{ihn} an mir \verbN{gesehen habt} \bindW{und} von mir \verbN{hört}.

\begin{block}[Gedanken zum Kapitel 1]
    \lineheight{0.5}
    \begin{itshape}
        1 - 10 Ein lange Begrüssung in der er alle Beteiligten begrüsst. Es ist immer wieder erstaunlich welche Gefühle Paulus für seine Glaubensgeschwister aufbringt. Mit vielen Emotionen ist dieser Gruss gespickt. Paulus wird nach den Erlebnissen in Philippi wohl eine spezielle Beziehung zu dieser Gemeinde haben.

        Die Philipper wissen, dass Paulus in Gefangenschaft ist und Paulus weiss, dass sie sich Sorgen um ihn und um das Evangelium machen. Durch das Erzählen von seine Erlebnissen, tröstet er die Gemeinde. Er zeigt ihnen auf, das Gottes Plan niemand stören oder unterbrechen kann. Gott kann alles und alle für seinen Plan gebrauchen, auch wenn dieser in Gefangenschaft und in Fesseln ist.
    \end{itshape}
\end{block}