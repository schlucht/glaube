
\section*{Kapitel 1}
\hh{1}\person{Paulus} und \person{Timotheus}, \person{Knechte Jesu Christi}, alle \person{Heiligen in Jesus}, der, Christus, die in \ort{Philippi} \verbN{sind}, zusammen mit \person{Aufsehern} und \person{Dienern}: \hh{2}Gnade euch und Friede von \person{Gott}, \person{unserem Vater}, und dem \person{Herrn Jesus, dem \-Gesalbten}! \hh{3}Ich \verbN{danke} meinem \person{Gott} bei jedem Gedenken an euch, \hh{4}allezeit in all meinem Beten für euch alle, dabei das Gebet mit Freuden \verbN{verrichtend} \hh{5}wegen eurer Teilnahme am Evangelium vom ersten Tag an bis jetzt, \hh{6}weil ich davon überzeugt \verbN{bin}, dass der, der ein gutes Werk in euch \verbN{angefangen hat}, \es{es} zu Ende \verbN{führen} wird bis zum Tag \person{Jesu Christi}; \hh{7}so wie es für mich recht \verbN{ist}, dies über euch alle zu \verbN{denken}, weil ich euch im Herzen \verbN{habe}, da ihr alle sowohl in meinen \ort{Fesseln} als auch in der Verteidigung und Bekräftigung des Evangeliums zusammen mit mir Teilhaber \verbN{seid} an der Gnade. \hh{8}Denn Gott \verbN{ist} mein \person{Zeuge}, wie ich mich nach euch allen \verbN{sehne} mit dem herzlichen Empfinden \person{Jesu, des Gesalbten}. \hh{9}Und dieses \verbN{erbete} ich, dass eure Liebe noch mehr und mehr \verbN{zunehme} in der Erkenntnis und allem Empfinden, \hh{10}sodass ihr \verbN{prüfen könnt}, was das Vorzuziehende \verbN{sei} damit ihr lauter und ohne Anstoss \verbN{seid} am \ort{Tag Christi}, \hh{11}\verbN{erfüllt} mit der Frucht der Gerechtigkeit, die durch \person{Jesus, den Gesalbten}, \es{ist}, zur Herrlichkeit und zum Lob \person{Gottes}.

\hh{12}Ich \verbN{will} aber, dass ihr \verbN{wisst}, \person{Brüder}, dass meine Umstände mehr zum Fortschreiten des Evangeliums \verbN{geführt haben}, \hh{13}sodass meine Fesseln \es{als Fesseln} in \person{Christus} offenbar \verbN{geworden sind} im ganzen \ort{Prätorium} und den übrigen allen, \hh{14}und dass die meisten der \person{Brüder}, da sie im \person{Herrn} Vertrauen \verbN{haben} durch meine Fesseln, umso mehr \verbN{wagen}, das Wort Gottes zu \verbN{sagen} ohne Furcht. \hh{15}Zwar \verbN{verkündigen} einige den \person{Christus} gar aus Neid und Streit, andere dagegen aus gutem Willen. \hh{16}Die einen aus Liebe, das sie \verbP{wissen}, dass ich zur Verteidigung des Evangeliums bestimmt \verbN{bin}; \hh{17}die anderen \verbN{verkünden} \person{Christus} aus Eigennutz, nicht lauter, da sie \verbN{meinen}, \es{mir} in meinen Fesseln Begrängnis zu \verbN{erwecken}. \hh{18}Doch was \es{\verbN{tut}\textquotesingle s}? Jedenfalls \verbN{wird} auf alle Weise, sei es zum Vorwand oder in Wahrheit, \person{Christus} \verbN{verkündet}, und darüber freue ich mich, ich \verbN{werde} mich auch \es{weiterhin} freuen. \hh{19}Ich \verbN{weiss} nähmlich: \enquote{Dies wird mir zum Heil \verbN{ausgehen}} durch euer Bitten und durch die Unterstützung des \person{Geistes Jesu, des Gesalbten}, \hh{20}gemäss meinem erwartungsvollen Harren und der Hoffnung, dass ich in nichts \verbN{werde} beschämt \verbP{werden}, sondern mit allem Freimut, wie allezeit, so auch jetzt \person{Christus} gross \verbN{gemacht wird} an meinem Leib, ob durch Leben oder Tod. \hh{21}Denn zu \verbN{leben ist} für mich \person{Christus} und zu \verbN{sterben} Gewinn. \hh{22}Wenn aber im Fleisch zu \verbN{leben} -- das \es{\verbN{hiesse}} für mich Frucht aus \es{weiterem} Wirken. Und was ich \verbN{wählen} soll, \verbN{weiss} ich nicht. \hh{23}Ich \verbN{werde bedrängt} von beidem, da ich Lust \verbN{habe}, \verbN{aufzubrechen} und bei \person{Christus} zu \verbN{sein}, denn \es{das \verbN{wäre}} um vieles besser; \hh{24}doch das Verbleiben im Fleisch \verbN{ist} nötiger euretwegen. \hh{25}Weil ich von diesem überzeugt \verbN{bin}, \verbN{weiss} ich: Ich \verbN{werde} bleiben und bei euch allen \verbN{verbleiben} zu eurem Fortschreiten und eurer Freude im Glauben, \hh{26}damit euer Rühmen an mir in \person{Jesus, dem Gesalbten}, \verbN{zunehme} durch meine erneute Ankunft bei euch.

\hh{27}Nur: \verbN{Führt} euer Leben \es{im Gemeinwesen} würdig des Evangeliums des \person{Christus}, damit, ob ich \verbN{ankomme} und euch \verbN{erblicke} oder abwesend \verbN{bin}, ich von euren Umständen \verbN{höre}, dass ihr \verbN{\es{fest}steht} in einem Geist, mit einer Seele zusammen \verbN{kämpfend} für den Glauben des Evangeliums \hh{28}und durch nichts \verbN{eingeschüchtert} von den Widerstreitenden, was für sie ein Anzeichen des Verderbens \verbN{ist}, aber eures Heils -- und das von Gott her; \hh{29}denn euch \verbN{ist} es hinsichtlich \person{Christi} \verbP{geschenkt worden}, nicht allein an ihn zu \verbN{glauben}, sondern auch für ihn zu \verbN{leiden}, \hh{30}die ihr ja den gleichen Kampf \verbN{habt}, so \verbN{beschaffen}, wie ihr \es{ihn} an mir \verbN{gesehen habt} und von mir \verbN{hört}.

