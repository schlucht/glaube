\section*{Kapitel 4}
\hh{1}Daher meine geliebten und \verbN{ersehnten} \person{Brüder}, meine \person{Freude} und mein Siegeskranz: Auf diese Weise \verbN{steht} \es{fest} im \person{Herrn}, \person{Geliebte}!

\hh{2}\person{Evodia} \verbN{rufe} ich auf, und \person{Syntyche} \verbN{rufe} ich auf, das Gleiche zu \verbN{sinnen} im \person{Herrn}. \hh{3}Ja, ich \verbI{bitte} auch dich, echter \person{Jochgenosse}, \verbN{stehe} ihnen bei, die im Evangelium mit mir \verbN{gekämpft \verbN{habe}n}, samt \person{Clemens} und meinen übrigen \person{Mitarbeitern}, deren Namen im Buch des Lebens \es{\verbN{stehen}}.

\hh{4}\verbI{Freut} euch im \person{Herrn} allezeit! Nochmals \verbN{will} ich \verbN{sagen}: \verbI{Freut} euch! \hh{5}Eure Milde \verbI{werde} allen \person{Menschen} \verbI{bekannt}! Der Herr \verbN{ist} nahe. \hh{6}\verbI{Macht} euch um nichts Sorgen, sondern in allem sollen eure Bitten durch Gebet und Flehen mit Danksagung vor \person{Gott} \verbN{kundwerden}, \hh{7}und der alles Denken überragende Friede \person{\gott[white]{Gottes}} \verbN{wird} eure Herzen und eure Gedanken in Gewahrsam \verbN{halten} in \person{\jesus[white]{Jesus, dem Gesalbten}}.

\hh{8}Des Weiteren, \person{Brüder}, alles, was wahr, was ehrbar, was gerecht, was rein, was liebenswert \verbN{ist}, was wohltuend \verbN{ist}, ob eine Tugend, ob ein Lob -- diese Dinge bedenkt. \hh{9} Was ihr auch \verbN{gelernt} und \verbN{übernommen} und \verbN{gehört} und an mir \verbN{gesehen} \verbN{habt}, das \verbI{tut}, und der \person{Gott des Friedens} \verbN{wird} mit euch \verbN{sein}.

\hh{10}Ich \verbN{habe} mich im \person{Herrn} hoch \verbN{gefreut}, dass ihr endlich wieder \verbN{aufgeblüht} \verbN{seid}, an mich zu \verbN{denken}; woran ihr zwar \verbN{dachtet}, aber ihr \verbN{hattet} keine \geist{Gelegenheit}. \hh{11}Nicht dass ich das aufgrund von Mangel \verbN{sage}, denn ich \verbN{habe gelernt}, worin ich \verbN{bin}, genügsam zu \verbN{sein}. \hh{12}Ich \verbN{weiss} erniedrigt zu \verbN{sein}, und ich \verbN{weiss} übrig zu \verbN{haben}. In jedes und in alles \verbN{bin} ich \verbN{eingeweiht}: satt \verbN{sein} und hungern, übrig \verbN{haben} und Mangel \verbN{leiden}. \hh{13}Alles \verbN{vermag} ich durch den, der mich \es{fortwährend} \verbN{kräftigt}. \hh{14}Und doch, ihr \verbN{habt} gut \es{daran} \verbN{getan}, an meiner Bedrängnis Anteil zu \verbN{nehmen}.

\hh{15}Ihr \verbI{wisst} auch selbst \es{liebe} \person{Philipper}, dass im Anfang \es{der Verkündigung} des Evangeliums, als ich \verbN{wegzog}, von \ort{Mazedonien}, keine Gemeinde Gemeinschaft \verbN{hatte} mit mir im  \es{gegenseitigen} Geben und Empfangen als nur ihr allein. \hh{16}Nämlich auch in \ort{Thessalonich} \verbN{schicktet} ihr mir einmal, sogar zweimal \es{etwas} für meinen Bedarf. \hh{17}Nicht dass ich die Gabe \verbN{suche}, sondern ich \verbN{suche} die sich für eure Rechnung mehrende Frucht. \hh{18}Ich \verbN{habe} alles \verbN{erhalten} und \verbN{habe} übrig; ich \verbN{bin} \verbN{erfüllt}, nachdem ich von \person{Epaphroditus} die \es{Gabe} von euch \person{empfangen} \verbN{habe}, einen lieblichen Geruch. Ein willkommenes Opfer, \person{Gott} wohgefällig. \hh{19}Mein \person{Gott} aber \verbN{wird} all euren Bedarf \verbN{erüllen} nach seinem Reichtum in Herrlichkeit in \person{Christus Jesus}. \hh{20}Unserem \person{Gott} und \person{Vater} \verbI{sei} die Herrlichkeit in alle Ewigkeit! Amen.

\hh{21}\verbI{Grüsst} jeden \person{Heiligen} in \person{Jesus, dem Gesalbten}. \hh{22}Alle \person{Heiligen} Grüssen euch, am meisten die dem \ort{Haus des Kaisers}.

\hh{23}Die Gnade des \person{Herrn Jesus, des Gesalbten}, \es{\verbN{sei}} mit eurem Geist!
\hr
\begin{block}[Gedanken zum Kapitel 4]
    \lineheight{0.5}
    \begin{itshape}
        Im Kapitel 4 bedankt sich Paulus nochmals bei seinen Mitstreitern in Philippi. Er spricht auch die Spenden an, die er von ihnen erhalten hat. Paulus nimmt diese auch dankend an und unterstreicht, dass es nicht selbstverständlich ist, dass er von den Gemeinden unterstützt wird. Es ist also nicht so, dass er unbedingt in Armut und Hunger leben will, sondern er freut sich, wenn er etwas bekommt, aber er hat auch gelernt mit wenigem auszukommen.
        
        Am Ende lässt er alle Grüssen und gibt den Segen unseres Herrn Jesus Christus weiter.
    \end{itshape}
\end{block}