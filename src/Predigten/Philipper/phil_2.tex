\section{2}
\hh{1}Wenn es also \es{so ist, dass es} Ermunterung gibt in Christus, wenn Zuspruch der Liebe, wenn Gemeinschaft des Geistes, wenn inniges Mitgefühl und Erbarmungen, \hh{2}dann macht meine Freude \es{damit} voll, dass ihr \es[auf] das Gleiche sinnt, indem ihr dieselbe Liebe habt, in einer Seele verbunden seid und indem ihr auf eines sinnt, \hh{3}indem ihr nichts aus Eigennutz oder leerer Ruhmsucht \es[tut], sondern in der Demut einer den anderen für höher hält als sich selbst, \hh{4}indem ein jeder auch auf das der anderen. \hh{5}Unter euch sei diese Gesinnung, die auch in Jesus, dem Gesalbten, war, \hh{6}der, obwohl in Gestalt Gottes seiend, das Gott Gleichsein nicht wie eine Beute ansah, \hh{7}sondern sich selbst entäusserte, indem er die Gestalt eines Knechtes annahm. Den Menschen gleich geworden und in der äusseren Erscheinung wie ein Mensch erfunden, \hh{8}erniedrigte er sich selbst, indem er gehorsahm wurde bis zum Tod, zum Tod an einem Kreuz. \hh{9}Darum erhöht Gott ihn auch über \es{alles} und gab ihm den Namen, der über jeden Namen ist. \hh{10}damit im Namen Jesu sich beuge jedes Knie, \es{der} Himmlischen der Irdischen und Unterirdischen, \hh{11}und jede Zunge bekenne, dass Jesus, der Gesalbte, Herr ist, zur Verherrlichung Gottes, des Vaters.

\hh{12}So denn, meine Geliebte, wie ihr allezeit gehorcht habt, nicht nur wie in meiner Anwesenheit, sondern jetzt vielmehr in meiner Abwesenheit, bringt euer eigenes Hiel hervor mit Frucht und Zittern; \hh{13}denn Gott ist der in euch Wirkende -- sowohl das Wollen als auch das Wirken -- wegen \es{seines} Wohlgefallens.

\hh{14}Tut alles ohne Murren und Bedenken, \hh{15}damit ihr untadelig und unverfälscht werdet, Kinder Gottes ohne Makel inmitten eines krummen und verdrehten Geschlechts, unter dem ihr aufscheint wie Lichter in der Welt, \hh{16}indem ihr festhaltet das Wort des Lebens, mir zum \es{Gegenstand des} Rühmens auf den Tag Christi, weil ich \es{dann} nicht vergeblich gelaufen bin, noch auch vergeblich gearbeitethabe. \hh{17}Wenn ich aber auch \es{als Gussopfer} ausgegossen werde über das Opfer und den Priesterdienst für euren Glauben, freue ich mich und freue mich mit euch allen. \hh{18}Ebenso freut auch ihr euch und freut euch zusammen mit mir.

\hh{19}Ichhoffe aber in dem Herrn Jesus, Timotheus bald zu euch zu senden, damit auch ich frohgemut sei, wenn ich eure Umstände erfahre.
 \hh{20} Ich habe nämlich niemand gleichgesinnt, der in echter Weise für das eure besorgt sein wird; \hh{21}denn alle suchen das Eigene, nicht das, \es{was} Jesu Christi \es[ist]. \hh{22} Aber seine Bewährtheit kennt ihr, dass er wie ein Kind dem Vater zusammen mit mir gedient hat im Evangelium. \hh{23}Diesen also hoffe ich, sofort zu schicken, sobald ich absehe wie es um mich steht. \hh{24}Doch ich bin zuversichtlich im Herrn, dass auch ich selbst bald kommen werden. \hh{25}Ich hielt es aber für notwendig, Epaphroditus meinen Bruder und Mitarbeiter und Mitkämpfer, aber euren Abgesandten und Diener meines Bedarfs, zu euch zu schicken, \hh{26}da er sich nach euch allen sehnte und in Unruhe war, weil ihr gehört hattet, dass er erkrankt, dem Tod nahe. Doch Gott erbarmte sich über ihn, und nicht nur über ihn, sondern auch über mich, damit ich nicht Kummer über Kummer bekäme. \hh{28}Also habe ich ihn \es{unso} eiliger geschickt, damit ihr, wenn ihr ihn seht, wieder froh werdet und ich weniger bekümmert sei. \hh{29}Nehmt ihn also auf im Herrn mit aller Freude, und haltet solche in Ehren; \hh{30}denn wegen des Werkes Christi kam er dem Tod nahe, indem er sein Leben gering achtete, um euren Mangel im Dienst für mich aufzufüllen.