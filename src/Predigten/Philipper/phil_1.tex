
\section{1}
\hh{1}Paulus und Timotheus, Knechte Jesu Christi, alle Heiligen in Jesus, de, Christus, die in Philippi sind, zusammen mit Aufsehern und Dienern: \hh{2}Gnade euch und Friede von Gott, unserem Vater, und dem Herrn Jesus, dem Gesalbten!\hh{3}Ich danke meinem Gott bei jedem Gedenken an euch, \hh{4}allezeit in all meinem Beten für euch alle, dabei das Gebet mit Freuden verrichtend \hh{5}wegen eurer Teilnahme am Evangelium vom ersten Tag an bis jetzt, \hh{6}weil ich davon überzeugt bin, dass der, der ein gutes Werk in euch angefangen hat, [es] zu Ende führen wird bis zum Tag Jesu Christi; \hh{7} so wie es für mich recht ist, dies über euch alle zu denken, weil ich euch im Herzen habe, da ihr alle sowohl in meinen Fesseln als auch in der Verteidigung und Bekräftigung des Evangeliums zusammen mit mir Teilhaber seid an der Gnade. \hh{8}Denn Gott ist mein Zeuge, wie ich mich nach euch allen sehne mit dem herzlichen Empfinden Jesu, des Gesalbten. \hh{9}Und dieses erbete ich, dass eure Liebe noch mehr und mehr zunehme in der Erkenntnis und allem Empfinden, \hh{10}sodass ihr prüfen könnt, was das Vorzuziehende sei damit ihr lauter und ohne Anstoss seid am Tag Christi, \hh{11} erfüllt mit der Frucht der Gerechtigkeit, die durch Jesus, den Gesalbten, \es{ist}, zur Herrlichkeit und zum Lob Gottes.

\hh{12}Ich will aber, dass ihr wisst, Brüder, dass meine Umstände mehr zum Fortschreiten des Evangeliums geführt haben, \hh{13}sodass meine Fesseln \es{als Fesseln} in Christus offenbar geworden sind im ganzen Prätorium und den übrigen allen,\hh{14} und dass die meisten der Brüder, da sie im Herrn Vertrauen haben durch meine Fesseln, umso mehr wagen, das Wort Gottes zu sagen ohne Furcht. \hh{15} Zwar verkündigen einige den Christus gar aus Neid und Streit, andere dagegen aus gutem Willen. \hh{16} Die einen aus Liebe, das sie wissen, dass ich zur Verteidigung des Evangeliums bestimmt bin; \hh{17} die anderen verkünden Christus aus Eigennutz, nicht lauter, da sie meinen, \es{mir} in meinen Fesseln Begrängnis zu erwecken. \hh{18} Doch was \es{tut's}? Jedenfalls wird auf alle Weise, sei es zum Vorwand oder in Wahrheit, Christus verkündet, und darüber freue ich mich, ich werde mich auch \es{weiterhin} freuen. \hh{19} Ich weiss nähmlich: \enquote{Dies wird mir zum Heil ausgehen} durch euer Bitten und durch die Unterstützung des Geistes Jesu, des Gesalbten, \hh{20}gemäss meinem erwartungsvollen Harren und der Hoffnung, dass ich in nichts werde beschämt werden, sondern mit allem Freimut, wie allezeit, so auch jetzt Christus gross gemacht wird an meinem Leib, ob durch Leben oder Tod. \hh{21} Denn zu leben ist für mich Christus und zu sterben Gewinn. \hh{22} Wenn aber im Fleisch leben -- das \es{hiesse} für mich Frucht aus \es{weiterem} Wirken. Und was ich wählen soll, weiss ich nicht. \hh{23} Ich werde bedrängt von beidem, da ich Lust habe, aufzubrechen und bei Christus zu sein, denn \es{das wäre} um vieles besser; \hh{24} doch das Verbleiben im Fleisch ist nötiger euretwegen. \hh{25} Weil ich von diesem überzeugt bin, weiss ich: Ich werde bleiben und bei euch allen verbleiben zu eurem Fortschreiten und eurer Freude im Glauben, \hh{26} damit euer Rühmen an mir in Jesus, dem Gesalbten, zunehme durch meine erneute Ankunft bei euch.

\hh{27}Nur: Führt euer Leben \es{im Gemeinwesen} würdig des Evangeliums des Christus, damit, ob ich ankomme und euch erblicke oder abwesend bin, ich von euren Umständen höre, dass ihr \es{fest}steht in einem Geist, mit einer Seele zusammen kämpfend für den Glauben des Evangeliums \hh{28}und durch nicht eingeschüchtert von den Widerstreitenden, was für sie ein Anzeichen des Verderbens ist, aber eures Heils -- und das von Gott her; \hh{29} denn euch ist es hinsichtlich Christi geschenkt worden, nicht allein an ihn zu glauben, sondern auch für ihn zu leiden, \hh{30}die ihr ja den gleichen Kampf habt, so beschaffen, wie ihr \es{ihn} an mit gesehen habt und von mir hört.

