\author{OTS}
\documentclass[12pt]{../../inc/mybib}

\setincpath{../../inc/}

\usepackage{bible_style}

\graphicspath{{../../assets/images/}}

\newcommand{\Name}{Thomas}
% ensure scrlayer-scrpage has sufficient footheight
\setlength{\footheight}{20.4pt}

\begin{document}

\section{Tabellarische Auflisung der Imperative im Philipperbrief}
\renewcommand{\arraystretch}{1.8}
\begin{longtable}{l c l p{7cm}}
\hline 
\rowcolor{gray!20}

\textbf{Vers} & \textbf{Imperativ} & \textbf{Grammatik} & \textbf{Bemerkung} \\ 

\hline
\endfirsthead

\hline 
\rowcolor{gray!20}
\textbf{Vers} & \textbf{Imperativ} & \textbf{Grammatik} & \textbf{Bemerkung} \\ 
\hline
\endhead
\verslink{Philipper}{1}{27}
& Führt 
& 2. Person Plural 
& Aufforderung ein Leben nach dem Evangelium von Christus zu führen. 
\\ 
\hline
\verslink{Philipper}{2}{2}
& macht voll
& 2. Person Plural 
& Wenn sie Gemeinschaft, inniges Mitgefühl und Erbarmungen haben, dann erfüllen sie Paulus mit Freude.
\\ 
\hline
\verslink{Philipper}{2}{5}
& sei
& 3. Person Singular 
& Er fordert sie auf die gleiche Gesinnung zu haben wie Christus, der sich selbst erniedrigt hat.
\\ 
\hline
\verslink{Philipper}{2}{12}
& bringt hervor
& 3. Person Singular 
& Sie sollen auch in der Abwesenheit Paulus, das von Paulus gelehrte Evangelium ohne Furcht und Zittern wirken lassen.
\\ 
\hline
\verslink{Philipper}{2}{14}
& tut
& 2. Person Plural
& Sie sollen es mit Freude tun, so dass sie ein Licht sind der verdrehten Welt.
\\ 
\hline
\verslink{Philipper}{2}{18}
& freut euch, freut euch
& 2. Person Plural
& Sie sollen sich einzeln und zusammen als Gemeinschaft mit Paulus auf den Tag Christi freuen
\\

\hline
\verslink{Philipper}{2}{29}
& nehmt auf, haltet
& 2. Person Plural
& Paulus schickt Epaphroditus wieder zu ihnen zurück und bittet sie ihn aufzunehmen und hochzuhalten, da er für das Evangelium sein Leben riskiert hat.
\\ 
\hline
\verslink{Philipper}{3}{1}
& freut
& 2. Person Plural
& Eine Wiederholung, dass sie sich am Herrn erfreuen sollen und treu im Evangelium bleiben sollen.
\\ 
\hline
\verslink{Philipper}{3}{2}
& 3 x habt
& 2. Person Plural
& Sie sollen wachsam sein, dass sie nicht von falschen Lehren verführt werden.
\\ 
\hline
\verslink{Philipper}{3}{15}
& lasst uns gesinnt sein
& 1. Person Plural
& Paulus will das sie ihr Glaubensleben nach vorne ausrichten und den Kampfpreis nicht aus den Augen verlieren.
\\ 
\hline
\verslink{Philipper}{3}{16}
& ausrichten
& 1. Person Plural
& Alle sollen am gleichen Strang ziehen, dieser Strang ist das Evangelium von Christus.
\\ 
\hline
\verslink{Philipper}{3}{17}
& seid
& 2. Person Plural 
& Sie sollen Paulus als ihr Vorbild nehmen
\\ 
\hline
\verslink{Philipper}{3}{17}
& achtet
& 2. Person Plural 
& Sie sollen darauf achten, dass sie solchen Lehrern folgen, die das gleiche Evangelium wie Paulus verkünden.
\\ 
\hline
\verslink{Philipper}{4}{1}
& steht fest
& 2. Person Plural 
& Wie in Kapitel 3 erklärt ermahnt Paulus sie im Glauben festzustehen.
\\ 
\hline
\verslink{Philipper}{4}{3}
& stehe ihne bei
& 1. Person Singular 
& Der "Jochgenosse" soll Syntyche und Euodia in ihren Differenzen beistehen.
\\ 
\hline
\verslink{Philipper}{4}{4}
& 2 x freut
& 2. Person Plural 
& Sie sollen sich allezeit Freuen egal was passiert.
\\ 
\hline
\verslink{Philipper}{4}{5}
& bekannt werde
& 3. Person Plural 
& Sie sollen ihre Freude nach Aussen bringen, so das jeder sieht, dass sie im Herrn Freude haben.
\\ 
\hline
\verslink{Philipper}{4}{6}
& Macht sollen kundwerden
& 2. Person Plural 
& Jedes Anliegen soll im Gebet vor Gott gebracht werden.
\\ 
\hline
\verslink{Philipper}{4}{8}
& bedenkt
& 2. Person Plural 
& Immer auf die Tugenden achten
\\ 
\hline
\verslink{Philipper}{4}{9}
& tut
& 2. Person Plural 
& Nicht nur Hörer, sondern sie sollen auch Täter sein, nach dem Vorbild von Paulus.
\\ 
\hline
\verslink{Philipper}{4}{19}
& sei
& 1. Person Singular 
& Paulus preist Gott
\\ 
\hline
\verslink{Philipper}{4}{21}
& Grüsst
& 2. Person Plural 
& Sie sollen alle Heiligen grüssen
\\ 
\hline
\verslink{Philipper}{4}{23}
& sei
& 1. Person Singular 
& Er wünscht ihnen die Gnade des Herrn Jesus Christus
\\ 
\hline
\end{longtable}



\end{document}