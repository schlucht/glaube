\section*{Kapitel 3}
\hh{1}Des Weiteren, meine \person{Brüder}, \verbI{freut} euch \es{stets} im \person{Herrn}! Euch das Gleiche \es{wiederholt} zu \verbN{schreiben}, macht mir keine Bedenken, auch aber \es{gibt es} Festigkeit.

\hh{2}\verbI{Habt} ein Auge auf die \person{Hunde}, \verbI{habt} ein Auge auf die bösen \person{Arbeiter}, \verbI{habt} ein Auge auf die Zerschneidung. \hh{3}Denn wir sind die Beschneidung, die im \person{Geist Gottes} \es{\person{Gott}} \verbN{dienen} und uns in \person{Jesus, dem Gesalbten}, \verbN{rühmen} und nicht auf Fleisch \verbN{vertrauen}, \hh{4}obwohl auch ich \es{Grund \verbN{hätte}}, auf Fleisch zu \verbN{vertrauen}. Wenn irgendein anderer \verbN{meint}, er \es{\verbN{habe} Grund}, auf Fleisch zu \verbN{vertrauen}, ich mehr: \hh{5}Beschneidung als \person{Achtjähriger}, aus dem Geschlecht \ort{Israel}, dem \ort{Stamm Benjamin}, \person{Hebräer} von \person{Hebräern}; dem Gesetz nach \person{Pharisäer}; \hh{6}dem Eifer nach \person{Verfolger} der Gemeinde; der Gerechtigkeit nach, die im Gesetz \es{\verbN{ist}}, untadelig \verbN{geworden}.

\hh{7}Jedoch, was irgend mit Gewinn \verbN{war}, das \verbN{habe} ich des \person{Gesalbten} wegen für Verlust \verbN{geachtet}; \hh{8}ja, vielmehr, ich \verbN{achte} noch alles für Verlust aufgrund des überragenden \es{Wertes} der Erkenntnis \person{Christi Jesu}, meines \person{Herrn}, dessentwegen ich alles \verbN{verloren habe}, und ich \verbN{halte} es für Unrat, damit ich Christus \verbN{gewinne} \hh{9}und in ihm erfunden \verbN{werde}, wobei ich nicht meine Gerechtigkeit \verbN{habe} -- die aus dem Gesetz --, sondern die durch den Glauben an den \person{Gesalbten}, die Gerechtigkeit aus \person{Gott} aufgrund des Glaubens, \hh{10}um die Erkenntnis zu \verbN{erlangen} von ihm und von der Kraft seiner Auferstehung und die Gemeinschaft mit seinen Leiden, womit ich seinen Tod gleichgestaltet \verbN{werde}, \hh{11}ob ich \es{vielleicht} \verbN{hingelange} zur Auferstehung aus den Toten.

\hh{12}Nicht das ich \es{es} schon \verbN{ergriffen habe} oder schon vollendet \verbN{bin}; ich \verbN{jage} \es{ihm} aber nach, ob ich es auch \verbN{ergreifen möge}, weil ich \es{ja} auch \verbN{ergriffen wurde} von \person{Jesus, dem Gesalbten}. \hh{13}\person{Brüder}, ich selbst \verbN{halte} mich nicht dafür, \es{es} \verbN{ergriffen zu haben}; eines aber: Indem ich \verbN{vergesse}, was dahinten \verbN{ist}, und indem ich mich \verbN{ausstrecke} nach dem, was vorn \verbN{ist}, \hh{14}\verbN{jage} ich nach dem Ziel, hin zum Siegespreis, dem Ruf \person{Gottes} nach oben in \person{Jesus, dem Gesalbten}. \hh{15}So viele also vollkommen \es{sind}, \verbN{lasst} uns so \verbN{gesinnt} \verbN{sein}! Und wenn ihr anders \verbN{gesinnt} \verbN{seid}, auch das \verbN{wird} \person{Gott} euch \verbN{aufdecken}. \hh{16}Doch wozu wir \verbN{gelangt sind}; Richten wir uns nach derseleben \es{Ordnung} aus!

\hh{17}\verbI{Seid} zusammen meine \person{Nachahmer}, \person{Brüder}, und \verbI{achtet} \es{stets} auf jene, die so \verbN{wandeln}, wie ihr uns zum \person{Vorbild} \verbN{habt}! \hh{18}Denn viele \verbN{wandeln}, von denen ich euch oft \verbN{gesagt habe}, jetzt aber sogar \verbN{weinend} \verbN{sage}: Sie \es{\verbN{sind}} die \person{Feinde} des Kreuzes Christi, \hh{19}deren Ende Verderben, deren \person{Gott} der Bauch und die Herrlichkeit in ihrer Schande \verbN{ist}, die auf das Irdische \verbN{sinnen}. \hh{20}Aber unser Gemeinwesen \verbN{ist} in den Himmeln, von woher wir auch als \person{Retter} den \person{Herrn Jesus, den Gesalbten}, \verbN{erwarten}, \hh{21}der unseren Leib der Niedrigkeit \verbN{umwandeln} \verbN{wird}, sodass er gleichgestaltet \verbN{wird} seinem Leib der Herrlichkeit, nach der Wirkkraft, mit der er sich auch Alles zu \verbN{unterwerfen} vermag.

\begin{block}[Gedanken zum Kapitel 3]
    \lineheight{0.5}
    \begin{itshape}
        In diesem Kapitel spricht Paulus wie er sein Glaubensleben und seine Tätigkeiten erlebt. Die Philipper kennen die Nöte von Paulus. Sie haben ja live miterlebt, wie er geschlagen und in den Kerkers geworfen wurde.

        Diese Kapitel soll den Philipper Mut machen standhaft zu bleibe. Trotz Widrigkeiten sollen sie auf den Herrn vertrauen. So in Vers 20. Paulus schreibt: \flqq Freut euch! Freut euch im Herrn.\frqq Verlasst euch nicht auf das Fleisch sondern auf den Herrn, wie auch Paulus sich auf den Herrn verlässt und sich auf ihn und an ihm erfreut.
    \end{itshape}
\end{block}