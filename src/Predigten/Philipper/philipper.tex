\author{OTS}
\documentclass[12pt]{../../inc/mybib}

\setincpath{../../inc/}

\usepackage{bible_style}
\graphicspath{{../../assets/images/}}
\usepackage{header}

\newcommand{\verbN}[1]{\colorbox{green}{\textcolor{black}{#1}}} 
\newcommand{\verbP}[1]{\verbN{#1}\textsuperscript{\textbf{P}}}

\newcommand{\verbI}[1]{\colorbox{yellow}{\textcolor{black}{#1}}} 
\newcommand{\ort}[1]{\colorbox[rgb]{1,0,1}{\textcolor{white}{#1}}} 
\newcommand{\person}[1]{\colorbox{blue}{\textcolor{white}{#1}}} 
\newcommand{\bindW}[1]{\colorbox[rgb]{1,0,0}{\textcolor{white}{#1}}} 
\newcommand{\hr}{\par\noindent\hrulefill\par}
\newcommand{\lineheight}[1]{\setlength{\baselineskip}{{#1}\baselineskip}}
\geometry{
   left=25mm,
   right=40mm,
   top=25mm,
   bottom=25mm,   
   marginpar=3cm,
   head=20cm
   }
\setlength{\footheight}{29pt} % oder etwas mehr, je nach Boxhöhe
\setlength{\footskip}{40pt}   % optional, falls Abstand zum Text zu klein
\author{Lothar Schmid}
\setlength{\baselineskip}{2.5\baselineskip}
\begin{document}
\section{Legende}

\renewcommand{\arraystretch}{1.8} % mehr Zeilenhöhe
\begin{table}[htbp]
    \centering
    \begin{tabular}{|p{3cm}|p{4cm}|}
        \hline
        \verbP{passiv Veb}  & \verbN{normal Verb} \\ \hline
        \verbI{imperativ Verb} & \ort{Orte} \\ \hline
        \person{Personen} & \bindW{Bindewort} \\ \hline
    \end{tabular}
    \caption{Übersicht der Farben}
\end{table}


\begin{spacing}{2}    
    % 
\section*{Kapitel 1}
\hh{1}\person{Paulus} und \person{Timotheus}, \person{Knechte Jesu Christi}, alle \person{Heiligen in Jesus}, der, Christus, die in \ort{Philippi} \verbN{sind}, zusammen mit \person{Aufsehern} und \person{Dienern}: \hh{2}Gnade euch und Friede von \person{Gott}, \person{unserem Vater}, und dem \person{Herrn Jesus, dem \-Gesalbten}! \hh{3}Ich \verbN{danke} meinem \person{Gott} bei jedem Gedenken an euch, \hh{4}allezeit in all meinem Beten für euch alle, dabei das Gebet mit Freuden \verbN{verrichtend} \hh{5}wegen eurer Teilnahme am Evangelium vom ersten Tag an bis jetzt, \hh{6}weil ich davon überzeugt \verbN{bin}, dass der, der ein gutes Werk in euch \verbN{angefangen hat}, \es{es} zu Ende \verbN{führen} wird bis zum Tag \person{Jesu Christi}; \hh{7}so wie es für mich recht \verbN{ist}, dies über euch alle zu \verbN{denken}, weil ich euch im Herzen \verbN{habe}, da ihr alle sowohl in meinen \ort{Fesseln} als auch in der Verteidigung und Bekräftigung des Evangeliums zusammen mit mir Teilhaber \verbN{seid} an der Gnade. \hh{8}Denn Gott \verbN{ist} mein \person{Zeuge}, wie ich mich nach euch allen \verbN{sehne} mit dem herzlichen Empfinden \person{Jesu, des Gesalbten}. \hh{9}Und dieses \verbN{erbete} ich, dass eure Liebe noch mehr und mehr \verbN{zunehme} in der Erkenntnis und allem Empfinden, \hh{10}sodass ihr \verbN{prüfen könnt}, was das Vorzuziehende \verbN{sei} damit ihr lauter und ohne Anstoss \verbN{seid} am \ort{Tag Christi}, \hh{11}\verbN{erfüllt} mit der Frucht der Gerechtigkeit, die durch \person{Jesus, den Gesalbten}, \es{ist}, zur Herrlichkeit und zum Lob \person{Gottes}.

\hh{12}Ich \verbN{will} aber, dass ihr \verbN{wisst}, \person{Brüder}, dass meine Umstände mehr zum Fortschreiten des Evangeliums \verbN{geführt haben}, \hh{13}sodass meine Fesseln \es{als Fesseln} in \person{Christus} offenbar \verbN{geworden sind} im ganzen \ort{Prätorium} und den übrigen allen, \hh{14}und dass die meisten der \person{Brüder}, da sie im \person{Herrn} Vertrauen \verbN{haben} durch meine Fesseln, umso mehr \verbN{wagen}, das Wort Gottes zu \verbN{sagen} ohne Furcht. \hh{15}Zwar \verbN{verkündigen} einige den \person{Christus} gar aus Neid und Streit, andere dagegen aus gutem Willen. \hh{16}Die einen aus Liebe, das sie \verbP{wissen}, dass ich zur Verteidigung des Evangeliums bestimmt \verbN{bin}; \hh{17}die anderen \verbN{verkünden} \person{Christus} aus Eigennutz, nicht lauter, da sie \verbN{meinen}, \es{mir} in meinen Fesseln Begrängnis zu \verbN{erwecken}. \hh{18}Doch was \es{\verbN{tut}\textquotesingle s}? Jedenfalls \verbN{wird} auf alle Weise, sei es zum Vorwand oder in Wahrheit, \person{Christus} \verbN{verkündet}, und darüber freue ich mich, ich \verbN{werde} mich auch \es{weiterhin} freuen. \hh{19}Ich \verbN{weiss} nähmlich: \enquote{Dies wird mir zum Heil \verbN{ausgehen}} durch euer Bitten und durch die Unterstützung des \person{Geistes Jesu, des Gesalbten}, \hh{20}gemäss meinem erwartungsvollen Harren und der Hoffnung, dass ich in nichts \verbN{werde} beschämt \verbP{werden}, sondern mit allem Freimut, wie allezeit, so auch jetzt \person{Christus} gross \verbN{gemacht wird} an meinem Leib, ob durch Leben oder Tod. \hh{21}Denn zu \verbN{leben ist} für mich \person{Christus} und zu \verbN{sterben} Gewinn. \hh{22}Wenn aber im Fleisch zu \verbN{leben} -- das \es{\verbN{hiesse}} für mich Frucht aus \es{weiterem} Wirken. Und was ich \verbN{wählen} soll, \verbN{weiss} ich nicht. \hh{23}Ich \verbN{werde bedrängt} von beidem, da ich Lust \verbN{habe}, \verbN{aufzubrechen} und bei \person{Christus} zu \verbN{sein}, denn \es{das \verbN{wäre}} um vieles besser; \hh{24}doch das Verbleiben im Fleisch \verbN{ist} nötiger euretwegen. \hh{25}Weil ich von diesem überzeugt \verbN{bin}, \verbN{weiss} ich: Ich \verbN{werde} bleiben und bei euch allen \verbN{verbleiben} zu eurem Fortschreiten und eurer Freude im Glauben, \hh{26}damit euer Rühmen an mir in \person{Jesus, dem Gesalbten}, \verbN{zunehme} durch meine erneute Ankunft bei euch.

\hh{27}Nur: \verbN{Führt} euer Leben \es{im Gemeinwesen} würdig des Evangeliums des \person{Christus}, damit, ob ich \verbN{ankomme} und euch \verbN{erblicke} oder abwesend \verbN{bin}, ich von euren Umständen \verbN{höre}, dass ihr \verbN{\es{fest}steht} in einem Geist, mit einer Seele zusammen \verbN{kämpfend} für den Glauben des Evangeliums \hh{28}und durch nichts \verbN{eingeschüchtert} von den Widerstreitenden, was für sie ein Anzeichen des Verderbens \verbN{ist}, aber eures Heils -- und das von Gott her; \hh{29}denn euch \verbN{ist} es hinsichtlich \person{Christi} \verbP{geschenkt worden}, nicht allein an ihn zu \verbN{glauben}, sondern auch für ihn zu \verbN{leiden}, \hh{30}die ihr ja den gleichen Kampf \verbN{habt}, so \verbN{beschaffen}, wie ihr \es{ihn} an mir \verbN{gesehen habt} und von mir \verbN{hört}.


    % \section*{Kapitel 2}
\hh{1}\bindW{Wenn} es also \es{so \verbN{ist}, \bindW{dass} es} Ermunterung \verbN{gibt} in \person{Christus}, \bindW{wenn} Zuspruch der Liebe, \bindW{wenn} Gemeinschaft des \person{Geistes}, \bindW{wenn} inniges Mitgefühl \bindW{und} Erbarmungen, \hh{2}\bindW{dann} \verbN{macht} meine Freude \es{damit} voll, \bindW{dass} ihr \es{auf} das Gleiche \verbN{sinnt}, \bindW{indem} ihr dieselbe Liebe \verbN{habt}, in einer Seele \verbN{verbunden seid} \bindW{und} indem ihr auf eines \verbN{sinnt}, \hh{3}\bindW{indem} ihr nichts aus Eigennutz oder leerer Ruhmsucht \es{\verbN{tut}}, \bindW{sondern} in der Demut einer den anderen für höher \verbN{hält} als sich selbst, \hh{4}\bindW{indem} ein jeder nicht auf das Seine \verbN{schaut}, \bindW{sondern} ein jeder auch auf das der anderen. \hh{5}Unter euch \verbN{sei} diese Gesinnung, die auch in \person{Jesus, dem Gesalbten}, \verbN{war}, \hh{6}der, \bindW{obwohl} in Gestalt Gottes \verbN{seiend}, das Gott Gleichsein nicht wie eine Beute \verbN{ansah}, \hh{7}\bindW{sondern} sich selbst \verbN{entäusserte}, indem er die Gestalt eines \person{Knechtes} \verbN{annahm}. Den \person{Menschen} \verbN{gleich geworden} \bindW{und} in der äusseren Erscheinung wie ein Mensch \verbP{erfunden}, \hh{8}\verbN{erniedrigte} er sich selbst, indem er \verbN{gehorsahm wurde} bis zum Tod, \bindW{zum} Tod an einem Kreuz. \hh{9}\bindW{Darum} \verbI{erhöht} Gott ihn auch über \es{alles} \bindW{und} \verbN{gab} ihm den Namen, der über jeden Namen \verbN{ist}, \hh{10}\bindW{damit} im Namen \person{Jesu} sich \verbN{beuge} jedes Knie, \es{der} Himmlischen der Irdischen \bindW{und} Unterirdischen, \hh{11}\bindW{und} jede Zunge \verbN{bekenne}, \bindW{dass} \person{Jesus, der Gesalbte}, Herr \verbN{ist}, \bindW{zur} Verherrlichung \person{Gottes, des Vaters}.

\hh{12}\bindW{So denn}, meine Geliebte, wie ihr allezeit \verbN{gehorcht} \verbN{habt}, nicht nur wie in meiner Anwesenheit, \bindW{sondern} jetzt vielmehr in meiner Abwesenheit, \verbN{bringt} euer eigenes Heil hervor mit Furcht \bindW{und} Zittern; \hh{13}\bindW{denn} Gott \verbN{ist} der in euch Wirkende -- \bindW{sowohl} das Wollen als auch das Wirken -- wegen \es{seines} Wohlgefallens.

\hh{14}\verbI{Tut} alles ohne Murren \bindW{und} Bedenken, \hh{15}\bindW{damit} ihr \verbN{untadelig} \bindW{und} \verbP{unverfälscht werdet}, \person{Kinder Gottes} ohne Makel inmitten eines krummen \bindW{und} verdrehten Geschlechts, unter dem ihr \verbN{aufscheint} wie Lichter in der Welt, \hh{16}\bindW{indem} ihr \verbN{festhaltet} das Wort des Lebens, mir zum \es{Gegenstand des} Rühmens auf den Tag \person{Christi}, \bindW{weil} ich \es{dann} nicht vergeblich \verbN{gelaufen bin}, \bindW{noch} auch vergeblich \verbN{gearbeitet habe}. \hh{17}\bindW{Wenn} ich aber auch \es{als Gussopfer} \verbN{ausgegossen werde} über das \person{Opfer} \bindW{und} den Priesterdienst für euren Glauben, \verbN{freue} ich mich mit euch allen. \hh{18}\bindW{Ebenso} \verbN{freut} auch ihr euch \bindW{und} \verbN{freut} euch zusammen mit mir.

\hh{19}Ich \verbN{hoffe} aber in dem \person{Herrn Jesus}, \person{Timotheus} bald zu euch zu \verbN{senden}, \bindW{damit} auch ich frohgemut \verbN{sei}, \bindW{wenn} ich eure Umstände \verbN{erfahre}.
 \hh{20}Ich \verbN{habe} nämlich niemand gleichgesinnt, der in echter Weise für das eure \verbN{besorgt sein wird}; \hh{21}\bindW{denn} alle \verbN{suchen} das Eigene, nicht das, \es{was} \person{Jesu Christi} \es{\verbN{ist}}. \hh{22} \bindW{Aber} seine Bewährtheit \verbN{kennt} ihr, \bindW{dass} er wie ein \person{Kind} dem \person{Vater} zusammen mit mir \verbN{gedient} hat im Evangelium. \hh{23}Diesen also \verbN{hoffe} ich, sofort zu \verbN{schicken}, sobald ich \verbN{absehe} wie es um mich \verbN{steht}. \hh{24}\bindW{Doch} ich \verbN{bin} zuversichtlich im \person{Herrn}, \bindW{dass} auch ich selbst bald \verbN{kommen} werden. \hh{25}Ich \verbN{hielt} es aber für notwendig, \person{Epaphroditus} meinen \person{Bruder} \bindW{und} \person{Mitarbeiter} \bindW{und} \person{Mitkämpfer}, \bindW{aber} euren \person{Abgesandten} \bindW{und} \person{Diener} meines Bedarfs, zu euch zu \verbN{schicken}, \hh{26}da er sich nach euch allen \verbN{sehnte} \bindW{und} in Unruhe war, \bindW{weil} ihr \verbN{gehört hattet}, \bindW{dass} er \verbN{erkrankt}, dem Tod nahe. \bindW{Doch} \person{Gott} \verbN{erbarmte} sich über ihn, \bindW{und} nicht nur über ihn, \bindW{sondern} auch über mich, \bindW{damit} ich nicht Kummer über Kummer \verbN{bekäme}. \hh{28}\bindW{Also} \verbN{habe} ich ihn \es{umso} eiliger \verbN{geschickt}, \bindW{damit} ihr, wenn ihr ihn \verbN{seht}, wieder \verbN{froh werdet} \bindW{und} ich weniger \verbN{bekümmert sei}. \hh{29}\verbI{Nehmt} ihn also auf im \person{Herrn} mit aller Freude, \bindW{und} \verbI{haltet} solche in Ehren; \hh{30}\bindW{denn} wegen des Werkes \person{Christi} \verbN{kam} er dem Tod nahe, \bindW{indem} er sein Leben \verbN{gering achtete}, \bindW{um} euren Mangel im Dienst für mich \verbN{aufzufüllen}.

 \begin{block}[Gedanken zum Kapitel 2]
    \lineheight{0.5}
    \begin{itshape}
        Im Gegensatz zum Korintherbrief muss Paulus die Gemeinde in Philippi nicht tadeln. Er lobt ihre Zusammenarbeit \bindW{und} der Umgang miteinander. Er ermutigt sie, so weiter zu machen \bindW{und} nicht auf falsche Lehrer zu hören \bindW{und} nicht überheblich zu werden. Er ermutigt sie Jesus Christus den Gesalbten zu bekennen \bindW{und} ihm treu zu bleiben.

        Paulus will ihnen Timotheus schicken. Auf diesen kann er sich verlassen. Bei ihm weiss er, \bindW{dass} er sich gut um die Philipper kümmert.

        Den Mitarbeiter Epaphroditus schickt er, sobald dieser Ges\bindW{und} war, wieder direkt zurück. Der Mut \bindW{und} den Einsatz von Epaphrodites bringt Trost \bindW{und} Zuversicht in die Gemeinde von Philippi.
    \end{itshape}
\end{block}
    % \section*{Kapitel 3}
\hh{1}Des Weiteren, meine \person{Brüder}, \verbI{freut} euch \es{stets} im \person{Herrn}! Euch das Gleiche \es{wiederholt} zu \verbN{schreiben}, macht mir keine Bedenken, auch aber \es{gibt es} Festigkeit.

\hh{2}\verbI{Habt} ein Auge auf die \person{Hunde}, \verbI{habt} ein Auge auf die bösen \person{Arbeiter}, \verbI{habt} ein Auge auf die Zerschneidung. \hh{3}Denn wir sind die Beschneidung, die im \person{Geist Gottes} \es{\person{Gott}} \verbN{dienen} und uns in \person{Jesus, dem Gesalbten}, \verbN{rühmen} und nicht auf Fleisch \verbN{vertrauen}, \hh{4}obwohl auch ich \es{Grund \verbN{hätte}}, auf Fleisch zu \verbN{vertrauen}. Wenn irgendein anderer \verbN{meint}, er \es{\verbN{habe} Grund}, auf Fleisch zu \verbN{vertrauen}, ich mehr: \hh{5}Beschneidung als \person{Achtjähriger}, aus dem Geschlecht \ort{Israel}, dem \ort{Stamm Benjamin}, \person{Hebräer} von \person{Hebräern}; dem Gesetz nach \person{Pharisäer}; \hh{6}dem Eifer nach \person{Verfolger} der Gemeinde; der Gerechtigkeit nach, die im Gesetz \es{\verbN{ist}}, untadelig \verbN{geworden}.

\hh{7}Jedoch, was irgend mit Gewinn \verbN{war}, das \verbN{habe} ich des \person{Gesalbten} wegen für Verlust \verbN{geachtet}; \hh{8}ja, vielmehr, ich \verbN{achte} noch alles für Verlust aufgrund des überragenden \es{Wertes} der Erkenntnis \person{Christi Jesu}, meines \person{Herrn}, dessentwegen ich alles \verbN{verloren habe}, und ich \verbN{halte} es für Unrat, damit ich Christus \verbN{gewinne} \hh{9}und in ihm erfunden \verbN{werde}, wobei ich nicht meine Gerechtigkeit \verbN{habe} -- die aus dem Gesetz --, sondern die durch den Glauben an den \person{Gesalbten}, die Gerechtigkeit aus \person{Gott} aufgrund des Glaubens, \hh{10}um die Erkenntnis zu \verbN{erlangen} von ihm und von der Kraft seiner Auferstehung und die Gemeinschaft mit seinen Leiden, womit ich seinen Tod gleichgestaltet \verbN{werde}, \hh{11}ob ich \es{vielleicht} \verbN{hingelange} zur Auferstehung aus den Toten.

\hh{12}Nicht das ich \es{es} schon \verbN{ergriffen habe} oder schon vollendet \verbN{bin}; ich \verbN{jage} \es{ihm} aber nach, ob ich es auch \verbN{ergreifen möge}, weil ich \es{ja} auch \verbN{ergriffen wurde} von \person{Jesus, dem Gesalbten}. \hh{13}\person{Brüder}, ich selbst \verbN{halte} mich nicht dafür, \es{es} \verbN{ergriffen zu haben}; eines aber: Indem ich \verbN{vergesse}, was dahinten \verbN{ist}, und indem ich mich \verbN{ausstrecke} nach dem, was vorn \verbN{ist}, \hh{14}\verbN{jage} ich nach dem Ziel, hin zum Siegespreis, dem Ruf \person{Gottes} nach oben in \person{Jesus, dem Gesalbten}. \hh{15}So viele also vollkommen \es{sind}, \verbN{lasst} uns so \verbN{gesinnt} \verbN{sein}! Und wenn ihr anders \verbN{gesinnt} \verbN{seid}, auch das \verbN{wird} \person{Gott} euch \verbN{aufdecken}. \hh{16}Doch wozu wir \verbN{gelangt sind}; Richten wir uns nach derseleben \es{Ordnung} aus!

\hh{17}\verbI{Seid} zusammen meine \person{Nachahmer}, \person{Brüder}, und \verbI{achtet} \es{stets} auf jene, die so \verbN{wandeln}, wie ihr uns zum \person{Vorbild} \verbN{habt}! \hh{18}Denn viele \verbN{wandeln}, von denen ich euch oft \verbN{gesagt habe}, jetzt aber sogar \verbN{weinend} \verbN{sage}: Sie \es{\verbN{sind}} die \person{Feinde} des Kreuzes Christi, \hh{19}deren Ende Verderben, deren \person{Gott} der Bauch und die Herrlichkeit in ihrer Schande \verbN{ist}, die auf das Irdische \verbN{sinnen}. \hh{20}Aber unser Gemeinwesen \verbN{ist} in den Himmeln, von woher wir auch als \person{Retter} den \person{Herrn Jesus, den Gesalbten}, \verbN{erwarten}, \hh{21}der unseren Leib der Niedrigkeit \verbN{umwandeln} \verbN{wird}, sodass er gleichgestaltet \verbN{wird} seinem Leib der Herrlichkeit, nach der Wirkkraft, mit der er sich auch Alles zu \verbN{unterwerfen} vermag.

\begin{block}[Gedanken zum Kapitel 3]
    \lineheight{0.5}
    \begin{itshape}
        In diesem Kapitel spricht Paulus wie er sein Glaubensleben und seine Tätigkeiten erlebt. Die Philipper kennen die Nöte von Paulus. Sie haben ja live miterlebt, wie er geschlagen und in den Kerkers geworfen wurde.

        Diese Kapitel soll den Philipper Mut machen standhaft zu bleibe. Trotz Widrigkeiten sollen sie auf den Herrn vertrauen. So in Vers 20. Paulus schreibt: \flqq Freut euch! Freut euch im Herrn.\frqq Verlasst euch nicht auf das Fleisch sondern auf den Herrn, wie auch Paulus sich auf den Herrn verlässt und sich auf ihn und an ihm erfreut.
    \end{itshape}
\end{block}
     \section*{Kapitel 4}
\hh{1}Daher meine geliebten und \verbN{ersehnten} \person{Brüder}, meine \person{Freude} und mein Siegeskranz: Auf diese Weise \verbN{steht} \es{fest} im \person{Herrn}, \person{Geliebte}!

\hh{2}\person{Evodia} \verbN{rufe} ich auf, und \person{Syntyche} \verbN{rufe} ich auf, das Gleiche zu \verbN{sinnen} im \person{Herrn}. \hh{3}Ja, ich \verbI{bitte} auch dich, echter \person{Jochgenosse}, \verbN{stehe} ihnen bei, die im Evangelium mit mir \verbN{gekämpft \verbN{habe}n}, samt \person{Clemens} und meinen übrigen \person{Mitarbeitern}, deren Namen im Buch des Lebens \es{\verbN{stehen}}.

\hh{4}\verbI{Freut} euch im \person{Herrn} allezeit! Nochmals \verbN{will} ich \verbN{sagen}: \verbI{Freut} euch! \hh{5}Eure Milde \verbI{werde} allen \person{Menschen} \verbI{bekannt}! Der Herr \verbN{ist} nahe. \hh{6}\verbI{Macht} euch um nichts Sorgen, sondern in allem sollen eure Bitten durch Gebet und Flehen mit Danksagung vor \person{Gott} \verbN{kundwerden}, \hh{7}und der alles Denken überragende Friede \person{\gott[white]{Gottes}} \verbN{wird} eure Herzen und eure Gedanken in Gewahrsam \verbN{halten} in \person{\jesus[white]{Jesus, dem Gesalbten}}.

\hh{8}Des Weiteren, \person{Brüder}, alles, was wahr, was ehrbar, was gerecht, was rein, was liebenswert \verbN{ist}, was wohltuend \verbN{ist}, ob eine Tugend, ob ein Lob -- diese Dinge bedenkt. \hh{9} Was ihr auch \verbN{gelernt} und \verbN{übernommen} und \verbN{gehört} und an mir \verbN{gesehen} \verbN{habt}, das \verbI{tut}, und der \person{Gott des Friedens} \verbN{wird} mit euch \verbN{sein}.

\hh{10}Ich \verbN{habe} mich im \person{Herrn} hoch \verbN{gefreut}, dass ihr endlich wieder \verbN{aufgeblüht} \verbN{seid}, an mich zu \verbN{denken}; woran ihr zwar \verbN{dachtet}, aber ihr \verbN{hattet} keine \geist{Gelegenheit}. \hh{11}Nicht dass ich das aufgrund von Mangel \verbN{sage}, denn ich \verbN{habe gelernt}, worin ich \verbN{bin}, genügsam zu \verbN{sein}. \hh{12}Ich \verbN{weiss} erniedrigt zu \verbN{sein}, und ich \verbN{weiss} übrig zu \verbN{haben}. In jedes und in alles \verbN{bin} ich \verbN{eingeweiht}: satt \verbN{sein} und hungern, übrig \verbN{haben} und Mangel \verbN{leiden}. \hh{13}Alles \verbN{vermag} ich durch den, der mich \es{fortwährend} \verbN{kräftigt}. \hh{14}Und doch, ihr \verbN{habt} gut \es{daran} \verbN{getan}, an meiner Bedrängnis Anteil zu \verbN{nehmen}.

\hh{15}Ihr \verbI{wisst} auch selbst \es{liebe} \person{Philipper}, dass im Anfang \es{der Verkündigung} des Evangeliums, als ich \verbN{wegzog}, von \ort{Mazedonien}, keine Gemeinde Gemeinschaft \verbN{hatte} mit mir im  \es{gegenseitigen} Geben und Empfangen als nur ihr allein. \hh{16}Nämlich auch in \ort{Thessalonich} \verbN{schicktet} ihr mir einmal, sogar zweimal \es{etwas} für meinen Bedarf. \hh{17}Nicht dass ich die Gabe \verbN{suche}, sondern ich \verbN{suche} die sich für eure Rechnung mehrende Frucht. \hh{18}Ich \verbN{habe} alles \verbN{erhalten} und \verbN{habe} übrig; ich \verbN{bin} \verbN{erfüllt}, nachdem ich von \person{Epaphroditus} die \es{Gabe} von euch \person{empfangen} \verbN{habe}, einen lieblichen Geruch. Ein willkommenes Opfer, \person{Gott} wohgefällig. \hh{19}Mein \person{Gott} aber \verbN{wird} all euren Bedarf \verbN{erüllen} nach seinem Reichtum in Herrlichkeit in \person{Christus Jesus}. \hh{20}Unserem \person{Gott} und \person{Vater} \verbI{sei} die Herrlichkeit in alle Ewigkeit! Amen.

\hh{21}\verbI{Grüsst} jeden \person{Heiligen} in \person{Jesus, dem Gesalbten}. \hh{22}Alle \person{Heiligen} Grüssen euch, am meisten die dem \ort{Haus des Kaisers}.

\hh{23}Die Gnade des \person{Herrn Jesus, des Gesalbten}, \es{\verbN{sei}} mit eurem Geist!
\hr
\begin{block}[Gedanken zum Kapitel 4]
    \lineheight{0.5}
    \begin{itshape}
        Im Kapitel 4 bedankt sich Paulus nochmals bei seinen Mitstreitern in Philippi. Er spricht auch die Spenden an, die er von ihnen erhalten hat. Paulus nimmt diese auch dankend an und unterstreicht, dass es nicht selbstverständlich ist, dass er von den Gemeinden unterstützt wird. Es ist also nicht so, dass er unbedingt in Armut und Hunger leben will, sondern er freut sich, wenn er etwas bekommt, aber er hat auch gelernt mit wenigem auszukommen.
        
        Am Ende lässt er alle Grüssen und gibt den Segen unseres Herrn Jesus Christus weiter.
    \end{itshape}
\end{block}
\end{spacing}
\end{document}