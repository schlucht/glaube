\section*{Kapitel 4}
\hh{1}Daher meine geliebten und ersehnten Brüder, meine Freude und mein Siegeskranz: Auf diese Weise steht \es{fest} im Herrn, Geliebte!

\hh{2}Evodia rufe ich auf, und Syntyche rufe ich auf, das Gleiche zu sinnen im Herrn. \hh{3}Ja, ich bitte auch dich, echter Jochgenosse, stehe ihnen bei, die im Evangelium mit mir gekämpft haben, samt Clemens und meinen übrigen Mitarbeitern, deren Namen im Buch des Lebens \es{stehen}.

\hh{4}Freut euch im Herrn allezeit! Nochmals will ich sagen: Freut euch! \hh{5}Eure Milde werde allen Menschen bekannt! Der Herr ist nahe. \hh{6}Macht euch um nichts Sorgen, sondern in allem sollen eure Bitten durch Gebet und Flehen mit Danksagung vor Gott kundwerden, \hh{7}und der alles Denken überragende Friede Gottes wird eure Herzen und eure Gedanken in Gewahrsam halten in Jesus, dem Gesalbten.

\hh{8}Des Weiteren, Brüder, alles, was wahr, was ehrbar, was gerecht, was rein, was liebenswert ist, was wohltuend ist, ob eine Tugend, ob ein Lob -- diese Dinge bedenkt. \hh{9} Was ihr auch gelernt und übernommen und gehört und an mir gesehen habt, das tut, und der Gott des Friedens wird mit euch sein.

\hh{10}Ich habe mich im Herrn hoch gefreut, dass ihr endlich wieder aufgeblüht seid, an mich zu denken; woran ihr zwar dachtet, aber ihr hattet keine Gelegenheit. \hh{11}Nicht dass ich das aufgrund von Mangel sage, denn ich habe gelernt, worin ich bin, genügsam zu sein. \hh{12}Ich weiss erniedrigt zu sein, und ich weiss übrig zu haben. In jedes und in alles bin ich eingeweiht: satt sein und hungern, übrig haben und Mangel leiden. \hh{13}Alles vermag ich durch den, der mich \es{fortwährend} kräftigt. \hh{14}Und doch, ihr habt gut \es{daran} getan, an meiner Bedrängnis Anteil zu nehmen.

\hh{15}Ihr wisst auch selbst \es{liebe} Philipper, dass im Anfang \es{der Verkündigung} des Evangeliums, als ich wegzog, von Mazedonien, keine Gemeinde Gemeinschaft hatte mit mir im  \es{gegenseitigen} Geben und Empfangen als nur ihr allein. \hh{16}Nämlich auch in Thessalonich schicktet ihr mir einmal, sogar zweimal \es{etwas} für meinen Bedarf. \hh{17}Nicht dass ich die Gabe suche, sondern ich suche die sich für eure Rechnung mehrende Frucht. \hh{18}Ich habe alles erhalten und habe übrig; ich bin erfüllt, nachdem ich von Epaphroditus die \es{Gabe} von euch empfangen habe, einen lieblichen Geruch. Ein willkommenes Opfer, Gott wohgefällig. \hh{19}Mein Gott aber wird all euren Bedarf erüllen nach seinem Reichtum in Herrlichkeit in Christus Jesus. \hh{20}Unserem Gott und Vater sei die Herrlichkeit in alle Ewigkeit! Amen.

\hh{21}Grüsst jeden Heiligen in Jesus, dem Gesalbten. \hh{22}Alle Heiligen Grüssen euch, am meisten die dem Haus des Kaisers.

\hh{23}Die Gnade des Herrn Jesus, des Gesalbten, \es{sei} mit eurem Geist!