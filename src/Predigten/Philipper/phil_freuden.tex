\author{OTS}
\documentclass[12pt]{../../inc/mybib}

\setincpath{../../inc/}

\usepackage{bible_style}

\graphicspath{{../../assets/images/}}

\newcommand{\Name}{Thomas}
% ensure scrlayer-scrpage has sufficient footheight
\setlength{\footheight}{20.4pt}

\begin{document}

\section{Freudenförderer im Philipperbrief}
\renewcommand{\arraystretch}{1.8}
\begin{longtable}{l p{15cm}}
\hline 
\rowcolor{gray!20}

\textbf{Vers} & \textbf{Freudenförderer} \\ 

\hline
\endfirsthead

\hline 
\rowcolor{gray!20}
\textbf{Vers} & \textbf{Freudenförderer} \\ 
\hline
\endhead
\verslink{Philipper}{1}{5}
& Teilnahme am Gottesdienst
\\ 
\hline
\verslink{Philipper}{1}{9}
& Das Zunehmen der Liebe und der Erkenntnis und allem Empfinden
\\ 
\hline
\verslink{Philipper}{1}{18}
& Das Verkündigen des Evangeliums egal aus welchen Motiven
\\ 
\hline
\verslink{Philipper}{1}{27}
& Einwürgiges Leben nach dem Evangelium von Christus führen
\\ 
\hline
\verslink{Philipper}{2}{3}
& nicht aus Eigennutz oder eitler Ruhmsucht handeln, sondern in Demut die anderen höher achten als sich selbst.
\\ 
\hline
\verslink{Philipper}{2}{8}
& Gehorsamkeit
\\ 
\hline
\verslink{Philipper}{2}{16}
& Festhalten am Wort des Lebens
\\ 
\hline
\verslink{Philipper}{2}{29}
& Hilfsbereitschaft und Fürsorge füreinander
\\ 
\hline
\verslink{Philipper}{3}{3}
& Vertrtauen auf Gott und nicht auf den Menschen
\\ 
\hline
\verslink{Philipper}{3}{9}
& Gottes Gerechtigkeit durch den Glauben
\\ 
\hline
\verslink{Philipper}{3}{12}
& Dieser Gerechtigkeit nachjagen
\\ 
\hline
\verslink{Philipper}{3}{17}
& Auf gute Vorbilder achten
\\ 
\hline
\verslink{Philipper}{4}{4}
& Die Freude im Herrn
\\ 
\hline
\verslink{Philipper}{4}{6}
& Bitten, Danksagung und Gebet an Gott
\\ 
\hline
\verslink{Philipper}{4}{8}
& Des Weiteren, Brüder, alles, wasc wahr, 
was ehrbar, was gerecht, was reind, was lie
benswert ist, was wohllautend ist, ob eine 
Tugend, ob ein Lob
\\ 
\hline
\verslink{Philipper}{4}{19}
& Geben, andere untestützen und helfen.
\\ 
\hline

\end{longtable}

\newpage
%%%%%%%%%%%%%%%%%%%%%%%%%%Section%%%%%%%%%%%%%%%%%%%%%%%%%%%%%%%%%%%%%
\section{Freudenkiller im Philipperbrief}
\begin{longtable}{l p{15cm}}
\hline 
\rowcolor{gray!20}

\textbf{Vers} & \textbf{Freudenkiller} \\ 

\hline
\endfirsthead
%%%%%%%%%%%%%%%%%%%%%%%%%Tabelle%%%%%%%%%%%%%%%%%%%%%%%%%%%%%%%%%%%%%
\hline 
\rowcolor{gray!20}
\textbf{Vers} & \textbf{Freudenkiller} \\ 
\hline
\endhead
\verslink{Philipper}{1}{12}
& Umstände in den Fesseln
\\ 
\hline
\verslink{Philipper}{1}{17}
& Verkündigen des Evangeliums aus falschen Motiven
\\ 
\hline
\verslink{Philipper}{1}{28}
& Einschüchterung durch Gegner
\\ 
\hline
\verslink{Philipper}{2}{14}
& Murren und Bedenken haben
\\ 
\hline
\verslink{Philipper}{2}{21}
& Ein eigennutziges Leben
\\ 
\hline
\verslink{Philipper}{3}{2}
& Falsche Lehrer und Lehren
\\ 
\hline
\verslink{Philipper}{3}{18}
& Falschen Vorbilden folgen
\\ 
\hline
%%%%%%%%%%%%%%%%%%%%%%%%%Tabelle%%%%%%%%%%%%%%%%%%%%%%%%%%%%%%%%%%%%%%
\end{longtable}


\end{document}