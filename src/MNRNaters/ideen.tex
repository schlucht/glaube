\author{Lothar}
\documentclass{../inc/mybib}

\setincpath{../inc/}

\usepackage{bible_style}
\usepackage{header}

\begin{document}    
    \section{MNR-Naters Bibellesen}
    \subsection{Idee}
    Die Idee ist ein zwangloses Bibellesen für Bibelinteressierten anzubieten. Egal ob wiedergeborene Christen oder einfach für Menschen die neugierig sind was so in der Bibel steht.

    Der Zutritt soll für alle möglich sein. Es sollen jederzeit neue Leute dazustossen können. Als Einstieg soll ein Vortrag über die Bibel sein. Dieser Vortrag soll öffentlich sein. Aus diesem Vortrag wird dann ein Bibellesen ins Leben gerufen.

    \subsection{Ablauf}
    Total Zeit: 45min
    Start: 20:00Uhr
    \begin{itemize}
        \item Gebet
        \item Abarbeitung Fragen aus dem letzten Leseabend.
        \item Ein Kaptiel aus einem Buch vorlesen oder vorlesen lassen.
        \item Eine Liste mit vorbereiteten Fragen durchgehen.
        \item Frage Runde. Fragen die nicht gelöst werden können, aufschreiben und bis zum nächsten Abend versuchen zu lösen.
        \item Gebet
    \end{itemize}
    \begin{enumerate}
        \item Keine theologischen Debatten führen, die keinen heilsbezogenen Bezug haben. Grundlage ist das Glaubensbekenntnis der MNR und ihr Verhältnis zu Israel. (Foliensatz)
        \item Die 45min müssen eingehalten werden.
        \item Es werden Bibeln zur Verfügung gestellt.
        \item Wasser wird zur Verfügung gestellt.
        \item Finanzieren auf Spendebasis wird dem MNR zugeschickt (Wasser, Bibeln)
        \item \dots       
    \end{enumerate}
    
    \subsection{Organisatorisch}
    \begin{enumerate}
        \item Wasser zur Verfügung stellen
        \item Schlachter Bibeln zur Verfügung stellen
        \item Leitung von einer Person aus der Gemeinde
        \item Wichtig! Jesus muss im Mittelpunkt sein. Nicht die Bibel an sich ist wichtig, sondern Botschaft in der Bibel.
        \item Start ab 1. Januar; Immer Dienstags alle 2 Wochen.
        \item Vortrag mitte November. Video Roger Liebi oder jemand von uns.
        \item Reklame und Inserate.
    \end{enumerate}
    \section{Evangelisation}
    \begin{itemize}
        \item Tracktate verteilen: Traktate mit Adressen versehen, so das die Leute Kontakt aufnehmen können.
        \item Kalender verteilen
        \item Tag der offenen Tür
        \item ...
    \end{itemize}
    \section{Öffentliche Vorträge}
    \begin{itemize}
        \item Ist die Bibel glaubwürdig
        \item ...
    \end{itemize}
    
    
\end{document}