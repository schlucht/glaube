
\section{Begrüssung}

Ich möchte euch alle und Andre Beitze herzlich zu diesem Gottesdienst begrüßen.
Schön Andre Beitze, dass du den langen Weg auf dich genommen hast, um uns das Wort Gottes zu bringen.
Heute feiern wir zusammen das Abendmahl, um daran zu gedenken, was Jesus für uns am Kreuz gemacht hat. Das Abendmahl ist ein fröhliches Fest, darum lasst uns jetzt beten und das erste Lied singen.\\
\colorbox{red}{\textcolor{white}{\textbf{GEBET}}}\\
Lasst uns beten und dann das erste Lied singen: \colorbox{yellow}{O Gott dir sei ehre der Grosses Getan}.

\section{Ankündigungen}
\begin{itemize}
    \item Donnerstag 10.10.2024 Bibel und Gebetsabend mit Thomas Lieth ab Römer 10.4.
    \item Nächster Gottesdienst: 13.10.2024 mit Thomas Lieth
    \item Info zum neuem Lokal. Wir haben ein neues Lokal. Der MNR übernimmt die Mietkosten. Am Samstag 12.Okt. werden wir schon teilweise mit dem Zügeln anfangen. Wer an diesem Samstag Zeit hat, kann sich gerne bei Ariane melden. Sie organisiert diesen Tag. Am 3.Nov. wird der erste Gottesdienst in den neuen Räumen stattfinden. Bitte betet, dass alles klappt und alles Sicher über die Bühne läuft.
    \begin{quote}
        Neue Adresse:
        Furkastrasse 46        
    \end{quote}
    Einfach hier gerade aus weiter Richtung Osten laufen, nach dem Denner auf der rechten Strassenseite.
\end{itemize}

\section{ Input }
\begin{spacing}{1.5}
\textbf{Wie werde ich glücklich?}\\
Meine Jugend habe ich in Birgisch verbracht. Mein Vater hatte so um die 8 Kühe, 5 Schweine, Hasen, Hühner, Katzen und einen Hund. Ausserdem noch Ackerland.\\
Das gab natürlich Arbeit. Als Jugendlicher hatte ich aber andere Flausen im Kopf, als meinem Vater auf dem Hof zu helfen. Die Arbeit als Bergbauer ist sehr streng und mein Vater ist noch einer vom alten Walliser Schlag.\\ So gab es zu Hause dann auch Regeln, die befolgt werden mussten. Mädels hatten diese Arbeit, Männer jene Arbeit.\\
Als rebellischer Jugendlicher habe ich dann auch gegen diese Regeln verstossen. Nicht nur, dass es danach Ärger gab, nein es plagte einen auch das schlechte Gewissen.\\ Ich war nicht wirklich glücklich, wenn ich statt die Arbeit zu machen, mit Freunden unterwegs war. Ich war am glücklichsten, wenn ich alle Regeln befolgt habe. Dann war ich zufrieden mit mir selber und natürlich auch mein Vater.\\
Bei dieser Geschichte kommt mir der Psalm 119 in den Sinn. Da steht in den ersten drei Versen:
\begin{bibelbox}{ELB}{Ps}{119:1-3}
1 Glücklich sind, die im Weg untadelig sind, die im Gesetz des Herrn wandeln\\
2 Glücklich sind, die seine Zeugnisse bewahren, die ihn vom ganzen Herzen suchen\\
3 Die auch kein Unrecht tun, die auf seinen Wegen wandeln.
\end{bibelbox}
Da steht nicht, \grqq Es kommen die in den Himmel, die im Gesetz des Herrn wandeln. \glqq{} Sondern, dass man glücklich ist, wenn man in seinem Gesetz wandelt. In vielen psychologischen Büchern steht drin, dass man mit guten Taten glücklich wird.\\
Wir brauchen keine separaten Bücher. Dafür haben wir die Bibel. Da steht drin, wie wir uns verhalten sollen um ein glückliches Leben zu führen. Viel Geld im Leben alleine, macht uns nicht glücklich. Sonst müssten wir hier alle Glücklich sein. 99\% der Menschen auf der Welt haben weniger Geld als wir.\\
Darum sagt Jesus:
\begin{bibelbox}{ELB}{Lk}{11:28}
Glückselig, die das Wort Gottes hören (lesen) und \colorbox{red}{\textcolor{white}{befolgen}}
\end{bibelbox}
Es gibt noch viele Bibelstellen die hervorheben, wie wichtig das Wort und die Regeln Gottes sind. Darum lest und hört das Wort Gottes und setzt um, was es uns sagt.
Zum Beispiel:
\begin{bibelbox}{ELB}{1Kor}{13:23}
Nun aber bleibt Glaube, Hoffnung, Liebe, diese drei; das Grösste aber von diesen ist die Liebe.
\end{bibelbox}

% \begin{bibelbox}{Sch2}{Kol}{1:18}
% Und er (Christus Jesus) ist das Haupt der Gemeinde...
% \end{bibelbox}

\end{spacing}
Darum singen wir jetzt unser \textbf{2. Lied} \colorbox{yellow}{Du bist mein Ziel mein Gott}.

Nach dem Lied möchte ich Andre bitten nach vorne zu kommen und uns das Wort des Herrn weiter zu geben.

\section{Predigt}
- Die wird von Andre verkündigt.

\textbf{Nach der Predigt}

Danke für die Predigt

Nach dem Beten singen wir unser \textbf{3. Lied}: \colorbox{yellow}{Gnade die Jesus uns zugewandt}.\\

\section{Abendmahl}
- Brotbrechen
\colorbox{yellow}{Blut des Lammes reinigt uns (Strophe 1)}.\\
- Wein
\colorbox{yellow}{Blut des Lammes reinigt uns (Strophe 2)}.\\

\section{Abschluss}
Vielen Dank für eure Teilnahme und das Gebet. Im Anschluss gibt`s noch Kaffee und hoffentlich gute Gespräche.
Ich will noch beten und dann mit dem Segen aus 1. Moses 28.15 abschliessen.
\begin{bibelbox}{SCHL}{1Mos}{28:15}
Gott spricht: Siehe, ich bin mit dir,
ich behüte dich, wohin du auch gehst.
Denn ich verlasse dich nicht,
bis ich vollbringe, was ich dir versprochen habe.
\end{bibelbox}
Maranatha Amen
