
\section{Begrüssung}
Ich möchte euch alle und Philipp herzlich zu diesem Gottesdienst begrüßen. Schön Philipp, dass du diese Reise auf dich genommen hast, um uns das Wort zu bringen. \\Lasst uns beten und dann das erste Lied singen: Halleluja, Halleluja, singt und jubelt.

\textbf{1. Lied}
\textit{Halleluja, Halleluja, singt und jubelt}

\section{Ankündigungen}
\begin{itemize}
    \item Nächster Bibel und Gebetsabend am Donnerstag: 25.7.2024 Römer 9 1-5
    \item Nächster Gottesdienst: 28.7.2024 mit Thomas Lieth
    \item Allgemein: Wer will kann am 18.8. an dem Gemeindefest des MNR Bern teilnehmen. Über diesen Flyer könnt ihr euch anmelden. An diesem Sonntag Nachmittag findet hier wie normal ein Gottesdienst statt.
\end{itemize}

\section{ Input }
\begin{spacing}{1.5}
Janina und ich haben Anfangs Juli an einer Freizeit vom Mitternachtsruf in Szépalma Ungarn teilgenommen. Das Wetter war ziemlich warm und schön. Wir waren diese sommerlichen Temperaturen noch nicht so gewöhnt aber Janina hat sie sehr genossen.

Das Essen und die Zimmer waren sehr gut. Das Hotel liegt abgelegen im Grünen. Neben dem Hotel ist dort auch noch eine Pferdepension mit aktuell gut 100 Pferde.

Wir waren 65 Personen. Täglich hat Norbert Lieth uns eine Stunde lang die Entstehung der Gemeinde in der Apostelgeschichte erklärt. Es war sehr Informativ und mit vielen Aha-Erlebnissen.

Wir haben in der Woche die Apostegschichte 1 - 13 durchgenommen. Er hat uns detailliert erklärt, dass man darauf achten muss, das Reich Gottes und das Reich der Gnade oder Herrlichkeit nicht zu verwechseln oder diese miteinander zu vermischen. Das erste geht um die Wiederkunft Jesus und sein Reich hier auf der Erde (das messianische Reich) und das zweite gilt aktuell für uns, die Jesus Christus von ganzem Herzen angenommen haben und ist auf den Himmel gerichtet.

Viele kleine Details wurden uns gezeigt und die Zusammenhänge erklärt. Ein kleines Beispiel:

In Lukas 13,6-9 gibt es ein Gleichnis von einem Feigenbaum und einem Gärtner. Ich lese es mal kurz vor:
Kurz gesagt, der Besitzer ist Gott, der Gärtner Jesus, der Feigenbaum Israel. Nachdem Jesus 3 1/2 Jahre in Israel gewirkt hatte, fand sich keine Frucht im Volk Israel, das Volk lehnte ihn ab und kreuzigte ihn. Der Gärtner Jesus bittet aber Gott dem Baum noch eine letzte Chance zu geben. Diese Chance wurde Israel nach der Auferstehung nochmals gegeben. Wäre Israel nun umgekehrt, dann hätte Jesus da schon sein messianisches Reich aufgebaut. Nach der Steinigung von Stefanus (Kap. 10) war dann endgültig Schluss. Israel hatte seine letzte Chance vertan. 70 nach Christus wurde Jerusalem und der Tempel zerstört und das Volk getötet, verschleppt oder vertrieben.

Erst ab Kapitel 10 in der Apostelgeschichte kommt dann die Gemeinde für uns Heiden ins Spiel. Das Pfingstfest in Kapitel 2 war rein jüdisch. Einfach Genial, wenn man die Bibel mit Blick auf die verschiedenen Heilszeitalter liesst. Es wird vieles so logisch und einfach.

Der MNR Verlag hat ein super Buch, was aussieht wie ein Kinderbuch, es aber sich hat. Das Buch \glqq Das Navi Gottes \grqq{} beschreibt und zeigt wie die  Geschehnisse in der Bibel in verschieden Zeitalter aufgeteilt werden kann. Vieles wird dadurch einfacher zu verstehen. ZB: wieso gilt uns der Sabat nicht mehr. Wie ist die Bergpredigt einzuordnen. Wie war es bei Abraham, was war da Gottes Plan usw.

Wenn ich es zusammenkriege, möchte ich euch gerne am 18.8 im Gottesdienst eine Zusammenfassung von dem geben, was Norbert uns dort gesagt hat, und euch so einen Überblick vermitteln wie die Gemeinde entstanden ist. Schwerpunkt wird das Ausgiessen des hl. Geistes an Pfingsten in Israel und dann bei der Bekehrung des Hauptmannes Kornelius sein.

% \begin{bibeltext}{Sch2}{Kol}{1:18}
% Und er (Christus Jesus) ist das Haupt der Gemeinde...
% \end{bibeltext}

\end{spacing}{1.5}
So wir singen jetzt unser \textbf{2. Lied} \textit{Herr füll mich neu}

- Oswald nimmt die Schriftlesung vor und betet vor der Predigt

\section{Predigt}
- Die wird von Philipp verkündigt.

\textbf{Nach der Predigt}

Danke für die Predigt

Nach dem Beten singen wir unser \textbf{3. Lied}: \textit{Das höchste meines Lebens}\\

\section{Abschluss}

\begin{bibelbox}{SCHL}{Eph}{6:24}
Die Gnade sei mit allen, die unseren Herrn Jesus Christus lieb haben mit unvergänglicher Liebe. Amen
\end{bibelbox}

Ich wünsche euch einen schönen restlichen Tag und eine gute Woche, bis Donnerstag..
