
\section{Begrüssung}
Ich möchte Nathanel und euch alle herzlich zu diesem Gottesdienst begrüssen.
\\
Lasst uns beten und dann ein Lied zu ehre Gottes singen, \textit{Nur durch Christus in mir}

\section{Ankündigungen}
\begin{itemize}
    \item Nächster Bibel und Gebetsabend am Donnerstag:
    27.06.2024 Römer 8 
    RÖMER 8,28-30    Samuel Rindlisbacher
    \item Nächster Gottesdienst: 30.06.2024 mit Eberhard Hanisch
\end{itemize}

\section{ Input }
Wisst ihr noch früher? Hier im Wallis? In unseren Dörfern? Da gab es gewöhnlich drei Obrigkeiten, bei denen man, wenn sie einem begegneten, den Hut ziehen musste und diese ehrerbietig grüssen. \glqq Grüss Gott, Herr Präsident\grqq{}, \glqq Grüss Gott, Herr Lehrer\glqq{}, \grqq Grüss Gott, Herr Hochwürden\glqq{}.

Hier in Naters  wird bei der Messe, vor dem Eintreten der Priester in die Kirche immer eine kleine Glocke geläutet und die Teilnehmer stehen auf. Weiss nicht, ob das zur Ehre Gottes gemacht wird. Ich jedenfalls hatte das Gefühl, dass es da eher um den Einmarsch der Priester ging.

Der Pfarrer im Dorf hatte früher viel Macht. Er wusste was so seine Schäfchen in seiner Gemeinde treiben und nicht wenige dieser Gottesdiener haben dies dann auch für sich ausgenutzt. Noch jetzt haben katholische Gläubige viel Respekt vor dem Pfarrer, da er ja derjenige ist, der ihnen vielleicht ein paar Jahre Fegefeuer ersparen kann. Mehr als vor Jesus, der der eigentliche Erlöser ist und nur er alleine das ewige Leben geben kann.

Aber auch in der evangelischen Landeskirche: Vor ein paar Jahren war ich in Eggenstein, in Deutschland, in einem Gottesdienst. Ich weiss noch, die Predigt war ein Geschichtsvortrag über Luther und seinen Freund Melanchton. Am Ende, beim Rausgehen, stand der Pfarrer am Ausgang und verabschiedete sich persönlich von den Teilnehmern was ja sehr nett ist. Er hatte dieses, den Priestern eigenes Lächeln und dieses typische Kopfnicken. Für mich eine Demut von oben herab. Wenigstens mir kam es so vor.

Ich kenne aber auch Freikirchen, bei denen der Pastor eine ziemliche Autorität ausstrahlt und es nicht so einfach ist, sich seinen Meinungen zu widersetzen. Auch Mitglieder die schon länger \glqq dabei sind\grqq{} zeigen das gerne. Auch ich und ich glaube jeder von uns ist in Gefahr, sich für etwas Besseres zu halten. \textbf{Ich bin mit Jesus unterwegs.} Das ist doch oft unser Stolz. Das Problem ist nur, dass es nicht unser Verdienst ist. Es ist rein die Gnade Gottes, die uns zu IHM geführt hat. Dies alleine sollte uns in Demut auf die Knie zwingen.

Bei Jesus kennen wir seine Demut, es zeigte sich auch bei der Fusswaschung. Bei den Jüngern war diese Demut nicht vorhanden. Die stritten, wer im Himmel neben Jesus sitzen darf. Was sagt aber Paulus dazu? Der gute Paulus hat ja zu allem etwas zu sagen.

Paulus schreibt an die Philipper 2 1 - 8.

So sollen wir sein: Das in der Demut einer den anderen höher achte und liebe. Wie wollen wir die Liebe Gottes unseren nicht-gläubigen Mitmenschen weitergeben, wenn wir diese nicht untereinander haben. Das ist das, was unsere Mitmenschen bei uns sehen sehen.

Darum singen wir jetzt das Lied \textit{Nimm mein Leben}
Danach nimmt Oswald die Schriftlesung vor zur Predigt vor.

\section{Predigt}
Das Wort wird von Nathanael verkündigt.
\\
\\
\textbf{Nach der Predigt}\\
Nathanel vielen Dank für deine Worte...

Wir singen unser letztes Lied \textit{Du bist mein Ziel mein Gott}

\section{Abschluss}
(Philipper 4 20.23):
\begin{bibelbox}{SCHL}{Phil}{4:20.23}
Unser Gott und Vater aber sei die Herrlichkeit von Ewigkeit zu Ewigkeit!
Die Gnade des Herrn Jesus Christus sei mit eurem Geist!
\end{bibelbox}
Amen! Maranatha, der Herr komme bald!
