
\section{Begrüssung}
Ich möchte euch alle und Nathanel herzlich zu diesem Gottesdienst Begrüßen. Schön das du diese Reise auf dich genommen hast um uns das Wort zu bringen. \\Lasst uns beten und dann das erste Lied, Jesus höchster Name singen.

\textbf{1. Lied}
\textit{Jesus höchster Name}

\section{Ankündigungen}
\begin{itemize}
    \item Nächster Bibel und Gebetsabend am Donnerstag: 
30.5.2024 Römer 8.14 - 18 Eberhard Hanisch
    \item Nächster Gottesdienst: 2.06.2024 mit Abendmahl mit Philip Ottenburg
    \item Allgemein: Am 8. Juni findet in Dübendorf ein Männertag, werde gerne mitkommen will, kann
sich bei Oswald nach dem Gottesdienst melden.
\end{itemize}

\section{ Input }
Eines möchte ich klarstellen, das was ich jetzt sagen möchte, hat nichts damit zu tun, ob wir in den Himmel kommen oder nicht. Den Eintritt in den Himmel, bekommen wir, wenn wir das Geschenk von Jesus, welches er uns durch sein Opfer am Kreuz gegeben hat, von Herzen annehmen. Haben wir das Geschenk angenommen, möchten wir doch gerne aus Dankbarkeit, für das Geschenk so leben, dass es unserem Jesus wohlgefällig ist. Schon die Opfer im AT sollten Gott wohlgefällig dargebracht werden. Unser Leben soll unser Rauchopfer für den Herrn sein, dass ihm wohlgefällig ist.\\
Beim Bibel lesen bin ich auf die Verse Matthäus 15.10 - 15 gestossen. Da steht unter anderem:
(Matthäus 15,10 - 11):
\begin{bibelbox}{ELB}{Matt}{15:10-11}
Und er (Jesus) rief die Volksmenge herbei und sprach zu ihnen: \glqq{}Hört und versteht! Nicht was in den Mund hineingeht, verunreinigt den Menschen, sondern was aus dem Mund herausgeht,verunreinigt den Menschen\grqq
\end{bibelbox}
Bevor Jesus diese Worte sagte, haben ihn die Pharisäer angemacht, weil seine Jünger sich ihre Hände vor dem Essen nicht gewaschen haben, und so mit unreinen Händen gegessen haben. Es ging nicht nur darum, dass jetzt die Speise schmutzig wird, sondern, dass es auch die Seele verschmutzt. Quasi waschen um sein inneres zu Reinigen.\\
Auf diese Zurechtweisung durch die Pharisäer sagte Jesus:
(Matthäus 15,18):
\begin{bibelbox}{ELB}{Matt}{15:18}
\glqq{}Was aber aus dem Mund herausgeht, kommt aus dem Herzen hervor, und das verunreinigt den Menschen. Denn aus dem Herzen kommen hervor böse Gedanken: Mord, Ehebruch, Unzucht, Diebstahl, falsche Zeugnisse, Lästerungen; diese Dinge sind es die den Menschen verunreinigen, aber mit ungewaschenen Händen zu essen, verunreinigt den Menschen nicht.\grqq
\end{bibelbox}
Auch wir waschen unsere Hände vor dem Essen, aber nicht um eine innere Reinheit zu bekommen, sondern um unser Essen nicht zu verschmutzen. Gibt es das aber noch in unseren Kreisen, so etwas wie unsere "Hände waschen" vor allem schauen wir nicht auch gerne auf die anderen ob die ihre Hände waschen? Sind wir gläubigen Christen nicht ein bisschen wie die Pharisäer? Ist es nicht auch bei uns die Gefahr, dass wir auf die Äußerlichkeiten achten? Vor allem bei anderen Menschen und Christen? Wie oft schauen wir auf unsere Mitchristen, ob diese sich auch korrekt verhalten und aufführen? Zb wenn wir andere Gemeinden und Christen beobachten: 
\begin{itemize}
    \item "Wieso machen diese Musik mit Schlagzeug?" 
    \item "Diese Person senkt ja nicht mal sein Haupt beim Beten!"
    \item "Die haben ja Traubensaft statt Wein beim Abendmahl
    \item "Wieso hebt er jetzt die Hände beim singen?"
    \item ...
\end{itemize}
Alles Rituale. Man kann es so machen, muss aber nicht, denn dass verunreinigt uns nicht vor dem Herrn. 
Was uns aber verunreinigt, ist dass was aus inneren unseres Herzen kommt. Jesus gibt uns da eine unvollständige tolle Liste mit.
\begin{itemize}
    \item Mord, 
    \item Ehebruch, 
    \item Unzucht, 
    \item Diebstahl, 
    \item falsche Zeugnisse, 
    \item Lästerungen
\end{itemize}
Also der zweite Teil der 10 Gebote. In dieser Aufzählung geht es darum, wie wir mit anderen Menschen umgehen. In der Bergpredigt hat Jesus darauf Bezug genommen. Prüfen wir uns nächste Woche selber, ob wir eher auf das Hände waschen schauen, oder ob wir auf unser Herz achten und wollen bewusst überprüfen was wir Denken und Reden.

So wir singen jetzt unser \textbf{2. Lied} \textit{Allein deine Gnade genügt}

- Oswald nimmt die Schriftlesung vor und betet vor der Predigt

\section{Predigt}
- Die wird von Nathanael Winkler verkündigt.

\textbf{Nach der Predigt}

Es wäre schön wenn ein Bruder oder Schwester kurz für Susanne, die Frau von Fredy Peter beten könnte. 
Bei ihr wurde Krebs im fortgeschrittenen Stadium diagnostiziert und befindet sich jetzt in der Chemo-Behandlung.

Und ein zweiter Teilnehmer:in für Nathanael und die Prediger vom Mitternachtsruf, die doch jeden Sonntag immer wieder eine lange Reise auf sich nehmen.

Nach dem Beten singen wir unser \textbf{3. Lied}: \textit{Niemand ist so treu und gut wie Jesus}
Wer will kann dann dazu aufstehen

\section{Abschluss}

Der Herr möge Euch für die kommende Woche segnen und beschützen, er möchte Euch durch den Alltag tragen und Euch vor Versuchungen bewahren. Der Herr ist treu und leite uns.

Wünsche einen schönen restlichen Tag und eine gute Woche bis Donnerstag..
