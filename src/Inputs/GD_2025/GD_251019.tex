\author{OTS}
\documentclass{../../inc/mybib}

\setincpath{../../inc/}

\usepackage{bible_style}
\graphicspath{{../../assets/images/}}
\usepackage{header}

% \newenvironment{block}[1][]{%
%   \vspace{1.5em}%
%   \noindent\textbf{#1}\par%
%   \vspace{0.0em}%
% }{%
%   \vspace{1em}%
% }

\begin{document}

\section{Begrüssung}
Ich möchte alle recht herzlich zu unserem heutigen Gottesdienst begrüssen. Schön, dass ihr alle hier seid. Samuel, du bist dieses Jahr ein häufiger Gast und wir freuen uns sehr das du hier bist.

\beten{}

\section{Ankündigungen}
\begin{itemize}
    \item \green{Bibel und Gebetsabend:} Do,  23.10.2025 Bibel und Gebetsabend mit Samuel Rindlisbacher Markus 2,13-17
    \item \green{Nächster Gottesdienst:} So, 26.10.2025 10:00 Uhr Video Vortrag    
    \item Die Kollekte wird hinten in der grünen Trommel gesammelt geht an MNR für den Bau dieser Gemeinde.    
    \item Ich habe eine Liste erstellt in der ihr euch eingtragen könnt. Es sollte eine Mitglieder Liste werden, so dass wir uns besser kontaktieren können. Es ist Freiwillig. Wer sich eintragen will kann das tun. Die Adressen und Telefonnummern werden vertraulich behandelt und nicht an dritte weiter gegeben.
\end{itemize}

Singen wir nun zusammen unser erstes Lied.
% \noindent
\lied{Die dem Herrn vertrauen}.


\section{ Input }
\begin{spacing}{1.5}    
    \begin{block}[Einleitung]
        Machen wir weiter mit den Verschiedenen Phasen in der Bibel die zum Reich Gottes führen. Diese Einteilung ist von Dr. Vlach und wir wollen uns heute die Phase 3, den Sündenfall anschauen.
    \end{block}

    \begin{enumerate}
    \item Schöpfung \bibleverse{IMos} (1-2:)
    \item Sündenfall \bibleverse{IMos} (3:)
    \item Verheissung 1. Mose 3,15 - Maleachi
    \item Ankunft des Königs Evangelien und Briefe
    \item Wiederherstellung Offenbarung
   \end{enumerate}        
    \begin{block}[Schöpfung Kapitel 3]
       Vor zwei Wochen haben wir zusammen gelesen, dass Gott den beiden Einwohner ein Paradies geschenkt hat, das sie verwalten sollen. Ausserdem hat er ihnen ein einiges Gebot gegeben haben.
       \begin{bibelbox}{SCHL}{1Mos}{2:16-17}
            Und Gott der Herr gebot dem Menschen und sprach: \flqq Von jedem Baum des Gartens darfst du nach Belieben essen; aber von dem Baum der Erkenntnis des Guten und des Bösen sollst du nicht essen; denn an dem Tag, da du davon isst, musst du gewisslich sterben!\frqq
        \end{bibelbox}
        Anschliessend haben wir zusammen rausgefunden, dass wir \betonung{alle} hier früher oder später sterben müssen.
        Das ist \betonung{der Beweis}, dass das Gebot von Gott nicht eingehalten wurde. Wie das passiert ist, könnt ihr in der Bibel in \bibleverse{IMos} (3:1-6) nachlesen. Und was ist passiert?
        \begin{bibelbox}{ELB}{IMos}{3:7}
            Da wurden ihre beiden Augen aufgetan, und sie erkannten, dass sie nackt waren; und sie hefteten Feigenblätter zusammen und machten sich Schurze.
        \end{bibelbox}
        Was hat Gott nun gemacht? Einfach die ganze Aktion Mensch und Erde abgeblasen?
       
        \begin{bibelbox}{ELB}{IMos}{1:31}
            Und Gott sah \betonung{alles}, was er gemacht hatte, und siehe, es war sehr gut. \leise{Und es wurde Abend, und es wurde Morgen: der sechste Tag}
        \end{bibelbox}
        \betonung{Dann ist auch alles sehr gut.} Gott weiss genau was die eben erschaffenen Menschen im Schilde führen. Gott ist in keiner Zeit. Für Gott ist alles Gegenwart. Keine Ahnung wie das funktioniert, aber er hat ja die Zeit geschaffen. Also gab es vor der Erschaffung keine Zeit. Trotzdem ist für unseren allmächtigen Gott alles sehr gut.

        Wir sind ja hier alle schon ein bisschen älter. Kennt ihr diesen Trickfilm La Linea? Wieso hat Gott es mit dem Menschen nicht so gemacht wie der Zeichner bei diesem Trickfilm? Einfach drüberwischen und alles wieder löschen.
        \begin{itemize}
            \item Gott erschafft Himmel und Erde mit dem Menschen
            \item Der Mensch sündigt
            \item Gott hat einen Plan
        \end{itemize}
        
        Diesen Plan Gottes lesen wir in 1Mos 3:15. 
        \begin{bibelbox}{SCHL}{1Mos}{3:15}
            Und ich werde Feindschaft setzen zwischen dir und der Frau, zwischen deinem Nachwuchs und ihrem Nachwuchs; er wird dir den Kopf zermalmen, und du, du wirst ihm die Ferse zermalmen.
        \end{bibelbox}
        In diesem Vers wird der ganze Plan Gottes aufgezeigt. Er gibt uns nicht auf. Er hat uns geschaffen um mit uns Gemeinschaft zu haben. Wenn Gott etwas macht, dann bringt er es auch zu Ende. Aber, Gott macht es so wie er will und nicht wie wir uns das Denken.

        Über diesen Vers sind ganze Bücher geschrieben worden. Eva wusste was Gott mit diesen Worten meinte, denn sie war dauernd in Erwartung auf diesen Erlöser, der, welcher der Schlange den Kopf zermalmt. 
    \end{block} 
    \begin{block}[ABER]
        Der Mensch musst das Paradies verlassen und unter Mühsal und Schweiss sein Unterhalt verdienen. Die Frau kann nur unter schmerzen gebären. Ich weiss nicht genau, aber ich habe viele Kühe kalbern sehen. Bei denen wo es keine komplikationen gab, sah das nicht aus als ob die Kühe schmerzen hatten. Sie frassen sogar wären dem Pressen noch Heu. Wenn das Kalb am Boden lagen, drehten sie sich kauend zu ihm hin. Oder Samuel? Hast du das auch so erlebt?
        
        Gott hat ihr Umfeld verändert, aber nicht den Menschen. Der Mensch darf immer noch frei und selbst bestimmen und entscheiden. \betonung{Gott will Menschen, die Freiwillig zu ihm kommen.} Aber die ganze Schöpfung ist dem Untergang geweiht. Es geht unaufhörlich Berg ab mit dieser Schöpfung. Nichts hat Bestand. Auch wenn uns die Evolutionstheoretiker erklären wollen, dass sich alles weiter entwickelt, sehen wir doch allzu gut, dass auf dieser Welt das Gegenteil passiert. Letzthin habe ich ein Foto gesehen, wo erwachsenen Männer mit Schnuller und Windeln in einem Umzug mit liefen. Die Krönung der Schöpfung.

       So lesen wir in:
        \begin{bibelbox}{ELB}{Rom}{8:20}
            Denn die Schöpfung ist der Nichtigkeit unterworfen worden - nicht freiwillg, sondern durch den, der sie unterworfen hat -- auf Hoffnung hin.
        \end{bibelbox}
        Oder:
        \begin{bibelbox}{NUE}{Rom}{8:20}
            Denn alles Geschaffene ist der Vergänglichkeit ausgeliefert -- undfreiwillig. Gott hat es so verfügt.
        \end{bibelbox}
        Und wie das aussieht lesen wir in:
        \begin{bibelbox}{ELB}{Rom}{8:22}
            Denn wir \betonung{wissen}, dass die \betonung{ganze} Schöpfung zusammen seufzt und zusammen in Geburtswehen liegt bis jetzt.
        \end{bibelbox}
    \end{block}
    \begin{block}
        Gott hat uns Menschen nicht aufgeben. Nein. Er führt seinen Plan mit uns zum Ende. Heute wissen wir wie er endet. Damals wussten Adam und Eva das noch. Sie lebten knapp tausend Jahre. Wie oft haben sie sich gegenseitig wohl Vorwürfe gemacht?
    \end{block}
    \begin{block}
        Nächstes Mal schauen zu sammen wie es weiter geht. Wir kommen dann zu Phase 3. Diese Phase ist eine lange Zeit und dauert bis zum ersten kommen von Jesus Christus unserem Retter. Und was in dieser Zeit Gott alles mit den Menschen angestellt, hat wird uns dann wohl den Rest des Jahres beschäftigen.
    \end{block}    
    
\end{spacing}
\lied{Ich traue auf dich}.

\section{Predigt}
Nach dem Lied möchte ich Samuel nach vorne Bitten.
%  \textbf{Nach der Predigt}
%  Danken für die Predigt.

\section{Abschluss}
Vielen Dank Samuel für die Predigt.

\lied{Denn ich bin gewiss}.

\beten{}

\begin{bibelbox}{SCHL}{1Mos}{28:15}
    Gott spricht: Siehe, ich bin mit dir,
    ich behüte dich, wohin du auch gehst.
    Denn ich verlasse dich nicht,
    bis ich vollbringe, was ich dir versprochen habe.
\end{bibelbox}

Maranatha Amen

\end{document}
