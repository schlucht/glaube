\author{OTS}
\documentclass{../../inc/mybib}

\setincpath{../../inc/}

\usepackage{bible_style}
\graphicspath{{../../assets/images/}}
\usepackage{header}

% \newenvironment{block}[1][]{%
%   \vspace{1.5em}%
%   \noindent\textbf{#1}\par%
%   \vspace{0.0em}%
% }{%
%   \vspace{1em}%
% }

\begin{document}

\section{Begrüssung}
Ich möchte alle recht herzlich zu unserem heutigen Gottesdienst begrüssen. Schön, dass ihr alle hier seid. Samuel, auch dich will ich recht herzlich Begrüssen. 

\beten{}

\section{Ankündigungen}
\begin{itemize}
    \item \green{Bibel und Gebetsabend:} Do,  23.10.2025 Bibel und Gebetsabend mit Thomas Lieth Mk 2:
    \item \green{Nächster Gottesdienst:} So, 26.10.2025 10:00 Uhr Video Vortrag    
    \item Die Kollekte wird hinten in der grünen Trommel gesammelt geht an MNR für den Bau dieser Gemeinde.    
\end{itemize}

Singen wir nun zusammen unser erstes Lied.
% \noindent
\lied{Danke für diesen guten Morgen}.

\section{ Input }
\begin{spacing}{1.5}    
    \begin{block}[Einleitung]
        Machen wir weiter mit den Verschiedenen Phasen in der Bibel die zum Reich Gottes führen. Diese Einteilung ist von Dr. Vlach und wir wollen uns heute die Phase 3 der Sündenfall anschauen.
    \end{block}

    \begin{enumerate}
    \item Schöpfung \bibleverse{IMos} (1-2:)
    \item Sündenfall \bibleverse{IMos} (3:)
    \item Verheissung 1. Mose 3,15 - Maleachi
    \item Ankunft des Königs Evangelien und Briefe
    \item Wiederherstellung Offenbarung
   \end{enumerate}        
    \begin{block}[Schöpfung Kapitel 3]
       Vor zwei Wochen haben wir zusammen gelesen, dass Gott den beiden Einwohner im Paradies ein Gebot gegeben haben.
       \begin{bibelbox}{SCHL}{1Mos}{2:16-17}
            Und Gott der Herr gebot dem Menschen und sprach: \flqq Von jedem Baum des Gartens darfst du nach Belieben essen; aber von dem Baum der Erkenntnis des Guten und des Bösen sollst du nicht essen; denn an dem Tag, da du davon isst, musst du gewisslich sterben!\frqq
        \end{bibelbox}
        Anschliessend haben wir rausgefunden, dass wir alle hier früher oder später sterben werden. Also der Beweis, dass die Frucht von dem verbotenen Baum gegessen wurde. Was hat Gott nun gemacht? Einfach die ganze Aktion Mensch und Erde abgeblasen? Wir sind ja hier alle schon ein bisschen älter. Kennt ihr diesen Trickfilm La Linea? Wieso hat Gott es mit dem Menschen nicht so gemacht wie der Zeichner bei diesem Trickfilm? Einfach drüberwischen und alles wieder löschen.

        Nein Gott hatte einen anderen Plan. Er hat in Vers ???? geschrieben, alles ist sehr gut. Dann ist es auch alles sehr gut. Den Plan Gottes lesen wir in 1Mos 3:15. In diesem Vers wird der ganze Plan Gottes aufgezeigt. Er gibt uns nicht auf. Er hat uns geschaffen um mit uns Gemeinschaft zu haben. Wenn etwas macht, dann bringt er es auch zu Ende. Das ist diese suveränität die mich überzeugt. Gott macht das nie so wie wir Menschen uns vorstellen. 

        Du baust dir eine Firma auf und stellst deinen ersten Arbeiter an. Du sagst ihm du darfst alle Geräte benutzen, ausser das Gerät da hinten. Wenn du das einschaltest dann geht die ganze Firma bachab. Am anderen Morgen kommst du nach Hause und erkennst, dass die die Maschine eingeschaltet wurde. Wie reagierst du? Feuerst du deinen neuen Arbeiter oder sagst du: "Ok, dann wollen wir nun gemeinsam mit diesem Schaden vorwärts gehen." Gott hat das zweite gemacht. Gott macht mit uns weiter. Er hat seinen Plan.         
    \end{block} 
    \begin{block}
        Der Mensch musste das Paradies verlassen und unter Mühsal und Schweiss sich das Essen erarbeiten. Die Frau kann nur unter Schmerzen Kinder bekommen. Gott hat das Umfeld verändert, aber nicht den Menschen. Der Mensch kann immer noch frei selbst bestimmen und entscheiden. Gott will Menschen die Freiwillig zu ihm kommen.
    \end{block}
    \begin{block}
        Adam und Eva sind jetzt nun in unserer bekannten Welt und mussten mit allen Problemen nun umgehen und leben. Wie geht es nun weiter? Welchen Plan hat Gott mit uns Menschen? Gibt es eine Zukunft für den Menschen? Wie es weiter geht werden wir in 2 Wochen erfahren wenn wir zur Phase 4 kommen.
    \end{block}
    \begin{bibelbox}{SCHL}{1Mos}{1:28}
        Und Gott segnete sie; und Gott sprach zu ihnen: \enquote{Seid fruchtbar und mehrt euch und füllt die Erde und macht euch sie untertan; und herrscht über die Fische im Meer und über die Vögel des Himmels und über alles Lebendige, das sich regt auf der Erde.}
    \end{bibelbox}  
\end{spacing}
\lied{Lobpreiset unseren Herrn}.

\section{Predigt}
Nach dem Lied möchte ich Obed nach vorne Bitten.
%  \textbf{Nach der Predigt}
%  Danken für die Predigt.

\section{Abschluss}
Vielen Dank Obed für die Predigt.

\lied{Gut das wir einnander haben}.

\beten{}

\begin{bibelbox}{SCHL}{1Mos}{28:15}
    Gott spricht: Siehe, ich bin mit dir,
    ich behüte dich, wohin du auch gehst.
    Denn ich verlasse dich nicht,
    bis ich vollbringe, was ich dir versprochen habe.
\end{bibelbox}

Maranatha Amen

\end{document}
