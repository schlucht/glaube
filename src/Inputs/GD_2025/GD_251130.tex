\author{OTS}
\documentclass{../../inc/mybib}

\setincpath{../../inc/}

\usepackage{bible_style}
\graphicspath{{../../assets/images/}}
\usepackage{header}

% ensure scrlayer-scrpage has sufficient footheight
\setlength{\footheight}{20.4pt}

\begin{document}

\section{Begrüssung}
Hallo Nathanael wir freuen uns, dich in Naters begrüssen zu können. Wir sind gespannt auf dein Wort.

Ich möchte auch euch alle recht herzlich zu diesem Gottesdienst begrüssen. Schön, dass ihr gekommen seid, um unserem \herr N zu Danken, ihn zu loben und zu preisen.

% \noindent
\beten{} Und anschliessend singen wir zusammen das Lied

% \noindent
\lied{Gross ist dein Name}.

\section{Ankündigungen}
\begin{itemize}
    \item \green{Bibel und Gebetsabend:} Do, 4.12.2025 20:00 Uhr Bibel und Gebetsabend .
    \item \green{Nächster Gottesdienst:} So, 07.12.2025 14:45 Uhr hier mit Nathanael Winkler.
    
    \item \green{Die Kollekte:} Wir haben jetzt alles gekauft, was wir brauchen und ich finde, es sieht super aus. Die Kosten belaufen sich auf 2828.50 Fr. Ein Spende von 500.- ist schon eingegangen, so bleibt noch ein Betrag von 2328.50. Ich lese hier jetzt nicht die Liste vor. Bei Interesse könnt ihr euch bei Janina die Liste anschauen und bei Bedarf die Quittungen kontrollieren. 
    
    Wer sich an den Kosten beteiligen möchte, kann das Geld entweder an Janina per Twint schicken oder heute und die nächsten zwei Sonntage hinten in die grüne Tonne tun. Alle Spenden werden anonym behandelt.
    
    Alle Möbel und Inventar gehören der Gremeinde. Wenn also jemand die Gemeinde verlässt, kann er/sie die Möbel nicht mehr mitnehmen.
\end{itemize}

\section{ Input }
\begin{spacing}{1.5}
    \begin{block}[Einführung]
        Die Phase 3 ist eine sehr lange Zeit. Sie beginnt, nachdem Adam und Eva aus dem Paradies vertrieben wurden. Von Adam und Eva aus wurde die ganze Erde bevölkert. Die Zeit, die der Mensch hatte, ist von 4119 bis 2463 v. Chr datiert, also 1656 Jahre. Die Menschen wurden in der Zeit immer gottloser. Dann hatte Gott genug und es kam zur Sintflut. Wir dürfen nicht vergessen, dass die Menschen zu der Zeit wissen konnten, dass es einen Schöpfergott gibt. Dadurch, dass die Menschen so alt wurden, konnten sie von Generation zu Generation das Wissen untereinnander weiter geben. In der Bibel werden zwei verschiedene Ahnentafeln aufgezeigt. Eine von Kain und die andere von Sem.

        In der Linie Kains werden Menschen beschrieben, die Erbauer einer Stadt waren, Erfinder der Musik, Eisenherstellung und die Vielehe erfunden haben. 

        Wenn wir zur Linie von Seth kommen, heisste es in Vers 26
        \begin{bibelbox}{SCHL}{1Mos}{4:26}
            Und dem Seth, auch ihm wurde ein Sohn geboren, und er gab ihm den Namen Enosch. Danach fing man an den Namen des \herr N anzurufen.
        \end{bibelbox}
        Wir sehen also, in der Linie Kains alles Weltliche beschrieben wird und bei Seth kommt der Gottesdienst mit ins Spiel. Dass die Menschen in dieser Zeit Gott verlassen haben, können wir ab Kapitel 6 lesen. Eine detailierte Genealogie von Adam bis Noah finden wir im Kapitel 5. \numcirc{1}
        Wie ihr auf der Folie seht, haben sich viele Generationen überlappt, denn es sind alle Menschen alt geworden und nicht nur diese, die hier aufgeführt werden. So wurde das Wissen immer weiter gegeben bzw. hätte weitergegeben werden können.

        Ist es heutzutage anders?
        \begin{bibelbox}{SCHL}{Lk}{17:26-27}
            Und wie es in den Tagen Noahs zuging, so wird es auch sein in den Tagen des Menschensohnes: Sie assen, sie tranken, sie heirateten und liessen sich heiraten bis zu dem Tag, als Noah in die Arche ging; und die Sintflut kam und vernichtete alle.
        \end{bibelbox}
        Wir sehen es doch in dieser Welt. Der Islam ist auf dem Vormarsch, Gott wird geleugnet und auf die Seite gestellt -- auch oder vor allem von Christen. Kinder werden in Massen getötet bevor diese überhaupt auf die Welt kommen dürfen. Das Konstrukt Familie wird hinterfragt. Es geht nur noch um die gesetzlichen Richtlinien. Neu ist, dass man nicht mehr wissen kann, ob man ein Mann oder eine Frau ist. Wohin das führt?
        \begin{bibelbox}{SCHL}{Mat}{24:39}
            \dots und nichts merkten, bis die Sintflut kam und sie alle dahinraffte, so wird auch die Wiederkunft des Menschensohnes sein.
        \end{bibelbox}

        Die Geschichte von der Sintflut kennen wir alle. Kommen wir nun dazu, was danach geschah.

        Nachdem die Arche gestrandet ist, hat Noah einen Altar gebaut und Gott gedankt und gepriesen. Gott sieht das und sprach in seinem Herzen:
        \begin{bibelbox}{SCHL}{1Mos}{8:21-22}
            Und der  \herr{} roch den lieblichen Geruch, und der \herr{} sprach in seinem Herzen: Ich will künftig den Erdboden nicht mehr verfluchen, um des Menschen willen, \betonung{obwohl das Trachten des menschlichen Herzens böse ist von seiner Jugend an;} auch will ich künftig nicht mehr alles Lebendige schlagen, wie ich es getan habe. Von nun an soll nicht aufhören Saat und Ernte, Frost und Hitze, Sommer und Winter, Tag und Nacht, solange die Erde besteht.
        \end{bibelbox}
        Das ist ein Versprechen von Gott und Gott hält seine Versprechen. Dieses Versprechen gilt für uns alle.
        Es ist aber noch nicht alles. Im Kapitel 9, Vers 1-2 gibt Gott uns nochmals den Auftrag die Erde zu bevölkern und zu beherrschen:
        \begin{bibelbox}{SCHL}{1Mos}{9:1-2}
            Und Gott segnete Noah und seine Söhne und sprach zu ihnen: Seid fruchtbar und mehrt euch und erfüllt die Erde! Furcht und Schrecken vor euch soll über alle Tiere der Erde kommen und über alle Vögel des Himmels, über alles, was sich regt auf dem Erdboden, und über alle Fische im Meer; in eure Hand sind sie gegeben.
        \end{bibelbox}
        Zuerst hat er die Tiere auf den Bösen Menschen vorbereitet, dann hat er uns erlaubt von diesen Tieren zu essen.
        \begin{bibelbox}{SCHL}{1Mos}{9:3}
            Alles, was sich regt und lebt, soll euch zur Nahrung dienen; wie das grüne Kraut habe ich euch alles gegeben.
        \end{bibelbox}
        Gott macht mit Noah und seinen Söhnen einen Bund. Dieser Bund gil ab da für alle Menschen. Dieser Bund ist den Versen 8-17 zu entnehmen.
        \bibleverse{1Mos}(9:8-17).

        Ist euch etwas aufgefallen? Gott sagt:
        \begin{bibelbox}{SCHL}{1Mos}{9:7}
            Ihr aber, seid fruchtbar und mehrt euch und breitet euch aus auf der Erde, dass ihr zahlreich werdet darauf!
        \end{bibelbox}
        Im Bund steht nirgends, dass Gott uns seine Verheissungen nur gibt, wenn wir uns brav vermehren und uns brav auf der Erde verhalten. Nein, dieser Bund ist an keine Bedingungen geknüpft und gilt für alle und ewig. Jedes Mal, wenn wir einen Regenbogen sehen, dürfen wir uns an diesen Bund erinnern und unserem \herr N danken.
    \end{block}
    \begin{block}
        Wie ist es mit uns hier? Wir sind eine kleine Gruppe Christen. Manchmal fragen wir uns, was Gott mit uns plant. Hat das überhaupt eine Zukunft hier? Was muss ich machen, dass mehr Menschen kommen? Wieso sehen denn die Menschen nicht, wie toll es hier ist?
        
        Überlassen wir es Gott wohin die Leute gehen sollen. Es ist egal, solange Jesus drin steht. Das Gleiche hat auch Paulus erlebt und seine Antwort darauf war:
        \begin{bibelbox}{SCHL}{Phil}{1:18}
            Was tut es? Jedenfalls wird auf alle Weise, sei es zum Vorwand oder in Wahrheit, Christus verkündigt, und darüber freue ich mich, ja, ich werde mich auch weiterhin freuen!
        \end{bibelbox}        
        Wenn ich diesen Bund mit Noah lese, habe ich die Zuversicht, dass Gott uns führt und leitet, wie er es auch mit Noah getan hat. Noah hat 150 Jahre lang an einem Schiff gebaut, ohne zu wissen wieso. Wir müssen nicht wissen, was mit dieser Gemeinde hier passiert. Wir müssen nur vertrauen, \betonung{dass Gott weiss}, was er mit dieser Gemeinde tut und vorhat. Das entspannt doch alles, oder? Lasst uns doch einfach miteinander Gott loben und preisen, ihn anbeten und ihm dienen. Lasst uns daran freuen, was wir hier zusammen aufgebaut haben. Lasst uns gegenseitig unsere Freundschaft pflegen und die Zeit geniessen mit dem Herrn. Ich für mich persönlich sehe mich von Gott hier eingesetzt. Ich plane nicht weit voraus, ich gucke immer von Sonntag zu Sonntag und freue mich am Sonntag, euch hier zu treffen und gemeinsam mit euch, mit immer besserer Musik, Gott zu danken, was er in der ganzen Woche mit mir und für mich gemacht hat.
    \end{block}    
    Singen wir zusammen ein Lied und nach dem Lied möchte ich Thomas das Wort übergeben.
   
\end{spacing}
\lied{Du bist Emmanuel}
\section{Predigt}



% \textbf{Nach der Predigt}

% Danken für die Predigt.

% \section{Abendmahl}

% Beten für das Brot

% \lied{Das Blut der Lammes 1. Strophe}

% Beten für den Wein

% \lied{Das Blut der Lammes 2. Strophe}

\section{Abschluss}

Jetzt wollen wir Gott mit dem Lied \lied{Denn ich bin gewiss} danken.

Vielen Dank für eure Teilnahme am Gottesdienst. Im Anschluss seid ihr zu Kaffee und guten Gesprächen eingeladen.
\beten{}

\begin{bibelbox}{SCHL}{1Mos}{28:15}
Gott spricht: Siehe, ich bin mit dir,
ich behüte dich, wohin du auch gehst.
Denn ich verlasse dich nicht,
bis ich vollbringe, was ich dir versprochen habe.
\end{bibelbox}

Maranatha Amen
\end{document}