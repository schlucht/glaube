\author{OTS}
\documentclass{../../inc/mybib}

\setincpath{../../inc/}

\usepackage{bible_style}
\graphicspath{{../../assets/images/}}
\usepackage{header}

% ensure scrlayer-scrpage has sufficient footheight
\setlength{\footheight}{20.4pt}

\begin{document}

\section{Begrüssung}
Hallo Thomas wir freuen uns, in Naters begrüssen zu können. Wir sind gespannt auf dein Wort und freuen uns darauf.

Ich möchte euch alle recht herzlich zu diesem Gottesdienst begrüssen. Schön das ihr gekommen seid, um unserem Herrn zu Danken ihn zu Loben und zu Preisen.

% \noindent
\beten{} Und anschliessend singen wir zusammen das Lied

% \noindent
\lied{Nur durch Christus in mir}.

\section{Ankündigungen}
\begin{itemize}
    \item \green{Bibel und Gebetsabend:} Do, 4.12.2025 20:00 Uhr Bibel und Gebetsabend .
    \item \green{Nächster Gottesdienst:} So, 09.11.2025 14:45 Uhr hier mit Fredy Peter.
    \item \green{Allgemein:} Janina und ich sind ab Dienstag zwei Wochen unterwegs in Israel. Gottesdienst und Gebetsabende laufen normal weiter. Es geht ja auch nicht um Janina und mir, sondern um unseren unvergleichlichen grossen Gott. 
    \item \green{Die Kollekte:} Wir haben jetzt alles gekauft was wir brauchen. Der Preis der Möbel beläuft sich auf . Wer Intresse kann sich bei Janina Melden und kann die Quittiungen anschauen. Wer sich an den Kosten beteiligen möchte kann das Geld entweder an Janina per PayPal schicken oder heute und die nächsten zwei Sonntagen hinten die grüne Tonne. Alles Spenden werden anonym gehandelt.
\end{itemize}

\section{ Input }
\begin{spacing}{1.5}
    \begin{block}[Einführung]
        Die Phase 3 ist eine sehr lange Zeit. Sie beginnt nachdem Adam und Eva aus dem Paradies vertrieben wurden. Von Adam und Eva aus, wurde die ganze Erde bevölkert. Die Zeit bis zur Sindflut war genau \red{Dauer bis Sintflut}. Dann hatte Gott genug und es kam zur Sintflut. Die Menschen wurden immer gottloser. Wir dürfen nicht vergessen, dass wie heute auch schon zu der alles wissen konnten, dass es einen Gott gibt. Dadurch, dass die Menschen so alt wurden, konnten sie von Generation zu Generation das Wissen untereinnander weiter geben. In der Bibel werden zwei verschiedene Ahnentafeln aufgezeigt. Eine von Kain und die andere von Sem.

        In der Linie Kain werden Menschen beschrieben, die Erbauer einer Stadt waren, Erfinder der Musik, Eisenherstellung und die Vielehe. Lamech hatte sich zwei Frauen genommen.

        Wenn wir zu Sem kommen, heisste es in Vers 26
        \begin{bibelbox}{SCHL}{1Mos}{4:26}
            Und dem Set auch ihm wurde ein Sohn geboren, und er gab ihm den Namen Enosch. Danach fing man an den Namen des \herr N anzurufen.
        \end{bibelbox}
        Wir sehen in der Linie Kain wird alles weltliche beschrieben und bei Seim kommt der Gottesdienst ins spiel. Das die Menschen mit der Zeit Gott verlassen haben können wir ja ab Kapitel 6 lesen. Eine detailierte Genealogie von Adam bis Noah finden wir in Kapitel 5. \numcirc{1}
        Wie ihr auf der Folie seht, haben sich viele Generationen überlappt und es sind sicher alle Menschen alt geworden und nicht nur diese, die hier aufgeführt werden.

        Es ist heute nicht anders als zu der Zeit. Darum heisst es in 
        \begin{bibelbox}{SCHL}{Mat}{24:}
            Sie werden ??? wir zur Zeit Noah...
        \end{bibelbox}
        Wir sehen es schön in dieser Welt. Der Islam ist auf dem Vormarsch, Gott wird geleugnet und auf die Seite gestellt. Kinder werden im Massen getötet bevor diese überhaupt auf die Welt kommen. Das Konstrukt Familie ist am bröckeln. Das neueste ist, dass man nicht mehr wissen kann ob man jetzt ein Mann oder eine Frau ist. Ich glaube darum sollte man die Texte gendern, so dass auch Männer die sich als Frauen fühlen angesprochen werden.

        Die Sintflut hat alles was auf dem Land und in der Luft lebte getötet. Die einzigen Überlebenen befanden sich auf ein Nussschale Namens Arche. Die Geschichte kennt ihr alle.

        Nachdem die Arche gestrandet ist, hat Noah einen Altar gebaut und Gott gedankt. Gott sieht das und sprach in seinem Herzen:
        \begin{bibelbox}{SCHL}{1Mos}{8:21-22}
            
        \end{bibelbox}
    \end{block}
    \numcirc{2}
    \begin{block}[Noah Bund]
        
    \end{block}
   
\end{spacing}
\lied{Du bist Emmanuel}
\section{Predigt}

Danach gebe ich das Wort an Nathanel weiter.

% \textbf{Nach der Predigt}

% Danken für die Predigt.

\section{Abendmahl}

Beten für das Brot

\lied{Das Blut der Lammes 1. Strophe}

Beten für den Wein

\lied{Das Blut der Lammes 2. Strophe}

\section{Abschluss}

Jetzt wollen wir Gott mit dem Lied \lied{Denn ich bin gewiss} danken.

Vielen Dank für eure Teilnahme am Gottesdienst. Im Anschluss seid ihr zu Kaffee und guten Gesprächen eingeladen.
\beten{}

\begin{bibelbox}{SCHL}{1Mos}{28:15}
Gott spricht: Siehe, ich bin mit dir,
ich behüte dich, wohin du auch gehst.
Denn ich verlasse dich nicht,
bis ich vollbringe, was ich dir versprochen habe.
\end{bibelbox}

Maranatha Amen
\end{document}