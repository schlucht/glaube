\author{OTS}
\documentclass{../../inc/mybib}

\setincpath{../../inc/}

\usepackage{bible_style}
\graphicspath{{../../assets/images/}}
\usepackage{header}

\begin{document}


\section{Begrüssung}

Ich möchte euch alle recht herzlich zu diesem Gottesdienst begrüssen. Schön das ihr gekommen seid, um unserem Herrn zu Danken ihn zu Loben und zu Preisen.
Hallo Hendrik wir freue uns, dich hier im Oberwallis in Naters begrüssen zu können. Wir sind gespannt auf dein Wort und freuen uns darauf.

% \noindent
\beten{} und anschliessend singen wir zusammen das Lied

% \noindent
\lied{Wir singen von Jesus}.

\section{Ankündigungen}
\begin{itemize}
    \item \green{Bibel und Gebetsabend:} Do 24.07.2025 20:00 Uhr Bibel und Gebetsabend mit Elia Morris  und legt uns Römerbrief 16 21-24 aus.
    \item \green{Nächster Gottesdienst:} So 27.07.2025 14:45 Uhr hier mit Samule Rindlisbacher.
    \item \green{Allgemein:} Am Sonntag 17.August findet der erste Gemeindetag statt. Der Gottesdienst wird mit Fredy Peter um 10 Uhr statt finden. Danach wollen wir uns zu einem gemütlichen Nachmittag mit Grill zusammensetzen. Genaue Details gebe ich nächsten Sonntag bekannt.
    \item \green{Die Kollekte:} Die Kollekte geht an den MNR und wird für den Bau dieser Gemeinde eingesetzt.
\end{itemize}

\section{ Input }
\begin{spacing}{1.5}
Vier Wochen waren Janina und ich auf der Alp Ärnergalen auf 2400müM. Mit uns da oben waren 2 Hütehunde und 120 Schafe. Unsere Arbeit war unter anderem, die Schafe jeden Abend zusammen zu treiben und die Hunde zu füttern. Dort oben ist es sehr ruhig und die Bergwelt ist grandios. Oft habe ich dort auf einem Stein gesessen, die Natur genossen und nachgedacht. Immer wieder kamen mir die Worte von Salomon in Prediger 1 in den Sinn.
\begin{bibelbox}{SCHL}{Pred}{1:2-3}
   O Nichtigkeit der Nichtigkeiten!, spricht der Prediger. O Nichtigkeit der Nichtigkeiten! Alles nichtig!

   Was bleibt dem Menschen von all seiner Mühe, womit er sich abmüht unter der Sonne!
\end{bibelbox}
Da oben abseits aller Hektik, Trubel und Weltgeschehen, sieht man wie klein und nichtig wir Menschen sind. Im Grunde sind wir machtlos gegen das Geschehen dieser Welt. Oft versuchen wir die Welt zu verbessern. Vor allem wir Christen denken oft, dass wir die besseren Menschen sind. Oft vergessen wir, dass wir nicht die Guten sind, sondern durch Christus Gut gemacht wurden. Aus reiner Gnade. 

Wenn du so zwischen diesen mächtigen Bergen sitzt und dir bewusst wird, wer das alles geschaffen hat und wer das alles wieder vernichtet, kann man und sollte man sich klein und mickrig vorkommen. Demut ist dort oben angesagt. Nichts können wir Menschen ohne das mit wirken Gottes machen. Gott stellt Hiob in den Kapitel 38 und 39 ein paar Fragen. Die einzige Antwort die Hiob auf diese Fragen hatte, war:
\begin{bibelbox}{SCHL}{Hi}{40:4-5}
    Siehe, ich bin zu gering; was soll ich dir erwidern? Ich will meine Hand auf meinen Mund legen!

    Ich habe einmal geantwortet, und noch ein zweites Mal, und ich will es nicht mehr tun!
\end{bibelbox}
    Trotz des Tadels von Gott an Hiob, hat Gott Hiob das nicht angerechnet, sondern ihn wieder hergestellt.
\begin{bibelbox}{SCHL}{Hi}{42:10}
    Und der \herr wendete Hiobs Geschick, als er für seine Freunde bat; und der \herr erstattet Hiob alles doppelt wieder, was er gehabt hatte.
\end{bibelbox}
Wir alle hatten vor unserer Bekehrung und Wiedergurt ein riesengrosses Paket an Sünden auf dem Rücken geschleppt. Durch die Wiedergeburt hat Jesus dieses Paket uns abgenommen. Wir wurden von ihm weiss gewaschen. Und wisst ihr was das gute ist? Das Paket wurde ins tiefste Meer versenkt. Unwiederbringlich. Nie mehr hält er uns diese Sünden vor.

Und wir? Wenn Jesus, der absolut sündlos war, uns alle Sünden wegnimmt, ohne diese uns wieder unter die Nase zu halten, sollten wir sündigen Menschen nicht erst recht gegenüber unseren Geschwistern so handeln? Machen wir nicht alle Fehler? Wer sind wir denn, dass wir den anderen verurteilen. Sollten wir nicht unsere Geschwister auf die Fehler aufmerksam machen, aber diese nicht verurteilen? Ist es richtig hintenrum über die Brüder und Schwester zu richten, aber wenn sie vor einem stehen, ein christliches Lächeln auf den Lippen haben? 

Paulus sagt dazu in \bibleverse{Gal}(6:1-6)
\begin{bibeltext}{SCHL}{Gal}{6:1-6}
    Geschwister, wenn auch ein Mensch von einer Übertretung übereilt würde,so helft ihr, die ihr geistlich seid, einem solchen im Geist der Sanftmut wieder zurecht; und gibt dabei acht auf dich selbst, dass du nicht auch versucht wirst! Einer trage des anderen Lasten, und so sollt ihr das Gesetz des Christus erfüllen! Denn wenn jemand meint, etwas zu sein, da er doch nichts ist, so betrügt er sich selbst.

    Jeder aber prüfe sein eigenes Werk, und dann wird er für sich selbst den Ruhm haben und und nicht für einen anderen; denn jeder Einzelne wird seine eigene Bürde zu tragen haben.
\end{bibeltext}
\end{spacing}
\lied{Die dem Herrn vertrauen}

\section{Predigt}

Danach gebe ich das Wort an Hendrick weiter.

\textbf{Nach der Predigt}

Danken für die Predigt.

% \section{Abendmahl}

% Beten für das Brot

% \lied{Das Blut der Lammes 1. Strophe}

% Beten für den Wein

% \lied{Das Blut der Lammes 2. Strophe}

\section{Abschluss}

Jetzt wollen wir Gott mit dem Lied \lied{Die Gott lieben werden sein wie die Sonne} danken.

Vielen Dank für eure Teilnahme am Gottesdienst. Im Anschluss seid ihr zu Kaffee und guten Gesprächen eingeladen.
\beten{}

\begin{bibelbox}{SCHL}{1Mos}{28:15}
Gott spricht: Siehe, ich bin mit dir,
ich behüte dich, wohin du auch gehst.
Denn ich verlasse dich nicht,
bis ich vollbringe, was ich dir versprochen habe.
\end{bibelbox}

Maranatha Amen
\end{document}