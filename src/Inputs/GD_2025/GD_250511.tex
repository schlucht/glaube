\author{OTS}
\documentclass{../../inc/mybib}

\setincpath{../../inc/}

\usepackage{bible_style}
\usepackage{header}

\begin{document}


\section{Begrüssung}

Ich möchte euch alle recht herzlich zu diesem besonderen Gottesdienst begrüssen. Schön das so viele gekommen sind, um unserem Herrn zu Danken ihn zu Loben und zu Preisen.
Auch möchte ich Philip begrüssen. Leider ist Oswald heute krank und kann nicht die Leitung übernehmen.

\noindent
\beten{} und anschliessend singen wir zusammen das Lied

\noindent
\lied{Gnade die Jesus}.

\section{Ankündigungen}
\begin{itemize}
    \item \green{Bibel und Gebetsabend:} Do 15.05.2025 20:00 Uhr Bibel und Gebetsabend mit Samuel Rindlisbacher der uns ein weiterer Teil vom Römerbrief 15 4-7 .
    \item \green{Nächster Gottesdienst:} So 25.05.2025 14:45 Uhr mit Nathanel    
    \item \green{Die Kollekte:} Die Kollekte geht an den Mitternachtsruf und wird dort für die Missionsarbeit in der Welt eingesetzt.
\end{itemize}

\section{ Input }
\begin{spacing}{1.5}
\subsection{Die Christen in Europa}
Letztes Wochenende hat in Hannover der 34. Kirchentag statt gefunden. Dieser Kirchentag wird von der evangelischen Kirche Deutschland organisiert. Dieser Event zeigt schön auf wie Christsein heute definiert, gepredigt und gelebt wird. Interessant fand ich eine Statistik zum Programm vom Kirchentag. Das Programm vom Kirchentag ist 255 Seiten lang. Also da war ordentlich was los.\\
Wenn man das Programm gelesen hat, viel einem auf dass in diesem Programm herzlich wenig von Jesus, seinem Sühnetod und Busse die Rede war. Das Evangelium der Bibel findet sich in dem Programm nirgends. Wenn man in diesem 255 Seiten Programm nach Evangelischen Stichworten sucht, bekommt man folgendes Resultat:
\begin{enumerate}
    \item Sünde = 0
    \item Erlösung = 0
    \item Christus = 5 Drei davon erwähnen eine Veranstaltung mit dem Titel "Der tanzende Christus"; der Rest eine Musikgruppe mit dem Namen "Entschieden für Christus"
    \item Reue = 0
    \item Busse = kommt 2 mal vor, weil eine Frau mit dem Namen Alexandra Busse im Programm vorkommt.
    \item Auferstanden = 0
    \item Auferstehung = 1 Eine Veranstaltung mit dem Namen "Auferstehung wirklich?"
    \item Evangelium = 0
\end{enumerate}
Was findet man denn im dem Programm wirklich.
\begin{enumerate}
    \item queer = 23
    \item Politik = 40
    \item Rassismus = 23
\end{enumerate}
Auf Rassismus aufmerksam zu machen ist ja gut.

Ironischer Weise gab es aber eine Veranstaltung die ausschließlich farbige Kinder besuchen durften. Alle weißen Kinder waren zu der Veranstaltung ausgeschlossen!
\begin{bibelbox}{SCHL}{Gal}{3:28}
Da ist weder Jude noch Grieche, da ist weder Knecht noch Freier, da ist weder Mann noch Frau; denn ihr seid alle einer in Christus Jesus.
\end{bibelbox}
Ich möchte hier jetzt aufhören. Der ganze evangelische Kirchentag, hat nichts mit dem Christentum, wie es in der Bibel steht zu tun. Es ist eine Volksveranstaltung mit Gotteslästerung. Was mit Völkern passiert, die Gott lächerlich machen und lästern könnt ihr gerne im alten Testament nachlesen.

Diese Entwicklung ist aber nicht überraschend. Ich glaube jedem fleißigen Bibelleser kommt da Römer Kapitel 1 ab Vers 18 in den Sinn. Auch Jesus spricht diese Problem in seiner Ölbergpredigt an.

Zum Abschluss möchte ich euch jetzt die Verse aus dem Römerbrief Kapitel 1, 18-32 vorlesen. Es soll uns wachsam halten aber auch ermutigen, dass Gott alles absolut souverän führt und leitet. Wer Wiedergeboren ist braucht sich keine Sorgen zu machen, abzufallen. Jesus ist unser Führsprecher und der hl. Geist unsere Stütze und Stärkung.
\end{spacing}

\lied{Nimm du mich ganz hin}
\newpage
\section{Predigt}

Danach gebe ich das Wort an Philipp weiter.

\textbf{Nach der Predigt}

Danken für die Predigt.

% \section{Abendmahl}

% Beten für das Brot

% \lied{Das Blut der Lammes 1. Strophe}

% Beten für den Wein

% \lied{Das Blut der Lammes 2. Strophe}

\section{Abschluss}

Jetzt wollen wir Gott mit dem Lied \lied{Lasten sie fallen auf Golgatha} danken.

Vielen Dank für eure Teilnahme am Gottesdienst. Im Anschluss seid ihr zu Kaffee und guten Gesprächen eingeladen.
\beten{}

\begin{bibelbox}{SCHL}{1Mos}{28:15}
Gott spricht: Siehe, ich bin mit dir,
ich behüte dich, wohin du auch gehst.
Denn ich verlasse dich nicht,
bis ich vollbringe, was ich dir versprochen habe.
\end{bibelbox}

Maranatha Amen
\end{document}