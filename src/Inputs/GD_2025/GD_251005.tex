\author{OTS}
\documentclass{../../inc/mybib}

\setincpath{../../inc/}

\usepackage{bible_style}
\graphicspath{{../../assets/images/}}
\usepackage{header}

% \newenvironment{block}[1][]{%
%   \vspace{1.5em}%
%   \noindent\textbf{#1}\par%
%   \vspace{0.0em}%
% }{%
%   \vspace{1em}%
% }

\begin{document}

\section{Begrüssung}
Ich möchte alle recht herzlich zu unserem heutigen Gottesdienst begrüssen. Schön, dass ihr alle hier seid. Obed, auch dich will ich recht herzlich Begrüssen. Wir freuen uns, dass du bei uns im Wallis bist und uns das Wort Gottes mitbringst. Ich glaube es ist am besten, wenn du dich nacher selber ein bisschen vorstellst.

\beten{}

Singen wir nun zusammen unser erstes Lied.
% \noindent
\lied{Alle Herren dieser Welt}.

\section{Ankündigungen}
\begin{itemize}
    \item \green{Bibel und Gebetsabend:} Do,  9.10.2025 Bibel und Gebetsabend
    \item \green{Nächster Gottesdienst:} So, 12.10.2025 14:45 Uhr Gottesdienst da werde ich das Wort verkündigen.    
    \item Die Kollekte wird hinten in der Grünen Trommel gesammelt und für den Bau dieser Gemeinde verwendet.
\end{itemize}

\section{ Input }
\begin{spacing}{1.5}    
    \begin{block}[Einleitung]
        Am Sonntag 14.9 ich das Thema vom Reich Gottes und den Bündnisse, die Gott mit uns Menschen gemacht hat angefangen. Als erstes haben wir 5 Phasen die man in der Bibel erkennen kann, mit denen Gott sein Königreich aufrichten will, angeschaut.
    \end{block}

    \begin{enumerate}
    \item Schöpfung \bibleverse{IMos}(1-2:)\\ 
    \item Sündenfall \bibleverse{IMos}(3:)\\
    \item Verheissung 1. Mose 3,15 - Maleachi\\
    \item Ankunft des Königs Evangelien Briefe\\
    \item Wiederherstellung Offenbarung\\
   \end{enumerate}
        
    \begin{block}[Schöpfung Kapitel 1]
       Heute wollen wir das Thema Schöpfung anschauen. Die Schöpfung wird in Kapitel 1 und Kapitel 2 vom 1. Mose behandelt. Oft wird behauptet, dass es zwei unterschiedliche Schöpfungen sind. Ich sehe das nicht so. Kapitel 1 wird die Schöpfung schön Chronologisch über 6 aufgezeichnet. Schön der Reihe nach. Erde, Universum, Tag und Nacht, Himmel. Wasser auf der Erde, Trockene Erde Gras, Samen, Früchte. Dann Sonne Mond und Sterne. Die Tiere im Wasser, in der Luft und auf der Erde. Und zuletzt der Mensch.

       Alles schön der Reihe nach. Somit ist auch geklärt was zuerst war, das Ei oder das Huhn. 
    \end{block}
  
   \begin{block}[Schöpfung Kapitel 2]
   Hier zoomen wir jetzt in die Schöpfung rein. Zuerst wird nochmals im Schnelldurchlauf die Schöpfung gezeigt. Dann detailiert das Paradies beschrieben, in dem der Mensch jetzt leben darf.
   \end{block}
   
\end{spacing}
\lied{Nur durch Christus in mir}

\section{Predigt}
Nach dem Lied möchte ich Samuel nach vorne Bitten.
%  \textbf{Nach der Predigt}
%  Danken für die Predigt.

%  \section{Abendmahl}
%  Beten Nick und Lothar evt. Ueli
%  Beten für das Brot 

%  \lied{Das Blut der Lammes 1. Strophe}

%  Beten für den Wein

%  \lied{Das Blut der Lammes 2. Strophe}

\section{Abschluss}
Vielen Dank Samuel für die Predigt.

\lied{Schau ich zurück}.

\beten{}

\begin{bibelbox}{SCHL}{1Mos}{28:15}
    Gott spricht: Siehe, ich bin mit dir,
    ich behüte dich, wohin du auch gehst.
    Denn ich verlasse dich nicht,
    bis ich vollbringe, was ich dir versprochen habe.
\end{bibelbox}

Maranatha Amen
Leider gibs am Heute nur noch Kapselkaffee. Aber ich freue mich auf die Gespräche.
\end{document}
