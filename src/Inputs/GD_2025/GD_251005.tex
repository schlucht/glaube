\author{OTS}
\documentclass{../../inc/mybib}

\setincpath{../../inc/}

\usepackage{bible_style}
\graphicspath{{../../assets/images/}}
\usepackage{header}

% \newenvironment{block}[1][]{%
%   \vspace{1.5em}%
%   \noindent\textbf{#1}\par%
%   \vspace{0.0em}%
% }{%
%   \vspace{1em}%
% }

\begin{document}

\section{Begrüssung}
Ich möchte alle recht herzlich zu unserem heutigen Gottesdienst begrüssen. Schön, dass ihr alle hier seid. Obed, auch dich will ich recht herzlich Begrüssen. Wir freuen uns, dass du bei uns im Wallis bist und uns das Wort Gottes mitbringst. Ich glaube, es ist am besten, wenn du dich nacher selber ein bisschen vorstellst.

\beten{}

Singen wir nun zusammen unser erstes Lied.
% \noindent
\lied{Alle Herren dieser Welt}.

\section{Ankündigungen}
\begin{itemize}
    \item \green{Bibel und Gebetsabend:} Do,  9.10.2025 Bibel und Gebetsabend
    \item \green{Nächster Gottesdienst:} So, 12.10.2025 14:45 Uhr Gottesdienst da werde ich die Predigt halten, und zwar möchte ich etwas über den Psalm 63 sagen.    
    \item Die Kollekte wird hinten in der Grünen Trommel gesammelt und für den Bau dieser Gemeinde verwendet.
    \item Noch etwas zu eigener Sache. Wie ihr vielleicht bemerkt habt, haben wir ein bisschen umdekoriert. Ich möchte mich allen die sich daran beteiligt haben, von ganzem Herzen bedanken. Nur durch eure tatkräftige Mithilfe konnten wir auf die Schnelle alles so schön herrichten. Lasst und diesen Gottesdienst benutzen, um unserem Herrn zu Danken für die Gnade die er uns gegeben hat.
\end{itemize}

\section{ Input }
\begin{spacing}{1.5}    
    \begin{block}[Einleitung]
        Am Sonntag 14.9 habe ich das Thema vom Reich Gottes und den Bündnissen, die Gott mit uns Menschen gemacht hat angefangen. Da haben wir gesehen, dass man fünf Phasen erkennen kann, mit denen Gott sein Königreich aufrichten will. Hier nochmals die Auflistung.
    \end{block}

    \begin{enumerate}
    \item Schöpfung \bibleverse{IMos} (1-2:)
    \item Sündenfall \bibleverse{IMos} (3:)
    \item Verheissung 1. Mose 3,15 - Maleachi
    \item Ankunft des Königs Evangelien Briefe
    \item Wiederherstellung Offenbarung
   \end{enumerate}        
    \begin{block}[Schöpfung Kapitel 1]
       Heute wollen wir das Thema Schöpfung anschauen. Die Schöpfung wird in Kapitel 1 und Kapitel 2 von 1. Mose behandelt. Oft wird behauptet, dass es zwei unterschiedliche Schöpfungen sind. Ich sehe das nicht so. Kapitel 1 wird die Schöpfung schön chronologisch über 6 Tage aufgezeichnet. Erde, Universum, Tag und Nacht, Himmel. Dann das Wasser auf der Erde, die Scheidung trockene Erde Gras, Samen, Früchte und Meer. Dann Sonne Mond und Sterne. Die Tiere im Wasser, in der Luft und auf der Erde. Und zuletzt der Mensch.
       Alles genau so der Reihe nach, wie die verschiedenen Arten voneinnander abhängig sind. Somit ist auch geklärt was zuerst war, das Ei oder das Huhn. 
    \end{block}

   \begin{block}[Schöpfung Kapitel 2]
    Hier zoomen wir jetzt in die Schöpfung rein. Zuerst wird nochmals im Schnelldurchlauf die Schöpfung aufgezeigt. Dann detailiert das Paradies beschrieben, in dem der Mensch jetzt leben darf. Hier wollen wir heute ansetzen. Gott hat jetzt nicht direkt ein Bund mit den zwei Menschen geschlossen, sondern hat ihnen Anweisungen und ein Gebot geben wie sie sich im Paradies verhalten sollen. 
    \begin{bibelbox}{SCHL}{1Mos}{1:27}
        Und Gott schuf den Menschen in seinem Bild, im Bild Gottes schuf er ihn; als Mann und Frau schuf er sie.
    \end{bibelbox}

    Als Bild Gottes sollen wir uns sich ihm hinwenden. Er will gemeinschaft mit uns.

    \begin{bibelbox}{SCHL}{1Mos}{1:28}
        Und Gott segnete sie; und Gott sprach zu ihnen: \enquote{Seid fruchtbar und mehrt euch und füllt die Erde und macht euch sie untertan; und herrscht über die Fische im Meer und über die Vögel des Himmels und über alles Lebendige, das sich regt auf der Erde.}
    \end{bibelbox}

    Klarer Auftrag. Wir sollen uns vermehren, und die Erde uns zu eigen machen. Oft wird gesagt, dass die Erde zu klein wird für die ganzen Menschen. Das Problem ist nicht die grösse, sondern dass wir hier im Westen alle Resourcen für uns beansprechen.
    Und als Drittes ein Gebot.

    \begin{bibelbox}{SCHL}{1Mos}{2:16-17}
        Und Gott der Herr gebot dem Menschen und sprach: \flqq Von jedem Baum des Gartens darfst du nach Belieben essen; aber von dem Baum der Erkenntnis des Guten und des Bösen sollst du nicht essen; denn an dem Tag, da du davon isst, musst du gewisslich sterben!\frqq
    \end{bibelbox}

    Gewisslich sterben. Heisst es da. Und was müssen wir jetzt alle? Gewisslich sterben. Wie es dazu kam werden wir, dann in zwei Wochen zusammen anschauen.
   \end{block}
   
\end{spacing}
\lied{Nur durch Christus in mir}

\section{Predigt}
Nach dem Lied möchte ich Samuel nach vorne Bitten.
%  \textbf{Nach der Predigt}
%  Danken für die Predigt.

%  \section{Abendmahl}
%  Beten Nick und Lothar evt. Ueli
%  Beten für das Brot 

%  \lied{Das Blut der Lammes 1. Strophe}

%  Beten für den Wein

%  \lied{Das Blut der Lammes 2. Strophe}

\section{Abschluss}
Vielen Dank Obed für die Predigt.

\lied{Schau ich zurück}.

\beten{}

\begin{bibelbox}{SCHL}{1Mos}{28:15}
    Gott spricht: Siehe, ich bin mit dir,
    ich behüte dich, wohin du auch gehst.
    Denn ich verlasse dich nicht,
    bis ich vollbringe, was ich dir versprochen habe.
\end{bibelbox}

Maranatha Amen

\end{document}
