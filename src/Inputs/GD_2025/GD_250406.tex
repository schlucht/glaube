
\section{Begrüssung}

Ich möchte euch alle recht herzlich zu diesem besonderen Gottesdienst begrüssen. Schön das so viele gekommen sind, um unserem Herrn zu Danken ihn zu Loben und zu Preisen.

Auch möchte ich Nathanael begrüssen, der im Rahmen seiner Schweizertoutnee den Weg hier her gefunden hat, um uns das Wort des Herrn zu verkünden. 

\noindent
\beten{} und anschliessend singen wir zusammen das Lied

\noindent
\lied{Zünde an dein Feuer}.

\section{Ankündigungen}
\begin{itemize}
    \item \green{Bibel und Gebetsabend:} Do 10.4.2025 20:00Uhr Bibel und Gebetsabend mit Samuel Rindlisbacher der uns ein weiterer Teil vom Römerbrief auslegt.
    \item \green{Nächster Gottesdienst:} So 13.4.2025 14:45 Uhr mit Norbert Lieth
    \item \green{Diverses:} Ostersonntag findet hier kein Gottestdienst statt. In Dübendorf findet die Osterkonferenz statt und viele von uns sind an diesem Wochenende dort.
    \item \green{Diverses:} Am Montag als Morgen Abend, hält Nathanael in diesem Raum am 19:30Uhr den Vortrag Israel -- wie weiter? Dieser Vortrag ist im Rahmen der CH-Tournee und wir haben die ehre diesen hier in Naters durchführen zu dürfen. Die letzten Jahre wurde dieser Anlass immer in Brig im Grünwald-Saal durchgeführt.
    \item \green{Die Kollekte:} Die Kollekte geht an den Mitternachtsruf und wird dort für die Missionsarbeit in der Welt eingesetzt.
\end{itemize}

\section{ Input }
\begin{spacing}{1.5}
\subsection{Was Glauben wir im MNR}
Heute möchte ich euch mal anfangen zu Erzählen, was der MNR überhaupt glaubt. Ich werde heute nicht das ganze Glaubendsbekenntins durchgehen sondern mit einem ersten Teil anfangen und dann an anderen Sonntagen weiter fahren. Wie ihr ja sicher wisst, werden wir von der freien evangelischen Gemeinde Mitternachtsruf unterstützt und geleitet.

Seit gut zwei Jahren kommen regelmässig Prediger von Zürich hier her ins Wallis und verkünden uns das Wort unseren \herr n. Aber wisst ihr auch was der MNR glaubt? Was erzählt ihr den Leuten, wenn sie euch fragen was denn der Mitternachtsruf ist? Was glaubt der MNR? Und wir? Was glauben wir?

Ich möchte euch die nächsten Sonntage das Glaubendsbekenntnis vom MNR näher bringen. Ihr werdet überrascht sein, was sie glauben.

Das Glaubendsbekenntnis vom MNR kann man auf der Webseite von ihnen abrufen. Ihr findet es wenn ihr auf die Startseite www.mnr.ch geht unter dem Link \frqq Unser Glaubensbekenntnis \flqq{}. Es wird in 6 Teile unterteilt.
\begin{enumerate}
    \item \uppercase{über die bibel}
    \item \uppercase{über gott und den Menschen}
    \item \uppercase{über die Erlösung}
    \item \uppercase{über die Gemeinde und Israel}
    \item \uppercase{über die Engel}
    \item \uppercase{über die letzten Dinge}
\end{enumerate}
Heute möchte ich mit euch kurz anschauen was der Mitternachtsruf über die Bibel sagt. Ich werde jetzt jetzt nicht jeden Punkt auseinnander nehmen, sondern euch die Bibelstellen die zu der jeweiligen Aussage zutreffen vorlesen. Es wichtig, dass nicht einfach etwas behauptet wird, sondern dass diese Behauptung auch von Gott getragen und mit der Bibel bezeugt wird.
\begin{enumerate}
    \item Wir glauben, dass die Bibel auf das unfehlbare, vom hl. Geist inspirierte Wort Gottes gründet
    \begin{itemize}
        \item 1. Kor 2.13
        \item 1. Thess 2.13
        \item 2. Tim 3.16
        \item 2. Petr 1.19-21
    \end{itemize}
    \item Wir glauben dass die ganze Bibel von Gott inspiriert ist.
    \begin{itemize}
        \item Jes 46.9-10
        \item Jes 55.11
        \item Mt 5.17-18
        \item Röm 3.3-4
        \item Röm 11
        \item Offb 22.6-20
    \end{itemize}
    \item Wir glauben, dass die Bibel in ihrem einfachsten Sinn zu verstehen und auszulegen ist.
    \begin{itemize}
        \item Ps 33.4
        \item Joh 5.31-47
        \item 2. Tim 3.14-17
        \item Offb 1.3
    \end{itemize}
\end{enumerate}

% \begin{bibelbox}{SCHL}{Gal}{2:20}
% Ich lebe, doch nun nicht ich, sondern Christus lebt in mir. Denn was ich jetzt lebe im Fleisch, das lebe ich im Glauben an den Sohn Gottes, der mich geliebt hat und sich selbst für mich hingegeben.
% \end{bibelbox}

\end{spacing}

\lied{Majestät, herrliche Majestät}

\section{Predigt}

Danach gebe ich das Wort an Nathanale weiter.

\textbf{Nach der Predigt}

Danken für die Predigt.

\section{Abendmahl}

Beten für das Brot

\lied{Das Blut der Lammes 1. Strophe}

Beten für den Wein

\lied{Das Blut der Lammes 2. Strophe}

\section{Abschluss}

Jetzt wollen wir Gott mit dem Lied \lied{Vater, unser Vater} danken.

Vielen Dank für eure Teilnahme am Gottesdienst. Im Anschluss seid ihr zu Kaffee und guten Gesprächen eingeladen.
\beten{}

\begin{bibelbox}{SCHL}{1Mos}{28:15}
Gott spricht: Siehe, ich bin mit dir,
ich behüte dich, wohin du auch gehst.
Denn ich verlasse dich nicht,
bis ich vollbringe, was ich dir versprochen habe.
\end{bibelbox}

Maranatha Amen
