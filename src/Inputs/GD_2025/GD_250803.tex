\author{OTS}
\documentclass{../../inc/mybib}

\setincpath{../../inc/}

\usepackage{bible_style}
\graphicspath{{../../assets/images/}}
\usepackage{header}

\begin{document}

\section{Begrüssung}

Ich möchte euch alle recht herzlich zu diesem Gottesdienst begrüssen. Schön das ihr gekommen seid, um unserem Herrn zu Danken ihn zu Loben und zu Preisen.
Hallo Thomas wir freue uns, dich hier im Oberwallis in Naters begrüssen zu können. Wir sind gespannt auf dein Wort und freuen uns darauf.

% \noindent
\beten{} und anschliessend singen wir zusammen das Lied

% \noindent
\lied{Du bist der Weg und das Leben}.

\section{Ankündigungen}
\begin{itemize}
    \item \green{Bibel und Gebetsabend:} Do 7.08.2025 20:00 Uhr Bibel und Gebetsabend mit Samuel Rindlisbacher  beginnt ein neues Buch Markus 1,1-4.
    \item \green{Nächster Gottesdienst:} So 10.08.2025 14:45 Uhr hier mit Paul Minder.
    \item \green{Allgemein:} Am Sonntag 17.August findet der erste Gemeindetag statt. Der Gottesdienst wird mit Fredy Peter um 10 Uhr statt finden. Danach wollen wir uns zu einem gemütlichen Nachmittag mit Grill zusammensetzen.\\    
    Das Essen findet in der Gartenwirtschaft vom Restaurant Melodie statt. Ich werde einen Grill aufstellen und für Getränke sorgen. Ihr dürft gerne etwas zum Grillieren mitbringen.\\
    Bitte meldet euch nach dem Gottesdienst bei mir an, damit ich weiss wie viele Personen kommen.\\
    \item \green{Die Kollekte:} Die Kollekte geht an den MNR und wird für den Bau dieser Gemeinde eingesetzt.
\end{itemize}

\section{ Input }
\begin{spacing}{1.5}
1. August der Nationalfeiertag der Schweiz. Der erste August wurde das erste mal 1891 gefeiert. Seit 1993 ist es in der Schweiz ein gesetzlicher Feiertag. Der 1. August datiert sich auf den Bundesbrief von 1291 der auf Anfang August datiert ist.

In der Bundesverfassung der Schweizerischen Eidgenossenschaft steht als Präambel:
\begin{quote}
Im Namen Gottes des Allmächtigen!\\
Das Schweizervolk und die Kantone, in der Verantwortung gegenüber der Schöpfung,...
\end{quote}
Diese Präambel wurde 1999 nach einer Volksabstimmung in die Bundesverfassung aufgenommen. Es gab zwar eine Disskussion darüber ob \enquote{Im Namen Gottes} mit aufgenommen werden soll, aber die Mehrheit hat sich dafür entschieden.\\
Heute wird wieder darüber diskutiert, ob diese Präambel in der Verfassung bleiben soll oder nicht. Aktuell ist sie noch drin, wohl weil sie für viele keine Bedeutung hat.

Glauben wir noch daran, ob Gott das Geschick eines Landes leitet? Oder denken wir auch auch die Regierung macht das? Also quasi wir machen das. Wenn man in der Bibel die Kaptel der Chronik und Könige studiert, war auch dort die genau gleiche Meinung. Propheten habe die Könige immer wieder ermahnt, aber die Könige vertrauten nicht mehr auf Gott.

Ist unser Klimawandel, die Kriege, das Elend auf dieser Welt nicht auch ein Zeichen dafür, dass wir Gott nicht mehr vertrauen?

Oder im kleinen Rahmen, in unserer Gemeinde? Glauben wir das Gott die Gemeinde baut? Oder vertrauen wir auf uns allein? Oder zu Hause in der Familie? Wie oft fragen wir Gott um Rat? Ja, nachdem wir uns selber in eine Sackgasse manövriert haben, dann, ja dann bitten wir Gott um Hilfe. Darum wollen wir doch in diesem Gottesdienst Gott um Weisheit und Führung für uns, unsere Gemeinde, unser Kanton und unser Land bitten. Aber auch darauf vertrauen, dass Gottes Wege höher sind als unsere Wege.

Mit dem nächsten Lied wollen wir also Gott danken.
\end{spacing}
\lied{Danke mein Vater}

\section{Predigt}

Danach gebe ich das Wort an Thomas weiter.

% \textbf{Nach der Predigt}

% Danken für die Predigt.

\section{Abendmahl}

Beten für das Brot

\lied{Das Blut der Lammes 1. Strophe}

Beten für den Wein

\lied{Das Blut der Lammes 2. Strophe}

\section{Abschluss}

Jetzt wollen wir Gott mit dem Lied \lied{Herr du gibst uns Hoffnung} danken.

Vielen Dank für eure Teilnahme am Gottesdienst. Im Anschluss seid ihr zu Kaffee und guten Gesprächen eingeladen.
\beten{}

\begin{bibelbox}{SCHL}{1Mos}{28:15}
Gott spricht: Siehe, ich bin mit dir,
ich behüte dich, wohin du auch gehst.
Denn ich verlasse dich nicht,
bis ich vollbringe, was ich dir versprochen habe.
\end{bibelbox}

Maranatha Amen
\end{document}