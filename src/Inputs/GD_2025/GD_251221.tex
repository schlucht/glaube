\author{OTS}
\documentclass{../../inc/mybib}

\setincpath{../../inc/}

\usepackage{bible_style}
\graphicspath{{../../assets/images/}}
\usepackage{header}

\newcommand{\Name}{Fredy}
% ensure scrlayer-scrpage has sufficient footheight
\setlength{\footheight}{20.4pt}

\begin{document}

\section{Begrüssung}
Hallo \Name{}, wir freuen uns, dich in Naters begrüssen zu können. Es ist schön, dass du mit uns diesen Weihnachtsgottesdienst feiern willst. Wir sind gespannt auf dein Wort.

Ich möchte auch euch alle recht herzlich zu diesem Gottesdienst begrüssen. Schön, dass ihr gekommen seid, um unserem \herr N zu Danken, ihn zu loben und zu preisen.

% \noindent
\beten{} Und anschliessend singen wir zusammen das Lied

% \noindent

\lied{In der Nacht von Betlehem}

\lied{Herbei o ihr Gläubigen}
\section{Informationen}
\begin{itemize}
    \item \bt[infos]{Bibel und Gebetsabend:} Do, 15.01.2026 20:00 Uhr Bibel und Gebetsabend mit Nathanael Winkler; zu Markus 4,1-20.
    \item \bt[infos]{Nächster Gottesdienst:} So, 04.01.2026 14:45 Uhr hier mit Nathanael Winkler und mit Abendmahl. 
    \item \bt[infos]{INFOS} Do, 25.12.2025 10:00 Uhr findet in der Gemeinde Bern ihr traditioneller Weihnachtsgottesdienst statt. Janina und werden mit dem Zug hinfahren. Wer mitkommen will, kann sich nach dem Gottesdienst gerne bei Janina melden.
    \item \bt[infos]{Die Kollekte:} Geht nach Dübendorf und wird für den Bau dieser Gemeinde verwendet. Ich möchte noch herzlich Danken für die grosszügigen Spenden. Es konnte alles bezahlt werden und nun sind wir zusammen stolze Besitzer eigener Möbel.\\
    Paulus würde es so ausdrücke; frei nach Philipper 4:18-19:
    \begin{bibelbox}{SCHL}{Phil}{4:18-19}
        Ich habe alles empfangen und bin reichlich versorgt, mit dem, was ihr gespendet habt. Mein Gott aber wird all eure Bedürfnisse erfüllen nach seinem Reichtum in Herrlichkeit in Christus Jesus.        
    \end{bibelbox}
\end{itemize}

\section{ Input }
\begin{spacing}{1.5}
    \begin{block}[Einführung]
    Nun kommen wir zum achten Teil der Heilsgeschichte Gottes mit uns Menschen. Letzten Sonntag haben wir den Bund mit Mose angeschaut. Der Bund mit Mose war kein bedingungsloser Bund mehr, wie die beiden vorherigen Bündnisse mit Noah und Abraham, sondern ein Bund, an den Bedingungen geknüpft sind. Gott hat Israel Gesetze gegeben und an deren Einhaltung hingen Fluch und Segen. Jetzt an Weihnachten schauen wir den Bund mit David an. Ihr fragt euch jetzt sicher was dieser Bund mit Weihnachten zu tun hat. Ganz einfach, ich lese es euch vor: \bibleverse{Mat}(1:6-16).

    Ab Vers 12 gibt es keine Könige mehr. Josef war auch arm. Gott hat aber David versprochen, dass seine Königsherrschaft \bt[bund]{ewig} dauern wird. Wie wir wissen hat die Königslinie um 570 v. Chr. aufgehört. Der letzte König aus der Linie David war Josia. Sein Sohn Zedekia wurde von den Babiloniern weggeführt und danach gab es keine Könige mehr in Israel. Aber Gott hat doch versprochen, dass die davidische Königslinie nie mehr aufhört? Zu lesen in 2. Sam 7:12-16
    \begin{bibelbox}{SCHL}{2Sam}{7:12-16}
        Wenn deine Tage erfüllt sind und du bei deinen Vätern liegst, so will ich deinen Samen nach dir erwecken, der aus deinem Leib kommen wird, und ich werde sein Königtum befestigen. Der wird meinem Namen ein Haus bauen, und ich werde den Thron seines Königreichs auf \bt[bund]{ewig} befestigen. Ich will sein Vater sein, und er soll mein Sohn sein. Wenn er eine Missetat begeht, will ich ihn mit Menschenruten züchtigen und mit Schlägen der Menschenkinder strafen. \bt[blau]{Bis hier die Zusagen für Salomo}. Nun geht es weiter:
        Aber meine Gnade soll nicht von ihm weichen, wie ich sie von Saul weichen liess, den ich vor dir beseitigt habe; sondern dein Haus und dein Königreich sollen \bt[bund]{ewig} Bestand haben vor deinem Angesicht; dein Thron soll auf \bt[bund]{ewig} fest stehen!
    \end{bibelbox}
    Es geht hier in den Versen 12 -- 14 um Salomo, aber es muss noch um jemand anderes gehen. Die Königsline wurde ja abgebrochen und Salomo hat sich am Ende seines Lebens von Gott abgewandt. Ein \bt[bund]{ewiges} davidisches Königreich gab es also noch nie, doch Gott verspricht, ab Vers 15, dass ein König kommen wird, der \bt[bund]{ewig} regiert.
    Auch in den Psalmen lesen wir davon. Hier ein paar Verse aus Psalm 89.
     \begin{bibelbox}{SCHL}{Ps}{89:4-5}
        Ich habe einen Bund geschlossen mit meinem Auserwählten, habe meinem Knecht David geschworen: Auf \bt[bund]{ewig} will ich deinen Samen fest gründen und für alle Geschlechter deinen Thron bauen.
    \end{bibelbox}
     \begin{bibelbox}{SCHL}{Ps}{89:30.34-36}
        Und ich setze seinen Samen auf \bt[bund]{ewig} ein und mache seinen Thron wie die Tage des Himmels. 
        Meine Gnade will ich ihm (Israel) nicht entziehen und meine Treue nicht verleugnen; meinen Bund will ich nicht ungültig machen und nicht ändern, was über meine Lippen gekommen ist. Einmal habe ich bei meiner Heiligkeit geschworen; niemals werde ich David belügen. Sein Samen soll \bt[bund]{ewig} bestehen und sein Thron vor mir wie die Sonne bleiben.
    \end{bibelbox}
    Und das ist Weihnachten. An Weihnachten kam dieser \bt[bund]{ewige} König als Mensch auf diese Welt. Das ist der Segen, den Gott Abraham versprochen hat. \bt[betonung]{Du wirst für alle ein Segen werden}. 

    Simeon sagte als Maria mit Jesus zum Tempel kam in Lukas 2:30-32:
    \begin{bibelbox}{SCHL}{Luk}{2:30-32}
        Denn meine Augen haben dein Heil gesehen, das du vor allen Völkern bereitet hast, ein Licht zur Offenbarung für die Heiden und zur Verherrlichung deines Volkes Israel!
    \end{bibelbox}
    Simeon war gerecht und gottesfürchtig und wartete auf den "Trost Israels". Das heisst, er lebte im Advent. Simeon wusste nicht wann, aber er wusste, dass der Messias noch kommen wird. Er hat das Versprechen, dass Gott David gegeben hat ernst genommen. In diesem Versprechen steckt noch eine zweite Wahrheit, und zwar die, die \bt[betonung]{uns} motivieren sollte im Advent zu leben. \bt[betonung]{Dein Königreich soll ewig Bestand haben und dein Thron soll auf ewig fest stehen.} Das steht noch aus, noch ist es nicht so weit. Aktuell sitzt Jesus zu rechten Gottes im Himmel. Aber es ist der Thron Davids, der von Jesus bei seiner zweiten Wiederkunft bestiegen wird und er sein Königreich hier auf Erden wieder aufrichten wird.

    Dass Jesus wiederkommt, das ist unsere Erwartung, nicht nur jetzt vor Weihnachten, sondern unser ganzes Leben lang. Wir, die wir Jesus Christus in uns haben, können mit Freuden diese Wiederkunft erwarten.
    In Apostelgeschichte 1.6, fragen die Jünger Jesus, wann er dieses Königreich aufrichten wird. Jesus antwortet ihnen:
    \begin{bibelbox}{SCHL}{Apg}{1:7}
        Es ist nicht eure Sache, Zeiten und Zeitpunkte zu kennen, die der Vater in seiner eigenen Vollmacht gesetzt hat.    
    \end{bibelbox}
    Jesus sagt also nicht, dass er \bt[betonung]{nicht} wieder kommen wird. Auch die Engel bestätigen in Apostelgeschichte 1.11, dass er wieder kommt.:
    \begin{bibelbox}{SCHL}{Apg}{1:11}
        Dieser Jesus, der von euch weg in den Himmel aufgenommen worden ist, wird in derselben weise wiederkommen, wie ihr ihn habt in den Himmel auffahren sehn.
    \end{bibelbox}

    Das ist meine Weihnachtsgeschichte, die Zusage von Gott höchst persönlich, dass er wiederkommt. Er kam schon einmal und wurde abgelehnt. Damals ist er für unsere Sünden am Kreuz gestorben. Das zweite Mal kommt er mit Macht und Herrlichkeit, und wir werden mit ihm zusammen kommen. Das ist unsere ganze Hoffnung und von nichts sollten wir uns diese Hoffnung trüben oder rauben lassen.

    Ein Teil des Versprechens, das Gott Abraham vor 4000 Jahren gegeben hat, wurde schon eingelöst: \bt[betonung]{du wirst für alle ein Segen werden}.

    Wenn ihr diesen Jesus noch nicht kennt, dann ladet ihn ein. Tut Busse, bekennt eure Sünden und nehmt das Opfer, das Jesus für dich und mich am Kreuz gebracht hat, an.     

\end{block}
    Singen wir zusammen ein Lied und nach dem Lied möchte ich \Name{} das Wort übergeben.

    \lied{Engel bringe frohe Kunde}.
   
\end{spacing}

\section{Predigt}

% \textbf{Nach der Predigt}

% Danken für die Predigt.

% \section{Abendmahl}

% Beten für das Brot

% \lied{Das Blut der Lammes 1. Strophe}

% Beten für den Wein

% \lied{Das Blut der Lammes 2. Strophe}

\section{Abschluss}

Jetzt singen wir zusammen noch die Weihnachtslieder

\lied{O du fröhliche} und


\beten{}

\begin{bibelbox}{SCHL}{1Mos}{28:15}
    Gott spricht: Siehe, ich bin mit dir,
    ich behüte dich, wohin du auch gehst.
    Denn ich verlasse dich nicht,
    bis ich vollbringe, was ich dir versprochen habe.
\end{bibelbox}

Maranatha, komm Herr Jesus! Amen\\
\lied{Stille Nacht}\\
\end{document}