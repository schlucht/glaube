\author{OTS}
\documentclass{../../inc/mybib}

\setincpath{../../inc/}

\usepackage{bible_style}
\graphicspath{{../../assets/images/}}
\usepackage{header}

\newcommand{\Name}{Fredy}
% ensure scrlayer-scrpage has sufficient footheight
\setlength{\footheight}{20.4pt}

\begin{document}

\section{Begrüssung}
Hallo \Name{} wir freuen uns, dich in Naters begrüssen zu können. Es ist schön, dass du mit uns diesen Weihnachtsgottesdienst feiern willst. Wir sind gespannt auf dein Wort.

Ich möchte auch euch alle recht herzlich zu diesem Gottesdienst begrüssen. Schön, dass ihr gekommen seid, um unserem \herr N zu Danken, ihn zu loben und zu preisen.

% \noindent
\beten{} Und anschliessend singen wir zusammen das Lied

% \noindent

\lied{Macht hoch die Tür}
\section{Ankündigungen}
\begin{itemize}
    \item \green{Bibel und Gebetsabend:} Do, 15.01.2026 20:00 Uhr Bibel und Gebetsabend mit Nathanael Winkler; zu Markus 4,1-20.
    \item \green{Nächster Gottesdienst:} So, 04.01.2026 14:45 Uhr hier mit Nathanael Winkler und Abendmahl. 
    \item \green{INFOS} Do, 25.12.2025 10:00 Uhr findet in Bern der traditionelle Weihnachtsgottesdienst statt. Janina und werden mit dem Zug hinfahren. Wer mitkommen will, kann sich nach dem Gottesdienst gerne bei Janina melden.
    \item \green{Die Kollekte:} Geht nach Dübendorf und wird für den Bau dieser Gemeinde verwendet. Ich möchte noch herzlich Danken für die grosszügigen Spenden für die Möbel. Es konnte alles bezahlt werden und wir haben jetzt eine eigene Ausstattung. 
\end{itemize}

\section{ Input }
\begin{spacing}{1.5}
    \begin{block}[Einführung]
    Der achte Teil der Heilsgeschichte Gottes mit uns Menschen. Letzten Sonntag haben wir den Bund mit Mose angeschaut. Der Bund mit Mose war kein bedingungsloser Bund mehr, wie die beiden vorherigen Bündnisse mit Noah und Abraham, sondern ein Bund an den Bedingungen geknüpft sind. Gott Mose und seinem Volk Gesetze gegeben und an der Einhaltung dieser Gesetze hingen Fluch und Segen. Heute ist Weihnachten und wir schauen den Bund mit David an. Ihr fragt euch jetzt sicher was dieser Bund mit Weihnachten zu tun hat. Ganz einfach ich lese es euch vor: \bibleverse{Mat}(1:6-16).

    Hier seht ihr, auch schon den Bund der Gott mit David geschlossen hat. Gott hat David versprochen, dass seine Königsherrschaft ewig dauern wird. Wie wir wissen hat die aber um 570 v. Chr. aufgehört. Der letzte König aus der Linie David war Josia. Sein Sohn Zedekia wurde von den Babiloniern weggeführt. Danach gab es keine Könige mehr aus der Linie David. Aber Gott hat doch versprochen? Gott hat David folgendes versprochen:
    \begin{bibelbox}{SCHL}{2Sam}{7:12-13}
        Wenn deine Tage erfüllt sind und du bei deinen Vätern liegst, so will ich deinen Samen nach dier erwecken, der aus deinem Leib kommen wird, und ich werde sein Königtum befestigen. Der wird meinem Namen ein Haus bauen, und ich werde den Thron seines Königreichs auf ewig befestigen. Ich will sein Vater sein, und er soll mein Sohn sein. Wenn er eine Missetat begeht, will ich ihn mit Menschenruten züchtigen und mit Schlägen der Menschenkinder strafen. Aber meine Gnade soll nicht von ihm weichen, wie ich sie von Saul weichen liess, den ich vor dir beseitigt habe; sonden dein Haus und dein Königreich sollen ewig Bestand haben vor deinem Angesicht; dein Thron soll auf ewig fest stehen!
    \end{bibelbox}
    Es geht hier in diesen Versen um Salomon, aber es muss noch um jemand anderes gehen. Die Königsline wurde ja unterbrochen. Salomon hat sich am Ende seines Lebens von Gott abgewandt. Gott verspricht hier aber, dass ein König kommen wird der ewig regieren wird.

    Und das ist Weihnachten. An Weihnachten kam dieser ewige König als Mensch auf diese Welt. Das ist auch der Segen den Gott Abraham versprochen hat. Du wirst ein Segen werden. 
    In Lukas 2:30-32 steht:
    \begin{bibelbox}{SCHL}{Luk}{2:30-32}
        Denn meine Augen haben dein Heil gesehen, das du vor allen Völkern bereitet hast, ein Licht zur Offenbarung für die Heiden und zur verherrlichung deines Volkes Israel!
    \end{bibelbox}

\end{block}
    Singen wir zusammen ein Lied und nach dem Lied möchte ich \Name{} das Wort übergeben.
    \lied{Tochter Zion freue dich}.
   
\end{spacing}

\section{Predigt}

% \textbf{Nach der Predigt}

% Danken für die Predigt.

% \section{Abendmahl}

% Beten für das Brot

% \lied{Das Blut der Lammes 1. Strophe}

% Beten für den Wein

% \lied{Das Blut der Lammes 2. Strophe}

\section{Abschluss}

Jetzt wollen wir Gott mit dem Lied \lied{Freue dich Welt} danken.

Vielen Dank für eure Teilnahme am Gottesdienst. Im Anschluss seid ihr zu Kaffee und guten Gesprächen eingeladen.
\beten{}

\begin{bibelbox}{SCHL}{1Mos}{28:15}
Gott spricht: Siehe, ich bin mit dir,
ich behüte dich, wohin du auch gehst.
Denn ich verlasse dich nicht,
bis ich vollbringe, was ich dir versprochen habe.
\end{bibelbox}

Maranatha, komm Herr Jesus! Amen
\end{document}