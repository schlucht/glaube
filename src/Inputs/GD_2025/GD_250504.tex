\author{OTS}
\documentclass{../../inc/mybib}

\setincpath{../../inc/}

\usepackage{bible_style}
\usepackage{header}

\begin{document}


\section{Begrüssung}

Ich möchte euch alle recht herzlich zu diesem besonderen Gottesdienst begrüssen. Schön das so viele gekommen sind, um unserem Herrn zu Danken ihn zu Loben und zu Preisen.

Auch möchte ich Elia begrüssen. Elia ist heute das erste mal hier bei uns in dem neuen Lokal, um uns das Wort des Herrn nahe zu bringen.

\noindent
\beten{} und anschliessend singen wir zusammen das Lied

\noindent
\lied{O Gott, dir sei ehre}.

\section{Ankündigungen}
\begin{itemize}
    \item \green{Bibel und Gebetsabend:} Do 08.05.2025 20:00 Uhr Bibel und Gebetsabend mit Nathanel der uns ein weiterer Teil vom Römerbrief auslegt 15 1-3.
    \item \green{Nächster Gottesdienst:} So 11.05.2025 14:45 Uhr mit Philip Ottenburg
    \item \green{Diverses:} Gemeinde Putztag  
    \item \green{Die Kollekte:} Die Kollekte geht an den Mitternachtsruf und wird dort für die Missionsarbeit in der Welt eingesetzt.
\end{itemize}

\section{ Input }
\begin{spacing}{1.5}
\subsection{Was Glauben wir im MNR}
Kommen wir zum zweiten Teil bei dem wir das Glaubensbekenntnis vom MNR anschauen. Ich finde es ist wichtig, dass wenn wir auf unseren Glauben angesprochen werden, auch wissen was wir überhaupt antworten sollen. Im ersten Teil ging es um die Bibel, heute im zweiten Teil, geht es um Gott und den Menschen.

Auch in diesem Teil geht es nicht drum was der Mitternachtsruf sagt, sondern was die Bibel über die Aussagen des Mitternachtsruf sagt. Es ist wichtig, dass wir jede Aussage immer wieder mit der Bibel überprüfen.

Das Glaubensbekenntnis vom MNR kann man auf ihrer Webseite abrufen. Ihr findet es auf der Startseite von www.mnr.ch, unter dem Link \frqq Unser Glaubensbekenntnis \flqq{}. Es wird in 6 Teile unterteilt.
\aus{folie}
\begin{enumerate}
    \item \uppercase{über die bibel} 06.04.2025
    \item \uppercase{über gott und den Menschen} 04.05.2025
    \item \uppercase{über die Erlösung}
    \item \uppercase{über die Gemeinde und Israel}
    \item \uppercase{über die Engel}
    \item \uppercase{über die letzten Dinge}
\end{enumerate}
Der zweite Punkt den wir heute anschauen, behandelt Gott und den Menschen. Wer ist Gott und wer ist der Mensch. Gott können wir nur durch das Wort Gottes erfahren. Darum ist die Bibel so wichtig, weil das die einzige Quelle ist, wo wir Information über ihn erfahren. 
\begin{enumerate}
    \item \aus{folie} Auf der Grundlage dieses Wortes Gottes glauben wir, dass es nur einen ewigen und vollkommenen Gott gibt, der der Schöpfer aller Dinge ist — wie unserer Welt, die Er in sechs Tagen erschuf. 
    \begin{bibelbox}{SCHL}{IMos}{1:3}
        Kapitel 1. Moses 1-3 die Schöpfungsgeschichte
    \end{bibelbox}
    \begin{bibelbox}{SCHL}{VMos}{6:4}
        Höre Israel, der HERR ist unser Gott, der HERR allein!
    \end{bibelbox}
    \begin{bibelbox}{SCHL}{Jes}{45:5-7}
        Ich bin der HERR und sonst ist keiner; denn außer mir gibt es keinen Gott. Ich habe dich gegürtet, ohne dass du mich kanntest, damit vom Aufgang der Sonne bis zu ihrem Niedergang erkannt werde, dass gar keiner ist außer mir. Ich bin der HERR, und sonst ist keiner, der ich das Licht mache und die Finsternis schaffe; der ich Frieden gebe und Unheil schaffe. Ich, der HERR, vollbringe dies alles.
    \end{bibelbox}
    Wir glauben, dass Er im Wesen eins ist und ewig in drei Personen existiert: in Gott dem Vater, Gott dem Sohn und Gott dem Heiligen Geist.
    \begin{bibelbox}{SCHL}{Matt}{28:19}
        So geht nun hin und macht zu Jüngern alle Völker, und tauft sie auf den Namen des Vaters und des Sohnes und des Heiligen Geistes
    \end{bibelbox}
    \begin{bibelbox}{SCHL}{IKor}{12:4-6}
        Es bestehen aber Unterschiede in den Gnadengaben, doch es ist derselbe Geist; auch gibt es unterschiedliche Dienste, doch es ist derselbe Herr; und auch die Kraftwirkungen sind unterschiedlich, doch es ist derselbe Gott, der alles in allen wirkt.
    \end{bibelbox}
    \begin{bibelbox}{SCHL}{IIKor}{13:13}
        Die Gnade des Herrn Jesus Christus und die Liebe Gottes und die Gemeinschaft des Heiligen Geistes sei mit euch allen! Amen.
    \end{bibelbox}    
    \item \aus{folie} Wir glauben, dass der Mensch von Gott Seinem Bild ähnlich geschaffen wurde, und zwar als Mann oder als Frau.
     \begin{bibelbox}{SCHL}{IMos}{1:26}
        Und Gott sprach: Lasst uns Menschen machen nach unserem Bild, uns ähnlich; die sollen herrschen über die Fische im Meer und über die Vögel des Himmels und über das Vieh und über die ganze Erde, auch über alles Gewürm, das auf der Erde kriecht!
    \end{bibelbox}
     \begin{bibelbox}{SCHL}{IMos}{2:25}
        Und sie waren beide nackt, der Mensch und seine Frau, und sie schämten sich nicht.
    \end{bibelbox}
     \begin{bibelbox}{SCHL}{Jak}{3:9}
        Mit ihr loben wir Gott, den Vater, und mit ihr verfluchen wir die Menschen, die nach dem Bild Gottes gemacht sind; 
    \end{bibelbox}
    Gott setzte die Ehe als Verbindung zwischen Mann und Frau ein und gab ihnen den Auftrag, sich die Erde untertan zu machen und sich zu vermehren. Doch die Sünde des ersten Menschen hat den Tod in die Welt gebracht, und seitdem wird das Gift der Sünde an jeden Menschen weitervererbt. Deswegen widerstreben wir in vielfältiger Weise der Schöpfungsordnung und dem Willen Gottes. Wir ziehen Seinen Zorn auf uns wegen unserer bösen Gedanken und Taten (Eph 2,1-3; 4,22; Jak 4,4).
     \begin{bibelbox}{SCHL}{IMos}{2:16-17}
        Und Gott der HERR gebot dem Menschen und sprach: Von jedem Baum des Gartens darfst du nach Belieben essen; aber von dem Baum der Erkenntnis des Guten und des Bösen sollst du nicht essen; denn an dem Tag, da du davon isst, musst du gewisslich sterben!
    \end{bibelbox}
     \begin{bibelbox}{SCHL}{Rom}{1:3}
        Kapitel Römer 1 - 3. Bitte zu Hause lesen. In diesen Kapitel wird von Paulus ausgiebig aufgezeigt, wer der sündlose Jesus ist und wer der sündige Mensch ist.
    \end{bibelbox}
     \begin{bibelbox}{SCHL}{Rom}{5:10}
        Denn wenn wir mit Gott versöhnt worden sind durch den Tod seines Sohnes, als wir noch Feinde waren, wie viel mehr werden wir als Versöhnte gerettet werden durch sein Leben! 
        \end{bibelbox}
        \begin{bibelbox}{SCHL}{Rom}{5:19}
        Denn gleichwie durch den Ungehorsam des einen Menschen die Vielen zu Sündern gemacht worden sind, so werden auch durch den Gehorsam des Einen die Vielen zu Gerechten gemacht.
    \end{bibelbox}
     \begin{bibelbox}{SCHL}{Eph}{2:1-3}
        auch euch, die ihr tot wart durch Übertretungen und Sünden, in denen ihr einst gelebt habt nach dem Lauf dieser Welt, gemäß dem Fürsten, der in der Luft herrscht, dem Geist, der jetzt in den Söhnen des Ungehorsams wirkt; unter ihnen führten auch wir alle einst unser Leben in den Begierden unseres Fleisches, indem wir den Willen des Fleisches und der Gedanken taten; und wir waren von Natur Kinder des Zorns, wie auch die anderen.
    \end{bibelbox}
     \begin{bibelbox}{SCHL}{Eph}{4:22}
        dass ihr, was den früheren Wandel betrifft, den alten Menschen abgelegt habt, der sich wegen der betrügerischen Begierden verderbte,
    \end{bibelbox}
     \begin{bibelbox}{SCHL}{Jak}{4:4}
        Ihr Ehebrecher und Ehebrecherinnen, wisst ihr nicht, dass die Freundschaft mit der Welt Feindschaft gegen Gott ist? Wer also ein Freund der Welt sein will, der macht sich zum Feind Gottes! 
    \end{bibelbox}
    \item \aus{folie} Wir glauben, dass Gottes ewiger Sohn Mensch wurde, geboren von einer Jungfrau, gezeugt durch den Heiligen Geist — und deswegen unbelastet von der Erbsünde war. 
    \begin{bibelbox}{SCHL}{Jes}{7:14}
        Darum wird euch der Herr selbst ein Zeichen geben: Siehe, die Jungfrau wird schwanger werden und einen Sohn gebären und wird ihm den Namen Immanuel geben.
    \end{bibelbox}   
    \begin{bibelbox}{SCHL}{Matt}{1:23}
        »Siehe, die Jungfrau wird schwanger werden und einen Sohn gebären; und man wird ihm den Namen Immanuel geben«, das heißt übersetzt: »Gott mit uns«.
    \end{bibelbox}   
    \begin{bibelbox}{SCHL}{Lk}{1:35}
        Und der Engel antwortete und sprach zu ihr: Der Heilige Geist wird über dich kommen, und die Kraft des Höchsten wird dich überschatten. Darum wird auch das Heilige, das geboren wird, Gottes Sohn genannt werden.
    \end{bibelbox}   
    \begin{bibelbox}{SCHL}{Rom}{5:0}
        ...
    \end{bibelbox}   
    \begin{bibelbox}{SCHL}{Gal}{4:4}
        Als aber die Zeit erfüllt war, sandte Gott seinen Sohn, geboren von einer Frau und unter das Gesetz getan,
    \end{bibelbox}   
        Jesus lebte als ganzer Gott und ganzer Mensch ein sündloses Leben und nahm am Kreuz von Golgatha die Sünden der Welt auf sich.
    \begin{bibelbox}{SCHL}{IIKor}{5:21}
        Denn er hat den, der von keiner Sünde wusste, für uns zur Sünde gemacht, damit wir in ihm [zur] Gerechtigkeit Gottes würden. 
    \end{bibelbox}
    \begin{bibelbox}{SCHL}{Phil}{2:6-11}
        der, als er in der Gestalt Gottes war, es nicht wie einen Raub festhielt, Gott gleich zu sein; sondern er entäußerte sich selbst, nahm die Gestalt eines Knechtes an und wurde wie die Menschen; und in seiner äußeren Erscheinung als ein Mensch erfunden, erniedrigte er sich selbst und wurde gehorsam bis zum Tod, ja bis zum Tod am Kreuz. Darum hat ihn Gott auch über alle Maßen erhöht und ihm einen Namen verliehen, der über allen Namen ist, damit in dem Namen Jesu sich alle Knie derer beugen, die im Himmel und auf Erden und unter der Erde sind, und alle Zungen bekennen, dass Jesus Christus der Herr ist, zur Ehre Gottes, des Vaters. 
    \end{bibelbox}
    Seine Auferstehung bewies Seine Gottessohnschaft und die Annahme des Opfers durch Gott den Vater (Röm 1,4). 
    \begin{bibelbox}{SCHL}{Rom}{1:4}
        und erwiesen ist als Sohn Gottes in Kraft nach dem Geist der Heiligkeit durch die Auferstehung von den Toten, Jesus Christus, unseren Herrn,
    \end{bibelbox}
    Jesus fuhr in den Himmel auf und sitzt nun zur Rechten Gottes des Vaters (Mk 16,19), wo Er sich als Fürsprecher und Hohepriester für Seine Erlösten einsetzt (Hebr 4,14; 1Joh 2,1).
    \begin{bibelbox}{SCHL}{Mk}{16:19}
    Der Herr nun wurde, nachdem er mit ihnen geredet hatte, aufgenommen in den Himmel und setzte sich zur Rechten Gottes.
    \end{bibelbox}
    \begin{bibelbox}{SCHL}{Hebr}{4:14}
        Da wir nun einen großen Hohenpriester haben, der die Himmel durchschritten hat, Jesus, den Sohn Gottes, so lasst uns festhalten an dem Bekenntnis!
    \end{bibelbox}
    \begin{bibelbox}{SCHL}{IJoh}{2:1}
        Meine Kinder, dies schreibe ich euch, damit ihr nicht sündigt! Und wenn jemand sündigt, so haben wir einen Fürsprecher bei dem Vater, Jesus Christus, den Gerechten;
    \end{bibelbox} 
\end{enumerate}

\end{spacing}

\lied{Gnade die Jesus uns zugewandt}
\newpage
\section{Predigt}

Danach gebe ich das Wort an Elia weiter.

\textbf{Nach der Predigt}

Danken für die Predigt.

\section{Abendmahl}

Beten für das Brot

\lied{Das Blut der Lammes 1. Strophe}

Beten für den Wein

\lied{Das Blut der Lammes 2. Strophe}

\section{Abschluss}

Jetzt wollen wir Gott mit dem Lied \lied{Auferstanden aus des Grabes Nacht} danken.

Vielen Dank für eure Teilnahme am Gottesdienst. Im Anschluss seid ihr zu Kaffee und guten Gesprächen eingeladen.
\beten{}

\begin{bibelbox}{SCHL}{1Mos}{28:15}
Gott spricht: Siehe, ich bin mit dir,
ich behüte dich, wohin du auch gehst.
Denn ich verlasse dich nicht,
bis ich vollbringe, was ich dir versprochen habe.
\end{bibelbox}

Maranatha Amen
\end{document}