\author{OTS}
\documentclass{../../inc/mybib}

\setincpath{../../inc/}

\usepackage{bible_style}
\usepackage{header}

\begin{document}


\section{Begrüssung}

Ich möchte euch alle recht herzlich zu diesem besonderen Gottesdienst begrüssen. Schön das so viele gekommen sind, um unserem Herrn zu Danken ihn zu Loben und zu Preisen.
Auch möchte ich Stephan Beitze begrüssen, der heute Morgen in Bern dem Wort diente und jetzt hier bei uns im Wallis. 

\noindent
\beten{} und anschliessend singen wir zusammen das Lied

\noindent
\lied{Zünde an dein Feuer}.

\section{Ankündigungen}
\begin{itemize}
    \item \green{Bibel und Gebetsabend:} Do 12.06.2025 20:00 Uhr Bibel und Gebetsabend mit Philip Ottenburg  der uns ein weiterer Teil vom Römerbrief Röm 15:17-21.
    \item \green{Nächster Gottesdienst:} So 15.06.2025 10:00 Uhr hier mit eine Übertragung vom Gottesdienst mit Nathanael.    
    % \item \green{Allgemein:} Am Sonntag 17.August planen wir einen Gemeinde Tag. Der Gottesdienst wird mit Fredy Peter um 10 Ùhr statt finden. Danach wollen wir uns zu einem gemütlichen Nachmittag mit Grill zusammensetzen. Genaues Programm wird dann noch bekannt gegeben.
    \item \green{Die Kollekte:} Die Kollekte geht an den MNR und wird für den Bau dieser Gemeinde eingesetzt.
\end{itemize}

\section{ Input }
\begin{spacing}{1.5}
Heute feiern wir Pfingstsonntag. Pfingsten wird als die Geburtsstunde der Gemeinde angesehen.  Ein interessanter Punkt zu Pfingsten hat uns Norbert Lieth letzten Sommer in Ungarn aufgezeigt. An Pfingsten, dessen Fest wir heute feiern, wurde der hl. Geist nur für Juden ausgeschüttet (Apg 1;1-13). Für die Heiden, also für uns wurde der hl. Geist erst in Apg 10;44 ausgegossen. Apg 1:8 sagt uns Jesus, dass er den Hl. Geist schicken wird um uns Kraft zu geben. Der hl. Geist führt und stärkt und in unserem Glaubensleben. Wenn ich mein Leben anschauen, sehe ich vorher und ein nachher. Kein Mensch konnte mich von der Richtigkeit des Evangeliums überzeugen. Erst als Jesus in mir zu wirken anfing.

Wie der Mitternachtsruf über die Gemeinde und Israel denkt, wollen wir heute gemeinsam anschauen. Wir fahren fort mit dem Glaubensbekenntnis der Gemeinde MNR.

\subsection{Was Glauben wir im MNR}
    \begin{enumerate}
        \item \uppercase{über die bibel} 06.04.2025
        \item \uppercase{über gott und den Menschen} 04.05.2025
        \item \uppercase{über die Erlösung} 18.5.2025
        \item \uppercase{über die Gemeinde und Israel} 08.06.2025
        \item \uppercase{über die Engel} 08.06.2025
        \item \uppercase{über die letzten Dinge}
    \end{enumerate}
    \begin{enumerate}      
      \item \textbf{Über die Gemeinde und Israel} \aus{folie}
        \begin{itemize}
            \item Wir glauben, dass alle erlösten Christen durch den Heiligen Geist Glieder am Leib Christi, der weltweiten Gemeinde, sind (1.Kor 12,12-13).\aus{folie}
            
            \item Wir glauben, dass Gott lokale Gemeinden eingesetzt hat, damit sich die Christen zum Gottesdienst, zur Gemeinschaft und zur Unterrichtung in der Lehre dort versammeln und durch ihre Aufgaben und Gaben den Herrn verherrlichen, der Gemeinde dienen und das Evangelium in aller Welt verbreiten (
            Mk 16,15; 
            Apg 2,42;
            Eph 3,21; (4,7-16;)
            1.Tim 3,15;
            Hebr 10,25).\aus{folie}
            
            \item Wir glauben, dass die Gemeinde Jesu Christi aus wiedergeborenen Juden und Heiden besteht (Gal 3,28; Kol 3,11) und Gottes Volk ist. Doch wir glauben auch, dass die Gemeinde nicht das einzige Volk des Herrn ist. Wir glauben, dass Gott die Nation Israel nicht verworfen (Jer 31; Röm 11), sondern lediglich beiseite gestellt hat, bis die Gemeinde vollzählig ist.\aus{folie}
            
            \item Wir glauben, dass Gott heute bereits dabei ist, das Volk Israel in seinem Heimatland zu sammeln (Hes 36,24;). Die Entstehung des Staates Israel 1948 zeigt uns, dass sich jahrtausendealte Prophetie erfüllt (Zef 2,1-2). Wir glauben, dass das Volk Israel bei der Wiederkunft Jesu in Herrlichkeit Ihn als den Messias erkennen, errettet werden und die Versprechungen erhalten wird, die Gott im Alten Testament gemacht und noch nicht eingelöst hat (Jes; Jer; Hes; Sach u.a.; vgl. Offb 21,5-6).\aus{folie}
            
            \item Wir glauben, dass bis zur Wiederkunft des Herrn geistlich gesehen jedoch nur die Juden «wahre Juden» und das «Israel Gottes» sind, die durch den Glauben an Jesus Christus aus Gnade errettet wurden (Röm 5,11). Diese sind in Christus eins mit den Gläubigen aus den Nationen und gehören zum geistlichen Volk der Gemeinde (Eph 2,11.13).            
            \aus{folie}
        \end{itemize}
        \item \textbf{Über die Engel}
        \begin{itemize}
            \item Wir glauben an die Existenz der heiligen Engel Gottes und der gefallenen Engel, nämlich Satan und seine Dämonen. Alle Engel sind erschaffene Wesen. Sie haben zwar einen höheren Rang als die Menschen und dürfen nicht gelästert, aber auch nicht angebetet werden (Eph 6,12; Hebr 1,6-7.14; 2Pt 2,11;) Und noch eine sehr wichtige Stelle Off 22,9. Johannes ist ein Engel erschienen und Johannes warf sich vor dem Engel nieder. Der Engel antwortet ziemlich energisch in Off 22.9
        \end{itemize}
    \end{enumerate}
    Das Glaubensbekenntnis vom MNR kann man auf ihrer Webseite abrufen. Ihr findet es auf der Startseite von www.mnr.ch, unter dem Link \frqq Unser Glaubensbekenntnis \flqq{}. Es wird in 6 Teile unterteilt.
\end{spacing}
\lied{Glauben heisst vertrauen}

\section{Predigt}

Danach gebe ich das Wort an Stephan weiter.

\textbf{Nach der Predigt}

Danken für die Predigt.

% \section{Abendmahl}

% Beten für das Brot

% \lied{Das Blut der Lammes 1. Strophe}

% Beten für den Wein

% \lied{Das Blut der Lammes 2. Strophe}

\section{Abschluss}

Jetzt wollen wir Gott mit dem Lied \lied{Ich hab einen herrlichen K"onig} danken.

Vielen Dank für eure Teilnahme am Gottesdienst. Im Anschluss seid ihr zu Kaffee und guten Gesprächen eingeladen.
\beten{}

\begin{bibelbox}{SCHL}{1Mos}{28:15}
Gott spricht: Siehe, ich bin mit dir,
ich behüte dich, wohin du auch gehst.
Denn ich verlasse dich nicht,
bis ich vollbringe, was ich dir versprochen habe.
\end{bibelbox}

Maranatha Amen
\end{document}