
\section{Begrüssung}

Ich möchte euch alle recht herzlich zu diesem besonderen Gottesdienst begrüssen. Schön das so viele gekommen sind, um unserem Herrn zu Danken ihn zu Loben und zu Preisen.

Auch möchte ich Fredy begrüssen, der den Weg von Zürich über Bern hier her gemacht hat, um uns das Wort des Herrn zu verkünden. 

\noindent
\beten{} und anschliessend singen wir zusammen das Lied

\noindent
\lied{Herr ich sehe deine Welt}.

\section{Ankündigungen}
\begin{itemize}
    \item \green{Bibel und Gebetsabend:} Do 3.4.2025 20:00Uhr Bibel und Gebetsabend mit Norbert Lieth der uns den Römerbrief auslegt.
    \item \green{Nächster Gottesdienst:} So 30.3.2025 10:00Uhr mit Live Stream aus Dübendorf
    \item \green{Diverses:} Ostersonntag findet hier kein Gottestdienst statt, da viele von uns an der Osterkonferenz in Dübendorf sind.
    \item \green{Die Kollekte:} Die Kollekte geht an den Mitternachtsruf und wird dort für die Missionsarbeit in der Welt eingesetzt.
\end{itemize}

\section{ Input }
\begin{spacing}{1.5}
\subsection{Wir in Jesus oder Jesus in uns?}
Wer sorgfältig die Bibel liesst, findet darin einmal die Aussage, dass wir \flqq In Jesus sind...\frqq{} und einmal \flqq Jesus in uns...\flqq{} ist. Was stimmt denn jetzt nun? Da beides von Paulus gepredigt wird und beides in Bibel steht, stimmt beides.

Ich erkläre mir das so: Du möchtest gerne auf eine Kreuzfahrt gehen. Du stehst am Hafen und das riesige Schiff steht vor dir. Was musst du machen um diese tolle teure Reise antreten zu können? Genau, du musst in das Schiff einsteigen. Bleibst du am Hafen stehen, kannst du eine Woche später das Schiff wieder sehen, glückliche Passagiere aussteigen sehen, aber du selber hattest nichts von der Reise.

Genau so ist es, in Jesus zu sein. Nur wenn wir in unserem Leben in Jesus Christus einsteigen, könne wir das volle Potenzial von Jesus erleben. Außerdem ist der Ort in Jesus ein Schutz. Man fährt nicht mit einem Gummiboot auf eine Kreuzfahrt, sondern begibt sich in den Schutz eines grossen Schiffes. In Jeremia 17:7 steht, dass wenn wir dem \herr n vertrauen, gesegnet sind. Das heisst beschützt, ausgesondert sind.
\begin{bibelbox}{ELB}{Jer}{17:7}
Gesegnet ist der Mann, der auf den HERRN vertraut und dessen Vertrauen der HERR ist! 
\end{bibelbox}
Wie auf einem Kreuzschiff haben wir in Jesus Geborgenheit und Schutz. In Jesus ist unser Weg mit Liebe gepflastert. Jesus ist die Liebe. Solgange wir in Ihm sind, können wir nicht anders, als zu Lieben.
\begin{bibelbox}{ELB}{Eph}{5:2}
und wandelt in Liebe, wie auch der Christus uns geliebt und sich selbst für uns hingegeben hat als Darbringung und Schlachtopfer, Gott zu einem duftenden Wohlgeruch.    
\end{bibelbox}
Auch gibt es uns Sicherheit vor Sünden. In Jesus ist es wie in einem Glasschrank. Ausserhalb des Glasschranks können wir nach drei Seiten mit Steinen schmeissen, sind wir aber in ihm drin sollten wir keine Steine schmeissen. Wenn wir in Jesus sind, wird es uns schwieriger werden zu sündigen.

Aber Jesus ist auch in uns. Das ist der Motor im Kreuzfahrtschiff. Ohne Motor würdest du im Hafen bleiben. Hast zwar allen Luxus die das Schiff zu bieten hat, aber du erlebst nichts neues. Es kann sicherer sein, aber es wird langweilig werden, lernst nichts dazu und hast später zu Hause nichts vorzuweisen ausser vielleicht ein paar Kilo mehr.

\flqq Jesus in uns \frqq{}, hilft uns unser Leben nach ihm auszurichten, als Christen Vorwärts zu gehen und als Christ das Leben zu erleben. Jesus sagt in Johannes 
\begin{bibelbox}{ELB}{Joh}{14:16}
und ich werde den Vater bitten, und er wird euch einen anderen Sachwalter geben, dass er bei euch sei in Ewigkeit,
\end{bibelbox}
Dieser Sachverwalter von dem Jesus hier spricht ist der heilige Geist. Dieser Sachverwalter hilft uns das Wort Gottes zu verstehen. Haben wir den Heiligen Geist nicht in uns, verstehen wir das Wort Gottes nicht und wir lernen Jesus nie kennen wie er wirklich ist. Paulus sagt in erste Korinther 2.12-14:
\begin{bibelbox}{SCHL}{IKor}{2:12-14}
Wir aber haben nicht den Geist der Welt empfangen, sondern den Geist aus Gott, damit wir wissen, was uns von Gott geschenkt ist. Und davon reden wir auch, nicht in Worten, gelehrt durch menschliche Weisheit, sondern in Worten, gelehrt durch den Geist, indem wir Geistliches durch Geistliches deuten. Der natürliche Mensch aber nimmt nicht an, was des Geistes Gottes ist; denn es ist ihm Torheit, und er kann es nicht erkennen, weil es geistlich beurteilt werden muss.
\end{bibelbox}
Wie wir uns nun von dem Schiff auch in Stürmen und Unwetter sicher führen lassen, lassen wir uns, von Jesus sicher durch unser stürmisches Leben führen.
\begin{bibelbox}{SCHL}{Gal}{2:20}
Ich lebe, doch nun nicht ich, sondern Christus lebt in mir. Denn was ich jetzt lebe im Fleisch, das lebe ich im Glauben an den Sohn Gottes, der mich geliebt hat und sich selbst für mich hingegeben.
\end{bibelbox}
Jesus lebt also in dir und führt dich. Wenn du in Jesus bist, bist du in einem sicheren Ort und strahlst die Liebe Jesus aus. Wenn Jesus in dir ist, hast du den Motor für deinen Lebensweg. Bitte jeden Morgen darum, dass dich Jesus Christus in deinem Leben leitet und führt.

\end{spacing}

\lied{Gott ist gegenwärtig}

\section{Predigt}

Danach gebe ich das Wort an Fredy weiter.

\textbf{Nach der Predigt}

Danken für die Predigt.

\section{Abschluss}

Jetzt wollen wir Gott mit dem Lied \lied{Bald schon kann es sein} danken.


Vielen Dank für eure Teilnahme am Gottesdienst. Im Anschluss seid ihr zu Kaffee und guten Gesprächen eingeladen.
\beten{}

\begin{bibelbox}{SCHL}{1Mos}{28:15}
Gott spricht: Siehe, ich bin mit dir,
ich behüte dich, wohin du auch gehst.
Denn ich verlasse dich nicht,
bis ich vollbringe, was ich dir versprochen habe.
\end{bibelbox}

Maranatha Amen
