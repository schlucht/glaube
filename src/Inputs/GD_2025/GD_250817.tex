\author{OTS}
\documentclass{../../inc/mybib}

\setincpath{../../inc/}

\usepackage{bible_style}
\graphicspath{{../../assets/images/}}
\usepackage{header}

\begin{document}

\section{Begrüssung}
Einen wunderschönen Sonntagmorgen wünsche ich euch allen. 10Uhr Gottesdienst ist eher ungewohnt für uns, aber wir möchten nicht nur den Gottesdienst, sondern auch den Sonntag gemeinsam verbringen. Ich will auch Fredy begrüssen. Fredy du bist für uns nicht nur \enquote{ein Prediger aus Dübendorf}, sondern du bist die Bezugsperson, der Mentor und Untestüzer von uns. Ich weiss wir Walliser tun uns schwer mit Zürcher, aber du machst das sehr gut und lernst immer mehr dazu.

Die Lieder von heute haben wir schon vor zwei Wochen miteinnander geübt darum können wir kräftig Mitsingen.

% \noindent
\beten{} Und anschliessend singen wir zusammen das Lied

% \noindent
\lied{Du bist der Weg die Wahrheit und das Leben}.

\section{Ankündigungen}
\begin{itemize}
    \item \green{Bibel und Gebetsabend:} Do, 21.08.2025 20:00 Uhr Bibel und Gebetsabend mit Fredy  Markus 1,9-13.
    \item \green{Nächster Gottesdienst:} So, 24.08.2025 14:45 Uhr treffen wir uns wieder hier. Speziell wird sein, dass ich versuchen werde die Predigt zu halten. Das Thema: \textbf{Ist der Turmbau von Babel beendet?}
    \item \green{Allgemein:} Das Programm von Heute: Nach dem Gottesdienst, gibt es einen kleinen Apero in der Cafeteria. Danach gehen wir gemeinsam zum Restaurant Melodie. Die Familie ??? hat uns dankenswerterweise erlaubt, die Gartenterrasse für unseren Anlass zur Verfügung gestellt.
    Ziel ist es, dass wir 13Uhr alle im Melodie beisammen sind und dann dort gemeinsam den Nachmittag verbringen.
\end{itemize}

\section{ Input }
\begin{spacing}{1.5}
    Heute ist ein spezieller Tag. Wir führen unseren ersten Gemeindetag durch. So ein Gemeinschaftstag ist absolut biblisch und von Gott gewollt. Für Gott ist es sehr wichtig, dass wir Menschen miteinnader Gemeinschaft haben und feiern. Schon im Alten Testament hat Gott diverse Feste eingeführt, die die Menschen miteinnander feiern sollen. So heisst es in 5.Mose:
\begin{bibelbox}{BDF}{5Mos}{16:14}
    Und du sollst fröhlich sein an deinem Fest, du und dein Sohn und deine Tochter und dein Knecht und deine Magd und der Levit und der Fremdling und die Waise und Witwe, die in deinen Toren sind!
\end{bibelbox}
Aber auch schon über die erste Gemeinde in der Apostelgeschichte steht:
\begin{bibelbox}{BDF}{Apg}{2:44}
    Alle Glaubenden pflegten beisammen zu sein und hatten alles gemeinsam.
\end{bibelbox}
In vielen Briefen spricht Paulus diese Gemeinschaft der Gläubigen an. Manchmal positiv, aber auch ermahnend, wenn solche Gemeinschaften in Festgelage ausgeartet sind.

In vielen ärmeren Ländern wird ein Gottesdienst den ganzen Tag mit Musik und Essen gemeinsam verbracht.

Am 25. Januar 2024 hatten wir den ersten Bibel- und Gebetsabend und am 28. Januar unseren ersten Gottesdienst in einem eigenem Raum, dem kleinen Fotostudio Fux. Inzwischen sind wir hier gelandet, was für mich ein Wink von Gott war weiter zu machen.

Letzten Sonntag hat uns Paolo Minder sehr schön aufgezeigt, was die Bibel zur Gemeinde Jesu sagt. Paolo sagte: \enquote{Wir finden in der Bibel nirgends, wie eine Gemeinde aussehen soll.} Ich habe ein bisschen in der Bibel nachgeforscht, und dann doch etwas gefunden was alle Gemeinden gemeinsam haben sollten. Gott hat uns nicht einen genauen Bauplan einer Gemeinde geben, aber Jesus hat uns genau gesagt wie eine Gemeinde sein soll. Eine Gemeinde besteht nicht aus Statuten, Ältesten, Diakonen, Putztruppen, Prediger usw. Eine Gemeinde besteht aus Menschen und diese Menschen führen ihren Dienst anhand ihrer Gaben aus. Jesus ist der Eckstein und jeder einzelne hier, der Jesus Christus als seinen Herrn angenommen hat, ist ein Stein in dem Gemeindbau. 

Was ist der Mörtel von diesen Mauern? Was hält die Gemeinde zusammen? Die Steine sind keine schönen viereckige Ziegel, die man so einfach aufeinander legen kann. Es sind oft kantige Steine die nicht immer so einfach zueinander passen. Was ist nun der Mörtel oder der Zement, der alles zusammenhält? Wisst ihr das? Es ist sehr wichtig, dass wir das wissen. \textbf{Es ist die Liebe Gottes}! Die Liebe ist es, die der Gemeinde eine Strucktur gibt, aus Liebe kommt der Dienst von den Ältesten, Diakone, Putzkolonnen, Musiker, Sänger, Prediger, Gottesdienstbesucher. Wenn so eine Struktur nicht aus der Liebe herhaus wächst, wird sie früher oder später zusammenfallen. Und viele fallen zusammen. Und wer ist diese Liebe? Gott ist die Liebe und er hat seine Liebe in unsere Herzen ausgegossen.
\begin{bibelbox}{BDF}{Rom}{5:5}
    Die Hoffnung lässt nicht zuschanden werden, weil die Liebe Gottes in unsere Herzen ausgegossen ist, durch den Heiligen Geist, der uns gegeben ist.
\end{bibelbox}

Und wir? Wir sind alle Sünder und es menschelet unter uns. Wir sind nicht Jesus, aber er hat uns den Heiligen Geist gebeben der uns hilft, Jesus immer ähnlicher zu werden. Gott will diese Gemeinschaft für die gegenseitige Unterstützung. Wenn jetzt irgendwo der Mörtel bröckelt, fällt nicht gleich alles zusammen, sondern die Liebe der Gemeinschaft hält die Mauer immer noch zusammen. Diese kann reapriert werden oder wenn es sein muss, kann auch mal ein Stein ersetzt werden.

Darum schreibt Paulus nach dem er die Gemeinde in \bibleverse{1Kor}(12:) beschrieben hat, das Kapitel 13, das auch das Liebeskapitel genannt wird.

\begin{bibelbox}{BDF}{1Kor}{13:4-7}
Die Liebe ist geduldig, ist freundlich. Die Liebe neidet nicht. Die Liebe tut nicht gross, ist nicht aufgebläht, gebärdet sich nicht in unanständiger Weise, sucht nicht das Ihre, ist nicht schnell gereizt, rechnet das Böse nicht an, freut sich nicht über die Ungerechtigkeit, freut sich aber mit der Wahrheit. Sie deckt alles zu, glaubt alles, hofft alles, erdultet alles.
\end{bibelbox}

Darum liebe Geschwister lasst uns einen fröhlichen Tag miteinnander verbringen, uns ein bisschen besser kennen- und liebenlernen und in dem allem nicht vergessen, dass die ehre Gott alleine gebührt. Aus diesem Grund singen wir jetzt zusammen das Lied Danke mein Vater.

Mit dem nächsten Lied wollen wir also Gott danken.
\end{spacing}
\lied{Danke mein Vater}

\section{Predigt}

Danach gebe ich das Wort an Fredy weiter.

% \textbf{Nach der Predigt}

% Danken für die Predigt.

% \section{Abendmahl}

% Beten für das Brot

% \lied{Das Blut der Lammes 1. Strophe}

% Beten für den Wein

% \lied{Das Blut der Lammes 2. Strophe}

\section{Abschluss}

Ich möchte allen danken, die zum Gelingen von diesem Tag mit Gebet, Essen, Gottestdienstbesuch und die Gemeinschaft beigetragen haben.\\
Dann möchte ich mich recht herzlich bei Nick und Janina für diese tolle Musikalische Begleitung bedanken. Vielen, vielen Dank für die Zeit die ihr dafür geopfert habt. Der Herr wird diese Gabe ganz sicher annehmen. So schön, dass das Klavier gebraucht wurde vielen Dank Nick.\\
Ich möchte mich bei der Familie ??, bei Armin und Marie-Therese bedanken, dass wir die Möglichkeit haben im Meldodie die Gartenwirtschaft benutzen zu dürfen.\\
Ausserdem möchte ich mich bei Fredy bedanken, für alles was er für unsere kleine Gruppe zwischen den grossen Bergen gemacht hat und auch für die gute Zusammenarbeit. Auch für dein Wort heute danke ich dir von Herzen.\\
Alle Gottesdienstbesucher sind für den Nachmittag eingeladen. Wir haben genug zu Essen und zu trinken. Also würde ich mich ehrlich freuen, wenn alle kommen würden.\\
Lasst uns nun gemeinsam in der Hoffnung, dass der Herr alles in seinen Händen hält und ein Plan mit uns hat das letzte Lied singen. \lied{Herr du gibst uns Hoffnung}.

\beten{}

\begin{bibelbox}{SCHL}{1Mos}{28:15}
    Gott spricht: Siehe, ich bin mit dir,
    ich behüte dich, wohin du auch gehst.
    Denn ich verlasse dich nicht,
    bis ich vollbringe, was ich dir versprochen habe.
\end{bibelbox}

Maranatha Amen
Bitte denkt daran, dass wir gemeinsam gegen 12:45Uhr runter zum Melodie laufen. 
\end{document}