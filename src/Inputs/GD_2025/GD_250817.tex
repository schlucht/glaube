\author{OTS}
\documentclass{../../inc/mybib}

\setincpath{../../inc/}

\usepackage{bible_style}
\graphicspath{{../../assets/images/}}
\usepackage{header}

\begin{document}

\section{Begrüssung}

Ich möchte euch alle recht herzlich zu diesem Gottesdienst begrüssen. Schön das ihr gekommen seid, um unserem Herrn zu Danken ihn zu Loben und zu Preisen.
Hallo Thomas wir freuen uns, dich hier im Oberwallis in Naters begrüssen zu können. Wir sind gespannt auf dein Wort und freuen uns darauf.\\
Ich möchte euch alle, die hier versammelt seid, herzlich von Andre und Ingrid Beitze aus Bolivien grüssen.

% \noindent
\beten{} Und anschliessend singen wir zusammen das Lied

% \noindent
\lied{Du bist der Weg und das Leben}.

\section{Ankündigungen}
\begin{itemize}
    \item \green{Bibel und Gebetsabend:} Do, 7.08.2025 20:00 Uhr Bibel und Gebetsabend mit Samuel Rindlisbacher  beginnt ein neues Buch Markus 1,1-4.
    \item \green{Nächster Gottesdienst:} So, 10.08.2025 14:45 Uhr hier mit Paul Minder.
    \item \green{Allgemein:} Nach dem Gottesdienst, trinken wir zusammen ein Kaffee und gehen dann gemeinsam zum Melodie zum Mittagessen.
\end{itemize}

\section{ Input }
\begin{spacing}{1.5}
1. August der Nationalfeiertag der Schweiz. Der erste August wurde das erste Mal 1891 gefeiert. Seit 1993 ist es in der Schweiz ein gesetzlicher Feiertag. Der 1. August beruft sich auf den Bundesbrief von 1291 der auf Anfang August datiert ist.

In der Bundesverfassung der Schweizerischen Eidgenossenschaft steht als Präambel:
\begin{quote}
Im Namen Gottes des Allmächtigen!\\
Das Schweizervolk und die Kantone, in der Verantwortung gegenüber der Schöpfung, usw.
\end{quote}
Diese Präambel wurde 1999 nach einer Volksabstimmung in die Bundesverfassung aufgenommen. Es gab zwar eine Disskussion darüber ob \enquote{Im Namen Gottes} mit aufgenommen werden soll, aber die Mehrheit hat sich dafür entschieden.\\
Heute wird wieder darüber diskutiert, ob diese Präambel in der Verfassung bleiben soll oder nicht. Aktuell ist sie noch drin, wohl weil sie für viele sowieso keine Bedeutung hat.

Glauben wir noch daran, das Gott die Geschicke eines Landes leitet? Oder denken wir auch, die Regierung macht das schon? Also quasi wir machen das. Wenn man in der Bibel die Kapitel der Chronik und Könige studiert, war auch dort genau diese gleiche Meinung. Die Propheten habe die Könige immer wieder ermahnt auf das Wort Gottes zu hören, aber die Könige vertrauten mehr auf sich selber als auf Gott. 

Ist unser Klimawandel, die Kriege, das Elend auf dieser Welt nicht auch ein Zeichen dafür, dass wir Gott nicht mehr vertrauen? Es passiert ja nichts ohne den Willen Gottes auf dieser Welt.

Oder im kleinen Rahmen, in unserer Gemeinde? Glauben wir, das Gott die Gemeinde baut? Oder vertrauen wir auf uns allein? Oder zu Hause in der Familie? Wie oft fragen wir Gott um Rat? Ja, nachdem wir uns selber in eine Sackgasse manövriert haben, dann, ja dann bitten wir Gott um Hilfe.

Wollen wir doch in diesem Gottesdienst Gott um Weisheit und Führung für uns, unsere Gemeinde, unser Kanton und unser Land bitten, aber dann auch darauf vertrauen, dass Gottes Wege höher sind als unsere Wege.

Das hat auch Bileam erkannt, als er es nicht schaffte das Volk Israel zu verfluchen.
\begin{bibelbox}{NUE}{4Mos}{23:19}
    Gott ist ja kein Mensch, der lügt, kein Menschensohn, der etwas bereut. Wenn er etwas sagt, dann tut er es auch, und was er verspricht, das hält er gewiss.
\end{bibelbox}

Mit dem nächsten Lied wollen wir also Gott danken.
\end{spacing}
\lied{Danke mein Vater}

\section{Predigt}

Danach gebe ich das Wort an Thomas weiter.

% \textbf{Nach der Predigt}

% Danken für die Predigt.

% \section{Abendmahl}

% Beten für das Brot

% \lied{Das Blut der Lammes 1. Strophe}

% Beten für den Wein

% \lied{Das Blut der Lammes 2. Strophe}

\section{Abschluss}

Jetzt wollen wir Gott mit dem Lied \lied{Herr du gibst uns Hoffnung} danken.

Bitte, denkt daran den Kaffee so zu bestellen und zu trinken, dass wir gemeinsam um 12:45Uhr runter zum Melodie laufen. Das Essen und Kuchen könnt ihr Janina mitgeben, die wird das dann mit dem Auto zum Melodie fahren.
\beten{}

\begin{bibelbox}{SCHL}{1Mos}{28:15}
Gott spricht: Siehe, ich bin mit dir,
ich behüte dich, wohin du auch gehst.
Denn ich verlasse dich nicht,
bis ich vollbringe, was ich dir versprochen habe.
\end{bibelbox}

Maranatha Amen
\end{document}