
\section{Begrüssung}

Ich möchte euch alle recht herzlich zu diesem besonderen Gottesdienst begrüssen. Schön das so viele gekommen sind, um unserem Herrn zu Danken ihn zu Loben und zu Preisen.

Besonders möchte ich Erich begrüssen, der den Weg von Zürich über Bern hier her gemacht hat, um uns das Wort des Herrn zu verkünden. 

\noindent
\beten{} und anschliessend singen wir zusammen das Lied

\noindent
\lied{Herr ich sehe deine Welt}.

\section{Ankündigungen}
\begin{itemize}
    \item \green{Bibel und Gebetsabend:} Do 13.3.2025 20:00Uhr Bibel und Gebetsabend mit Thomas Lieth der uns den Römerbrief auslegt.
    \item \green{Nächster Gottesdienst:} So 16.3.2025 14:45Uhr mit Thomas Lieth
    \item \green{Diverses:} Sonntag 30.3.2025 findet der Gottesdienst um 10:00Uhr statt. Ostersonntag findet hier kein Gottestdienst statt, das viele von uns an der Osterkonferenz in Dübendorf sind.
    \item \green{Die Kollekte:} Die Kollekte geht an den Mitternachtsruf und wird dort für die Missionsarbeit in der Welt eingesetzt.
\end{itemize}

\section{ Input }
\begin{spacing}{1.5}
\subsection{Gott ist gegenwärtig}
Werdet ihr auch manchmal von anderen Christen gefragt, an was ihr glaubt? Oder habt ihr schon mal andere gefragt, was sie glauben? Oft geht es ja darum zu erfahren von welcher Denomination unser gegenüber ist. Ob einer katholisch, reformiert, Freikirchler, Pfingstler, oder etwa Zeuge ist. Wenn eine von diesen Religionen als Antwort kommt, oder ihr so eine Antwort gebt, dann ist diese Antwort schlicht und einfach falsch.

Unsere Antwort sollte sein: \glqq An Jesus Christus, glaube ich! \grqq{}. Wir sollen nicht an eine Organisation glauben. Wir glauben nicht an den Mitternachtsruf, oder an die Bibel, oder an das Kreuz. Wir glauben, dass in der Bibel Gottes Wort steht und das Jesus am Kreuz für unsere Sünden gestorben ist, ja das glauben wir. Aber nicht an die Objekte selber. Apropo Kreuz: Ich habe gelesen, wenn man alle von dieser Welt stammenden Reliquien vom Kreuz, zusammenlegen würde, würde dieses Kreuz mehrere Tonnen schwer sein und Jesus musste diese Kreuz tragen.

Schon im alten Testament war Gott von seinem Volk enttäuscht, weil diese an die Opferung und Rituale glaubten, aber nicht an den \herr{} der sie aus Ägypten geführt hat. Hier ein paar Bibelstellen in den Gott sein Volk ermahnte, wieder zu ihm zurück zu kehren.
\begin{bibelbox}{ELB}{ISam}{15:22}
Samuel aber sprach: Hat der Herr so viel Lust an Brandopfer und Schlachtopfer wie daran, dass man der Stimme des \herr n gehorcht? Siehe, Gehorchen ist besser als Schlachtopfer, Aufmerken besser als das Fett der Widder.
\end{bibelbox}
\begin{bibelbox}{ELB}{Hos}{6:6}
 Denn an Güte habe ich Gefallen, nicht an Schlachtopfern, und an der Erkenntnis Gottes mehr als an Brandopfer.
\end{bibelbox}

Aber auch im neuen Testament war Jesus dieses Thema sehr wichtig.
Er klagte die Pharisäer oft an, dass sie Rituale vor Gott stellten. In Lukas 11:37 ff. konfrontiert Jesus die Pharisäer direkt mit diesem Problem. Weitere Stellen im Neuen Testament
\begin{bibelbox}{ELB}{Mt}{9:13}
Geht aber hin und lernt, was das ist: \glqq Ich will Barmherzigkeit und nicht Schlachtopfer.\grqq{} Denn ich bin nicht gekommen, Gerechte zu rufen sondern Sünder.
\end{bibelbox}
\begin{bibelbox}{ELB}{Mark}{7:7}
Er aber sprach zu ihnen: Treffend hat Jesaja über euch Heuchler geweissagt, wie geschrieben steht: \glqq Dieses Volk ehrt mich mit den Lippen, aber ihr Herz ist weit entfernt von mir. Vergeblich aber verehren sie mich, indem sie als Lehrer Menschengebote lehren.\grqq{}
\end{bibelbox}
Darum ist nicht dieser Gottesdienst an sich das wichtige, sondern mit welcher Herzenseinstellung wir an diesem Gottesdienst teilnehmen. Nicht der Gottesdienst ist das Ziel unseres Glaubens, sondern der allmächtige Gott, der uns all dies hier geschenkt hat. 

Dazu möchte ich euch noch einen Bericht von einem Chinesen, der in China wegen seines Glaubens ins Gefängnis kam vorlesen. Dieser Bericht hat mich zum Nachdenken gebracht, ob ich auch wirklich mit ganzem Herzen mich Gott übergebe oder es mir einfach nur wichtig ist, hier vorne toll auszusehen.

\green{Bericht}

Wollen wir doch nächste Woche, bewusst jeden Tag prüfen wo Gott uns leitet, hilft und uns seine Gnade schenkt.

\end{spacing}

\lied{Gott ist gegenwärtig}

\section{Predigt}

Danach gebe ich das Wort an Erich weiter.

\textbf{Nach der Predigt}

Danken für die Predigt.

\section{Abschluss}

Jetzt wollen wir Gott mit dem Lied \lied{Bald schon kann es sein} danken.


Vielen Dank für eure Teilnahme und das Gebet. Im Anschluss seid ihr zu Kaffee und guten Gesprächen eingeladen.
\beten{}

\begin{bibelbox}{SCHL}{1Mos}{28:15}
Gott spricht: Siehe, ich bin mit dir,
ich behüte dich, wohin du auch gehst.
Denn ich verlasse dich nicht,
bis ich vollbringe, was ich dir versprochen habe.
\end{bibelbox}

Maranatha Amen
