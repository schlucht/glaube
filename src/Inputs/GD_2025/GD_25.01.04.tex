
\section{Begrüssung}

Ich möchte euch alle recht herzlich zu diesem besonderen Gottesdienst begrüssen. Schön das so viele gekommen sind, um unserem Herrn zu Danken ihn zu Loben und zu Preisen.

Besonders möchte ich Nathanael begrüssen, der den Weg von Zürich über Bern hier her gemacht hat, um uns das Wort des Herrn zu verkünden. 

\noindent
\beten{} und anschliessend singen wir zusammen das Lied

\noindent
\lied{Herbei oh ihr Gläubigen}.

\section{Ankündigungen}
\begin{itemize}
    \item \green{Bibel und Gebetsabend:} Do 9.1.2025 20:00Uhr Es gibt keine Bibelauslegung aus Dübendorf. In Dübendorf findet eine Gebetswoche statt. Wir wollen an dem Abend im Gebet dem Herrn für 2024 Danken und ihn für Bewahrung 2025 bitten. Hat jemand Gebetsanliegen, können wir diese gerne mit aufnehmen.
    \item \green{Nächster Gottesdienst:} So 12.1.2025 15:00Uhr mit Philip Ottenburg
    \item \green{Diverses:} Am letzten Sonntag jeden Monats geht eine Delegation von hier nach Dübendorf an die Bibelschule. An diesem Sonntag sind sehr wenig Gottesdienst Besucher hier. Die Idee ist am dem Sonntag den Gottesdienste am Morgen um 10Uhr durchzuführen. Wir wollen gemeinsam den Livestream von Dübendorf als Gottesdienst anschauen. Eine gute Gelegenheit für Leute die keine Lust haben 15Uhr zum Gottesdienst zu kommen und den Mitternachtsruf kennen zu lernen.
    \item \green{Die Kollekte:} Die Kollekte geht an den Mitternachtsruf und wird dort für die Missionsarbeit in der Welt eingesetzt.
\end{itemize}

\section{ Input }
\begin{spacing}{1.5}
\subsection{ Jahres Aus- und Rückblick}

Das Jahr 2024 ist vorbei, das Jahr 2025 ist vor uns. Wir wissen nicht was uns erwartet und ich finde das ist auch gut so. Für den Jahresrückblick lese ich euch die Verse 1 und 4 von Psalm 23 vor.

\begin{bibelbox}{ELB}{Ps}{23:1}
Der Herr ist mein Hirte mir nichts mangeln.
\end{bibelbox}
\begin{bibelbox}{ELB}{Ps}{23:4}
Auch wenn ich wandere im Tal des Todesschattens, fürchte ich kein Unheil, denn du bist bei mir; dein Stecken und Stab, sie trösten mich.
\end{bibelbox}

Manche finden man soll nicht zurückblicken, ich aber finde man kann nur etwas lernen wenn man auf sein Tun zurückblickt. Macht das mal in eurer stillen Zeit. Nehmt die Verse 1 und 4 von Psalm 23 und geht durch das Jahr 2024. 

Schaut im Guten wie im Schlechten, wo der Stab Gottes sichtbar ist. Man übersieht es so gerne. Wenn ihr erkannt habt, wo Gottes Wirken im 2024 war, werden auch schlechte Erfahrungen zu dankbaren Gottessegnungen, was wiederum einem Zuversicht für 2025 gibt.

Für das Jahr 2025 habe ich für jeden einzelnen von euch eine Botschaft. Die möchte ich euch gerne Vorlesen. Es ist eine Andacht aus dem Buch "Sei guten Mutes" von Norbert Lieth.
Nach diesem Text singen wir zusammen ein Lied und danach möchte ich Nathanel bitten uns sein Wort weiter zu geben.

\end{spacing}

\lied{Majestät, herrliche Majestät}

\section{Predigt}
\green{Schriftlesung}

Danach gebe ich das Wort an Nathanael weiter.

\textbf{Nach der Predigt}

Danken für die Predigt.

\section{Abendmahl}

Beten für das Brot

\lied{Das Blut der Lammes 1. Strophe}

Beten für den Wein

\lied{Das Blut der Lammes 2. Strophe}


\section{Abschluss}

Jetzt wollen wir Gott mit dem Lied \lied{Vater, unser Vater } danken.


Vielen Dank für eure Teilnahme und das Gebet. Im Anschluss seid ihr zu Kaffee und guten Gesprächen eingeladen.
\beten{}

\begin{bibelbox}{SCHL}{1Mos}{28:15}
Gott spricht: Siehe, ich bin mit dir,
ich behüte dich, wohin du auch gehst.
Denn ich verlasse dich nicht,
bis ich vollbringe, was ich dir versprochen habe.
\end{bibelbox}

Maranatha Amen
