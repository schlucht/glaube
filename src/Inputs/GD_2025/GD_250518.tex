\author{OTS}
\documentclass{../../inc/mybib}

\setincpath{../../inc/}

\usepackage{bible_style}
\usepackage{header}

\begin{document}


\section{Begrüssung}

Ich möchte euch alle recht herzlich zu diesem besonderen Gottesdienst begrüssen. Schön das so viele gekommen sind, um unserem Herrn zu Danken ihn zu Loben und zu Preisen.
Auch möchte ich Nathanael begrüssen. 

\noindent
\beten{} und anschliessend singen wir zusammen das Lied

\noindent
\lied{Halleluja lobet Gott in seinem Heiligtum}.

\section{Ankündigungen}
\begin{itemize}
    \item \green{Bibel und Gebetsabend:} Do 22.05.2025 20:00 Uhr Bibel und Gebetsabend mit Thomas Lieth der uns ein weiterer Teil vom Römerbrief .
    \item \green{Nächster Gottesdienst:} So 25.05.2025 10:00 Uhr hier mit eine Übertragung vom Gottesdienst mit Nathanael.    
    \item \green{Allgemein:} Am Sonntag 17.August planen wir einen Gemeinde Tag. Der Gottesdienst wird mit Fredy Peter um 10 Ùhr statt finden. Danach wollen wir uns zu einem gemütlichen Nachmittag mit Grill zusammensetzen. Genaues Programm wird dann noch bekannt gegeben.
    \item \green{Die Kollekte:} Die Kollekte geht an den Mitternachtsruf und wird dort für die Missionsarbeit in der Welt eingesetzt.
\end{itemize}

\section{ Input }
\begin{spacing}{1.5}
\subsection{Was Glauben wir im MNR}
Heute kommen wir zum 3. Teil vom Glaubensbekenntnis, das der MNR auf ihrer Webseite verlinkt hat. Auch hier wieder, es ist wichtig, dass wir die einzelnen Aussagen mit der Bibel belegen. Wollen wir das zusammen anschauen. Dieses mal werde ich nicht mehr alle Bibelstellen vorlesen, sondern nur die, von denen ich denke, dass diese die Aussage besonders gut unterstreichen. Auf den Folien sind darum viel mehr Bibelstellen aufgeführt. 

Das Glaubensbekenntnis vom MNR kann man auf ihrer Webseite abrufen. Ihr findet es auf der Startseite von www.mnr.ch, unter dem Link \frqq Unser Glaubensbekenntnis \flqq{}. Es wird in 6 Teile unterteilt.
\aus{folie}
\begin{enumerate}
    \item \uppercase{über die bibel} 06.04.2025
    \item \uppercase{über gott und den Menschen} 04.05.2025
    \item \uppercase{über die Erlösung} 18.5.2025
    \item \uppercase{über die Gemeinde und Israel}
    \item \uppercase{über die Engel}
    \item \uppercase{über die letzten Dinge}
\end{enumerate}
Der dritte Punkt den wir heute anschauen, behandelt die Erlösung. Die Erlösung ist darum weil, hier beschrieben wird wie wir zu Gott kommen können. Im Alten Testament wird Gott oft gefragt wie sie ihn sehen können. Wir können unsere Bibel fragen. Jesus Chrsitus ist unser Weg. 
\begin{enumerate}      
      \item \textbf{Erlöst allein durch Glaube} \aus{folie}
        \begin{itemize}
            \item Wir glauben, dass jeder Mensch -- ob Jude oder Nichtjude -- allein durch den Glauben an Jesus Christus und nur aus Gnade mit Gott versöhnt und erlöst werden kann (Apg 4,12; Eph 2,1-10).
        \end{itemize}
        \item \textbf{Jeder wird errettet der den Herrn anruft}\aus{folie}
            \begin{itemize}
                \item Wir glauben, dass jeder errettet wird, der den Herrn anruft und glaubt und bekennt, dass Gottes Sohn, Jesus Christus, für seine Sünden gestorben, zu seiner Errettung auferstanden und sein Herr ist (Apg 2,21; Röm 1,16-17; 10,9-13; ).
            \end{itemize}            
        \item \textbf{Jeder Erlöste ist von Gott gerecht gesprochen}\aus{folie}
            \begin{itemize}
                \item Wir glauben, dass Gott alle Menschen zum Heil bestimmt hat (Hes 18,23; 1Tim 2,4; 2Pt 3,9). der Mensch kann sich zwar nur durch Gottes Wirken bekehren (Mt 11,27; Joh 6,44.65;), ist dann aber in seinem Willen gefordert, darauf einzugehen (Mk 16,16;). Gott hat bereits von Ewigkeit her in Seiner Allwissenheit die ersehen, die sich bekehren würden und all diejenigen dazu bestimmt, gerettet zu werden (Röm 8,29-30). Gott hat vor Grundlegung der Welt ausnahmslos jeden Christus dazu erwählt, gerettet zu werden, der sich Ihm zuwendet.
            \end{itemize}
        \item \textbf{Mit der Errettung für Zeit und Ewigkeit wiedergeboren und versiegelt}        
            \begin{itemize}
                \item Wir glauben, dass jeder mit der Errettung aus Gnade in Christus sogleich für Zeit und Ewigkeit wiedergeboren und mit dem Heiligen Geist versiegelt wird. UNd als Miterbe Christi ist jedem von Gott Erretteten die Erlösung sowie ein Erbteil im Himmel sicher (Joh 10,27-29; Röm 8,17; Tit 3,5).               
            \end{itemize}              
            \newpage
        \item \textbf{Jeder Erlöste ist in Christus von Gott völlig gerecht gesprochen und geheiligt }\aus{folie}
            \begin{itemize}
                \item Wir glauben, dass jeder Erlöste in Christus von Gott völlig gerecht gesprochen und geheiligt ist (1Kor 6,11). Der Erlöste erhält die Gerechtigkeit Christi (1Kor 1,30), weil Christus alle Sünden und Schuld des Erlösten auf sich genommen und bezahlt hat (2Kor 5,21;). Auch wenn der Erlöste in der Stellung vor Gott bereits völlig geheiligt ist (Hebr 10,14), ist er dennoch dazu aufgerufen, in der Praxis nach der Heiligung zu streben (1Pt 1,15), würdig der Berufung und gemäss dem Bild Gottes und Seiner Schöpfungsordnung als Tempel des Heiligen Geistes zu leben.              
            \end{itemize} 
\end{enumerate}
\end{spacing}
\lied{Ich weiss nicht, warum Gottes Gnad}

\section{Predigt}

Danach gebe ich das Wort an Nathanael weiter.

\textbf{Nach der Predigt}

Danken für die Predigt.

% \section{Abendmahl}

% Beten für das Brot

% \lied{Das Blut der Lammes 1. Strophe}

% Beten für den Wein

% \lied{Das Blut der Lammes 2. Strophe}

\section{Abschluss}

Jetzt wollen wir Gott mit dem Lied \lied{Ich hab einen herrlichen K"onig} danken.

Vielen Dank für eure Teilnahme am Gottesdienst. Im Anschluss seid ihr zu Kaffee und guten Gesprächen eingeladen.
\beten{}

\begin{bibelbox}{SCHL}{1Mos}{28:15}
Gott spricht: Siehe, ich bin mit dir,
ich behüte dich, wohin du auch gehst.
Denn ich verlasse dich nicht,å
bis ich vollbringe, was ich dir versprochen habe.
\end{bibelbox}

Maranatha Amen
\end{document}