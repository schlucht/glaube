
\section{Begrüssung}

Ich möchte euch alle recht herzlich zu diesem besonderen Gottesdienst begrüssen.

Auch möchte ich Florian recht herzlich begrüßen, der den Weg von Bern hier her gemacht hat, um uns das Wort des Herrn zu verkünden. 

\noindent
\beten{} und anschließend singen wir zusammen das Lied

\noindent
\lied{Herr du hast uns gerufen}.

\section{Ankündigungen}
\begin{itemize}
    \item \green{Bibel und Gebetsabend:} Do 13.2.2025 20:00 Uhr mit Saumel Rindlisbacher  ab Römer 12:14-16
    \item \green{Nächster Gottesdienst:} So 16.1.2025 14:45 Uhr Hier Gottesdienst mit Thomas Lieth
    \item \green{Diverses:} Am Donnerstag ab 17:30 Uhr ist hier der Büchertisch und für Informationen geöffnet, falls ihr jemanden kennt, der sich gerne über uns Informieren möchte.
    \item \green{Die Kollekte:} Die Kollekte geht an den Mitternachtsruf und wird dort für die Missionsarbeit in der Welt eingesetzt.
\end{itemize}

\section{ Input }
\begin{spacing}{1.5}
\subsection{ Wieso Gottesdienstbesuch? }
Als Kind sind wir regelmäßig jeden Sonntag in die Kirche gegangen und wenn wir am Sonntag keine Zeit hatten, dann am Samstag Abend. Wobei ich mich jetzt nicht daran erinnern kann, ob mein Vater auch Samstag Abend zum Gottesdienst ging. Irgend wann habe ich dann auch gefragt, wieso dass ich jeden Sonntag zum Gottesdienst muss. So richtig erklären konnte er mir das auch nicht. "Das sei halt mal so. Es wird halt gesagt, dass man in den Himmel kommt." Ok, in den Gottesdienst gehen um in den Himmel zu kommen klang logisch. Ich war in jungen Jahren ein fleißiger Kirchgänger und Messdiener. Aber irgendwann zog das Argument, mit in den Himmel kommen nicht mehr. Vor allen in Birgisch. Nach dem Amt ging es dann neben an ins Restaurant zum Apero und da ging es nicht immer so Christlich zu und her. Ab 17-18 ließ ich es dann mit den Gottesdienst besuchen. Zu Beerdigungen oder andere Anlässe, sonst ging ich nicht mehr hin.

Wieso seid ihr denn heute hier? Um in den Himmel zu kommen? Hoffentlich nicht! Wegen Kaffee und Kuchen? Kann ein Grund sein! Vielleicht um eine Lebensweisheiten zu kriegen? Psychologische Ratschläge? Oder wegen der Gesellschaft? Ist noch ein besserer Grund! Was sagt denn Gottes Wort wieso wir denn in den Gottesdienst gehen sollen? (1Tim 3,15; 2Tim 2,2; Tit 1,5) Sagt sie überhaupt was dazu? In der Bibel steht nicht explizit drin, wie der Gottesdienst auszusehen hat, oder das er immer am Sonntag stattfinden soll, aber die Bibel sagt uns, dass wir uns regelmäßig versammeln sollen und zwar um Gott zu danken, ihn zu loben und zu preisen. Aber auch für die Gemeinschaft untereinander. Wir sollen einander unterstützen und helfen.( Hebr 10,25) Für einander beten. Außerdem sollen wir den Gottesdienst besuchen um etwas über Gott und unseren Glauben zu lernen. Der Gottesdienst ist in erster Linie für wiedergeborenen Christen um zu lernen, Gemeinschaft zu pflegen und erst in zweiter Linie zur Evangelisation. Zur Evangelisation sind wir alle berufen und nicht nur die Gemeinde.(Mt 28,18-20; Mk 16,15; Apg 2,42;) Aber wir können nur evangelisieren, wenn wir wissen um was es überhaupt geht. Darum ist es wichtig zu lernen. Ihr kennt doch sicher alle das Glaubensbekenntnis vom Mitternachtsruf da steht:
\begin{quote}
    Wir glauben, dass Gott lokale Gemeinden eingesetzt hat, damit sich die Christen zum Gottesdienst, zur Gemeinschaft und zur Unterrichtung in der Lehre dort versammeln und durch ihre Aufgaben und Gaben den Herrn verherrlichen, der Gemeinde dienen und das Evangelium in aller Welt verbreiten 
\end{quote}
(1Kor 11,18-20; Eph 3,21;).
Und weil wir zum Mitternachtsruf gehören, stehen wir auch zu dieser Aussage. Also lernen wir das Wort Gottes fleißig und pflegen die Gemeinschaft untereinander. Wer Gebetsanliegen hat, kann die gerne uns über die Email Adresse mnr-naters@jagolo.ch Mitteilen. So dass wir am Donnerstag Abend gemeinsam Beten über die Anliegen beten können.

Bevor uns aber Florian unterrichtet, wollen wir unserem Herrn nochmals ein Lied singen. Danach möchte ich Florian nach vorne bitten.

\glqq Vor alledem \grqq{} 

\end{spacing}

\lied{Gut das wir einander haben}

\section{Predigt}
\green{Schriftlesung}

Danach gebe ich das Wort an Florian weiter.

\textbf{Nach der Predigt}

Danken für die Predigt

\section{Abschluss}

Jetzt wollen wir Gott mit dem Lied \lied{Wir haben eine Hoffnung} danken.


Vielen Dank für eure Teilnahme und das Gebet. Im Anschluss seid ihr zu Kaffee und guten Gesprächen eingeladen.
\beten{}

\begin{bibelbox}{SCHL}{Eph}{3:20-21}
Dem aber, der weit über die Maßen mehr zu tun vermag als wir bitten oder verstehen, gemäß der Kraft, die in uns wirkt, ihm sei die Ehre in der Gemeinde in Christus Jesus, auf alle Geschlechter der Ewigkeit der Ewigkeiten! Amen.
\end{bibelbox}

Maranatha Amen
