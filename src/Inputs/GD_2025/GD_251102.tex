\author{OTS}
\documentclass{../../inc/mybib}

\setincpath{../../inc/}

\usepackage{bible_style}
\graphicspath{{../../assets/images/}}
\usepackage{header}

\begin{document}

\section{Begrüssung}

Ich möchte euch alle recht herzlich zu diesem Gottesdienst begrüssen. Schön das ihr gekommen seid, um unserem Herrn zu Danken ihn zu Loben und zu Preisen.
Hallo Nathanel wir freuen uns, dich hier im Oberwallis in Naters begrüssen zu können. Wir sind gespannt auf dein Wort und freuen uns darauf.\\
Heute feiern wir zusammen das Abendmal, mit dem wir an den Bund mit Gott gedenken und nicht vergessen wollen, was Jesus für uns am Kreuz vollbracht hat.

% \noindent
\beten{} Und anschliessend singen wir zusammen das Lied

% \noindent
\lied{Du bist der Weg und das Leben}.

\section{Ankündigungen}
\begin{itemize}
    \item \green{Bibel und Gebetsabend:} Do, 6.11.2025 20:00 Uhr Bibel und Gebetsabend mit Nathanel Winkler mit Markus 2,23-28.
    \item \green{Nächster Gottesdienst:} So, 09.11.2025 14:45 Uhr hier mit Fredy Peter.
    \item \green{Allgemein:} Janina und ich sind ab Dienstag zwei Wochen unterwegs in Israel. Gottesdienst und Gebetsabende laufen normal weiter. Es geht ja auch nicht um Janina und mir, sondern um unseren unvergleichlichen grossen Gott. 
    \item \green{Die Kollekte:} Die Kollekte geht an den MNR und wird für den Bau dieser Gemeinde eingesetzt.
\end{itemize}

\section{ Input }
\begin{spacing}{1.5}
    \begin{block}[Paradies]
        Vom Paradies bis zum Sündenfall und der Vertreibung aus dem Paradies haben wir uns an den letzten Sonntagen beschäftigt. Da haben wir gelernt, dass Gott uns Menschen einen Auftrag gegeben hat
        \begin{bibelbox}{SCH}{IMos}{1:28}
            Und Gott segnete sie; und Gott sprach zu ihnen: Seid fruchtbar und mehret euch und füllt die Erde und macht sie euch untertan; und herrscht über die Fische im Meer und über die Vögel des Himmels und über alles Lebendige, das sich regt auf der Erde.
        \end{bibelbox}
        Das ist der Auftrag an uns. Das Gebot, welches Gott den beiden Menschen damals gab, nicht von der Frucht vom Baum der Erkentniss zu essen wurde nicht eingehalten. Zur Strafe mussten sie das Paradies verlassen, es kam der Tod und die Sünde ins Leben. Der Satan wurde der König der Welt. Den Auftrag der Gott uns gegeben hat, gilt immer noch, aber er wurde erschwert. Wir müssen jetzt im Schweisse des Angesichts die Erde bebauen und die Frauen müssen unter Schmerzen gebären. Aber der Auftrag steht immer noch.

        Gott hat den Menschen nicht verworfen, sondern er hat weiterhin einen Plan mit ihm. Diesen Plan was beschreibt er in 
        \begin{bibelbox}{SCH}{IMos}{3:15}
            Und ich will Feindschaft setzten zwischen dir und der Frau; zwischen deinem Samen und ihren Samen: Er wird dir den Kopf zertreten und du wirst ihn in die Ferse stechen.
        \end{bibelbox}
        Heute wissen wir, dass dieser Fürst unser Herr Jesus Christus ist. Damals wussten sie zwar, dass jemand kommen wird, aber sie wussten nicht, wer oder wann dieser ist und kommt.
        Hier kommen wir nun zu der Phase 3 in Gottes Plan sein Königreich aufzurichten. Diese Phase dauert in der Bibel von der Vertreibung aus dem Paradies bis zu Maleachi und zum Auftritt von Jesus Christus. Eine Dauer von \red{XXXXX} Jahren laut Roger Liebi.

        In dieser Zeit hat Gott aber nicht tatenlos zugesehen wie die Menschheit verkommt und ins Verderben rennt, sondern er hat einen Plan, den er durchzieht. Und dieser Plan wird uns in den nächsten Sonntagen beschäftigen. Es gibt verschiedene Bünde die Gott mit uns Menschen geschlossen hat um zum Ziel zu kommen. Das kommen von Jesus wird mit der Zeit immer konkreter.
        \begin{itemize}
            \item 
        \end{itemize}
    \end{block}
    \begin{block}[Noah Bund]
    \end{block}
    \begin{block}[Abraham Bund]
    \end{block}
    \begin{block}[Mose Bund]
    \end{block}
    \begin{block}[David Bund]
    \end{block}
    \begin{block}[Neuer Bund]
    \end{block}

\end{spacing}
\lied{Danke mein Vater}

\section{Predigt}

Danach gebe ich das Wort an Nathanel weiter.

% \textbf{Nach der Predigt}

% Danken für die Predigt.

\section{Abendmahl}

Beten für das Brot

\lied{Das Blut der Lammes 1. Strophe}

Beten für den Wein

\lied{Das Blut der Lammes 2. Strophe}

\section{Abschluss}

Jetzt wollen wir Gott mit dem Lied \lied{Herr du gibst uns Hoffnung} danken.

Vielen Dank für eure Teilnahme am Gottesdienst. Im Anschluss seid ihr zu Kaffee und guten Gesprächen eingeladen.
\beten{}

\begin{bibelbox}{SCHL}{1Mos}{28:15}
Gott spricht: Siehe, ich bin mit dir,
ich behüte dich, wohin du auch gehst.
Denn ich verlasse dich nicht,
bis ich vollbringe, was ich dir versprochen habe.
\end{bibelbox}

Maranatha Amen
\end{document}