\author{OTS}
\documentclass{../../inc/mybib}

\setincpath{../../inc/}

\usepackage{bible_style}
\graphicspath{{../../assets/images/}}
\usepackage{header}

\begin{document}

\section{Begrüssung}
Ich möchte alle recht herzlich zu unserem heutigen Gottesdienst begrüssen. Schön, dass ihr alle hier seid. Wollen wir gemeinsam Gott Loben und Preisen. Heute feiern wir gemeinsam Abendmahl. Was immer etwas besonderes ist. 
\begin{block}
Wie ihr schon gesehen habt, ist heute nicht wie eigentlich geplant Nathanael hier, sondern Fredy Peter. Wir freuen uns Fredy, dass du heute hier bist. Fredy ist unser Mentor und Ältester und so auch das Bindeglied zwischen dem MNR -- Hauptsitz und uns hier im Wallis Du hast für uns eine Organisatorische Ansage zu machen. Und so möchte ich dich nach meinem Gebet nach vorne bitten und uns deine Informationen weitergeben
\end{block}

\beten{}
\red{Fredy Peter}\\

% \noindent
Vielen Dank Fredy, obwohl du mitten in den Vorbereitungen deiner Schweizer Tour bist, hast du diese Nachricht persönlich gebracht. Ich finde, dass eine riesen Wertschätzung gegenüber unseren kleinen Gruppe ist.\\

Singen wir nun zusammen unser erstes Lied.
% \noindent
\lied{Niemand ist so true und gut wie Jesus}.

\section{Ankündigungen}
\begin{itemize}
    \item \green{Bibel und Gebetsabend:} Do, 11.09.2025 20:00 Uhr Bibel und Gebetsabend Markus 1,29-34 mit Norbert.
    \item \green{Nächster Gottesdienst:} So, 14.09.2025 14:45 Uhr Gottesdienst mit Samuel Rindlisbacher
    \item \green{Allgemein:} Wie ihr hier auf dem Flyer gesehen habt, startet Fredy 9.09 seine Schweizer Tour. Auf dieser Tour wird er am 18.09.2025 um 19:30Uhr wieder hier bei uns in diesem Lokal sein und seinen Vortrag mit dem Thema \textbf{Die herrlichen Auswirkungen der Entrückung}. Tönt spannend Fredy. Schlagt also kräftig die Werbetrommel, damit viele Leute auch von diesen herrlichen Auswirkungen der Entrückung erfahren dürfen.
    \item \green{Allgemein:} Die Israel Konferenz steht wieder vor der Tür. Sie findet vom 27.09 bis 28.09.2025 in Zürich statt. An diesem Sonntag findet hier bei uns keinen Gottesdienst. Janina und ich werden dort teilnehmen. Wer von euch auch daran teilnehmen möchte, kann sich gerne bei mir oder Janina melden und wir können uns am Sonntag in Dübendorf nach dem Gottesdienst zum Mittagessen treffen.
    \item Die Kollekte wir hinten in der Grünen Trommel gesammelt und für den Bau dieser Gemeinde verwendet.
\end{itemize}

\section{ Input }
\begin{spacing}{1.5}    
   \textbf{\enquote{YES WE CAN}} der Wahlslogan von Barack Obama 2008. Dann ist da noch die Angela Merkel die 2005 mit dem Satz \enquote{Wir schaffen das}, die Flüchtlingskrise in Deutschland meistern wollte. Donald Trump der 2016 mit dem Slogan \enquote{Make America Great Again} die Präsidentschaftswahl in den USA gewann. Ein Milei mit der Kettensäge der Argentinen reformieren will. Alles Männer und Frauen, die an der Spitze von Nationen stehen und Verantwortung über Millionen von Menschen haben.
    \begin{block}
    Grosse Worte, die grosse Taten versprechen und den Menschen Hoffnung geben sollen. Wie weit gingen diese Versprechungen in Erfüllung? Das Gesundheitssystem in den USA ist immer noch schlecht. Die Flüchtlingskrise in Deutschland ist immer noch nicht bewältigt und in Argentinen sind wieder vermehrt unruhen.
    \end{block}
    \begin{block}
    Aber doch möchte auch ich heute diese Worte aussprechen \enquote{textbf{YES WE CAN}}. Das WIR meint nicht nur uns, dass WIR die, die wir hier sitzen, schaffen das nicht alleine. Nur mit Jesus im Boot werden wir es schaffen. 
    \end{block}
    Als der Sturm auf dem See Genezaret tobte, war Jesus mit auf dem Boot und hat seelenruhig geschlafen. Die Jünger hatten Angst und weckten Jesus auf. Jesus stand auf und sprach zu dem Sturm \enquote{Schweig, sei still} und der Sturm legte sich. Wir können es schaffen, wenn Jesus mit im Boot ist. Die Jünger mussten zuerst Jesus aufwecken. Auch wir müssen Jesus aufwecken. Jesus war aber da, er war bereit den Sturm zu stillen. Jesus sagte zu den Jüngern \enquote{Warum habt ihr Angst, ihr Kleingläubigen?} Ohne Jesus wären das Schiff versunken. Lass uns Jesus mit unseren Taten, Worten und Gebeten aufwecken. So wird er uns durch stürmische Zeiten bringen.

    \textbf{\enquote{JA, wir können es, nur mit Jesus!}}

\end{spacing}
\lied{Näher zu dir}

\section{Predigt}

Fredy: Kein Grund zur Entmutigung!
Dazu möchte ich euch die Bibelverse \bibleverse{IIKor}(4:16-18) vorlesen.

%  \textbf{Nach der Predigt}
%  Danken für die Predigt.

 \section{Abendmahl}
 Beten Nick und Lothar evt. Ueli
 Beten für das Brot 

 \lied{Das Blut der Lammes 1. Strophe}

 Beten für den Wein

 \lied{Das Blut der Lammes 2. Strophe}

\section{Abschluss}

 \lied{Wenn Friede mit Gott}.

\beten{}

\begin{bibelbox}{SCHL}{1Mos}{28:15}
    Gott spricht: Siehe, ich bin mit dir,
    ich behüte dich, wohin du auch gehst.
    Denn ich verlasse dich nicht,
    bis ich vollbringe, was ich dir versprochen habe.
\end{bibelbox}

Maranatha Amen
Leider gibs am Heute nur noch Kapselkaffee. Aber ich freue mich auf die Gespräche.
\end{document}
