\author{OTS}
\documentclass{../../inc/mybib}

\setincpath{../../inc/}

\usepackage{bible_style}
\graphicspath{{../../assets/images/}}
\usepackage{header}

% ensure scrlayer-scrpage has sufficient footheight
\setlength{\footheight}{20.4pt}

\begin{document}

\section{Begrüssung}
Hallo Nathanael wir freuen uns, dich in Naters begrüssen zu können. Wir sind gespannt auf dein Wort.

Ich möchte auch euch alle recht herzlich zu diesem Gottesdienst begrüssen. Schön, dass ihr gekommen seid, um unserem \herr N zu Danken, ihn zu loben und zu preisen.

% \noindent
\beten{} Und anschliessend singen wir zusammen das Lied

% \noindent

\lied{Gott wurde arm für uns}
\section{Ankündigungen}
\begin{itemize}
    \item \green{Bibel und Gebetsabend:} Do, 11.12.2025 20:00 Uhr Bibel und Gebetsabend mit Fredy Peter; zu Markus 3,20-30.
    \item \green{Nächster Gottesdienst:} So, 14.12.2025 14:45 Uhr hier mit Philipp Ottenburg.
    \item \green{Allgemein:} So, 21.12.2025 14:45 Uhr wollen wir einen Weihnachtsgottesdienst feiern mit anschliessendem, gemütlichen Beisammensein.
    \item \green{Die Kollekte:} Die Kollekte wird diesen und noch nächsten Sonntag für den Kauf der neuen Möbel verwendet. Ihr könnt etwas hinten in die grüne Tonne legen oder mit Twint an Janina schicken. Alle Spenden werden anonym behandelt. Die Liste und die Quittungen können gerne eingesehen werden.
\end{itemize}

\section{ Input }
\begin{spacing}{1.5}
    \begin{block}[Einführung]
        Wir sind immer noch in der Phase 3, in der langen Zeit von Adam und Eva bis Maleachi. Letzten Sonntag haben wird den Bund mit Noah besprochen und gesehen, wie dieser Bund noch heute für uns Gültigkeit hat und uns Mut gibt.

        Heute wollen wir uns mit dem Bund, den Gott mit Abraham geschlossen hat, beschäftigen. Nicht lange nachdem Noah gestorben war, hat Gott Abraham als sein Werkzeug für seine Heilsgeschichte auserwählt. Abraham war 75 Jahre alt, als Gott ihn erwählt hat.
        
        Gott wollte von ihm, dass er den folgenden Bedingungen Folge leistet:
        \begin{itemize}
        	\item Sich von der Religion und seiner Kultur lösen.
        	\item Sich von seiner Verwandtschaft lösen.
        	\item Sich von seiner Familie lösen.
        \end{itemize}
        \begin{bibelbox}{SCHL}{1Mos}{12:1}
            Der \herr{} aber hatte zu Abraham gesprochen: Geh hinaus aus deinem Land und aus deiner Verwandtschaft und aus dem Haus deines Vaters in das Land, das ich dir zeigen werde!
        \end{bibelbox}
        Mit 75? Seine Familie war ja jetzt nicht gerade Gross. Hätte Abraham dem \herr N gehorcht, so wären sie gerade mal -- ohne Angestellte -- zwei Personen gewesen. Aber wie es halt so ist, Gott gibt konkrete Anweisungen und der Mensch wandelt diese dann ein bisschen zu seinen Gunsten um. So auch hier. Wobei ich da absolutes Verständnis habe. Es waren ja nicht gerade sichere Zeiten und so ist man sicherer, wenn man eine grössere Gruppe ist. Es ist auch einfacher das Unbekannte zu bewältigen.
        \begin{bibelbox}{SCHL}{1Mos}{12:5}
            Und Abraham, nahm seine Frau Sarai und Lot, den Sohn seines Bruders, samt all ihrer Habe, die sie erworben hatten, und den Seelen, die sie in Haran gewonnen hatten; und sie zogen aus, um ins Land Kanaan zu gehen; und sie kamen in das Land Kanaan.
        \end{bibelbox}
        Die Probleme, die Abraham dann mit Lot hatte, kennt ihr alle. Da waren die Probleme mit den Hirten, die Entführung Lots, Sodom und Gomorra und später die Inzucht.
        
        Gott hatte aber mit Abraham etwas Grosses vor: Gott wollte mit Abraham einen Bund schliessen. Einen Bund, der sowohl für alle Menschen gilt, als auch Versprechen, die das Volk Israel betreffen, enthalten.

        Abraham bekommt von Gott folgendes Verheissen:
        \begin{itemize}
            \item Er wird ein Land bekommen, das Gott ihm zeigen wird.
            \item Er wird eine grosse Nation werden.
            \item Er wird einen grossen Namen haben.
            \item Er wird gesegnet werden.
            \item Er wird ein Segen sein.
        \end{itemize}
        \begin{bibelbox}{SCHL}{1Mos}{12:2-3}
            Und ich will dich zu einem grossen Volk machen und dich segnen und deinen Namen gross machen, und du sollst ein Segen sein. Ich will segnen, die dich segnen, und verfluchen, die dich verfluchen; und in dir soll gesegnet werden \betonung{alle} Geschlechter auf der Erde!
        \end{bibelbox}
        Der Segen und Fluch der Einzelnen und der Nationen hängen untrennbar mit ihrer Stellung zu Abraham und seinen Nachkommen zusammen.

        Es gibt noch weitere Versprechen unseres \herr N an Abraham und seinen Nachkommen. Abraham hatte ja zwei Söhne, Ismael und Isaak. Mit Isaak ging die Linie weiter zum Volk Israel. Aber auch Ismael wurde von Gott gesegnet. So lesen wir in:
        \begin{bibelbox}{SCHL}{1Mos}{17:20}
            Wegen Ismael aber habe ich dich auch erhört. Siehe, ich habe ihn reichlich gesegnet und will ihn fruchtbar machen und sehr mehren. Er wird zwölf Fürsten zeugen, und ich will ihn zu einem grossen Volk machen.
        \end{bibelbox}
        Ismael wird heute als Vorfahre der Araber angesehen. Darum sehen auch die Moslems Abraham als ihren Stammvater an.

        Der Bund, den wir vorher betrachtet haben, die Landverheissung, der Segen usw. wird aber nur Abraham und seinen Nachkommen gegeben.
        \begin{bibelbox}{SCHL}{1Mos}{17:21}
            Meinen Bund aber will ich mit Isaak aufrichten, den dir Sarah um diese bestimmte Zeit im nächsten Jahr gebären soll!
        \end{bibelbox}
        Nachlesen könnt ihr das alles diese Woche in 1. Mose, 12.13.15 und 17
    \end{block}
  	\begin{block}[Fazit]
  		Wie schon der Bund mit Noah, ist dieser Bund einseitig. Es gibt keine Bedingungen. Gott hat diesen Bund mit Abraham, mit dem alleinigen Durchschreiten der Tierhälften besiegelt. Der Bund mit uns Menschen durch Noah wurde damit aber nicht aufgelöst.

        Abraham war 75 Jahre alt, als er von unserem \herr N berufen wurde. Ich war 48 als mich Jesus Christus durch meine Janina zu sich gezogen hat. Vor zwei Jahren, also mit 56, habe ich hier in dieser Gemeinde angefangen. Das ist für mich eine Berufung von Gott. Abraham und Sarah waren zwei Personen, die nicht nach eigenem Willen, sondern nach Gottes Willen und durch Glauben gehandelt haben. So heisst es:
        \begin{bibelbox}{SCHL}{Heb}{11:8}
            Durch Glauben gehorchte Abraham, als er berufen wurde, nach dem Ort auszuziehen, den er als Erbteil empfangen sollte; und er zog aus, ohne zu wissen, wohin er kommen werde.
        \end{bibelbox}
        \dots Und
        \begin{bibelbox}{SCHL}{Heb}{11:11}
            Durch Glauben erhielt auch Sarah selbst die Kraft, schwanger zu werden, und sie gebar, obwohl sie über das geeignete Alter hinaus war, weil sie den für true achtete, der es verheissen hatte.
        \end{bibelbox}
        Also, nicht weil sie so gebildet, oder so speziel waren, sondern weil sie unserem \herr N geglaubt haben und ihm Treu waren. Das Resultat sehen wir jetzt. Zwar ein halsstarriges Volk, das die ganze Welt in Atem hält, aber von Gott gesegnet. Und Gott wird auch mit seinem Volk zum Ziel kommen.
        
        Was müssen \betonung{wir} machen, um hier in Naters vorwärtszukommen? Schönere Gottesdienste? Noch besserer Prediger? (Wenn es sowas gibt) Super Musiker und Sänger? Viele Anlässe? Guten Kaffee? Alles wichtige Dinge und gute Sachen, aber unsere Gemeinde wird nicht darum \betonung{(vorallen in die richtige Richtung)} wachsen, sondern durch den Glauben an unseren \herr N und Heiland Jesus Christus. In seinem Namen sollen wir hier Schritt für Schritt vorwärtsgehen. Auf ihn schauend bleiben wir Jesus zentriert. Und wenn wir im Glauben treu bleiben, kann Gott mit uns etwas anfangen. Und nur so werden wir an den Segnungen, die Abraham versprochen wurden, teilhaben.

        Auch wir sind nicht mehr jung. Das ist Gott aber egal, und uns erst recht. Oder? 
  	\end{block} 
    Nun Singen wir zusammen ein Lied und nach dem Lied möchte ich Nathanael das Wort übergeben.
    \lied{In dir ist Freude}.
   
\end{spacing}

\section{Predigt}



% \textbf{Nach der Predigt}

% Danken für die Predigt.

\section{Abendmahl}

Beten für das Brot

\lied{Das Blut der Lammes 1. Strophe}

Beten für den Wein

\lied{Das Blut der Lammes 2. Strophe}

\section{Abschluss}

Jetzt wollen wir Gott mit dem Lied \lied{Schau ich zurück} danken.

Vielen Dank für eure Teilnahme am Gottesdienst. Im Anschluss seid ihr zu Kaffee und guten Gesprächen eingeladen.
\beten{}

\begin{bibelbox}{SCHL}{1Mos}{28:15}
Gott spricht: Siehe, ich bin mit dir,
ich behüte dich, wohin du auch gehst.
Denn ich verlasse dich nicht,
bis ich vollbringe, was ich dir versprochen habe.
\end{bibelbox}

Maranatha, komm Herr Jesus! Amen
\end{document}