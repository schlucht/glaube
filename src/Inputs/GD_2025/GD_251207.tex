\author{OTS}
\documentclass{../../inc/mybib}

\setincpath{../../inc/}

\usepackage{bible_style}
\graphicspath{{../../assets/images/}}
\usepackage{header}

% ensure scrlayer-scrpage has sufficient footheight
\setlength{\footheight}{20.4pt}

\begin{document}

\section{Begrüssung}
Hallo Nathanael wir freuen uns, dich in Naters begrüssen zu können. Wir sind gespannt auf dein Wort.

Ich möchte auch euch alle recht herzlich zu diesem Gottesdienst begrüssen. Schön, dass ihr gekommen seid, um unserem \herr N zu Danken, ihn zu loben und zu preisen.

% \noindent
\beten{} Und anschliessend singen wir zusammen das Lied

% \noindent
\lied{Gross ist dein Name}.

\section{Ankündigungen}
\begin{itemize}
    \item \green{Bibel und Gebetsabend:} Do, 11.12.2025 20:00 Uhr Bibel und Gebetsabend .
    \item \green{Nächster Gottesdienst:} So, 14.12.2025 14:45 Uhr hier mit Philipp Ottenburg.
    
    \item \green{Die Kollekte:} Die Kollekte wird diesen und nächsten Sonntag für den Kauf der neuen Möbel verwendet. Ihr könnt etwas hinten in die grüne Tonne legen oder mit Twint an Janina schicken. Alle Spenden werden anonym behandelt. Die Liste und die Quittungen können gerne eingesehen werden.
\end{itemize}

\section{ Input }
\begin{spacing}{1.5}
    \begin{block}[Einführung]
        Wir sind immer noch in der Phase 3, in der langen Zeit von Adam und Eva bis Maleachi. Letzten Sonntag haben wird den Bund mit Noah besprochen und gesehen, wie dieser Bund noch heute für uns Gültigkeit hat und uns Mut gibt.

        Heute wollen wir gemeinsam den Bund mit Abraham anschauen. Dieser Bund wird in den Kapiteln 1. Mose 14 und 1. Mose \red{Bibelstelle} beschrieben. Wir wollen gemeinsam diesen Text lesen.
        \begin{bibelbox}{SCHL}{1Mos}{4:26}
            Und dem Seth, auch ihm wurde ein Sohn geboren, und er gab ihm den Namen Enosch. Danach fing man an den Namen des \herr N anzurufen.
        \end{bibelbox}
        
    \end{block}    
    Singen wir zusammen ein Lied und nach dem Lied möchte ich Nathanael das Wort übergeben.
   
\end{spacing}
\lied{Du bist Emmanuel}
\section{Predigt}



% \textbf{Nach der Predigt}

% Danken für die Predigt.

% \section{Abendmahl}

% Beten für das Brot

% \lied{Das Blut der Lammes 1. Strophe}

% Beten für den Wein

% \lied{Das Blut der Lammes 2. Strophe}

\section{Abschluss}

Jetzt wollen wir Gott mit dem Lied \lied{Denn ich bin gewiss} danken.

Vielen Dank für eure Teilnahme am Gottesdienst. Im Anschluss seid ihr zu Kaffee und guten Gesprächen eingeladen.
\beten{}

\begin{bibelbox}{SCHL}{1Mos}{28:15}
Gott spricht: Siehe, ich bin mit dir,
ich behüte dich, wohin du auch gehst.
Denn ich verlasse dich nicht,
bis ich vollbringe, was ich dir versprochen habe.
\end{bibelbox}

Maranatha Amen
\end{document}