\author{OTS}
\documentclass{../../inc/mybib}

\setincpath{../../inc/}

\usepackage{bible_style}
\usepackage{header}

\begin{document}


\section{Begrüssung}

Ich möchte euch alle recht herzlich zu diesem besonderen Gottesdienst begrüssen. Schön das ihr gekommen seid, um unserem Herrn zu Danken ihn zu Loben und zu Preisen.
Auch möchte ich Thomas Lieth begrüssen, der heute Morgen in Bern dem Wort diente und jetzt hier bei uns im Wallis ist. 

% \noindent
\beten{} und anschliessend singen wir zusammen das Lied

% \noindent
\lied{Zünde an dein Feuer}.

\section{Ankündigungen}
\begin{itemize}
    \item \green{Bibel und Gebetsabend:} Do 3.07.2025 20:00 Uhr Bibel und Gebetsabend mit Thomas Lieth  der uns ein weiterer Teil vom Römerbrief auslegt.
    \item \green{Nächster Gottesdienst:} So 06.07.2025 10:00 Uhr hier mit eine Übertragung vom Gottesdienst mit Nathanael.
    \item \green{Allgemein:} Am Sonntag 17.August planen wir einen Gemeinde Tag. Der Gottesdienst wird mit Fredy Peter um 10 Uhr statt finden. Danach wollen wir uns zu einem gemütlichen Nachmittag mit Grill zusammensetzen. Genaues Programm wird dann noch bekannt gegeben.
    \item \green{Die Kollekte:} Die Kollekte geht an den MNR und wird für den Bau dieser Gemeinde eingesetzt.
\end{itemize}

\section{ Input }
\begin{spacing}{1.5}

Kommen wir zum letzten Teil aus unserer Reihe, das Glaubensbekenntnis vom Mitternachtsruf kennen zu lernen. Ziel ist oder war es, zu lernen was der Mitternachtsruf überhaupt lehrt und glaubt. Wie immer ist uns in erster Linie nicht wichtig, was die Leute vom MNR sagen, sondern was die Bibel dazu sagt. Aus diesem Grund, versuche ich auch heute wieder, die Aussagen mit Bibelversen zu belegen.

In diesem Teil geht es um die \flqq Die letzten Dinge \frqq{}. Das heisst es geht darum was am Ende der Zeit passiert. Diese Zeit wird teilweise von den Propheten im alten Testament beschrieben, aber vor allem im Buch der Offenbahrung.

\subsection{Was Glauben wir im MNR}
    \begin{enumerate}
        \item \uppercase{über die bibel} 06.04.2025
        \item \uppercase{über gott und den Menschen} 04.05.2025
        \item \uppercase{über die Erlösung} 18.5.2025
        \item \uppercase{über die Gemeinde und Israel} 08.06.2025
        \item \uppercase{über die Engel} 08.06.2025
        \item \uppercase{über die letzten Dinge}
    \end{enumerate}
    \begin{enumerate}      
      \item \textbf{Über die letzen Dinge} \aus{folie}
        \begin{itemize}
            \item Wir glauben, dass jeder Mensch sterben muss (Hebr 9,27; ausser jene, die entrückt werden). Die Seele des erlösten Menschen ist nach dem Tod sogleich in der Gegenwart Jesu (Phil 1,23) und die Seele eines Menschen, der nicht an Christus geglaubt hat, ist an einem Ort der Qual.

            \item Wir glauben, dass Jesus Christus eines Tages zurückkommen und Seine Gemeinde zu sich hin in den Wolken entrücken wird. Die Verstorbenen werden auferstehen und die noch Lebenden werden verwandelt. Jeder Christ erhält einen Auferstehungsleib, wird sich vor dem Preisrichterstuhl Christi verantworten und für immer bei Christus sein (Joh 14,1-3; 1.Kor 15,51-53; 2.Kor 5,10;). Wir glauben nach unserem Stand der Erkenntnis, dass die Entrückung vor dem Tag des Herrn erfolgen wird (1.Thess 1,10; Tit 2,13) und dass sie jederzeit geschehen kann.
            
            \item Wir glauben, dass nach der Entrückung ein antichristlicher Weltherrscher auftreten wird, der einen falschen Friedensbund mit Israel eingeht, und damit der Tag des Herrn (Gottes Gericht) eingeleitet wird (Dan 9,27; 2.Thess 2,3ff.). In dieser Zeit wird Gott Seinen Zorn über die Erde ausgiessen und Israel für Jesu Wiederkunft vorbereiten (\textbf{Jer 30,7;}).
        
            \item Wir glauben, dass am Ende des Tages des Herrn der Herr Jesus Christus in grosser Macht und Herrlichkeit zusammen mit Seiner Gemeinde und Seinen Engeln zurückkommen und die Völker und Israel richten wird. Israel wird seinen Erlöser erkennen und gerettet werden (Jes 59,19-20;). Der Herr Jesus wird im national und geistlich wiederhergestellten Israel, in Jerusalem, eine tausendjährige Herrschaft über diese Erde antreten (Mt 25,31;). Durch diese messianische Herrschaft werden alle noch ausstehenden Verheissungen für Israel erfüllt (Röm 11,1-2).
        
            \item Wir glauben, dass Satan und seine gefallenen Engel (Dämonen) während des Tausendjährigen Friedensreiches Jesu Christi gebunden sind. Doch am Ende werden sie losgelassen, um die Nationen der Erde zu verführen (Offb 20,7ff.). Christus wird ihrer Rebellion mit Feuer ein Ende setzen und danach werden Satan und seine Engel in den Feuersee, die Hölle, geworfen (Mt 25,41; Offb 20,10). Dann werden alle Menschen, die nicht durch den Glauben an Christus errettet wurden, auferweckt und von Gott nach ihren Werken gerichtet und zu einer ewigen Strafe im Feuersee verurteilt.
        
            \item Wir glauben, dass nach allem diesem der alte Himmel und die alte Erde aufgelöst werden (2.Petr 3,10; Offb 20,11). Gott wird einen neuen Himmel und eine neue Erde schaffen, in der Gerechtigkeit wohnen und Gott mit allen Seinen Erlösten aller Zeitalter leben wird (Eph 5,5;). Das himmlische Jerusalem wird vom Himmel herabkommen (Offb 21,2) und Jesus Christus wird das Reich Gott dem Vater übergeben. Dann werden alle Erlösten in ihren Auferstehungsleibern mit ihrem auferstandenen Herrn für immer auf einer auferstandenen Erde leben.
        \end{itemize}
    \end{enumerate}
    Das Glaubensbekenntnis vom MNR kann man auf ihrer Webseite abrufen. Ihr findet es auf der Startseite von www.mnr.ch, unter dem Link \frqq Unser Glaubensbekenntnis \flqq{}.
    Wir haben in den letzten Monaten dieses Glaubensbekenntnist angeschaut und die Aussagen des Mitternachtsrufs mit Bibelstellen belegt. Es ist wichtig, dass man in seinem Glauben ein Fundament hat. Nur wenn wir die Bibel und die Heilsgeschichte Gottes kennen, können wir vieles verstehen und einordnen, was in dieser gefallen Welt passiert.
    
    \textbf{Zusammenfassung:} \aus{Inhaltsverzeichnis Folie}
    \begin{enumerate}
        \item Die Bibel gibt uns das Wort Gottes wieder, nur durch dieses Buch wissen wir etwas von Gott
        \item wer Gott ist und wer wir Menschen sind, hilft uns unseren Glauben und die Welt zu verstehen
        \item unser Erlösung, gibt uns Trost und Zuversicht für unser Leben
        \item Mit der Gemeinde und Israel können wir lernen wie Gottes Heilsplan aussieht. 
        \item Die Engelwelt gibt uns einen Blick in den 3. Himmel. Auch sie sind von Gott geschaffene Wesen.
        \item Die letzten Dinge zeigen uns auf, dass Gott seinen Plan zu Ende führen wird und wie dieses Ende aussehen wird. Es soll uns nicht abschrecken sondern uns ermutigen und zuversichtlich machen.
    \end{enumerate}
    Ich empfehle euch die Bibelstellen, in dem Glaubensbekenntnis nachzulesen und mit den Aussagen von MNR zu vergleichen. Es geht nicht darum was der Mitternachtsruf sagt, sondern was Gott sagt.
\end{spacing}
\lied{Glauben heisst vertrauen}

\section{Predigt}

Danach gebe ich das Wort an Stephan weiter.

\textbf{Nach der Predigt}

Danken für die Predigt.

% \section{Abendmahl}

% Beten für das Brot

% \lied{Das Blut der Lammes 1. Strophe}

% Beten für den Wein

% \lied{Das Blut der Lammes 2. Strophe}

\section{Abschluss}

Jetzt wollen wir Gott mit dem Lied \lied{Ich hab einen herrlichen K"onig} danken.

Vielen Dank für eure Teilnahme am Gottesdienst. Im Anschluss seid ihr zu Kaffee und guten Gesprächen eingeladen.
\beten{}

\begin{bibelbox}{SCHL}{1Mos}{28:15}
Gott spricht: Siehe, ich bin mit dir,
ich behüte dich, wohin du auch gehst.
Denn ich verlasse dich nicht,
bis ich vollbringe, was ich dir versprochen habe.
\end{bibelbox}

Maranatha Amen
\end{document}