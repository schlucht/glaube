\author{OTS}
\documentclass{../../inc/mybib}

\setincpath{../../inc/}

\usepackage{bible_style}
\graphicspath{{../../assets/images/}}
\usepackage{header}

% \newenvironment{block}[1][]{%
%   \vspace{1.5em}%
%   \noindent\textbf{#1}\par%
%   \vspace{0.0em}%
% }{%
%   \vspace{1em}%
% }

\begin{document}

\section{Begrüssung}
Ich möchte alle recht herzlich zu unserem heutigen Gottesdienst begrüssen. Schön, dass ihr alle hier seid. Samuel auch dich will ich recht herzlich Begrüssen. Wir freuen uns, dass du wieder bei uns im Wallis bist und uns das Wort Gottes mitbringst. 

\beten{}

Singen wir nun zusammen unser erstes Lied.
% \noindent
\lied{Alle Herren dieser Welt}.

\section{Ankündigungen}
\begin{itemize}
    \item \green{Bibel und Gebetsabend:} Keiner
    \item \green{Nächster Gottesdienst:} So, 21.09.2025 10:00 Uhr Video Gottesdienst aus Dübendorf
    \item \green{Allgemein:} Fredy ist aktuell auf der Schweizer Tour. Auf dieser Tour ist er am 18.09.2025 also nächsten Donnerstag um 19:30Uhr hier bei uns in diesem Lokal sein und seinen Vortrag mit dem Thema \textbf{Die herrlichen Auswirkungen der Entrückung} halten. Schlagt also kräftig die Werbetrommel, damit viele Leute auch von diesen herrlichen Auswirkungen der Entrückung erfahren dürfen.
    \item \green{Allgemein:} Die Israel Konferenz steht wieder vor der Tür. Sie findet vom 27.09 bis 28.09.2025 in Zürich statt. An diesem Sonntag findet hier bei uns keinen Gottesdienst. Janina und ich werden dort teilnehmen. Wer von euch auch daran teilnehmen möchte, kann sich gerne bei mir oder Janina melden und wir können uns am Sonntag in Dübendorf nach dem Gottesdienst zum Mittagessen treffen.
    \item Die Kollekte wird hinten in der Grünen Trommel gesammelt und für den Bau dieser Gemeinde verwendet.
\end{itemize}

\section{ Input }
\begin{spacing}{1.5}    
  \begin{block}[Einleitung]
    Ich möchte gerne die nächsten Sonntage etwas über das Thema vom Reich Gottes erzählen. Aktuell machen Janina und ich eine Bibelschule und haben dieses Thema durchgenommen. Die erklärungen kommen von Dr. Michael J. Vlach. Er ist Dozent in Kalifornien und hat sich jahrzehntelang mit diesem Thema beschäftigt.
    
  \end{block}
  
   \textbf{Das Reich Gottes als Thema in der Schrift}\\
   Was ist das Haupttheme in der Bibel? Wenn ihr die Bibel von Anfang bis Ende durchlest, was fällt einem da auf?

   Die gesamte Bibel spricht vom Reich Gottes. Bereits im Schöpfungsbericht zeigt sich der
   Herr als der regierende König, über seine zuvor geschaffene Welt (1Mo 1-2). Auch der
   Mensch soll in diesem Reich Gottes eine Schlüsselstellung einnehmen   
    \begin{bibelbox}{SCHL}{1Mos}{1:26-28}
      
    \end{bibelbox}
   Dann aber wird der Mensch zum Sünder \bibleverse{1Mos}(3:). Diese Sünde überwindet Gott indem er
   bereits vor Grundlegung der Welt geplant hat, wie die Sünde überwunden und das Reich
   Gottes trotzdem aufgerichtet werden kann. Dieser Stufenplan, welcher sich über mehrere
   Jahrtausende hinzieht, wird in der Bibel detailliert beschrieben.
   Das Ziel Gottes besteht darin, alle ihm ergebenen Menschen wie ursprünglich
   vorgesehen als Verwalter über seine Schöpfung zu stellen (Offb 22,3-5).
   \begin{bibelbox}{SCHL}{Offb}{22:3-5}

   \end{bibelbox}
   Vlach teilt die Entfaltung des Reiches Gottes in 5 Phasen ein:
   
   \begin{enumerate}
    \item Schöpfung \bibleverse{IMos}(1-2:)\\
     Der Mensch wird als Verwalter einer vollendeten Schöpfung geschaffen. Gott der König 
beauftragt die Menschen, in seinem Namen über die Schöpfung zu herrschen (1Mo 
1,26-28). 
    \item Sündenfall \bibleverse{IMos}(3:)\\
    Der Sündenfall zeigt über das menschliche Versagen, dass der Mensch nicht willens ist, 
die Aufgabe als Herrscher über die Erde anzunehmen. Die Schöpfung kommt dadurch 
unter den Einfluss Satans. Beides, die Schöpfung und die Menschen fallen unter den 
Fluch der Sünde
    \item Verheissung 1. Mose 3,15 - Maleachi\\
    Gott verheißt, dass der „Same“ der Frau letztendlich über das Böse siegen wird (1Mo 
3,15). Damit wird der Sündenfall überwunden werden, und auch der Mensch wird 
seiner ursprünglichen Berufung entsprechend über die Schöpfung herrschen können.
    \item Ankunft des Königs Evangelien Briefe\\
     Über seinen Tod am Kreuz erfüllt Jesus Christus alle Bedingungen, die zur Aufrichtung 
des Reiches Gottes nötig sind. Zugleich werden die Gläubigen dieser geistlichen 
Realität in direkter Weise teilhaftig. 
    \item Wiederherstellung Offenbarung\\
     Mit der heute noch ausstehenden Wiederkunft Jesu, wird das Reich Gottes auch 
physisch vollends aufgerichtet werden. Dieses Reich wird am Ende seinerseits durch ein 
 vollkommenes Reich abgelöst, in welchem die Erlösten die ursprüngliche Absicht 
Gottes freiwillig und mit großer Freude wahrnehmen werden.
   \end{enumerate}
Das ist jetzt mal ein kurzer Abriss von 5 Phasen, die schlussendlich zum Reich Gottes führen.

Wenn ihr beim Bibellesen nur einzelne Verse aussucht, werden ihr nie diesen roten Faden finden. Wenn ihr aber die Bibel von Anfang bis Ende durchliest, könnt ihr schön sehen, wie die einzelnen Bücher ineinander eingehen. Und das ist einer der Wunder dieses Buches.

Menschen können nicht über jahrhunderte sogar jahrtausende ein Buch schreiben, dass vollständig in sich stimmig ist. Verstehen können wir das aber nur, wenn wir Jesus haben. Darum singen wir nun gemeinsam das Lied. Nur Christus in mir.

\end{spacing}
\lied{Nur durch Christus in mir}

\section{Predigt}

%  \textbf{Nach der Predigt}
%  Danken für die Predigt.

%  \section{Abendmahl}
%  Beten Nick und Lothar evt. Ueli
%  Beten für das Brot 

%  \lied{Das Blut der Lammes 1. Strophe}

%  Beten für den Wein

%  \lied{Das Blut der Lammes 2. Strophe}

\section{Abschluss}

\lied{Schau ich zurück}.

\beten{}

\begin{bibelbox}{SCHL}{1Mos}{28:15}
    Gott spricht: Siehe, ich bin mit dir,
    ich behüte dich, wohin du auch gehst.
    Denn ich verlasse dich nicht,
    bis ich vollbringe, was ich dir versprochen habe.
\end{bibelbox}

Maranatha Amen
Leider gibs am Heute nur noch Kapselkaffee. Aber ich freue mich auf die Gespräche.
\end{document}
