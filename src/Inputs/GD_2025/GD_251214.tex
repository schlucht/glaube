\author{OTS}
\documentclass{../../inc/mybib}

\setincpath{../../inc/}

\usepackage{bible_style}
\graphicspath{{../../assets/images/}}
\usepackage{header}

\newcommand{\Name}{Lothar}
% ensure scrlayer-scrpage has sufficient footheight
\setlength{\footheight}{20.4pt}

\begin{document}

\section{Begrüssung}
Hallo \Name{} wir freuen uns, dich in Naters begrüssen zu können. Wir sind gespannt auf dein Wort.

Ich möchte auch euch alle recht herzlich zu diesem Gottesdienst begrüssen. Schön, dass ihr gekommen seid, um unserem \herr N zu Danken, ihn zu loben und zu preisen.

% \noindent
\beten{} Und anschliessend singen wir zusammen das Lied

% \noindent

\lied{Macht hoch die Tür}
\section{Ankündigungen}
\begin{itemize}
    \item \green{Bibel und Gebetsabend:} Do, 18.12.2025 20:00 Uhr Bibel und Gebetsabend mit Norbert Lieth; zu Markus 3,31-35.
    \item \green{Nächster Gottesdienst:} So, 21.12.2025 14:45 Uhr hier mit Fredy Peter Weihnachtsgottesdienst. Im Anschluss möchten wir gerne mit euch zusammen die Gemeinschaft geniessen und etwas zusammen essen.
    \item \green{Die Kollekte:} Heute ist der letzte Sonntag wo wir die Spende für die Möbel annehmen. Entweder hinten in die grüne Box oder an Janina mit TWINT.
\end{itemize}

\section{ Input }
\begin{spacing}{1.5}
    \begin{block}[Einführung]
    Der siebte Teil der Heilsgeschichte Gottes mit uns Menschen. Letzten Sonntag haben wir den Bund mit Abraham angeschaut. Folgendes wurde ihm verheissen:
    \begin{itemize}
        \item Er wird ein Land bekommen, das Gott ihm zeigen wird.
        \item Er wird eine grosse Nation werden.
        \item Er wird einen grossen Namen haben.
        \item Er wird gesegnet werden.
        \item \betonung{Er wird ein Segen sein.}
    \end{itemize}
    Wir sind jetzt mitten in der Zeit sogenannten Patriarchen. Abraham, Isaak und Jakob, auch genannt Israel. Dazwischen ist noch die Geschichte mit Joseph. Joseph der Sohn von Jakob, wurde der zweite Mann in Ägypten.  Heute Morgen habe ich gerade diese Geschichte gelesen, und ich musst schmunzeln als der Pharao, seinem hungrigen Volk den Tip gab: \frqq Geht zu Joseph! Was er euch sagt das tut!\flqq{} Wer hat das mit Neuen Testament auch gesgt?
    
    Während der siebenjährigen Hungersnot im Land ist Jakob mit seinem ganzen Anhang nach Ägypten gezogen. Für 400 Jahre war Israel dort. Das hat Gott schon Abraham so prophezeit.
    \begin{bibelbox}{SCHL}{1Mos}{16:13}
        Da sprach Er (Gott) zu Abraham: Du sollst mit Gewissheit wissen, dass dein Same ein Fremdling sein wird in einem Land das ihm nicht gehört; und man wird sie dort zu Knechten machen und demütigen 400 Jahre.
    \end{bibelbox}
    Nach diesen 400 Jahren hat Gott Mose aus dem Volk berufen. Nachdem Gott dem Pharao gezeigt hat, wer der \herr ist, ist Mose mit dem ganzen Volk aus Ägypten Richtung Rotes Meer geflohen um in das gelobtes und verheissenes Land, in dem Milch und Honig fliesst zu gelangen.

    In der Wüste beim Berg Sinai, hat Gott Mose die Gesetze gegeben und einen Bund mit Mose und dem Volk geschlossen. Im Gegensatz zu den Bündnissen mit Noah und Abraham war dieser Bund nicht mehr einseitig und bedingungslos. Das Volk Israel konnte nur auf die daran hängenden Segnungen hoffen, wenn es die Gebote, die Gott dem Mose gegeben hat, einhalten. Es hing Fluch und Segen an diesen Geboten. Dazu könnt ihr 5. Mose zu Hause lesen.

    Mose bekam auf dem Berg Sinai zwei Tafeln mit den 10 Geboten. Diese zwei Tafeln wurden von Gott persönlich zurechtgehauen und beschrieben. Als Mose vom Berg stieg und sah wie sein Volk um das Goldene Kalb tanzte, zerbrach er diese Tafeln.
    \begin{bibelbox}{SCHL}{IIMos}{32:19}
        Es geschah aber, das er nahe zum Lager kam und das Kalb und die Reigentänze sah, da entbrannte Moses Zorn, und er warf die Tafeln weg und zerschmetterte sie unten am Berg.
    \end{bibelbox}
    Mose hat das Gesetz noch nicht kommuniziert, und schon hat das Volk dieses gebrochen. Die zweiten Tafeln, waren dann nicht mehr das Werk Gottes, sondern Mose musste diese selber zurechthauen. Sie wurden aber wieder von Gott beschriftet.
    
    Wir wissen, das diese Gesetze dem Volk Israel galten, es gab Gesetze für den Gottesdienst, für die Hygiene oder für das soziale Leben. Ein paar von denen verstehen wir nicht mehr, oder sind für unsere Gesellschaft nicht mehr relevant, aber viele würden wir heute noch brauchen um unser soziales Leben erträglich zu machen. Zwar werden oft die 10 Gebote zitiert. Wenn wir aber eherlich sind, hält sich unsere Gesellschaft nicht mehr an diese Gebote. Du sollst nicht töten! 100000 Kinder werden in Deutschland pro Jahr getötet. Ehre dein Vater und deine Mutter. Viele Väter und Mütter versauern in Altersheimen. Lügen, Betrügen, Neid und Missgunst, das gab\singlespacing s schon immer, aber heute, ist es salonfähig und fällt unter die Rubrik Tolleranz und Freiheit.\\
    Jesus sagt uns:
     \begin{bibelbox}{SCHL}{Mat}{5:18}
        Denn wahrlich, ich sage euch: \frqq{} Bis Himmel und Erde vergangen sind, wird nicht ein Buchstabe noch ein einziges Strichlein vom Gesetz vergehen bis alles geschehen ist.
    \end{bibelbox}
    Wenn wir die Gebote von Gott nicht halten, hat das schwerwiegende Konsequenzen.
    \begin{bibelbox}{SCHL}{IIMos}{20:5-6}
        Denn ich, der \herr, dein Gott bin ein eifersüchtiger Gott, der die Schuld der Väter heimsucht an den Kindern bis in das dritte und vierte Glied derer, die mich hassen, der aber Gnade erweisst an vielen Tausenden, die mich lieben und meine Gebote halten.
    \end{bibelbox}
    Die Bergpredigt von Jesus, zeigt uns, dass wir die Gebote nicht halten können.
    In seinem Buch \enquote{Der grosse Katechismus} erklärt Luther, wie er die 10 Gebote sieht. Ich war erst kurz gläubig als ich das Buch gelesen habe und es hat mich tief betroffen gemacht. Zum Beispiel zum 5. Gebot \enquote{Du sollst nicht töten} schreibt er:
    \begin{quotation}[Zitat Luther:]
        Man tötet nicht nur, indem man Böses tut, sondern auch, indem man Gutes unterlässt: z.B. einem Hungernden nichts zu essen gibt, sodass er stirbt
    \end{quotation}
    Habt ihr auch die Bilder von verhungernden Kinder vor Augen und zur gleichen Zeit das Bild wie der Migro innen aussieht? Sind wir jetzt alle verloren?

    Paulus sagt in Galater 3.18. \textbf{Tip:} Lest zu Hause das ganze Kapitel 3 durch.
    \begin{bibelbox}{SCHL}{Gal}{3:18}
       Denn wenn das Erbe durchs Gesetz käme, so käme es nicht mehr durch Verheissung; dem Abraham aber hat es Gott durch Verheissung geschenkt.
   \end{bibelbox}
    Wenn wir nur errettet werden könnten, wenn wir die Gesetze halten, dann würde ja der Bund mit Abraham auch nicht mehr gelten.
    Wir können die Gebote nicht halten. Wir als Menschen sind dazu nicht imstande. Ist jetzt die Heilsgeschichte zu Ende? Das sei ferne! würde jetzt Paulus ausrufen. Wie es jetzt genau weiter geht, mit dem Plan Gottes, das schauen uns am nöchsten Sonntag an unserem Weihnachtsgottesdienst an. Für heute nur so viel:
    
    Es gibt und gab nur ein Mensch der es geschafft hat alle Gebote zu halten, dieser Mensch ist unser Erlöser und Retter Jesus, der Gesalbte. Sein Tod am Kreuz ist die einzige Rettung vor dem Zorn Gottes. Das ist unser Glaube.
     \begin{bibelbox}{SCHL}{Joh}{14:6}
        Jesus spricht zu ihm: Ich bin der Weg und die Wahrheit und das Leben; niemand kommt zum Vater als nur durch mich.
    \end{bibelbox}
    Und dazu einer meiner Lieblingsverse:
      \begin{bibelbox}{SCHL}{Joh}{11:26}
        \betonung{jeder}, der lebt und an mich glaubt, wird in Ewigkeit nicht sterben. \betonung{Glaubst du das?}
    \end{bibelbox}
    \textbf{\betonung{Die alles entscheidende Frage! Für uns alle! Vom Prediger bis zum einfachen Beter.}}\\
    \enquote{\betonung{GLAUBST DU DAS?}}
  	\end{block} 
    Singen wir zusammen ein Lied und nach dem Lied möchte ich \Name{} das Wort übergeben.
    \lied{Tochter Zion freue dich}.
   
\end{spacing}

\section{Predigt}

% \textbf{Nach der Predigt}

% Danken für die Predigt.

% \section{Abendmahl}

% Beten für das Brot

% \lied{Das Blut der Lammes 1. Strophe}

% Beten für den Wein

% \lied{Das Blut der Lammes 2. Strophe}

\section{Abschluss}

Jetzt wollen wir Gott mit dem Lied \lied{Freue dich Welt} danken.

Vielen Dank für eure Teilnahme am Gottesdienst. Im Anschluss seid ihr zu Kaffee und guten Gesprächen eingeladen.
\beten{}

\begin{bibelbox}{SCHL}{1Mos}{28:15}
Gott spricht: Siehe, ich bin mit dir,
ich behüte dich, wohin du auch gehst.
Denn ich verlasse dich nicht,
bis ich vollbringe, was ich dir versprochen habe.
\end{bibelbox}

Maranatha, komm Herr Jesus! Amen
\end{document}