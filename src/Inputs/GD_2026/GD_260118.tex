\author{OTS}
\documentclass[14pt]{../../inc/mybib}

\setincpath{../../inc/}

\usepackage{bible_style}
\graphicspath{{../../assets/images/}}
\usepackage{header}

\newcommand{\Name}{Samuel}
% ensure scrlayer-scrpage has sufficient footheight
\setlength{\footheight}{20.4pt}

\begin{document}

\section{Begrüssung}
Hallo \Name{}, wir freuen uns, dich in Naters begrüssen zu können. Wir sind gespannt auf dein Wort.

Ich möchte auch euch alle recht herzlich zum Gottesdienst begrüssen. Schön, dass ihr gekommen seid, um unserem \herr N zu Danken, ihn zu loben und zu preisen.

% \noindent
\beten{} Und anschliessend singen wir zusammen das Lied

% \noindent

\lied{Seid fröhlich in der Hoffnung}

\section{Informationen}
\begin{itemize}
    \item \bt[infos]{Bibel und Gebetsabend:} Do, 22.01.2026 20:00 Uhr Bibel und Gebetsabend mit Norbert Lieth; zu Markus 4,21-25.
    \item \bt[infos]{Nächster Gottesdienst:} So, 25.01.2026 14:45 Uhr hier Video Predigt vom MNR und ich werde ein paar Fotos von der Israelreise zeigen. Dieses mal zu Betlehem. 
    \item \bt[infos]{Kollekte:} Die Kollekte wird hinten in der grünen Box gesammelt und wird vollumfänglich für den Bau dieser Gemeinde eingesetzt.
\end{itemize}

\section{ Input }
\begin{spacing}{1.5}
    \begin{block}[Einführung]    
    Der neue Bund wird in \bibleverse{Jer}(31:31-40) beschrieben.\\    
    Folgende Verheissungen werden hier aufgelistet. Wir sehen sie auch auf der Folie:
    \begin{enumerate}
        \item \textbf{Die Herzensbeschneidung und das neue Herz}: Das totale Versagen Israels gegenüber Gott hat gezeigt, dass der Mensch keinen Gehorsam gegenüber einem lebendigen Gott hat.         
        \item \textbf{Die Ausgießung des Heiligen Geistes}: Das errettete Volk wird dadurch befähigt den Willen Gottes von innen heraus gerne zu tun. Damit verbunden ist die Erkenntnis und Belehrung durch den Geist Gottes.   
        \item \textbf{Vergebung der Sünden}: Gott selbst wird ihre Schuld hinwegnehmen.         
        \item \textbf{Die Sammlung und Wiederherstellung Israels}:  Zuerst wird das Volk äusserlich wieder hergestellt, danach wird es innerlich erneuert und der Geist eingehaucht.        
    \end{enumerate}
\end{block}

    \begin{block}[Die Gültigkeit des neuen Bundes]
        Schauen wir uns diesen Neuen Bund etwas genauer an.
        Für wen ist dieser Neue Bund gültig? Dazu lesen wir in \bibleverse{Jer}(31:31).
        \begin{bibelbox}{SCHL}{Jer}{31:31}
           Siehe, es kommen Tage, spricht der \herr{}, da ich mit dem Haus Israel und mit dem Haus Juda einen neuen Bund schliessen werde;
        \end{bibelbox} 
        Also schon mal mit dem Volk Israel und weiter in \bibleverse{Jes}(49:6).
        \begin{bibelbox}{SCHL}{Jes}{49:6}
           Der \herr{} spricht: Es ist zu gering, dass du mein Knecht bist, um die Stämme Jakobs aufzurichten und die Bewahrten aus Israel wiederzubringen; sondern ich habe dich auch zum Licht für die Heiden gesetzt, damit du mein Heil seist bis an das Ende der Erde.
        \end{bibelbox}
        Gott hat Israel auserwählt, um ein Heil und Licht für die Heiden zu sein.\\
        Eigentlich sollten wir von dem jüdischen Volk lernen, woher das Heil kommt und somit werden wir in diesen Neuen Bund mit aufgenommen. Paulus beschreibt in den Kapiteln \bibleverse{Rom}(9-11:), dass die Juden das Heil verworfen haben, aber trotzdem bleibt es das Volk Gottes. Darum sagt Paulus in \bibleverse{Rom}(11:11), dass wir durch unser Verhalten und den Glauben an unseren Herrn Jesus Christus die Juden zur Eifersucht reizen sollen, sodass diese erkennen, dass Jesus der ihnen versprochene Messias, ihr Christus ist.  
        \begin{bibelbox}{SCHL}{Rom}{11:11}
            Ich frage nun: Sind sie denn gestrauchelt, damit sie fallen sollen? Das sei ferne! Sondern durch ihren Fall wurde das Heil der Heiden zuteil, um sie zur Eifersucht zu reizen.
        \end{bibelbox}
        In \bibleverse{Rom}(11:15) sagt Paulus weiter:
        \begin{bibelbox}{SCHL}{Rom}{11:15}
            Denn wenn ihre Verwerfung die Versöhnung der Welt [zur Folge hatte], was wird ihre Annahme anderes [zur Folge haben], als Leben aus den Toten?
        \end{bibelbox}
    \end{block}     
        \begin{block}[Schlussfolgerung]
            Wenn also für die Juden im Neuen Bund gilt, dass sie erneuert werden und Vergebung der Sünden erhalten, also das Leben, erhalten wir durch die Aufnahme in den Bund auch diese Segnungen. Wenn wir also diesen Bund annehmen, werden wir in den Ölbaum mit eingepfropft. Der Ölbaum selber ist das Volk Gottes. Das heisst nicht, dass wir Juden werden oder das Volk ersetzen. Dieser Punkt ist sehr wichtig. Wir sind nicht das neue Volk Gottes, sondern wir gehören zur Gemeinde Gottes. Auch die Juden, die sich für den Herrn Jesus Christus entscheiden, gehören jetzt in der Gnadenzeit zu der Gemeinde.
    
            Die Theologie, dass wir Christen das neue Israel sind und wir nun die Juden abgelöst hätten, führte in der Geschichte immer wieder zu Antisemitismus und Judenmord. Der Fachbegriff ist \betonung{Substitutionstheologie}. Dieses komplizierte Wort bedeutet, dass die Christenheit Israel als das Volk Gottes ersetzt hat. Diese Theologie ist relativ weit verbreitet und wächst seit den letzten Ereignissen in der Welt wieder. Was das für eine Theologie ist, wie gefährlich und wie falsch diese Theologie ist, möchte ich hier in zwei Wochen zeigen. Vor allem in der heutigen Zeit ist es sehr wichtig, dass wir diese Problematik verstehen. Ich, aber auch der MNR distanziert sich klar von dieser Lehre, dass wir Christen die Juden ersetzt hätten.
    
            Wenn wir hier den Heilsplan Gottes ansehen, würde das ja heissen, dass nach dem Bund mit David, Gott einen Strich gezogen hätte und Israel verworfen hätte. Das würde heissen, dass er die Versprechen, die er dem Volk Gottes gegeben hat, nicht einhält und das hätte riesige Konsequenzen für uns. Würde Gott dann sein Versprechen einhalten, das wir durch seine Gnade errettet sind? \betonung{Aber Gott hält seine Versprechen. Das ist Gewiss.} Wenn du also Jesus Christus in dein Herz aufgenommen hast und deine Sünden vor ihn gelegt hast, dann bist du für ewig sein Kind und gerettet.
    
            Wenn du diesen Schritt noch nicht getan hast, dann tue es heute.             
        \end{block}

    Singen wir zusammen ein Lied und nach dem Lied möchte ich \Name{} das Wort übergeben.

    Dieses Lied ist neu. Aber es ist relativ einfach und ist ein Kinderlied. Der Text ist aber wunderschön und passt perfekt zu unserem Thema heute.
    \lied{Wie wissen nicht den Tag noch die Stunde}.
   
\end{spacing}

\section{Predigt}
\textbf{Nach der Predigt}\newline
Danken für die Predigt.\newline
% \section{Abendmahl}
% Beten für das Brot\newline
% Beten für den Wein\newline

\section{Abschluss}

\lied{Wunderbar grosser Erlöser}

\beten{}

\begin{bibelbox}{SCHL}{IIKor}{13:13}
    Die Gnade des Herrn Jesus Christus und die Liebe Gottes und die Gemeinschaft des Heiligen Geistes sei mit euch allen! Amen
\end{bibelbox}

\end{document}