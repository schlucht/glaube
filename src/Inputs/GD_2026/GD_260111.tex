\author{OTS}
\documentclass{../../inc/mybib}

\setincpath{../../inc/}

\usepackage{bible_style}
\graphicspath{{../../assets/images/}}
\usepackage{header}

\newcommand{\Name}{Thomas}
% ensure scrlayer-scrpage has sufficient footheight
\setlength{\footheight}{20.4pt}

\begin{document}

\section{Begrüssung}
Hallo \Name{}, wir freuen uns, dich in Naters begrüssen zu können. Wir sind gespannt auf dein Wort.

Ich möchte auch euch alle recht herzlich zum zweiten Gottesdienst 2026 begrüssen. Schön, dass ihr gekommen seid, um unserem \herr N zu Danken, ihn zu loben und zu preisen.

% \noindent
\beten{} Und anschliessend singen wir zusammen das Lied

% \noindent

\lied{Ich habe einen herrlichen König}

\section{Informationen}
\begin{itemize}
    \item \bt[infos]{Bibel und Gebetsabend:} Do, 15.01.2026 20:00 Uhr Bibel und Gebetsabend mit Nathanael Winkler; zu Markus 4,1-20.
    \item \bt[infos]{Nächster Gottesdienst:} So, 18.01.2026 14:45 Uhr hier mit Samuel Rindlisbacher. 
    \item \bt[infos]{Kollekte:} Die Kollekte wird hinten in der grünen Box gesammelt und wird vollumfänglich für den Bau dieser Gemeinde eingesetzt.\newline
    Wenn jemand Dauerspender an den Mitternachtsruf ist, kann er diese Spende mit dem Vermerk \enquote{MNR Gemeinde Naters} versehen. Dann wird diese Spende für den Bau dieser Gemeinde eingesetzt und man bekommt einen Steuerbescheid.    
\end{itemize}

\section{ Input }
\begin{spacing}{1.5}
    \begin{block}[Einführung]
    Janina und ich waren diese Woche an einem Seminar in Berlin mit dem Thema die Glaubwürdigkeit der Bibel.
    
    Themen waren unteranderem, 
    \begin{itemize}
        \item Die Bibel ist notwendig
        \begin{itemize}
            \item Um Gott zu erkennen
            \item Eine objektive Sicht von Gott zu erhalten
            \item Eine vollständige Sicht zu erhalten
            \item Eine irrtumslose Sicht zu erhalten
        \end{itemize}
        \item Die Bibel hat ihre Glaubwürdigkeit von Gott
        \begin{itemize}
            \item Nur Gott kann der Bibel Glaubwürdigkeit geben 
        \end{itemize}
        \item Die Bibel hat ihre Glaubwürdigkeit durch die Bibel
        \begin{itemize}
            \item Glaubwürigkeit des AT wird durch Gott, Christus, den Autoren aus dem NT bezeugt
            \item Glaubwürdigkeit des NT wird durch Christus, die Apostel 
        \end{itemize}
        \item Die Bibel ist irrtumslos 
        \begin{itemize}
            \item Die Bibel kommt von Gott
            \item Die Orginalhandschriften sind irrtumslos
            \item Die irrtumslosigkeit betrifft alles nicht nur der Glaube und Praxis
        \end{itemize}
        \item ...
    \end{itemize}
    Weitere Themen waren noch, wie dieses Buch entstanden ist, woher die Bücher kommen, wer hat die Bibel zusammengestellt usw. Es war sehr viel Stoff und sehr intressant.
    Am Ende des Seminars haben wir herausgefunden, dass die Bibel irrtumslos und fehlerlos ist. Das die Schreiber vom hl. Geist geführt wurden und das jedes Wort bis zum kleinsten Strichlein richtig ist. 
    Noch ein wichtiges Detail, Bibelkritik ist nicht gleich Textkritik. Textkritik ist eine Wissenschaft, die untersucht, wie die Texte entstanden sind, wer sie geschrieben hat, wann sie geschrieben wurden usw. Bibelkritik hingegen stellt die Glaubwürdigkeit der Bibel infrage.

    Die Theologen von heute stellen weniger die Herkunftt der Bibel infrage, sondern ob alles, was in der Bibel steht, auch wahr ist. Die Wunder muss man ja nicht so ernst nehmen. Ob es von Gott kommt oder vom Menschen. Aber nicht nur die Theologen haben dieses Problem, sondern immer mehr die evangelikalen bibeltreuen Christen. Man hört immer wieder, dass man den Schöpfungsbericht nicht so genau nehmen soll, so wegen der 6 Tage, die Wissenschaft ist doch heute viel weiter. Oder auch das der Gott im alten Testament nicht der gleiche ist wie der im neuen Testament. Wie kann ein Gott, der die Liebe ist, den Auftrag geben alle Menschen in Jericho zu töten?

    Das hier ist eine Bibel wie wir sie kennen. Wenn wir jetzt anfangen, hier ein Teil herauszunehmen und dort ein Teil zu entfernen, haben wir plötzlich eine Bibel in dieser Grösse. Also haben so eine neue Bibel aus einer Bibel hergestellt, die  besser zu mir passt.
    
    Vielleicht denkt ihr jetzt, ach das Problem habe ich nicht. Aber passt auf, dieses Streichen passiert langsam. Am Anfang fällt es nicht so auf. In der heutigen Zeit, wo wir überflutet werden mit Youtube-Predigten, von Pastoren,  studierten Theologen bis hin zu Laien die ihre Erkenntnisse loswerden wollen ist es wichtig zu prüfen. \betonung{Prüft alles}. Zum Beispiel Johannes Hartl, er glaubt nicht das alles, was in der Bibel steht, auch so stimmt. Oder der Kanal Worthaus, wo ein hochdekorierter Dr. Prof. die Gottheit Jesus leugnet, oder eine Frau Joyce Meyer, die das Evangelium zu ihren Gunsten ummodelt, um ihren Reichtum verteidigen zu können und noch viele mehr. 

    Auch Paulus wurde geprüft. In \bibleverse{Apg}(17:11) steht, dass die Beröer in den Schriften geprüft haben, ob das was Paulus sagte, auch so stimmt.

    \begin{bibelbox}{SCHL}{Apg}{17:11}
    Diese waren edler gesinnt als die in Thessalonich und nahmen das Wort mit aller Breitwilligkeit auf; und sich forschten täglich in der Schrift, ob es sich so verhalte.
    \end{bibelbox}

    Das gfährliche ist die Bibelkritik. Darauf müsst ihr achten.
    
    Aber auch wenn ich hier stehe, oder Thomas, vor allem ich, prüft alles. Wir alle sind Menschen und machen Fehler. Prüft die Worte anhand der Bibel, denn die Bibel ist irrtumslos und hat die Autorität zur Lehre direkt von Gott bekommen.

    Dieses Buch hat eine Macht wie kein anderes und kann Herzen verändern. Aber bevor ihr den Inhalt dieses Buch Leuten, die nicht an Gott glauben, um die Ohren haut, denkt daran. Nur die, welchen den Hl. Geist haben, können die Bibel verstehen. Die Bibel kann aber Menschen die Herzen öffnen, sodass diese verstehen lernen.
    \bibleverse{IKor}{2:14}
    \end{block}

    \lied{Schöpfer aller Himmel}.
   
\end{spacing}

\section{Predigt}
\textbf{Nach der Predigt}\newline
Danken für die Predigt.\newline
% \section{Abendmahl}
% Beten für das Brot\newline
% Beten für den Wein\newline

\section{Abschluss}

\lied{Seliges Wissen Jesus}

\beten{}

\begin{bibelbox}{SCHL}{IIKor}{13:13}
    Die Gnade des Herrn Jesus Christus und die Liebe Gottes und die Gemeinschaft des Heiligen Geistes sei mit euch allen! Amen
\end{bibelbox}

\end{document}