\author{OTS}
\documentclass{../../inc/mybib}

\setincpath{../../inc/}

\usepackage{bible_style}
\graphicspath{{../../assets/images/}}
\usepackage{header}

\newcommand{\Name}{Nathanael}
% ensure scrlayer-scrpage has sufficient footheight
\setlength{\footheight}{20.4pt}

\begin{document}

\section{Begrüssung}
Hallo \Name{}, wir freuen uns, dich in Naters begrüssen zu können. Wir sind gespannt auf dein Wort.

Ich möchte auch euch alle recht herzlich zum ersten Gottesdienst 2026 begrüssen. Schön, dass ihr gekommen seid, um unserem \herr N zu Danken, ihn zu loben und zu preisen.

% \noindent
\beten{} Und anschliessend singen wir zusammen das Lied

% \noindent

\lied{Halleluja! Lobet Gott in seinem Heiligtum}

\section{Informationen}
\begin{itemize}
    \item \bt[infos]{Bibel und Gebetsabend:} Do, 15.01.2026 20:00 Uhr Bibel und Gebetsabend mit Nathanael Winkler; zu Markus 4,1-20.
    \item \bt[infos]{Nächster Gottesdienst:} So, 11.01.2026 14:45 Uhr hier mit Thomas Lieth. 
    \item \bt[infos]{Kollekte:} Die Kollekte wird hinten in der grünen Box gesammelt und wird vollumfänglich für den Bau dieser Gemeinde eingesetzt.\newline
    Wenn jemand Dauerspender an den Mitternachtsruf ist, kann er diese Spende mit dem Vermerk \enquote{MNR Gemeinde Naters} versehen. Dann wird diese Spende für den Bau dieser Gemeinde eingesetzt und man bekommt einen Steuerbescheid.
    \item \bt[infos]{Jahresvorschau:}
    \begin{itemize}
        \item 5. April Osterkonferenz kein Gottesdienst
        \item 13. April Montag CH-Tournee mit Nathanel
        \item 23. August Gemeindetag
        \item 17. Sept Donnerstag CH-Tournee mit Fredy
        \item 27. Oktober Israelkonferenz kein Gottesdienst        
    \end{itemize}
    Ausserdem entfällt der Gottesdienst um 10:00Uhr. Einmal im Monat 2025 hatten wir um 10Uhr einen Gottesdienst. Es kam niemand ausser die Familie Perren zu diesem Gottesdienst. Also ist auch der Bedarf nicht da. Es ist aber passiert, dass ein paar Mal Personen am Nachmittag vor verschlossener Tür standen. Wenn die GBSM stattfindet, kommt zwar kein Prediger, aber wir werden den Gottesdienst gestalten und nicht nur die Übertragung laufen lassen. Es wird nur die Predigt vom Morgen aus Dübendorf übertragen.
\end{itemize}

\section{ Input }
\begin{spacing}{1.5}
    \begin{block}[Einführung]
    Ein neues Jahr hat begonnen. Viele Glückwünsche machen die Runde. \enquote{Gutes Neues Jahr}, \enquote{Viel Glück im neuen Jahr}, oder \enquote{Gott segne dich im neuen Jahr}, \enquote{Gesundheit und Glück}. Heute im Zeitalter vom Handis sind solche Wünsche schnell verschickt. Entweder im Status oder per Whatsapp ein Bildchen oder eine Nachricht. Gut gemeinte Wünsche. Aber ob sich das Leben nach diesen Wünschen richtet? Wohl eher nicht. Wie wir es gerade aktuell an dem schrecklichen Ereignis in Montana sehen, kann sich das Leben in einem Augenblick verändern. Aus Glückwünschen wird plötzlich Trauer.

    Auch wenn wir nicht wissen, was uns das neue Jahr an Gesundheit und Glück bringt, wissen wir, dass unser Gott einen Plan hat. Einen Plan mit uns und mit dieser Welt. An seinem Plan für die Welt wollen wir uns auch dieses Jahr weiter entlangangeln. Letztes Jahr haben wir bei der Schöpfung angefangen, dann den Bund mit Noah, Abraham, Mose und David angeschaut. Heute zum Jahresbeginn wollen wir gemeinsam den Neuen Bund anschauen.

    Dafür müssen wir zusammen die Bibel aufschlagen, und lesen die Verse \bibleverse{Jer}(31:31-40).\\    
    Folgende Verheissungen werden hier aufgelistet:
    \begin{enumerate}
        \item \textbf{Die Herzensbeschneidung und das neue Herz} Das Totale versagen Israels gegenüber Gott, hat gezeigt, dass der Mensch keinen Gehorsam gegenüber einem lebendigen Gott hat. Nur durch die Erneuerung des Herzens kann der Mensch Gott erkennen.
        \begin{bibelbox}{SCHL}{VMos}{30:6}
            Und der Herr dein Gott, wird dein Herz und das Herz deiner Nachkommen beschneiden, dass du den Herrn, deinen Gott, liebst von ganzem Herzen und von ganzer Seele, damit du lebst.            
        \end{bibelbox}
        In Hesekiel wird von einer regelrechten Herztransplantation geredet.
        \begin{bibelbox}{SCHL}{Hes}{36:26}
            Und ich will euch ein neues Herz geben und einen neuen Geist in euer Inneres legen; ich will das steinerne Herz aus eurem Fleisch wegnehmen und euch ein fleischernes Herz geben;
        \end{bibelbox}
        Auch jeder wiedergeborene Christ muss diese Erneuerung durchleben. 
        \begin{bibelbox}{SCHL}{IIKor}{5:17}
            Darum: Ist jemand in Christus, so ist er eine neue Schöpfung; das Alte ist vergangen; siehe es ist alle neu geworden.
        \end{bibelbox}
        \item \textbf{Die Ausgießung des Heiligen Geistes} Das errettete Volk wird dadurch befähigt den Willen Gottes von innen heraus gerne zu tun. Damit verbunden ist die Erkenntnis und Belehrung durch den Geist Gottes.
        \begin{bibelbox}{SCHL}{Joel}{3:1-5}
            Und nach diesem wird es geschehen, dass ich meinen Geist ausgiesse über alles Fleisch; und eure Söhne und eure Töchter werde weisssagen, eure Ältesten werden Täume haben, eure jungen Männer werde Gesichte sehen:
            \textit(und der Vers 5)
            Und es wird geschehen: Jeder, der den Namen des Herrn anruft, wird gerettet werden; denn auf dem Berg Zion und in Jerusalem wird Errettung sein, wie der \herr{} verheissen hat, und bei den Übriggeblieben, die der \herr{} beruft.
        \end{bibelbox}
        Jesus sagt
        \begin{bibelbox}{SCHL}{Joh}{14:16}
           Und ich will den Vater bitten, und er wird euch einen anderen Beistand geben, dass er bei euch bleibt in Ewigkeit, der Geist der Wahrheit, den die Welt nicht empfangen kann, denn sie beachtet ihn nicht und erkennt ihn nicht; ihr aber erkennt ihn, denn er bleibt bei euch und wird in euch sein.
        \end{bibelbox} 
        Wir können den Heiligen Geist nur erkennen, wenn wir zuvor ein neues Herz bekommen haben. Nicht wiedergeborene Christen erkennen den Geist nicht. Darum tun sie sich so schwer mit der Bibel.
        \item \textbf{Vergebung der Sünden} Gott selbst wird ihre Schuld hinwegnehmen.
        \begin{bibelbox}{SCHL}{Hes}{36:25}
           Und ich will reines Wasser über euch sprengen, und ihr werdet rein sein; von aller eurer Unreinheit und von allen euren Götzen will ich euch reinigen.
        \end{bibelbox} 
        \begin{bibelbox}{SCHL}{Hes}{36:33}
           So spricht Gott der Herr: Zu jener Zeit, wenn ich euch reinigen werde von allen euren Missetaten, da will ich euch wieder in den Städten wohnen lassen, und die Trümmer sollen wieder aufgebaut werden.
        \end{bibelbox} 
        Jesus kam auf die Welt um als Opferlamm für die Sünden der Welt am Kreuz zu sterben.
        \begin{bibelbox}{SCHL}{Joh}{3:16}
           Denn so sehr hat Gott die Welt geliebt, dass er seinen eingeborenen Sohn gab, damit jeder der an ihn glaubt, nicht verloren geht, sondern ewiges Leben hat.
        \end{bibelbox} 
        \item \textbf{Die Sammlung und Wiederherstellung Israels} In Hesekiel 37, (das ist die Geschichte mit den Gebeinen, die wieder zum Leben erweckt wurden), wird die Sammlung in zwei Phasen beschrieben. Zuerst wird das Volk äusserlich wieder hergestellt, danach wird es innerlich erneuert und der Geist eingehaucht.
        \bibleverse{Hes}(37:8-12)
    \end{enumerate}
\end{block}

    \begin{block}[Die Gültigkeit des neuen Bundes]
        Für wen ist dieser neue Bund gültig? Dazu lesen wir weiter in \bibleverse{Jer}(31:31).
        \begin{bibelbox}{SCHL}{Jer}{31:31}
           Siehe, es kommen Tage, spricht der \herr{}, da ich mit dem Haus Israel und mit dem Haus Juda einen neuen Bund schliessen werde;
        \end{bibelbox} 
        Also schon mal mit dem Volk Israel und weiter in \bibleverse{Jes}(49:6).
        \begin{bibelbox}{SCHL}{Jes}{49:6}
           Der \herr{} spricht: Es ist zu gering, dass du mein Knecht bist, um die Stämme Jakobs aufzurichten und die Bewahrten aus Israel wiederzubringen; sondern ich habe euch zum Licht für die Heiden gesetzt, damit du mein Heil seist bis an das Ende der Erde.
        \end{bibelbox}
        Gott hat Israel Auserwählt um ein Heil und Licht für die Heiden zu sein. Also auch für uns gilt dieser neue Bund. Der Unterschied besteht nur im Zeitpunkt des Inkrafttretens des neuen Bundes, für Israel und die Gemeinde aus Juden und Heiden.
    \end{block}
    \begin{block}[Die Einsetzung des neuen Bundes]
        Gott ist der Schenkende und Handelnde in diesem neuen Bund. Er ist allein in Gottes Verheissung und Tat begründet. Damit werden wir unwillkürlich in das Zentrum des Heilsplans Gottes geführt: \betonung{Jesus Christus} und sein vollbrachtes Werk. Er selbst hat den neuen Bund für uns eingesetzt zu lesen in \bibleverse{Luk}(22:17-20).
        \begin{bibelbox}{SCHL}{Luk}{22:17-20}
            Und er nahm den Kelch, dankte und sprach: Nehmt diesen und teilt ihn unter euch! Denn ich sage euch: Ich werde nicht mehr von dem Gewächs des Weinstocks trinken, bis das Reich Gottes gekommen ist. Und er nahm das Brot, dankte, brach es, gab es ihnen und sprach: Dies ist mein Leib der für euch gegeben wird; das tut zu meinem Gedächtnis! Desgleichen nahm er auch den Kelch nach dem Mahl und sprach: Dieser Kelch ist der neue Bund in meinem Blut, das für euch vergossen wird.
        \end{bibelbox}
        Paulus sagt in \bibleverse{2Kor}(3:6)
        \begin{bibelbox}{SCHL}{2Kor}{3:6}
            Er hat uns fähig gemacht, Diener des neuen Bundes zu sein, nicht des Buchstabens, sondern des Geistes; denn der Buchstabe tötet, der Geist aber macht lebendig.
        \end{bibelbox}
        So sehen wir, dass dieser neue Bund von Christus selbst eingesetzt wurde. Zu Pfingsten wurde der Heilige Geist für alle Gläubigen ausgegossen. Wir alle sind nun ein Teil dieses neuen Bundes. Dieser neue Bund ist nicht an Bedingungen geknüpft wie der Bund mit Mose, sondern an die Person und das Werk von Jesus Christus. Durch seinen Tod und seine Auferstehung sind wir erlöst und haben Vergebung der Sünden. Durch den Heiligen Geist sind wir befähigt, Gottes Willen zu tun. Durch die Beschneidung des Herzens sind wir fähig, Gott von ganzem Herzen und ganzer Seele zu lieben. Dieser neue Bund gilt für alle Menschen, die an Jesus Christus glauben, Juden und Heiden.

        Jetzt beschäftigt uns noch eine wichtige Frage. Hat denn jetzt die Gemeinde Israel abgelöst? Ist die Gemeinde das neue Israel? \betonung{Eine wichtige Frage in der heutigen Zeit ist:} Hat die Gemeinde Israel abgelöst? Dieser Frage wollen wir uns gemeinsam nächsten Sonntag widmen. 
    \end{block}    

    Singen wir zusammen ein Lied und nach dem Lied möchte ich \Name{} das Wort übergeben.

    \lied{Stern auf den ich schaue}.
   
\end{spacing}

\section{Predigt}
\textbf{Nach der Predigt}\newline
Danken für die Predigt.\newline
\section{Abendmahl}
Beten für das Brot\newline
Beten für den Wein\newline

\section{Abschluss}

\lied{Zünde an dein Feuer}

\beten{}

\begin{bibelbox}{SCHL}{IIKor}{13:13}
    Die Gnade des Herrn Jesus Christus und die Liebe Gottes und die Gemeinschaft des Heiligen Geistes sei mit euch allen! Amen
\end{bibelbox}

\end{document}