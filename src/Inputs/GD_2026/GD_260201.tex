\author{OTS}
\documentclass{../../inc/mybib}

\setincpath{../../inc/}

\usepackage{bible_style}
\graphicspath{{../../assets/images/}}
\usepackage{header}

\newcommand{\Name}{Erich}
% ensure scrlayer-scrpage has sufficient footheight
\setlength{\footheight}{20.4pt}

\begin{document}

\section{Begrüssung}
Hallo \Name{}, wir freuen uns, dich in Naters begrüssen zu können. Wir sind gespannt auf dein Wort.

Ich möchte auch euch alle recht herzlich zum ersten Gottesdienst 2026 begrüssen. Schön, dass ihr gekommen seid, um unserem \herr N zu Danken, ihn zu loben und zu preisen.

% \noindent
\beten{} Und anschliessend singen wir zusammen das Lied

% \noindent

\lied{Nur durch Christus in mir}

\section{Informationen}
\begin{itemize}
    \item \bt[infos]{Bibel und Gebetsabend:} Do, 05.02.2026 20:00 Uhr Bibel und Gebetsabend mit Thomas Lieth; zu Markus 4,30-34.
    \item \bt[infos]{Nächster Gottesdienst:} So, 08.02.2026 14:45 Uhr Videopredigt Elia Morris kann leider nicht kommen. 
    \item \bt[infos]{Kollekte:} Die Kollekte wird hinten in der grünen Box gesammelt und wird vollumfänglich für den Bau dieser Gemeinde eingesetzt.   
\end{itemize}

\section{ Input }
\begin{spacing}{1.5}
    \begin{block}[Glaubensbekenntnis]
        Wir glauben, dass die Gemeinde Jesu Christi aus wiedergeborenen Juden und Heiden besteht (Gal 3,28; Eph 2,11-18; Kol 3,11) und Gottes Volk ist (Röm 9,26; 1Pt 2,9-10). Doch wir glauben auch, dass die Gemeinde nicht das einzige Volk des Herrn ist. Wir glauben, dass Gott die Nation Israel nicht verworfen (Jer 31; Röm 11), sondern lediglich beiseitegestellt hat, bis die Gemeinde vollzählig ist.
    \end{block}
    \begin{block}[Substitutionstheorie]
        % Definition
        Dispensationalismus, Substitutionstheorie oder Ersatztheorie. Alle drei bedeuten das gleiche. Es geht um die Frage hat die Gemeinde Israel als Volk Gottes ersetzt? Sind jetzt wir das neu Israel, das neu Volk Gottes?

        Vielleicht denkt ihr jetzt so was im Gottesdienst verloren? Ich bin der Meinung, dass ein Gottesdienst zur Erbauung, zur Gemeinschaft und zur Lehre dienen soll. Erich sorgt für die Erbauung, in der Cafeteria die Gemeinschaft und ich etwas zur Lehre.

        % Wieso ist es wichtig?
        Wir sind ja in dem Heilsplan Gottes und sind beim neuen Bund stehen geblieben. Letztes mal haben wir darüber geredet für wen dieser neue Bund gilt und wir haben anhand der Bibel herausgefunden, dass dieser neue Bund für die Heiden und für die Juden gilt. Nachdem die Juden Jesus als den Messias abgelehnt haben, hat sich Gott den Heiden zugewandt und es die Gemeinde entstanden. Die Gemeinde ist kein Volk. Sonder ein Leib, ein Organismus mit vershiedenen Völkern. Jesus ist unser Haupt und jeder der Jesus Christus als sein Herrn und Retter annimmt, wird Teil dieses Leibes. Paulus ging zuerst immer in eine Synagoge um die Juden zu erreichen. Meistens wurde er abgelehnt und er wandte sich den Heiden zu. Immer mehr Heiden und weniger Juden wurden Teil der Gemeinde.

        Dann war da die Zerstörung Jerusalems 70 Nach Chr. und 130 nach Chr. der Bar Kochba aufstand. Viele Juden wurden getötet, vertrieben und versklavt. Heiden Christen sahen darin die Strafe Gottes und sahen sich nun als neues Volk Gottes. So hat sich die Erstztheorie schon früh in der Kirchengesichte gebildet.

        Auch später in der Reformationszeit blieb diese Theorie bestehen. Wobei viele der Reformatoren sich nicht so intensiv mit diesem Thema auseinandersetzten. Die katholische Kirche hat diese Theorie übernommen und vertritt sie noch heute.

        Was diese Theorie bewirken kann, sehen wir daran, dass die Juden verfolgt und unterdrückt wurden, das vorallem von Christen.
        % Was sagt die Bibel dazu?

        Was aber sagt die Bibel dazu? Sind wir als Gemeinde das neue Volk Gottes? Paulus würde jetzt ausrufen: Das sei ferne!.
        Aber wollen wir anhand an ein paar Bibelstellen bekräftigen.

        Vor allem in der heutigen Zeit ist es wichtig, sich von dieser Theorie zu distanzieren. Nach den aktuellen Ereignissen im Nahen Ostern, fasst diese Theorie auch in evangelikalen Kreisen wieder Fuss.
        
        Wir sind nicht das neue Volk Gottes. Gott hat Israel nicht verworfen. Das sehen wir an dem Aufbau und Besiedelung des Landes Israel. Bei seiner zweiten Wiederkunft wird Gott sein Werk mit diesem Volk vollenden. Wie das geschieht werden wir auch noch sehen wenn wir zum letzen Stadium der Heilsgeschichte kommen. Wenn Jesus wiederkommt, wird es keine Gemeinde mehr geben sondern diese wird in den Himmel entrückt sein.

        Da wir nicht wissen, wann diese Entrückung stattfindet, ist es wichtig, dass wir immer bereit sind, dass wir würdig sind von Jesus empfangen zu werden. 

        Über dieses Thema haben wir am Büchertisch ein Buch von Dr. Vlach Hat die Gemeinde Israel ersetzt? Wir haben noch ein Exemplar können aber gerne noch ein paar bringen lassen.
        
    \end{block}



    \lied{Denn ich bin gewiss}.
   
\end{spacing}

\section{Predigt}
\textbf{Nach der Predigt}\newline
Danken für die Predigt.\newline

\section{Abendmahl}
Beten für das Brot\newline
Beten für den Wein\newline

\section{Abschluss}

\lied{Gross ist dein Name}

\beten{}

\begin{bibelbox}{SCHL}{IIKor}{13:13}
    Die Gnade des Herrn Jesus Christus und die Liebe Gottes und die Gemeinschaft des Heiligen Geistes sei mit euch allen! Amen
\end{bibelbox}

\end{document}