\author{OTS}
\documentclass{../../inc/mybib}

\setincpath{../../inc/}

\usepackage{bible_style}
\graphicspath{{../../assets/images/}}
\usepackage{header}

\newcommand{\Name}{Erich}
% ensure scrlayer-scrpage has sufficient footheight
\setlength{\footheight}{20.4pt}

\begin{document}

\section{Begrüssung}
Hallo \Name{}, wir freuen uns, dich in Naters begrüssen zu dürfen. Wir sind gespannt auf dein Wort.

Ich möchte auch euch alle recht herzlich zum ersten Gottesdienst 2026 begrüssen. Schön, dass ihr gekommen seid, um unserem \herr N zu Danken, ihn zu loben und zu preisen.

% \noindent
\beten{} Und anschliessend singen wir zusammen das Lied

% \noindent

\lied{Nur durch Christus in mir}

\section{Informationen}
\begin{itemize}
    \item \bt[infos]{Bibel und Gebetsabend:} Do, 05.02.2026 20:00 Uhr Bibel und Gebetsabend mit Thomas Lieth; zu Markus 4,30-34.
    \item \bt[infos]{Nächster Gottesdienst:} So, 08.02.2026 14:45 Uhr Videopredigt Elia Morris kann leider nicht kommen. Wir werden die Predigt vom Morgen aus Zürich zeigen.
    \item \bt[infos]{Kollekte:} Die Kollekte wird hinten in der grünen Box gesammelt und wird vollumfänglich für den Bau dieser Gemeinde eingesetzt.   
\end{itemize}

\section{Input}
\begin{spacing}{1.5}
    \begin{block}[Glaubensbekenntnis]
        Wir glauben, dass die Gemeinde Jesu Christi aus wiedergeborenen Juden und Heiden besteht (Gal 3,28; Eph 2,11-18; Kol 3,11) und Gottes Volk ist (Röm 9,26; 1Pt 2,9-10). Doch wir glauben auch, dass die Gemeinde nicht das einzige Volk des Herrn ist. Wir glauben, dass Gott die Nation Israel nicht verworfen (Jer 31; Röm 11), sondern lediglich beiseitegestellt hat, bis die Gemeinde vollzählig ist.
    \end{block}
    \begin{block}[Definition]
        % Definition
        Superzessionismus, Substitutionstheorie oder Ersatztheorie. Alle drei bedeuten das Gleiche. Es geht um die Frage, hat die Gemeinde Israel das Volk Gottes ersetzt? Sind jetzt wir das neue Israel, das neue Volk Gottes?

        Vielleicht denkt ihr jetzt, was hat das im Gottesdienst verloren? Ich bin der Meinung, dass ein Gottesdienst zur Erbauung, zur Gemeinschaft \betonung{und zur Lehre} dienen soll. Erich sorgt für die Erbauung, in der Cafeteria dann die Gemeinschaft und ich trage etwas Weniges zur Lehre bei. 
        \begin{bibelbox}{SCHL}{IITim}{3:16-17}
        Alle Schrift ist von Gott eingegeben und nützlich zur Lehre, zur Überführung, zur Besserung, zur Erziehung in der Gerechtigkeit, damit der Mensch Gottes ganz zubereitet sei, zu jedem guten Werk völlig ausgerüstet.
        \end{bibelbox}
    \end{block}
    \begin{block}[Biblische Betrachtung]    
        % Wieso ist es wichtig?
        Wir sind ja gegen Ende im Heilsplan Gottes und sind beim neuen Bund stehen geblieben. Letztes Mal haben wir darüber geredet, für wen dieser neue Bund gilt und wir haben anhand der Bibel herausgefunden, dass dieser neue Bund für die Heiden \betonung{und} für die Juden gilt. Nachdem die Juden Jesus als den Messias abgelehnt haben, hat sich Gott den Heiden zugewandt und es ist daraus die Gemeinde entstanden. Die Gemeinde selber ist kein Volk. Sondern ein Leib, ein Organismus aus vershiedenen Völkern. Jesus ist unser Haupt und jeder der Jesus Christus als sein Herr und Retter annimmt, wird Teil dieser weltweiten Gemeinde. Paulus ging zuerst immer in eine Synagoge, um die Juden zu erreichen. Meistens wurde er abgelehnt und danach wandte er sich den Heiden zu. Immer mehr Heiden und immer weniger Juden wurden Teil dieser Gemeinde.

        Dazu kamen die Zerstörungen Jerusalems 70 n. Chr. und 130 n. Chr. der Bar Kochba Aufstand. Viele Juden wurden getötet, vertrieben und versklavt. Heiden Christen sahen darin die Strafe Gottes, dass sie den Messias verworfen haben und sahen sich nun als das neue Volk Gottes. So hat sich die Ersatztheorie schon früh in der Kirchengeschichte gebildet. Man findet sie schon bei den Kirchenvätern wie z.B. bei Justinus dem Märtyrer (100-165 n.Chr.) und Irenäus von Lyon (130-202 n.Chr.).

        Auch später in der Reformationszeit blieb diese Theorie bestehen. Wobei viele der Reformatoren sich nicht so intensiv mit diesem Thema auseinandersetzten. Luther hat gegen Ende seines schaffens die Theorie vertreten. Die katholische Kirche hat diese Theorie übernommen und vertritt sie noch heute.

        Was diese Theorie bewirken kann, sehen wir daran, dass die Juden verfolgt und unterdrückt wurden, das vor allem von Christen. Und auch heute sehen wir wie der Anthisemitismus in Europa wieder zunimmt. Weil viele Christen die Juden als das verwerfene Volk Gottes sehen.
        % Was sagt die Bibel dazu?

        Was aber sagt die Bibel dazu? Sind wir als Gemeinde das neue Volk Gottes? Paulus würde jetzt ausrufen: Das sei ferne!.
        Aber wollen wir anhand an ein paar Bibelstellen bekräftigen.
        \begin{bibelbox}{SCHL}{Rom}{11:1-2}
        Ich frage nun: Hat Gott sein Volk verstoßen? Das sei ferne! Denn auch ich bin ein Israelit, aus dem Samen Abrahams, vom Stamm Benjamin. Gott hat sein Volk nicht verstoßen, das er zuvor erkannt hat. 
        \end{bibelbox}
        \begin{bibelbox}{SCHL}{Rom}{11:5}
            So ist nun auch in der jetzigen Zeit ein Überrest vorhanden aufgrund der Gnadenwahl.
        \end{bibelbox}
        \begin{bibelbox}{SCHL}{Rom}{11:25-26}
            Denn ich will nicht, meine Brüder, dass euch dieses Geheimnis unerkannt bleibt, damit ihr euch nicht selbst klug haltet: Israel ist zum Teil Verstockung widerfahren, bis die Vollzahl der Heiden eingegangen ist; und so wird ganz Israel gerettet werden, wie geschrieben steht: Es wird kommen aus Zion der Erlöser; er wird die Gottlosigkeiten von Jakob abwenden.
        \end{bibelbox}
        Aber lest selber die Kapitel 9 bis 11 im Römerbrief. Paulus macht da ganz klar, dass Gott sein Volk Israel nicht verworfen hat. 

        Vor allem in der heutigen Zeit ist es wichtig, sich von dieser Theorie zu distanzieren. Nach den aktuellen Ereignissen im Nahen Osten fasst diese Theorie auch in evangelikalen Kreisen wieder Fuss.
    \end{block}
    \begin{block}[Schlussfolgerung]
        Wir sind nicht das neue Volk Gottes. Gott hat Israel nicht verworfen. Das sehen wir konkret an dem Aufbau und Besiedelung des Landes Israel. Bei seiner zweiten Wiederkunft wird Gott sein Werk mit diesem Volk Israel vollenden. Wie das geschehen wird, werden wir sehen, wenn wir zum letzen Stadium der Heilsgeschichte kommen. Wenn Jesus wiederkommt, wird es keine Gemeinde mehr geben, sondern diese wird in den Himmel entrückt sein.

        Da wir nicht wissen, \betonung{wann} diese Entrückung stattfindet, ist es wichtig, dass wir immer bereit sind, dass wir jederzeit würdig sind von Jesus empfangen zu werden. 

        Wen das Thema interessiert, für den haben wir am Büchertisch ein Buch von Dr. Michael J. Vlach \enquote{Hat die Gemeinde Israel ersetzt?} 
        
    \end{block}



    \lied{Denn ich bin gewiss}.

\end{spacing}

\section{Predigt}
\begin{itemize}
\item[] \textbf{Nach der Predigt}
\item[] Danken für die Predigt.
\end{itemize}

\section{Abendmahl}
\begin{itemize}
\item[] Beten für das Brot
\item[] Beten für den Wein
\end{itemize}

\section{Abschluss}
\begin{itemize}
\item[] \lied{Gross ist dein Name}
\item[] \beten{}
\end{itemize}

\begin{bibelbox}{SCHL}{IIKor}{13:13}
    Die Gnade des Herrn Jesus Christus und die Liebe Gottes und die Gemeinschaft des Heiligen Geistes sei mit euch allen! Amen
\end{bibelbox}

\end{document}