\author{OTS}
\documentclass{../../inc/mybib}

\setincpath{../../inc/}

\usepackage{bible_style}
\graphicspath{{../../assets/images/}}
\usepackage{header}

\newcommand{\Name}{Nathanael}
% ensure scrlayer-scrpage has sufficient footheight
\setlength{\footheight}{20.4pt}

\begin{document}

\section{Begrüssung}
Hallo \Name{}, wir freuen uns, dich in Naters begrüssen zu können. Wir sind gespannt auf dein Wort.

Ich möchte auch euch alle recht herzlich zum ersten Gottesdienst 2026 begrüssen. Schön, dass ihr gekommen seid, um unserem \herr N zu Danken, ihn zu loben und zu preisen.

% \noindent
\beten{} Und anschliessend singen wir zusammen das Lied

% \noindent

\lied{In der Nacht von Betlehem}

\section{Informationen}
\begin{itemize}
    \item \bt[infos]{Bibel und Gebetsabend:} Do, 15.01.2026 20:00 Uhr Bibel und Gebetsabend mit Nathanael Winkler; zu Markus 4,1-20.
    \item \bt[infos]{Nächster Gottesdienst:} So, 11.01.2026 14:45 Uhr hier mit Thomas Lieth. 
\end{itemize}

\section{ Input }
\begin{spacing}{1.5}
    \begin{block}[Einführung]
    Ein neues Jahr hat begonnen. Viele Glückwünsche machen die Runde. \enquote{Gutes Neues Jahr}, \enquote{Viel Glück im neuen Jahr}, oder \enquote{Gott segne dich im neuen Jahr}, \enquote{Gesundheit und Glück} Heute im Zeitalter vom Handy sind solche Wünsche schnell verschickt. Entweder im Status oder per Whatsapp ein Bildchen oder eine Nachricht. Gut gemeinte Wünsche. Aber ob das sich das Leben nach diesen Wünschen richtet? Wohl eher nicht. Am 31. Dezember 2026 wissen wir dann mehr. Auch wenn wir nicht wissen, was uns das neue Jahr an Gesundheit und Glück bringt, wissen wir, dass unser Gott einen Plan hat. Einen Plan mit uns und mit dieser Welt. An diesem Plan wollen wir uns auch dieses Jahr daran entlang angeln. Letztes Jahr haben wir uns an der Heilsgeschichte Gottes mit uns Menschen entlang gearbeitet. Angefangen bei der Schöpfung, über den Bund mit Noah, Abraham, Mose und David. Heute zum Jahresbeginn wollen wir gemeinsam den Neuen Bund anschauen.

    Dafür müssen wir zusammen die Bibel aufschlagen, und zwar \bibleverse{Jer}(31:31-40).\\
    Folgende Verheissungen werden hier aufgelistet:
    \begin{enumerate}
        \item \textbf{Die Herzensbeschneidung und das neue Herz} Das Totale versagen Israels gegenüber Gott, hat gezeigt, dass der Mensch keinen Gehorsam gegenüber einem lebednigen Gott hat. Nur durch die Erneuerung des Herzens kann der Mensch Gott erkennen.\\
        \begin{bibelbox}{SCHL}{VMos}{30:6}
            Und der Herr dein Gott, wird dein Herz und da Herz deiner Nachkommen beschneiden, dass du den Herrn, deinen Gott, liebst von ganzem Herzen und von ganzer Seele, damit du lebst.            
        \end{bibelbox}
        \begin{bibelbox}{SCHL}{Hes}{36:26}
            Und ich will euch ein neues Herz geben und einen neuen Geist in euer Inneres legen; ich will das steinerne Herz aus eurem Fleisch wegnehmen und euch ein fleischernes Herz geben;
        \end{bibelbox}
        \item \textbf{Die Ausgießung des Heiligen Geistes} Das errettete Volk wird dadurch befähigt den Willen Gottes von innen heraus gerne zu tun. Damit verbunden ist die Erkenntnis und Belehrung durch den Geist Gottes.\\
        \begin{bibelbox}{SCHL}{Joel}{3:1-5}
            Und nach diesem wird es geschehen, dass ich meinen Geist ausgiesse über alles Fleisch; und eure Söhne und eure Töchter werde weisssagen, eure Ältesten werden Täume haben, eure jungen Männer werde Gesichte sehen:
            \textit(und der Vers 5)
            Und es wird geschehen: Jeder, der den Namen des Herrn anruft, wird gerettet werden; denn auf dem Berg Zion und in Jerusalem wird Errettung sein, wie der \herr{} verheissen hat, und bei den Übriggeblieben, die der \herr{} beruft.
        \end{bibelbox}
        \item \textbf{Vergebung der Sünden} Gott selbst wird ihre Schuld hinwegnehmen.\\
        \begin{bibelbox}{SCHL}{Hes}{36:25}
           Und ich will reines Wasser über euch sprengen, und ihr werdet rein sein; von aller eurer Unreinheit und von allen euren Götzen will ich euch reinigen.
        \end{bibelbox} 
        \begin{bibelbox}{SCHL}{Hes}{36:33}
           So spricht Gott der Herr: Zu jener Zeit, wenn ich euch reinigen werde von allen euren Missetaten, da will ich euch wieder in den Städten wohnen lassen, und die Trümmer sollen wieder aufgebaut werden.
        \end{bibelbox} 
        \item \textbf{Die Sammlung und Wiederherstellung Israels} In Hesekiel 37, das ist die Geschichte, mit den Gebeinen die wieder zum Leben erweckt wurden, wird die Sammlung in zwei Phasen beschrieben. Zuerst wird das Volk äusserlich wieder hergestellt, danach wird es innerlich erneuert.
    \end{enumerate}
\end{block}

    \begin{block}[Die Gültigkeit des neuen Bundes]
        Für wen ist dieser neue Bund gültig? 
        \begin{bibelbox}{SCHL}{Jer}{31:31}
           Siehe, es kommen Tage, spricht der \herr{}, da ich mit dem Haus Israel und mit dem Haus Juda einen neuen Bund schliessen werde;
        \end{bibelbox} 
        Also schon mal mit dem Volk Israel.
        \begin{bibelbox}{SCHL}{Jes}{49:6}
           Der \herr{} spricht: Es ist zu gering, dass du mein Knecht bist, um die Stämme Jakobs aufzurichten und die Bewahrten aus Israel wiederzubringen; sondern ich habe euch zum Licht für die Heiden gesetzt, damit du mein Heil seist bis an das Ende der Erde.
        \end{bibelbox}
        Gott hat Israel Auserwählt um ein Heil und Licht für die Heisen zu sein. Also auch für uns gilt dieser neue Bund. Der Unterschied besteht nur im Zeitpunkt des Inkrafttretens des neuen Bundes, für Israel und die Gemeinde aus Juden und Heiden.
    \end{block}
    \begin{block}[Die Einsetzung des neuen Bundes]
        Gott ist der Schenkende und Handelnde in diesem neuen Bund. Er ist allein in Gottes Verheissung und Tat begründet. Damit werden wir unwillkürlich in das Zentrum des Heilsplans Gottes geführt: \betonung{Jesus Christus} und sein vollbrachtes Werk. Er selbst hat den neuen Bund für uns eingesetzt.
        \begin{bibelbox}{SCHL}{Luk}{22:17-20}
            Und er nahm den Kelch, dankte und sprach: Nehmt diesen und teilt ihn unter euch! Denn ich sage euch: Ich werde nicht mehr von dem Gewächs des Weinstocks trinken, bis das Reich Gottes gekommen ist. Und er nahm das Brot, dankte, brach es, gab es ihnen und sprach: Dies ist mein Leib der für euch gegeben wird; das tut zu meinem Gedächtnis! Desgleichen nahm er auch den Kelch nach dem Mahl und sprach: Dieser Kelch ist der neue Bund in meinem Blut, das für euch vergossen wird.
        \end{bibelbox}
        Beim Abendmahl mit den Jüngern vor seinem Tod hat Jesus den neuen Bund eingesetzt. 
    \end{block}
   
    \begin{bibelbox}{SCHL}{Apg}{1:11}
        Dieser Jesus, der von euch weg in den Himmel aufgenommen worden ist, wird in derselben weise wiederkommen, wie ihr ihn habt in den Himmel auffahren sehn.
    \end{bibelbox}  

    Singen wir zusammen ein Lied und nach dem Lied möchte ich \Name{} das Wort übergeben.

    \lied{Engel bringe frohe Kunde}.
   
\end{spacing}

\section{Predigt}

\textbf{Nach der Predigt}\\
Danken für die Predigt.\\
\section{Abendmahl}\\
Beten für das Brot\\
Beten für den Wein\\

\section{Abschluss}

Jetzt singen wir zusammen noch die Weihnachtslieder

\lied{O du fröhliche} und


\beten{}

\begin{bibelbox}{SCHL}{1Mos}{28:15}
    Gott spricht: Siehe, ich bin mit dir,
    ich behüte dich, wohin du auch gehst.
    Denn ich verlasse dich nicht,
    bis ich vollbringe, was ich dir versprochen habe.
\end{bibelbox}

Maranatha, komm Herr Jesus! Amen\\
\lied{Stille Nacht}\\
\end{document}