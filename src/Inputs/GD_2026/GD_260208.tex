\author{OTS}
\documentclass[14pt]{../../inc/mybib}

\setincpath{../../inc/}

\usepackage{bible_style}
\graphicspath{{../../assets/images/}}
\usepackage{header}

\newcommand{\Name}{Samuel}
% ensure scrlayer-scrpage has sufficient footheight
\setlength{\footheight}{20.4pt}

\begin{document}

\section{Begrüssung}
Ich möchte auch euch alle recht herzlich zum Gottesdienst begrüssen. Schön, dass ihr gekommen seid, um unserem \herr N zu Danken, ihn zu loben und zu preisen.

% \noindent
\beten{} Und anschliessend singen wir zusammen das Lied

% \noindent

\lied{Herr du gibst uns Hoffnung}

\section{Informationen}
\begin{itemize}
    \item \bt[infos]{Bibel und Gebetsabend:} Do, 12.02.2026 20:00 Uhr Bibel und Gebetsabend mit Nathanael Winkler; zu Markus 4,35-41.
    \item \bt[infos]{Nächster Gottesdienst:} So, 15.02.2026 14:45 Uhr mit Thomas Lieth
    \item \bt[infos]{Kollekte:} Die Kollekte wird hinten in der grünen Box gesammelt und wird vollumfänglich für den Bau dieser Gemeinde eingesetzt.
\end{itemize}

\section{ Input }
\begin{spacing}{1.5}
    \begin{block}[Einführung]
        Der neue Bund wird in \bibleverse{Jer}(31:31-40) beschrieben. Da heute kein Prediger auf den Zug muss, habe ich alle Zeit der Welt. 
        Da es schon eine Zeit her ist seit wir uns mit diesem Thema beschäftigt haben, möchte ich euch den Text nochmals vorlesen.
        Folgende Zusagen macht Gott seinem Volk im Neuen Bund:
        \begin{enumerate}
            \item \textbf{Die Herzensbeschneidung und das neue Herz}: Das totale Versagen Israels gegenüber Gott hat gezeigt, dass der Mensch keinen Gehorsam gegenüber einem lebendigen Gott hat.
            \item \textbf{Die Ausgießung des Heiligen Geistes}: Das errettete Volk wird dadurch befähigt den Willen Gottes von innen heraus gerne zu tun. Damit verbunden ist die Erkenntnis und Belehrung durch den Geist Gottes.
            \item \textbf{Vergebung der Sünden}: Gott selbst wird ihre Schuld hinwegnehmen.
            \item \textbf{Die Sammlung und Wiederherstellung Israels}:  Zuerst wird das Volk äusserlich wieder hergestellt, danach wird es innerlich erneuert und der Geist eingehaucht.
        \end{enumerate}
    \end{block}

    \begin{block}[Der neuen Bundes]
        Was bedeutet der Neue Bund für uns Christen und der Gemeinde. Letzten Sonntag haben wir darüber gesprochen ob wir Christen das neue Volk Gottes sind, oder ob wir Israel ersetzt haben. Die Hausaufgaben war Rönmer 9 - 11 zu lesen und somit habt ihr gelesen, das dem nicht so ist. Aber was haben wir dann mit dem neuen Bund zu tun?

        Für wen ist dieser Neue Bund gültig? Dazu lesen wir in \bibleverse{Jer}(31:31).
        \begin{bibelbox}{SCHL}{Jer}{31:31}
            Siehe, es kommen Tage, spricht der \herr{}, da ich mit dem Haus Israel und mit dem Haus Juda einen neuen Bund schliessen werde;
        \end{bibelbox}

    \end{block}
    \begin{block}[Der neuen Bundes für Christen]
        In Hebräer 8,8-13 wird deutlich, dass der Neue Bund auch für die Gemeinde eine grundlegende Bedeutung hat. Auch die Einsetzungsworte des Abendmahls lese (vgl. Mt 26,26-29 und Lk 20,17-22). So ist der neue Bund, den Christus mit seinem Blut für geschlossen hat, nicht nur der Grund für Israels zukünftige Errettung. Er ist zugleich auch das Fundament der Gemeinde Jesu. Sie steht heute schon unter dem Neuen Bund, während Israel mit der Wiederkunft Jesu und seiner Errettung in den Bund eintreten wird.

        Gott ist der Schenkende und Handelnde im neuen Bund. Er ist allein in Gottes Verheissung und Tat begründet. Damit werden wir automatisch in das Zentrum des Heilsplanes Gottes Geführt: \betonung{Jesus Christus und sein vollbrachtes Werk}.

        Er selbst hat in der Nacht vor der Kreuzigung den neuen Bund für uns eingesetzt. Lukas 20,17-22:
        \begin{bibelbox}{ELB}{Luk}{22:17-20}
            Und er nahm einen Kelch, dankte und sprach: Nehmt diesen und teilt ihn unter euch. Denn ich sage euch, dass ich von jetzt an nicht von dem Gewächs des Weinstocks trinken werde, bis das Reich Gottes kommt. Und er nahm Brot, dankte, brach und gab ihnen und sprach: Dies ist mein Leib, der für euch gegeben wird; dies tut zu meinem Gedächtnis! Ebenso auch den Kelch nach dem Mahl und sagte: Dieser Kelch ist der \betonung{neue Bund} in meinem Blut, das für euch vergossen wird.
        \end{bibelbox}
        So ist in dem neuen Bund nicht nur die zukünftige Errettung Israels begründet, sondern die \betonung{Errettung aller Menschen}, die durch das Evangelium zur Erkenntnis der Wahrheit und zur Gemeinde Jesu hinzukommen. In Matthäus 26,28 ist deshalb die Rede von dem Blut des neuen Bundes, welches zur Vergebung für viele vergossen wird. Das heisst nicht, dass das Blut Christi nicht für alle vergossen worden ist, sondern dass es nicht alle annehmen.

        Wie erwähnt ist der neue Bund allein in Christus und seinem vollbrachten Werk begründet. Er hat die Gerechtigkeit vollbracht, die wir benötigen um mit Gott Gemeinschaft haben zu können (Röm 3,22).
        \begin{bibelbox}{SCHL}{Rom}{3:22}
            die Gerechtigkeit Gottes durch den Glauben an Jesus Christus für alle und auf alle, die glauben. Denn es ist kein Unterschied:
        \end{bibelbox}
        Er hat die Sühnung und Vergebung für unsere Sünden vollbracht (Eph 1,7; 1Joh 2,4).
        \begin{bibelbox}{SCHL}{Eph}{1:7}
            In ihm haben wir die Erlösung durch sein Blut, die Vergebung der Sünden, nach dem Reichtum seiner Gnade,
        \end{bibelbox}
        \begin{bibelbox}{SCHL}{1Joh}{2:2}
            und er ist das Sühneopfer für unsere Sünden, aber nicht nur für die unseren, sondern auf für die ganze Welt.
        \end{bibelbox}
        Und um sein Werk zu vollenden hat er an Pfingsten den Heiligen Geist gesandt (Joh 16,7),
        \begin{bibelbox}{SCHL}{Joh}{16:7}
            Aber ich sage euch die Wahrheit: Es ist gut für euch, dass ich hingehe; denn wenn ich nicht hingege, so kommt der Beistand nicht zu euch. Wenn ich aber hingegangen bin, will ich ihn euch senden.
        \end{bibelbox}
        welcher Menschen zur Wiedergeburt bringt und befähigt, ein Leben zur Ehre Gottes zu führen Zugleich ist es der Geist der Wahrheit, welcher uns das Wort Gottes aufschliesst und Christus verherrlicht (Joh 16,13-14). Paulus nennt sich in 2.Korinther 3,6 ein Diener des neuen Bundes. In diesem Kapitel macht er in Vers 7-18 nochmals den Unterschied zwischen dem alten Bund (Mosebund) und dem neuen Bund deutlich. Dabei können wir erkennen, dass es um eine und denselben Bund geht, durch welchen Menschen schon heute errettet werden und auch Israel in Zukunft zur Erkenntnis Christi kommt und seine Erneuerung und Errettung erfährt. Paulus war das Evangelium für die Nationen aufgetragen. Zugleich konnte er aber auch sagen, dass er um der Hoffnung Israels willen gefangen war (Apg 26,6; 28,20).
        \begin{bibelbox}{SCHL}{Apg}{26:6}
            Und jetzt stehe ich vor Gericht wergen der Hoffnung auf die Verheissung, die von Gott an die Väter ergangen ist,
        \end{bibelbox}
        Es gibt noch viel mehr Stellen in der Bibel die beweisen, dass der neue Bund sowohl für die Gemeinde wie auch für Israel Gültigkeit hat. Der Unterschied liegt einzig darin, dass die Gemeinde heute schon im neuen Bund lebt und Israel erst in Zukunft in diesen Bund eintreten wird. Das werden wir dann in den nächsten Sonntagen anschauen.

    \end{block}

    \lied{}.

\end{spacing}

\section{Predigt}
\textbf{Nach der Predigt}\newline
Danken für die Predigt.\newline
% \section{Abendmahl}
% Beten für das Brot\newline
% Beten für den Wein\newline

\section{Abschluss}

\lied{}

\beten{}

\begin{bibelbox}{SCHL}{IIKor}{13:13}
    Die Gnade des Herrn Jesus Christus und die Liebe Gottes und die Gemeinschaft des Heiligen Geistes sei mit euch allen! Amen
\end{bibelbox}

\end{document}