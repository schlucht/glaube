\author{OTS}
\documentclass[14pt]{../../inc/mybib}

\setincpath{../../inc/}

\usepackage{bible_style}
\graphicspath{{../../assets/images/}}
\usepackage{header}

\newcommand{\Name}{Samuel}
% ensure scrlayer-scrpage has sufficient footheight
\setlength{\footheight}{20.4pt}

\begin{document}

\section{Begrüssung}
Ich möchte auch euch alle recht herzlich zum Gottesdienst begrüssen. Schön, dass ihr gekommen seid, um unserem \herr N zu Danken, ihn zu loben und zu preisen.

% \noindent
\beten{} Und anschliessend singen wir zusammen das Lied

% \noindent

\lied{Seid fröhlich in der Hoffnung}

\section{Informationen}
\begin{itemize}
    \item \bt[infos]{Bibel und Gebetsabend:} Do, 12.02.2026 20:00 Uhr Bibel und Gebetsabend mit Nathanael Winkler; zu Markus 4,35-41.
    \item \bt[infos]{Nächster Gottesdienst:} So, 15.02.2026 14:45 Uhr mit Thomas Lieth
    \item \bt[infos]{Kollekte:} Die Kollekte wird hinten in der grünen Box gesammelt und wird vollumfänglich für den Bau dieser Gemeinde eingesetzt.
\end{itemize}

\section{ Input }
\begin{spacing}{1.5}
    \begin{block}[Einführung]    
    Der neue Bund wird in \bibleverse{Jer}(31:31-40) beschrieben.\\    
    Folgende Verheissungen werden hier aufgelistet. Wir sehen sie auch auf der Folie:
    \begin{enumerate}
        \item \textbf{Die Herzensbeschneidung und das neue Herz}: Das totale Versagen Israels gegenüber Gott hat gezeigt, dass der Mensch keinen Gehorsam gegenüber einem lebendigen Gott hat.         
        \item \textbf{Die Ausgießung des Heiligen Geistes}: Das errettete Volk wird dadurch befähigt den Willen Gottes von innen heraus gerne zu tun. Damit verbunden ist die Erkenntnis und Belehrung durch den Geist Gottes.   
        \item \textbf{Vergebung der Sünden}: Gott selbst wird ihre Schuld hinwegnehmen.         
        \item \textbf{Die Sammlung und Wiederherstellung Israels}:  Zuerst wird das Volk äusserlich wieder hergestellt, danach wird es innerlich erneuert und der Geist eingehaucht.        
    \end{enumerate}
\end{block}

    \begin{block}[Der neuen Bundes]
        Schauen wir uns diesen Neuen Bund etwas genauer an.
        Für wen ist dieser Neue Bund gültig? Dazu lesen wir in \bibleverse{Jer}(31:31).
        \begin{bibelbox}{SCHL}{Jer}{31:31}
           Siehe, es kommen Tage, spricht der \herr{}, da ich mit dem Haus Israel und mit dem Haus Juda einen neuen Bund schliessen werde;
        \end{bibelbox} 
               
        \end{block}

    Singen wir zusammen ein Lied und nach dem Lied möchte ich \Name{} das Wort übergeben.

    Dieses Lied ist neu. Aber es ist relativ einfach und ist ein Kinderlied. Der Text ist aber wunderschön und passt perfekt zu unserem Thema heute.
    \lied{Wie wissen nicht den Tag noch die Stunde}.
   
\end{spacing}

\section{Predigt}
\textbf{Nach der Predigt}\newline
Danken für die Predigt.\newline
% \section{Abendmahl}
% Beten für das Brot\newline
% Beten für den Wein\newline

\section{Abschluss}

\lied{Wunderbar grosser Erlöser}

\beten{}

\begin{bibelbox}{SCHL}{IIKor}{13:13}
    Die Gnade des Herrn Jesus Christus und die Liebe Gottes und die Gemeinschaft des Heiligen Geistes sei mit euch allen! Amen
\end{bibelbox}

\end{document}