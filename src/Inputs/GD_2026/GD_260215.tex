\author{OTS}
\documentclass[14pt]{../../inc/mybib}

\setincpath{../../inc/}

\usepackage{bible_style}
\graphicspath{{../../assets/images/}}
\usepackage{header}

\newcommand{\Name}{Thomas}
% ensure scrlayer-scrpage has sufficient footheight
\setlength{\footheight}{20.4pt}

\begin{document}

\section{Begrüssung}
Ich möchte auch euch alle recht herzlich zum Gottesdienst begrüssen. Schön, dass ihr gekommen seid, um unserem \herr N zu Danken, ihn zu loben und zu preisen.
Auch dich \name will ich herzlich begrüssen. Schön, dass du wieder bei uns bist und mit uns Gottesdienst feierst.
% \noindent
\beten{} Und anschliessend singen wir zusammen das Lied

% \noindent

\lied{Herr du gibst uns Hoffnung}

\section{Informationen}
\begin{itemize}
    \item \bt[infos]{Bibel und Gebetsabend:} Do, 19.02.2026 20:00 Uhr Bibel und Gebetsabend ; zu Markus 4,35-41.
    \item \bt[infos]{Nächster Gottesdienst:} So, 22.02.2026 14:45 Uhr Video und Israelbilder mit Lothar
    \item \bt[infos]{Kollekte:} Die Kollekte wird hinten in der grünen Box gesammelt und wird vollumfänglich für den Bau dieser Gemeinde eingesetzt.
\end{itemize}

\section{ Input }
\begin{spacing}{1.5}
    \begin{block}[Die verstockung Israels]
        Römer 1,1-32
        In den letzten Sonntagen haben wir darüber gesprochen, was der Grund der Gemeinde ist. Ob wir jetzt als Gemeinde Israel verdrängt oder ersetzt haben. Dazu lesen wir  zuerst den Text aus Römer 11,25-27.
        \begin{bibelbox}{Römer 11,25-27}
            ....
        \end{bibelbox}
        Wie erwähnt verstockt Gott Israel für eine Zeit. Sie werden vom Hauptgleis auf das Nebengleis der Heilsgeschichte gestellt. Die Verstockung ist Folge von Israels Ab        lehnung des Messias. Zugleich gebraucht der Herr den Fall Israels, damit sich das Heil den Heiden zuwendet. Nun macht Paulus deutlich, wie Gott sich durch alles hin        durch verherrlicht. Sogar durch die Verstockung Israels. In Vers 12 spricht er davon, dass schon der Fall Israels zum Reichtum der Welt wurde. Er wendet sich dadurch       den Nationen zu und baut seine Gemeinde aus Juden und Heiden. So wird ihr Fall zum Reichtum, womit Israel aber nicht entschuldigt ist. Wie viel mehr wird dann ihre        Fülle, oder ihre Errettung, der Welt zum Heil werden. Damit richtet Paulus den Blick auf das Tausendjährigen Reiches den damit verbunden Segnungen für die Nationen.        Hier wird die ganze Weisheit des göttlichen Heilsplans deutlich, die unser menschliches Denken sprengt und zur Anbetung führt. Gott allein bekommt darüber alle Ehre,       wie wir in Römer 11,33-35 lesen. Er verherrlicht sich sogar durch sein Gerichtshandeln und alles dient dazu, dass er zum Ziel kommt.
    \end{block}

    \lied{}.

\end{spacing}

\section{Predigt}
\textbf{Nach der Predigt}\newline
Danken für die Predigt.\newline
% \section{Abendmahl}
% Beten für das Brot\newline
% Beten für den Wein\newline

\section{Abschluss}

\lied{}

\beten{}

\begin{bibelbox}{SCHL}{IIKor}{13:13}
    Die Gnade des Herrn Jesus Christus und die Liebe Gottes und die Gemeinschaft des Heiligen Geistes sei mit euch allen! Amen
\end{bibelbox}

\end{document}