\documentclass[12pt,a4paper]{scrarticle}

\usepackage[headsepline, footsepline]{scrlayer-scrpage}
\usepackage{blindtext}% nur für Fülltext
\usepackage[ngerman]{babel}
\usepackage[utf8]{inputenc}
\usepackage[T1]{fontenc}
\usepackage{lmodern}
\usepackage{blindtext}
\usepackage{graphicx}
\usepackage{xcolor}
\usepackage[most]{tcolorbox}
\usepackage{geometry}
\usepackage[version=4]{mhchem}
\usepackage{bibleref-german}
\usepackage{setspace}
%%%%%%%%%%%%%%%%%%%%%%Link Formatierung%%%%%%%%%%%%%%%%%%%%%%%%%%%
\usepackage[colorlinks = true,
linkcolor = black,
urlcolor  = blue,
citecolor = black,
anchorcolor = blue]{hyperref}
%\documentclass[xcolor=dvipsnames]{beamer}

\pagestyle{scrheadings}

\chead{\headmark}
\automark{section}



\cfoot{Seite \pagemark}



%%%%%%%%%%%%%%%%%%%%%Griechische Zeichen%%%%%%%%%%%%%%%%%%%%%%%%%%%%%%%%%%%
%\begin{otherlanguage}{polutonikogreek} 
%	Δοκεῖ δέ μοι καὶ Καρχηδόνα μὴ εἶναι. 
%\end{otherlanguage} 



%%%%%%%%%%%%%%%%%%%%%%%%%%%%%%%%%%%%%%%%%%%%%%%%%%%%%%%%%%%%%%%%%%
%
% Makro zur Namensvervollstädigung
% Parameter: Kürzel --- Bibel
%			LuN			Neue Luther Übersetzung
%			Sch2		Schlachter 2000
%			Elb			Elbfelder
%%%%%%%%%%%%%%%%%%%%%%%%%%%%%%%%%%%%%%%%%%%%%%%%%%%%%%%%%%%%%%%%%%%
	\newcommand{\bib}[1]{%
	\ifthenelse{\equal{#1}{EI}}{Einheitsübersetzung}{%
		\ifthenelse{\equal{#1}{Sch2}}{Schlachter 2000}{%
			\ifthenelse{\equal{#1}{HFA}}{Hoffnung für Alle}{%
				\ifthenelse{\equal{#1}{ELB}}{Elbfelder}{%
					\ifthenelse{\equal{#1}{Neü}}{neue ev. Übersetzung}{%
						\ifthenelse{\equal{#1}{Lut}}{Luther}{#1}%
					}%
				}%
			}%
		}%
	}%
}%

%%%%%%%%%%%%%%%%%%%%%%%%%%%%%%%%%%%%%%%%%%%%%%%%%%%%%%%%%%%%%%%%%%
%
% Makro für Bibelzitate
% 
% Beispiel: 
%		\begin{bibeltext}{ELB}{Matt}{1:1-4}
%			Ich bin der zitierte Bibeltext.
%		\end{bibeltext}
%1Petr, Matt, Offb, 1Mos, Joh, IThess, 2Thess, Kol, Ps, Jak, ITim
%%%%%%%%%%%%%%%%%%%%%%%%%%%%%%%%%%%%%%%%%%%%%%%%%%%%%%%%%%%%%%%%%%
\newcommand{\bibtit}[1]{\large\bib{#1}}
\newcommand{\tmpbib}{Hallo}
\newcommand{\herr}{HERR}

\newcommand{\lied}[1]{\colorbox{yellow}{\textcolor{blue}{Lied: #1}}}
\newcommand{\green}[1]{\colorbox{green}{\textcolor{black}{#1}}}
\newcommand{\red}[1]{\colorbox{red}{\textcolor{white}{#1}}}
\newcommand{\beten}{\red{Ich möchte noch beten!}}
\newcommand{\hh}[1]{\textsuperscript{#1}}
\newenvironment{bibeltext}[3]{%	
        \renewcommand{\tmpbib}{\textbf{\bibleverse{#2}( #3)}}
		\biblerefformat{lang}
        \begin{tcolorbox}[
        title=\tmpbib \hspace{1.5cm} (\bibtit{#1}),
        colback=black!2!white,
        colframe=black!55!black]        
	}{%	
		\newline		
        \end{tcolorbox}	
	\endquote		
}
% \begin{tcolorbox}[title=Bibeltext,
%     title filled=false,
%     colback=blue!5!white,
%     colframe=blue!75!black]
% \end{tcolorbox}


\title{\includegraphics[height=48pt]{assets/images/logo.png}\\Kaleb eine Entscheidung}
\author{Lothar Schmid}
\begin{document}
\maketitle
\section{Kaleb}
Kaleb wird nur wenig in der Bibel erwähnt. Trotzdem ist er ein Beispiel von Mut und Treue in der Nachfolge Gottes. Kaleb wurde auserwählt um für den Stamm Juda, das von Gott versprochene Land zu erkunden. Alle waren sich einig, dass das Land gut war, in dem Milch und Honig fliesst. Aber sie fürchteten sich vor den Einwohner des Landes.10 Fürsten der anderen Stämme die an der Erkundung mit dabei waren rebellierten. In diese Rebellion hinein sprach Kaleb:
\begin{bibeltext}{Sch2}{4Mos}{13:30}
    Was siehst du aber den Splitter im Auge deines Bruders, und den Balken in deinem
Auge bemerkst du nicht?
\end{bibeltext}
Niemand hörte auf ihn
\begin{bibeltext}{Sch2}{4Mos}{14:1-4}
    Was siehst du aber den Splitter im Auge deines Bruders, und den Balken in deinem
Auge bemerkst du nicht?
\end{bibeltext}
Kaleb musste eine Entscheidung treffen. Josua die Rechte Hand Mose und Kaleb waren die einzigen der 12 Kundschaft-er, waren die einzigen die überzeugt waren, dass sie es mit der Hilfe von schaffen können, diese Land einzunehmen. Das ganze Volk aber, die Mehrheit entschied sich dagegen. Dieses Rebellion gegen Gott hatte weitreichende folgen.
Auch wir müssen uns entscheiden. 
\begin{bibeltext}{Sch2}{Joh}{3:36}
    Was siehst du aber den Splitter im Auge deines Bruders, und den Balken in deinem
Auge bemerkst du nicht?
\end{bibeltext}
Treffen wir die Entscheidung für Jesus, gehören wir ab diesem Zeitpunkt einer Minderheit an. Kaleb entging wegen seiner Entscheidung nur knapp einer Steinigung. Bei uns ist es hier nicht so extrem, trotzdem stösst man auf Unverständnis oder auch auf Spott.

In 
\begin{bibeltext}{Sch2}{4Mos}{14:24}
    Was siehst du aber den Splitter im Auge deines Bruders, und den Balken in deinem
Auge bemerkst du nicht?
\end{bibeltext}
Ist das nicht wunderbar Gott das von dir und mir sagen würde? Kaleb blieb Gott treu. Trotzdem wurde er nicht aus der Gemeinschaft gerissen und ins gelobte Land gesetzt, sondern musste wie alle anderen 40Jahre auf die Besitznahme des Landes warten. Aber Gott hat ihm versprochen, dass er hinein kommt. 
Auch und hat Gott versprochen das wir eines Tages in den Himmel kommen wo er viele Wohnungen hat.
\begin{bibeltext}{Sch2}{Joh}{6:51-52}
    Was siehst du aber den Splitter im Auge deines Bruders, und den Balken in deinem
Auge bemerkst du nicht?
\end{bibeltext}
Wenn sich unser Mut und vertrauen auf Gottes Wort stützen, werden wir seine Hilfe und seinen Segen persönlich erleben. Durch unser verhalten können wir andere Menschen ermutigen, sich für Gott zu entscheiden, aber auch sie entmutigen ihm zu folgen, wie es die anderen 10 Fürsten der Stämme Israel getan haben.

Darum lasst uns beten und dem Herrn danken, dass er uns auserwählt hat.
\end{document}