\author{OTS}
\documentclass[12pt]{../../inc/mybib}

\setincpath{../../inc/}

\usepackage{bible_style}
\usepackage{header}

\author{Lothar Schmid}
\begin{document}

\section{Glaubensbekenntnis}
\subsection{Über die Bibel}
Unser Glaube ist auf die Bibel als das unfehlbare, vom Heiligen Geist inspirierte Wort Gottes gegründet (1Kor 2,13; 1Th 2,13; 2Tim 3:16; 2Pt 1,19-21). Darum sind wir auch bemüht, den \glqq ganzen Ratschluss Gottes\grqq, der uns in Seinem Wort gegeben ist zu verkündigen (Mt 28,18-20; Apg 20,27).

Weil wir glauben, dass die ganze Bibel von Gott inspiriert ist, glauben wir auch, dass sich alle noch offenen biblischen Prophezeiungen sicher so erfüllen werden wie von Gott versprochen (Jes 46,9-10; 55,11; Mt 5, 17-18; Röm 3,3-4;11; Offb 22,6-20).

Wir glauben, dass die Bibel in ihrem einfachsten Sinn zu verstehen und auszulegen ist (ps 33,4; Joh 5,31-47; 2Tim 3,14-17; Offb 1,3). Darum glauben wir auch, dass die Prophetie der Bibel das aussagt, was sie bei einfacher Interpretation sagt, und nicht vergeistigt, werden muss (Offb 1,3;22,6). Wir glauben, dass man keine höhere Erkenntnis oder Deutungskunst braucht, um die zu verstehen, sonden den Heilien Geist (Joh 7,17; Apg 1,18; 1Kor 2,7-15).

\subsection{Über Gott und den Menschen}
Auf der Grundlage dieses Wortes Gottes glauben wir, dass es nur einen ewigen und vollkommen Gott gibt, der der Schöpfer aller Dinge ist -- wie unserer Welt, die Er im Wesen eins ist und ewig in drei Personen existiert: in Gott dem Vater, GOtt dem Sohn dem Heiligen Geist (Mt 28,19; 1Kor 12,4-6; 2Kor 13,13).

Wir glauben, dass der Mensch von Gott Seinem Bild ähnlich geschaffen wurde, und zwar als Mann oder als Frau (1Mo 1,26-2-25; Jak 3,9). Gott setzte die Ehe als Verbindung zwischen Mann und Frau ein und gab ihnen den Auftrag, sich dei Erde untertan zu machen und sich zu vermehren. Doch die Sünde des ersten Menschen hat den Tod in die Welt gebracht, und seitdem wird das Gift der Sünde an jeden Menschen weiterverebt. Deswegen widerstreben wir in vielfältiger Weise der Schöpfungsordnung und dem Willen Gottes. Wir ziehen Seinen Zorn auf uns wegen unserer bösen Gedanken und Taten (1Mo 2,16-17;Röm 1-3;5,10.19; Eph 2,1-3; 4,22; Jak 4,4).

Wir glauben, dass Gottes ewiger Sohn Mensch wurde, geboren von einer Jungfrau, gezeugt durch den Heiligen Geist -- und deswegen unbelastet von der Erbsünde war (Jes 7,14; Mt 1,23; Lk 1,35; Röm 5; Gal 4,4) Er lebte als ganzer Gott und ganzer Mensch ein sündloses Leben und nahm am Kreuz von Golgatha die Sünden der Welt auf sich (2Kor 5,21; Phil 2,6-11). Seine Auferstehung bewies Seine Gottessohnschaft und die Annahme des Opfers durch Gott den Vater (Röm 1,4). Er fuhr in den Himmel auf und sitzt nun zur Rechten Gottes des Vaters (Mk 16,19), wo Er sich als Fürsprecher und Hohepriester für Seine Erlöste einsetzt (Hebr 4,14; 1Joh 2,1).
\subsection{Über die Erlösung}
Wir glauben, dass jeder Mensch -- ob Jude oder Nichtjude -- allein durch den Glauben an Jesus Christus und nur aus Gnade mit Gott versöhnt und erlöst werden kann (Apg 4,12; Röm 3,21-31; 5,1-11; Eph 2,1-10).

Wir glauben, dass jeder errettet wird, der den Herrn anruft und glaubt und bekennt, dass Gottes Sohn, Jesus Christus, für seine Sünden gestorben, zu seiner Errettung auferstanden und sein Herr ist (Apg 2,21; Joh 3,16; Röm 1,2-7.16-17; 10,9-13; 1Kor 1,18; 2Kor 5,21).

Wir glauben, dass Gott alle Menschen zum Heil bestimmt hat (Hes 18,23; Röm 11,32; 1Tim 2,4; 2Pt 3,9). der Mensch kann sich zwar nur durch Gottes Wirken bekehren (Mt 11,27; Joh 3,27; 6,44.65; 1Kor 12,3), ist dann aber in seinem Willen gefordert, darauf einzugehen (Mk 16,16; Lk 9,23; Mt 23,37; 1Th 1,9). Gott hat bereits von Ewigkeit her in Seiner Allwissenheit die ersehen, die sich bekehren würden und all diejenigen dazu bestimmt, gerettet zu werden (Röm 8,29-30). Gott hat vor Grundlegung der Welt ausnahmslos jeden Christus dazu erwählt, gerettet zu werden, der sich Ihm zuwendet (Eph 1,4ff).

Wir glauben, dass jeder mit der Errettung aus Gnade in Christus sogleich für Zeit und Ewigkeit wiedergeboren und mit dem Heiligen Geist versiegelt wird. UNd als Miterbe Christi ist jedem von Gott Erretteten die Erlösung sowie ein Erbteil im Himmel sicher (Joh 3,3-7; 10,27-29; Röm 8,17; Eph 1,3-14; 2; Tit 3,5).

Wir glauben, dass jeder Erlöste in Christus von Gott völlig gerecht gesprochen und geheiligt ist (1Kor 6,11; Eph 1,4). Der Erlöste erhält die Gerechtigkeit Christi (1Kor 1,30), weil Christus alle Sünden und Schuld des Erlösten auf sich genommen und bezahlt hat (2Kor 5,21; Kol 2,13-15; 1Pt 2,24). Auch wenn der Erlöste in der Stellung vor Gott bereits völlig geheiligt ist (Hebr 10,10.14), ist er dennoch dazu aufgerufen, in der Praxis nach der Heiligung zu streben (1Th 4,3; Hebr 12,14; 1Pt 1,15), würdig der Berufung (Eph 4,1ff; vgl. Röm 12,1ff) und gemöss dem Bild Gottes und Seiner Schöpfungsordnung als Tempel des Heiligen Geistes zu leben (1Kor 6).
\subsection{Über die Gemeinde und Israel}
Wir glauben, dass alle erlösten Christen durch den Heiligen Geist Glieder am Leib Christi, der vor Grundlegung der Welt d Gemeinde, sind (1Kor 12,12-13; Eph 1-2).

Wir glauben, dass Gott lokale Gemeinden eingesetzt hat, damit sich die Christen zum Gottesdienst, zur Gemeinschaft und zur Unterrichtung in der Lehre dort versammeln und durch ihre Aufgaben und Gaben den Herrn verherrlichen, der Gemeinde dienen und das Evangelium in aller Welt verbreiten (Mt 28,18-20; Mk 19,15; Apg 2,42; 14,23.27; 1Kor 11,18-20; Eph 3,21; 4,7-16; 1Tim 3,15; 2Tim 2,2; Tit 1,5; Hebr 10,25).

Wir glauben, dass die Gemeinde Jesu Christi aus wiedergeborenen Juden und Heiden besteht (Gal 3,28; Eph 2,11-18; Kol 3,11) und Gottes ist (Röm 9,26; 1Pt 2,9-10). Doch wir glauben auch, dass die Gemeinde nicht das einzige Volk des Herrn ist. Wir glauben, dass Gott die Nation Israel nicht verworfen (Jer 31; Röm 11), sondern lediglich beiseitegestellt hat, bis die Gemeinde vollzählig ist.

Wir glauben, dass Gott heute bereits dabei ist, das Volk Israel in seinem Heimatland zu sammeln (Hes 22,17-22; 36,22-24; 37,1-13). Die Entstehung des Staates Israel 1948 zeigt uns, dass sich jahrtausendealte Prophetie erfüllt (vgl. Jes 11,11-12; Zef 2,1-2). Wir glauben, dass das Volk Israel bei der Wiederkunft Jesu in Herrlichkeit Ihn als den Messias erkennen, errettet werden und die Versprechungen erhalten wird, die Gott im Alten Testament gemacht und noch nicht eingelöst hat (Jes; Jer; Hes; Sach u.a.; vgl. Offb 21,5-6).

Wir glauben, dass bis zur Wiederkunft des Herrn geistlich gesehen jedoch nur die Juden \glqq wahre Juden\grqq{} und das \glqq Israel Gottes\grqq{} sind, die durch den Glauben an Jesus Christus aus Gnade errettet wurden (Röm 2,17-29; 5,11; Gal 6,16). Diese sind in Christus eins mit den Gläubigen aus den Nationen und gehören zum geistlichen Volk der Gemeinde (Eph 2,11-18).
\subsection{Über die Engel}
Wir glauben an die Existenz der heiligen Engel Gottes und der gefallenen Engel, nähmlich Satan und seinen Dämonen. Alle Engel sind erschaffene Wesen. Sie haben zwar einen höheren Rang als die Menschen und dürfen nicht gelästert, aber auch nicht angebetet werden (Eph 6,12; Hebr 1,6-7.14; Jud 8-9; 2Pt 2,10-11; Offb 5,11-14; 19,10; 22,9).
\subsection{Über die letzten Dinge}
Wir glauben, dass jeder Mensch sterben muss (Hebr 9,27); ausser jene, die entrückt werden. Die Seele des erlösten Menschen ist nach dem Tod sogleich in der Gegenwart Jesu (Phil 1,23) und der Mensch, der nicht an Christus geglaubt hat, ist im Totenreich (vgl. Lk 16,19-26).

Wir glauben, dass Jesus Christus eines Tages zurückkommen und Seine Gemeinde in den Wolken zu sich hin entrücken wird. Die Verstorbenen werden auferstehen und die noch Lebenden werden verwandelt. Jeder Christ erhält einen Auferstehungsleib, wird sich vor dem Preisgerichterstuhl Christi verantworten und für immer bei Christus sein (Joh 14,1-3; 1Kor 3,11-15; 15,51-53; 2Kor 5,10; 1Th 4,15-5,11). Wir glauben nach unserem Stand der Erkenntnis, dass die Entrückung vor dem Tag des Herrn erfolgen wird (1Th 1,10; 4,16-5,11; Tit 2,13) und dass sie jederzeit geschehen kann.

Wir glauben, dass nach der Entrückung ein antichristlicher Weltherrscher auftreten wird, der einen falschen Friedensbund mit Israel eingeht, und damit der Tag des Herrn (Gottes Gericht) eingeleitet wird (Jes 13,9ff; 28,14-29; Dan 9,27; 2Th 2,3ff.). In dieser Zeit wird Gott Seinen Zorm über die Erde ausgiessen und Israel für Jesu Wiederkunft vorbereiten (Jer 30,7; Dan 9,24-27; Joel 1-2; Zef; Mt 24,15-31; Mt 25,31-46; 2th 2,7-12; Offb 6-19).

Wir glauben, dass am Ende des Tages des Herrn der Herr Jesus Christus in grosser Macht und Herrlichkeit zusammen mit Seiner Gemeinde und Seinen Engeln zurückkommen und die Völker und Israel richten wird (Hes 20; Mt 25; Jud 14; Offb 19). Israel wird seinen Erlöser erkennen und gerettet werden (Jes 59, 19-20; Hes 21,30-32; Sach 14). Der Herr Jesus wird im national und geistlich wiederhergestellten Israel, in Jerusalem, eine tausendjährige Herrschaft über diese Erde antreten (jer 31,31-24; Dan 7,17-22; Mt 25,31; Offb 19,11-16; 20,1-7). Durch diese messianische Herrschaft weren alle noch ausstehenden Verheissungen für Israel erfüllt (Röm 11,2-2).

Wir glauben, dass Satan und seine gefallenen Engel (Dämonen) während des Tausendjährigen Friedensreiches Jesu Christi gebunden sind. Doch am Ende werden sie losgelassen, um die Nationen der Erde zu verführen (Offb 20,7ff.). Christus wird ihrer Rebellion mit Feuer eine Ende setzten und danach werden Satan und seine Engel in den Feuersee, die Hölle, geworfen (Mt 25,41; Offb 20,10). Dann werden alle Menschen, die nicht durch den Glauben an Christus errettet wurde, auferweckt und von Gott nach ihren Werken gerischtet und zu einer Strafe im Feuersee verurteilt (Offb 20,11-15).

Wir glauben, dass nach all diesem der alte Himmel und die alte Erde aufgelöst werden (2pt 3,10; Offb 20,11). Gott wird einen neuen Himmel und eine neue Erde schaffen, in der Gerechtigkeit wohnen und Gott mit allen Seinen Erlösten aller Zeitalter leben wird (Eph 5,5; Offb 21-22). Das himmlische Jerusalem wird vom Vater übergeben (1Kor 15,25-28). Dann werden alle Erlösten in ihren Auferstehungsleibern mit ihren auferstandenen Herrn für immer auf einer auferstandenen Erde leben.

\end{document}
