\documentclass[aspectratio=43]{beamer}

\makeatletter
\def\input@path{{../../inc/}}
\makeatother
% Pakete für das Dokument
\usepackage{blindtext}
\usepackage{beamerthememnrstyle}


% Eigenes Theme laden
\usetheme{mnrstyle}

% Eigene Definition von enumerate
% \setlist[enumerate, 2]{label=\roman*., format=\itshape, leftmargin=2cm}

% Metadaten für das Dokument
\title{Glaubensbekenntnis MNR}
\author{OTS}
\date{2025}

\begin{document}

% Titelframe
\begin{frame}
    \frametitle{MNR Naters}
    \maketitle    
\end{frame}

\begin{frame}{Inhalt}
    \begin{enumerate}
        \item \uppercase{über die bibel} $\longrightarrow$ \textit{06.04.2025}
        \item \uppercase{über gott und den Menschen}
        \item \uppercase{über die Erlösung}
        \item \uppercase{über die Gemeinde und Israel}
        \item \uppercase{über die Engel}
        \item \uppercase{über die letzten Dinge}
    \end{enumerate}
\end{frame}

% Frametitel setzen
\begin{frame}
    \frametitle{Über die Bibel}  % Frametitel
    \parbox{\textwidth}{%
        \vspace{0.55cm}
            \begin{enumerate}
                \item \textbf{Unfehlbarkeit der Bibel}
                    \begin{itemize}              
                        \item Vom Heiligen Geist inspiriert, Grundlage des Glaubens\\
                        (1Kor 2,13; 1Th 2,13; 2Tim 3:16; 2Pt 1,19-21; Mt 28,18-20; Apg 20,27)
                    \end{itemize}
                    % \pause
                    \vspace{0.25cm}
                \item \textbf{Erfüllung biblischer Prophezeiungen}
                    \begin{itemize}
                        \item Alle Prophezeiungen werden sich sicher erfüllen\\
                        (Jes 46,9-10; 55,11; Mt 5, 17-18; Röm 3,3-4;11; Offb 22,6-20)
                    \end{itemize}
                    % \pause
                    \vspace{0.25cm}
                \item \textbf{Einfache und klare Bibelauslegung}
                    \begin{itemize}
                        \item Verständlich durch den Heiligen Geist, keine mystische Deutung erforderlich\\
                        (Ps 33,4; Joh 5,31-47; 2Tim 3,14-17; Offb 1,3;Offb 1,3;22,6; Joh 7,17; Apg 1,18; 1Kor 2,7-15)
                    \end{itemize}    
                   
            \end{enumerate} 
            }
\end{frame}
\begin{frame}
%%%% LEER ENDE
\end{frame}
%%%%%%%%%%%%%%%%%%%%%%%%% Folie 3 %%%%%%%%%%%%%%%%%%%%%%%%%%%%%%%%
\begin{frame}
    \frametitle{Über Gott und den Menschen}  % Frametitel
    \vspace{0.80cm}
    \begin{enumerate}
        \item \textbf{Der eine ewige und vollkommene Gott}
            \begin{itemize}              
                \item Gott ist der Schöpfer aller Dinge und existiert ewig in drei Personen: Vater, Sohn und Heiliger Geist.\\
                (1Mos 1-3; 5Mos 6.4; Jes 45.5-7; Mt 28.19; 1. Kor 12.4-6; 2. Kor 13.13;)                
            \end{itemize}
            \pause
            \vspace{0.20cm}
        \item \textbf{Der Mensch und die Schöpfungsordnung Gottes}
            \begin{itemize}
                \item Der Mensch wurde nach Gottes Bild geschaffen, doch die Sünde brachte Trennung von Gott und Unordnung in die Schöpfung.\\
                (1 Mos 1.26-2-25; Jak 3.9; 1 Mos 2.16-17; Röm 1-3; Röm 5.10.19; Eph 2.1-3; Jak 4.4)
            \end{itemize}
            \pause
            \vspace{0.20cm}
        \item \textbf{Jesus Christus: ewiger Gottes Sohn und unser Erlöser}
            \begin{itemize}
                \item Jesus Christus wurde Mensch, nahm die Sünden der Welt auf sich und wurde durch Seine Auferstehung zum Weg der Erlösung.\\
                ( Jes 7.14; Mt 1.23; Lk 1.35; Röm 5; Gal 4.4; 2 Kor 5.21; Phil 2.6-11; Röm 1.4; Mk 16.19; Hebr 4.14; 1Joh 2.1)
            \end{itemize}              
    \end{enumerate}   
\end{frame}


\end{document}
