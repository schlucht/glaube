\documentclass[12pt]{mybib}

\begin{document}
\begin{table}[t]
    \centering
    \begin{tabular}{l|c | c}
        \hline
        \textbf{Name} & \textbf{Vorname} & \textbf{Anzahl} \\ [0.5ex]
        \hline\hline
      Hallo   &  Lothar & 120 \\
      Schmid   & Birgisch & 25
    \end{tabular}
    \caption{Beschreibung zur Tabelle}
    \label{tab:my_label}
\end{table}
\begin{quote}
    $f=1/3+25sin(25)$
\end{quote}
\begin{tikzpicture}
    % Achsen zeichnen
    \draw[->] (0,0) -- (4,0) node[right] {x-Achse};
    \draw[->] (0,0) -- (0,4) node[above] {y-Achse};
    
    % Ursprung markieren
    \filldraw[red] (0,0) circle (2pt) node[anchor=north east] {0,0};
    
    % Weitere Punkte einzeichnen
    \filldraw[blue] (2,2) circle (2pt) node[anchor=south west] {(2,2)};
    \filldraw[green] (-1,-1) circle (2pt) node[anchor=north east] {(-1,-1)};
\end{tikzpicture}
\end{document}
